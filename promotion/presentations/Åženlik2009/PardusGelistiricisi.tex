\documentclass{beamer}

\usepackage[utf8]{inputenc}
\usepackage[turkish]{babel}

\usepackage{beamerthemesplit}
\usepackage[amssymb,cdot]{SIunits}

\title{Nasıl Pardus Geliştiricisi Olunur ?}
\author[Onur Küçük \texttt{onur@pardus.org.tr}]{}
\date{Bilgi Üniversitesi, İstanbul\\ 18 Nisan, 2009}
\institute{
  Ulusal Elektronik ve Kriptoloji Araştırma Enstitüsü\\
  TÜBİTAK}

\setbeamertemplate{navigation symbols}{}
\usetheme{Warsaw}
\usecolortheme{seagull}

%images
\pgfdeclareimage[width=8cm]{2009}{2009}
\pgfdeclareimage[width=9cm]{2009tech}{2009tech}
\pgfdeclareimage[width=7cm]{bilesen}{bilesen}
\pgfdeclareimage[width=2cm]{pardus-logo}{logo}
\logo{\pgfuseimage{pardus-logo}}

\begin{document}

\frame{\titlepage}

\section{Kaynak Kodun Açık Olması}

\subsection{Kaynak Kodun Açık olması}
\frame
{
    \frametitle{Açık Kodlu Yazılım Geliştirme}
    \begin{itemize}
       \item Paylaşım
       \item İletişim
       \item Takım çalışması ve işbölümü
       \item Ortaya koyduğunuz ürünün kullanılmasından duyulan haz
    \end{itemize}
}

\subsection{Bir Linux Dağıtımının Parçaları}
\frame
{
    \frametitle{Bir Linux Dağıtımının Parçaları}
    \begin{itemize}
       \item Yapılandırma Araçları
       \item Paket Yönetim Sistemi
       \item Açılış Sistemi
       \item Kurulum Yazılımları
       \item Paketler, paketler, paketler
    \end{itemize}
}

\section{Pardus Teknolojileri}

\subsection{Pardus Nedir ?}
\frame
{
    \frametitle{Pardus Nedir ?}
    \begin{block}{Ana sözleşme}
        Pardus, UEKAE tarafından, bilişim okur-yazarlığına sahip bilgisayar kullanıcılarının temel masaüstü ihtiyaçlarını hedefleyerek; mevcut Linux dağıtımlarının üstün taraflarını kavram, mimari ya da kod olarak kullanan; otonom sisteme evrilebilecek bir yapılandırma çerçevesi ve araçları ile kurulum, yapılandırma ve kullanım kolaylığı sağlamak üzere geliştirilen bir GNU/Linux dağıtımıdır
    \end{block}
}

\subsection{Pardus Yaklaşımı}
\frame
{
    \frametitle{Pardus Yaklaşımı}
    \only<1>{
    \begin{itemize}
       \item Yapılandırma Araçları : \textbf{Çomar}
       \begin{itemize}
            \item Yapılandırma arayüzleri görev tabanlı olmalı
            \item Gündelik işler için komut satırı gerekmemeli
            \item Bilgisayar kendi işini kendi yapmalı
            \item Uygulamalar bir arada çalışabilmeli
       \end{itemize}
       \item Paket Yönetim Sistemi
       \item Açılış Sistemi
       \item Kurulum Yazılımları
       \item Grafik Arayüzler
    \end{itemize}
    }
    \only<2>{
    \begin{itemize}
       \item Yapılandırma Araçları : \textbf{Çomar}
       \item Paket Yönetim Sistemi : \textbf{Pisi}
       \begin{itemize}
            \item Geniş özellik kümesi, sorunsuz kurulum ve güncelleme
            \item Standartlara uygun, basit paket tanımları
            \item Geliştiriciler zamanlarını paket \emph{yapmaya} değil, paket sorunlarını \emph{sorunlarını} çözmeye harcamalı
       \end{itemize}
       \item Açılış Sistemi
       \item Kurulum Yazılımları
       \item Grafik Arayüzler
    \end{itemize}
    }
    \only<3>{
    \begin{itemize}
       \item Yapılandırma Araçları : \textbf{Çomar}
       \item Paket Yönetim Sistemi : \textbf{Pisi}
       \item Açılış Sistemi : \textbf{Müdür}
       \begin{itemize}
            \item Hızlı, basit
            \item Paralel servis açılışı desteği
            \item Yönetimi kolay
       \end{itemize}
       \item Kurulum Yazılımları
       \item Grafik Arayüzler
    \end{itemize}
    }
    \only<4>{
    \begin{itemize}
       \item Yapılandırma Araçları : \textbf{Çomar}
       \item Paket Yönetim Sistemi : \textbf{Pisi}
       \item Açılış Sistemi : \textbf{Müdür}
       \item Kurulum Yazılımları : \textbf{Yalı}
       \begin{itemize}
           \item Görev temelli, insan odaklı
           \item PiSi ve Çomar altyapısı
           \item Ufak ve hızlı
           \item \emph{Herkes için kolay ve hızlı bir kurulum deneyimi} 
       \end{itemize}
       \item Grafik Arayüzler
    \end{itemize}
    }
    \only<5>{
    \begin{itemize}
       \item Yapılandırma Araçları : \textbf{Çomar}
       \item Paket Yönetim Sistemi : \textbf{Pisi}
       \item Açılış Sistemi : \textbf{Müdür}
       \item Kurulum Yazılımları : \textbf{Yalı}
       \item Grafik Arayüzler : \textbf{Tasma}
       \begin{itemize}
           \item Ağ Yöneticisi
           \item Paket Yöneticisi
           \item Görüntü Yöneticisi
           \item Kullanıcı Yöneticisi
           \item Açılış Yöneticisi
           \item Disk Yöneticisi
           \item Geçmiş Yöneticisi
           \item Servis Yöneticisi
           \item Kaptan
       \end{itemize}
    \end{itemize}
    }
    \only<6>{
    \begin{center}{\pgfuseimage{2009}}\end{center}
    }
}

\section{Nası Katkı Sağlarım ?}

\subsection{Kullanıcı Olarak Yapılabilecekler}
\frame
{
    \frametitle{Kullanıcı Olarak Yapılabilecekler}
	\begin{itemize}
        \item Çeviri çalışmaları
        \item Yeni kullanıcılara yardımcı olmak;
        \begin{itemize}
            \item E-Posta Listeleri : \textbf{liste.pardus.org.tr}
            \item Forumlar
            \item IRC
            \item Komünite Siteleri : \textbf{www.ozgurlukicin.com}
        \end{itemize}
        \item Belge yazmak
        \item Tema hazırlamak
        \item Hata bildirmek : \textbf{hata.pardus.org.tr}
	\end{itemize}
}

\subsection{Geliştirici Olarak Yapılabilecekler}
\frame
{
    \frametitle{Geliştirici Olarak Yapılabilecekler}
	\begin{itemize}
        \item Teknoloji Geliştirmek
        \begin{itemize}
            \item Yeni teknolojiler geliştirmek
            \item Mevcut teknolojileri geliştirmek
            \begin{itemize}
                \item Yeni özellikler eklemek
                \item Hataları gidermek
            \end{itemize}
        \end{itemize}
        \item Paket Geliştirmek
        \begin{itemize}
            \item Bakıcısız paketleri devralmak
            \item Mevcut paketlerin hatalarını gidermek
            \item Yeni paket isteklerine cevap vermek
        \end{itemize}
	\end{itemize}
}

\subsection{Geliştirici Araçları}
\frame
{
    \frametitle{Geliştirici Araçları}
	\begin{itemize}
        \item Programlama dilleri : \textbf{python, c/c++, asm, java, ruby, perl, scheme ...}
        \item Editörler : \textbf{vim, emacs, kate, eclipse, ed ...}
        \item Sürüm kontrol sistemleri : \textbf{svn, git, mercurial ...}
        \item Yorumlayıcılar : \textbf{ipython, iruby ...}
        \item Grafik arayüz araçları : \textbf{qt-designer, qt-creator, glade ...}
        \item GNU araçları : \textbf{bash, sed, grep ...}
        \item Derleyiciler : \textbf{gcc, g++ ...}
        \item Hata ayıklayıcılar : \textbf{gdb, pdb, kdbg ...}
        \item Takip araçları : \textbf{strace, ltrace ...}
	\end{itemize}
}

\section{Öğrenciler için fırsatlar}

\subsection{Pardus'ta Staj}
\frame
{
    \frametitle{Pardus'ta Staj}
    \begin{block}{Staj}
        Pardus 2009 yazında da genç geliştirici adaylarına, kendi projelerini Pardus çatısı altında oluşturma fırsatı veriyor. Tamamlanan staj projelerinin Pardus ürünü ile dağıtılacak ya da kullanıcılara ulaştırılacak alt ürünler ve/veya iyileştirmeler olması hedeflenmekte.
        Staj programına başvurmanız için özgeçmişinizi, ilgi alanlarınızı, önümüzdeki günlerde açıklanacak proje başlıklardan hangisinde çalışmak istediğinizi ve bu projelerle ilgili fikir ve önerilerinizle birlikte eğer varsa daha önce üzerinde çalıştığınız projelerle bunlara ait incelenmesi mümkün kaynak kodların yer aldığı detaylı bir CV'yi \textbf{bilgi@pardus.org.tr} adresine göndermeniz gerekmekte.
    \end{block}
}

\subsection{Google Summer of Code}
\frame
{
    \frametitle{Google Summer of Code}
    \begin{block}{GSOC}
        2005 yılından beri 100 ülkeden 2500 öğrenci bu öğrencilere hocalık yapan 2500 mentor (sorumlu) Google tarafından destekleniyor ve özgür yazılım dünyasına hatırı sayılır katkılar sağlıyor. Pardus geçen sene olduğu gibi bu yılda kabul edilen projelerin arasında yerini aldı. Öğrencilere projelerini başarılı bir şekilde bitirmeleri halinde 4500 \$ ödül verecek olan GSOC'u takip etmek için \textbf{socghop.appspot.com} adresinden ayrıntılı bilgi alabilirsiniz.
    \end{block}
}


%\section{The End}
\frame
{
	\frametitle{Teşekkürler}
	\begin{itemize}
        \item Sorular ?
	\end{itemize}

	\begin{itemize}
        \item Yeni başlarken : \textbf{www.ozgurlukicin.com}
        \item E-Posta Listeleri : \textbf{liste.pardus.org.tr}
        \item Hata takip sistemi : \textbf{hata.pardus.org.tr}
        \item Komünite Wiki : \textbf{tr.pardus-wiki.org}
        \item Geliştirici adayları : \textbf{developer.pardus.org.tr}
	\end{itemize}

}

\end{document}
