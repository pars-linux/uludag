\documentclass[a4paper,11pt]{article}

\title{P\'IS\'I Packages: Version Policy}
\date{\today}
\author{T. Bar\i{}\c s Metin}

\begin{document}
\maketitle

\section{Introduction}

We'll document the \emph{version policy} that applies to the P\'IS\'
packages.

\section{Source Version}

Source version is the version number provided is provided by the
upstream maintainer of the source archive used in package. It must be
always the same as the upstream version used.

\textbf{Example}: If the upstream archive name is
\emph{bash-3.0.tar.gz} the version number of the package is \emph{3.0}

\subsection{Version Suffixes}

There is a pre-defined list of suffixes a package version can
take.

\begin{itemize}
  \item \textbf{alpha} Source/Package is in alpha state
  \item \textbf{beta} Source/Package is in beta state
  \item \textbf{pre} Source/Pacgage passed the beta state but stable
    version is not relased yet.
  \item \textbf{rc} Source/Package is a release-candidate.
  \item \textbf{p} Source/Package is released and some patches are
    applied after the release. This is the patch level.
\end{itemize}

The suffix should be written after the special separator
character \textbf{\_}. And there must allways be a number after a
suffix. \textbf{Example}: packagename-1.0\_beta1

The basic order of the priorities for suffixes is:\newline
\emph{p $>$ (no suffix) $>$ rc $>$ pre $>$ beta $>$ alpha}.


\section{Release Number}

Release number is the number of the changes that is made to the
package source. A change can be a patch applied to the source archive,
modification in the actions.py, pspec.xml or any file in the source
package directory.

\emph{Detailed explanation is needed}

\section{Build Number}

\emph{Detailed explanation is needed}


\end{document}
