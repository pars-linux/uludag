\documentclass[a4paper,10pt]{article}
\usepackage[left=1cm,top=1cm,right=2cm,bottom=1cm]{geometry}
\usepackage[turkish]{babel}
\usepackage[utf8]{inputenc}
\title{Kullanıcı Yöneticisi Test Aşamaları}
\author{Semen Cirit}
\renewcommand{\labelenumi}{\arabic{enumi}.}
\renewcommand{\labelenumii}{\arabic{enumi}.\arabic{enumii}.}
\renewcommand{\labelenumiii}{\arabic{enumi}.\arabic{enumii}.\arabic{enumiii}.}
\renewcommand{\labelenumiv}{\arabic{enumi}.\arabic{enumii}.\arabic{enumiii}.\arabic{enumiv}.}

\begin{document}

\maketitle

\begin{enumerate}

\item Kullanıcı yöneticisini Menü $\rightarrow$ Sistem ayarları yolunu izleyerek açmaya çalışın.
	
Açılabildiğini gözlemleyin.

\item Menü $\rightarrow$ Sistem Kullanıcı yöneticisini Menü $\rightarrow$ sistem yolunu izleyerek açmaya çalışın.

Açılabildiğini gözlemleyin.

\item Yeni ekle düşen kutusundan kullanıcı ekleyi seçin.

   Ve yeni bir kullanıcı ekleyin.
    Aşağıdaki komutu çalıştırın:
    \begin{verbatim}
        $ getent passwd 
    \end{verbatim}
    Kullanıcının eklenmiş olduğunu gözlemleyin.

    \begin{enumerate}
    \item Parolayı değiştirmeye çalışın.
        \begin{enumerate}
        \item Parola ve parola onay alanlarına aynı olmak üzere yeni bir parola yazın.
	
	Parolanın değiştirilebildiğini gözlemleyin.

	Bilgisayarınızı restart edip yeni şifre ile login olabildiğinizi gözlemleyin. 
        \item Parola alanına yeni bir parola ve parola onay alanına farklı bir parola yazın. 

	Bunun için bir uyarı vermiş olduğunu ve şifre değiştirmeye izin vermediğini gözlemleyin.
	
        \end{enumerate}
    \end{enumerate}


    \item Yönetici hakları ver'i seçin

          Yeni eklenen kullanıcı ile giriş yapın ve bu kullanıcı ile root olmaya çalışın. 

	  Root olabildiğinizi gözlemleyin.
    \item Yönetici yetkilerini geri alın.

        Yeni eklenen kullanıcı ile giriş yapın ve bu kullanıcı ile root olmaya çalışın. 
	
	Root olamadığınızı gözlemleyin.

    \item Listeden bir yetki seçin ve kullacıyı yetkilendirin.

        Aşağıdaki komutu çalıştırın: (root olarak)
    \begin{verbatim}
        # cat /var/lib/PolicyKit/user-<username>.auths
    \end{verbatim}
        Kullanıcıya bu yetkinin verilmiş olduğunu gözlemleyin.

    
\item Yeni ekle düşen kutusundan Grup ekle'yi seçin.

     Ve bu bir grup ekleyin.
     Aşağıdaki komutu çalıştırın:
\begin{verbatim}
     cat /etc/group
 \end{verbatim}
     Grubun eklenmiş olduğunu gözlemleyin.

\item Kullanıcıları listeledikten sonra, bir kullanıcı seçin. 

  Eklediğiniz grubu bu kullanıcı için seçin.

          Aşağıdaki komutu çalıştırın:
    \begin{verbatim}
        $ groups <username> 
    \end{verbatim}

        Kullanıcının bu verdiğiniz grupta olduğunu gözlemleyin.

\item Bir grubu silin.

\begin{enumerate}
    \item Aşağıdaki komutu çalıştırın:

\begin{verbatim}
     $ getent group
\end{verbatim} 

     Grubun silindiğini gözlemleyin.

    \item Bu gruba eklenmiş kullanıcıyı inceleyin.

     Aşağıdaki komutu çalıştırın:
\begin{verbatim}
     $ groups <username>
\end{verbatim}
     Kullanıcın artık bu gruba ait olmadığını gözlemleyin.
\end{enumerate}

\item Kullanıcıyı bir gruptan çıkarın.

       Aşağıdaki komutu çalıştırın:
    \begin{verbatim}
        $ groups <username> 
    \end{verbatim}
        Kullanıcın artık bu gruba ait olmadığını gözlemleyin.

\item Yeni eklenmiş kullanıcının bilgilerini değiştirmeye çalışın.

Bu durumu gözlemleyin.
\item Kullanıcıyı silmeye çalışın.

      Aşağıdaki komutu çalıştırın:
\begin{verbatim}
     $ getent passwd
\end{verbatim}
     Kullanıcının silindiğini gözlemleyin.

\item Tüm kullanıcıları listelemeye çalışın.

Sorunsuz bir şekilde listelendiklerini gözlemleyin.
\item Tüm grupları listelemeye çalışın.

Sorunsuz bir şekilde listelendiklerini gözlemleyin.
\item Kullanıcı ekleme bilgi değiştirme gibi işlemleri, parolamı hatırla dedikten sonra tekrar deneyin.

Ve sorunsuz bir şekilde istediğiniz işlemin gerçekleştiğini gerekli komut çıktılarını kullanarak gözlemleyin.

\end{enumerate}

\end{document}


