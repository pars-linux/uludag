\documentclass[a4paper,10pt]{article}
\usepackage[turkish]{babel}
\usepackage[utf8]{inputenc}
\usepackage[left=1cm,top=1cm,right=2cm,bottom=2cm]{geometry}

\title{Network Test Cases}
\author{Semen Cirit}

\renewcommand{\labelenumi}{\arabic{enumi}.}
\renewcommand{\labelenumii}{\arabic{enumi}.\arabic{enumii}.}
\renewcommand{\labelenumiii}{\arabic{enumi}.\arabic{enumii}.\arabic{enumiii}.}
\renewcommand{\labelenumiv}{\arabic{enumi}.\arabic{enumii}.\arabic{enumiii}.\arabic{enumiv}.}

\begin{document}

\maketitle
\section{p2p sub Component}
\begin{enumerate}
 \item Following packages are subject to installation test:

\begin{verbatim}
gift
gift-ares
gift-fasttrack
gift-gnutella
gift-openft
giftcurs
\end{verbatim}

 \item After installation linuxdcpp package:
Enter a nick from Preferences tab. 

Open the application from Kmenu and try to connect a crowded hub. Observe that connection established without any problem.
\end{enumerate}
\section{Download sub component}
\begin{enumerate}
 \item After installation youtube-dl package:

Execute following command and observe a file with .flv extension is downloaded.
\begin{verbatim}
# youtube-dl http://www.youtube.com/watch?v=5u2q3P60XEk
\end{verbatim}

\end{enumerate}

\section{Chat sub component}
\begin{enumerate}
\item After choqok package:

Run the program and if you have a twitter account, save your account information and observe you will be connected twitter correctly.	
\item After konversation and konversation-docs packages:

Run the program and observe you can connect default channels.

\item After installation pidgin package:
Run the program and save your account information and observe you can connect correctly.

\end{enumerate}
\section{Web sub component}
\begin{enumerate}
\item After installation links package:

  Run the below command and observe that the google main page opened on console.
	\begin{verbatim}
	 # links www.google.com
	\end{verbatim}

\item After installation bilbo package:

Open the application from Kmenu and enter a blog title and subject and save it locally. Observe that it runs correctly.

If you have a blog, follaw Blog $\rightarrow$ Add Blog path and add your bolg and try to send a new blog.

\item After installation firefox and arora package:
\begin{itemize}
 \item Observe buttons work correctly on page below.
	\begin{verbatim}
	 http://www.croczilla.com/~alex/old-site/dom2.xml
	\end{verbatim}
 \item (for Firefox) Observe .firefox directory which is on Home directory had not regenerated.
	
	Observe your bookmarks had not lost.
	
 	Observe pages which are opened recently opens corrently when reload.

\item Open the site below and try watch video full screen, observe sound and display are correct.
	\begin{verbatim}
	http://www.dailymotion.com/video/x3akre_loreena-mckennitt-all-souls-night-l 
	\end{verbatim}
\item Try download a document under \begin{verbatim}http://cekirdek.pardus.org.tr/~semen/dist/test/office/openoffice/\end{verbatim} and observe the save as window is appeared.

Download this file and observe the download window is appeared correctly.
\item Run a few video from \begin{verbatim}http://cekirdek.pardus.org.tr/~semen/dist/test/multimedia/video/cokluortam/\end{verbatim} and observe they can run on firefox.

\end{itemize}

\end{enumerate}

\section{Monitor sub component}
\begin{enumerate}
 \item After installation wireshark package:
	Run Wireshark, select eth0 from interface list and observe related packages with this interface will be listed.
\end{enumerate}

\section{Mail sub component}
\begin{enumerate}
 \item After installation thunderbird and sylpheed packages:
\begin{itemize}
\item Create an e-mail account.
\item Observe can receive Mail.
\item Observe can send Mail.
\item Observe can create filter.
\item If we are already using thunderbird, observe the .thunderbird directory is not deleted under Home.
\end{itemize}
 \item After spamassassin and spamd packages:
\begin{enumerate}
	\item After installation related packages:
	\item Run Applications $\rightarrow$ Kmail from Menu and for to activate spam filter;
		
	\begin{enumerate}
		\item Follow Tools $\rightarrow$  Spam Blocking Wizard path from Kmail menubar.
		\item Open the spam filter wihich is you downloaded.
		\item Right click an email and follow Apply filter $\rightarrow$ Categorize filter as junk mail path from combobox.
		Observe this spam is gone to spam folder.( If you have not change, default spam folder is recycle folder.)

		\item Donwload gtube.txt from link below.
		\begin{verbatim}
 		http://spamassassin.apache.org/gtube/
		\end{verbatim}
		\item  Execute the following command:
		\begin{verbatim}
 		cat  gtube.txt | spamc 
		\end{verbatim}
		
		The commmand will return an output like this:
		
		\emph{If your spam filter supports it, the GTUBE provides a test by which you
	    	can verify that the filter is installed correctly and is detecting incoming
    		spam. You can send yourself a test mail containing the following string of
    		characters (in upper case and with no white spaces and line breaks):}
		\begin{verbatim}
 		XJS*C4JDBQADN1.NSBN3*2IDNEN*GTUBE-STANDARD-ANTI-UBE-TEST-EMAIL*C.34X
		\end{verbatim}
    		\emph{You should send this test mail from an account outside of your network.}

		\item After that copy this encrypted part and mail to yourself.
		
		Obverse this mail is gone to spam folder directly.
	\end{enumerate} 
\end{enumerate} 
\end{enumerate}
\section{Plugin sub component}
\begin{enumerate}
\item After installation flashplugin package:

Install the Opera package and observe video is worked correctly which is on link below.

Can it be full-screen? Is there any sound problem?
\begin{verbatim}
http://www.dailymotion.com/relevance/search/lorena+mckennit/video/xd9s3_princesse
-mononoke-studioslorenna 
\end{verbatim}
\item After installation gecko-mediaplayer package:
\begin{itemize}
  \item Observe gecko-mediaplayer is being added on Firefox $\rightarrow$ Edit $\rightarrow$ Options $\rightarrow$ Manage add-ons $\rightarrow$ Plugins.
  \item Run videos on firefox which are in directory below and observe they run correctly.
  \begin{verbatim}
  http://cekirdek.pardus.org.tr/~semen/dist/test/multimedia/video/cokluortam/  
  \end{verbatim}
\end{itemize}
\end{enumerate}

\section{Ftp sub component}
\begin{enumerate}
 \item After installation lftp package:
\begin{verbatim}
 # lftp http://ftp.pardus.org.tr/pub/
 # ls 
 # cd pardus
\end{verbatim}
Observe these connands run correctly.

\end{enumerate}

\section{Connection sub component}
\begin{enumerate}
\item After installation ifplugd package:

Observe that the network manager can list the networks and can activate and inactivate the network without any problem.

 \item Following packages subject to installation test.
\begin{verbatim}
iw
wireless-regdb
\end{verbatim}


\end{enumerate}


\end{document}

