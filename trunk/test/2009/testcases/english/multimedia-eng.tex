\documentclass[a4paper,10pt]{article}
\usepackage[turkish]{babel}
\usepackage[utf8]{inputenc}
\usepackage[left=1cm,top=1cm,right=2cm,bottom=2cm]{geometry}

\title{Multimedia Test Cases}
\author{Semen Cirit}

\renewcommand{\labelenumi}{\arabic{enumi}.}
\renewcommand{\labelenumii}{\arabic{enumi}.\arabic{enumii}.}
\renewcommand{\labelenumiii}{\arabic{enumi}.\arabic{enumii}.\arabic{enumiii}.}
\renewcommand{\labelenumiv}{\arabic{enumi}.\arabic{enumii}.\arabic{enumiii}.\arabic{enumiv}.}

\begin{document}

\maketitle

\section{Sound sub component}
\begin{enumerate}
\item After installation frescobaldi package:
 \begin{verbatim}
  # wget http://cekirdek.pardus.org.tr/~semen/dist/test/multimedia/sound/test_frescobaldi.ly
 \end{verbatim}
Open the file above with frescobaldi, click the Lilypond button at leftside and observe that a pdf document is generated.
\item After installation pulseaudio package:
\begin{itemize}
 \item Restart your system and be sure you hear opening sound.
 \item Do the multimedia-eng.pdf amarok package test.
\end{itemize}
\item After installation sox package:

Do the hardware-eng.pdf k3b test.
\item After installation amarok package:
 
Download the zip file below and observe sound is clear.
 \begin{verbatim}
  # wget wget http://cekirdek.pardus.org.tr/~semen/dist/test/multimedia/sound/sound.tar
 \end{verbatim}

\item After installation listen package: 

Run the program, go to /usr/kde/3.5/share/sounds/ directory from file manager and observe musics are listed correctly.

\item After installation vorbis-tools package: 
 \begin{verbatim}
  # wget http://cekirdek.pardus.org.tr/~semen/dist/test/multimedia/sound/sound/game.ogg
  # oggdec game.ogg
  # mplayer game.wav
  # oggenc game.wav
  # mpleyer game.ogg
 \end{verbatim}

\item After installation qpitch package:

Run the program from Kmenu and observe it works correctly.

\item After installation qjackctl package: 

Run qjackct from Kmenu. (meanwhile be sure that all sound devices are not open.)

Click Start button, observe Jack sound server has been started from status and messages.

\item After installation lame ve lame-docs packages: 

Observe following commands are worked correctly:
\begin{verbatim}
# wget http://cekirdek.pardus.org.tr/~semen/dist/test/multimedia/sound/sound/music.mp3
# lame music.mp3 music.mpeg
# mplayer music.mpeg
\end{verbatim}

\item After installation mpg123 package:

Observe following commands are worked correctly:
 \begin{verbatim}
  # wget http://cekirdek.pardus.org.tr/~semen/dist/test/multimedia/sound/sound/music.mp3
  # mpg123 music.mp3
 \end{verbatim}


\end{enumerate}

\section{Video sub component}
\begin{enumerate}
 \item After installation mplayer, vlc, kaffeine packages:
 \begin{verbatim}
  # wget http://cekirdek.pardus.org.tr/~semen/dist/test/multimedia/video/cokluortam.tar
 \end{verbatim}
Download the file and run all type files with the program. Observe it work correctly.
 \item After installation vlc-firefox package:
 \begin{itemize}
  \item Observe vlc plugin is being added on Firefox $\rightarrow$ Edit $\rightarrow$ Options $\rightarrow$ Manage add-ons $\rightarrow$ Plugins.
  \item Open the files on firefox at that link below and observe they work correctly.
  \begin{verbatim}
  http://cekirdek.pardus.org.tr/~semen/dist/test/multimedia/video/cokluortam/  
  \end{verbatim}
 \end{itemize}
\item After installation ffmpeg package:
 
Observe the second command returns correct outputs and third command can run those outputs.
\begin{verbatim}
  # wget http://cekirdek.pardus.org.tr/~semen/dist/test/multimedia/video/cokluortam.tar 
  # ffmpeg -i <multimedia dosyası> -r 24 <test dosyası>
  # ffplay <multimedia dosyası>
  \end{verbatim}
\item After installation x264 package:

Observe following commands are worked correctly:
\begin{verbatim}
# wget  http://cekirdek.pardus.org.tr/~semen/dist/test/multimedia/video/example.y4m.bz2
# x264 -o test.mp4 example.y4m.bz2 300x300 
# mplayer test.mp4
\end{verbatim}

\end{enumerate}

\section{Converter sub component}
\begin{enumerate}
 \item Following packages are related only installation test:
\begin{verbatim}
 nrg2iso
 vnc2swf
\end{verbatim}

 \item After installation amrwb and amrnb packages:

 Do multimedia-eng.pdf sox and mplayer tests.

\item After installation ccd2iso package:
\begin{verbatim}
 # wget http://cekirdek.pardus.org.tr/~semen/dist/test/multimedia/converter/default.img
 # ccd2iso default.img test.iso
\end{verbatim}

Observe the test.iso file is generated correctly.

\item After installation dvdbackup package:
\begin{itemize}
 \item Burn the following iso to DVD. 
\begin{verbatim}
 # wget http://cekirdek.pardus.org.tr/~semen/dist/test/hardware/optical/boot.iso
\end{verbatim}
 \item If DVD is RW, follow /dev/dvdrw. if not, follow /dev/dvd inputs and execute following command and observe DVD is backuped correctly.
\begin{verbatim}
 # dvdbackup -i <input_path> -o <output_path> -M
 Example 
 # dvdbackup -i /dev/dvdrw -o /home/pardus/dvd -M
\end{verbatim}
 
\end{itemize}
\item After installation emovix package:

Download the following file:
\begin{verbatim}
 # wget http://cekirdek.pardus.org.tr/~semen/dist/test/multimedia/converter/default.img
\end{verbatim}
 
Observe following commands run correctly.
\begin{verbatim}
 # movix-version
 # movix-files
 # movix-conf
 # mkmovixiso default.img --output-file=default.iso
\end{verbatim}

\item After installation ffmpeg2theora package:
\begin{verbatim}
 # wget http://cekirdek.pardus.org.tr/~semen/dist/test/multimedia/video/cokluortam/DVD.mpg
 # ffmpeg2theora DVD.mpg
\end{verbatim}
\begin{itemize}
 \item Observe the command above can create DVD.ogv correctly.
 \item Run the file with mplayer and observe it works correctly.
\end{itemize}
\item After installation icns2png package:
\begin{verbatim}
 # wget http://cekirdek.pardus.org.tr/~semen/dist/test/multimedia/converter/lazarus.png
 # icns2png lazarus.icns
\end{verbatim}
\begin{itemize}
 \item Observe that command generate lazarus.png correctly.
 \item Open this file with gwenview and observe it opens corretly.
\end{itemize}
\item After installation kaudiocreator package:
\begin{itemize}
 \item Burn an auido CD with k3b by using following sound files.
\begin{verbatim}
 # wget http://cekirdek.pardus.org.tr/~semen/dist/test/multimedia/video/cokluortam/
\end{verbatim}
 \item Run the kaudiocreator:

Observe it runs correctly and it can list files on CD.

Select the files and click Rip button.

Observe a folder is generated on your Home directory (its name can be mp3 or wav or ogg.) and observe these files can be run with mplayer correctly.
\end{itemize}

\item After installation libnut package:

Do multimedia-eng.pdf mplayer and ffmpeg tests.

\item After installation mkvtoolnix package:

Run Applications $\rightarrow$ Multimedia $\rightarrow$ mkvmerge GUI program:

Use the program to convert following files to .mkv format and observe the files can run on mplayer: (Click Add button to add the file and click to start muxing button.)
\begin{verbatim}
  # wget http://cekirdek.pardus.org.tr/~semen/dist/test/multimedia/video/cokluortam.tar
 \end{verbatim}

\item After installation mpeg2vidcodec package:
\begin{verbatim}
  # mkdir flower
  # cd flower
  # wget http://cekirdek.pardus.org.tr/~semen/dist/test/multimedia/converter/flowgard.mpg
  # mpeg2decode -b flowgard.mpg -f -r -o0 sflowg.%d
  # cd ..
  # wget http://cekirdek.pardus.org.tr/~semen/dist/test/multimedia/converter/flower2.par
  # mpeg2encode flower2.par flowgard.m2v
  # mplayer flowgard.m2v
\end{verbatim}

Observe the commands above work correctly.
\item After installation ogmtools package:

Observe the following commands work correctly.
\begin{verbatim}
# wget http://cekirdek.pardus.org.tr/~semen/dist/test/multimedia/sound/sound/music.mp3 
# ogmmerge music.mp3 -o test.ogg
# mplayer test.ogg
\end{verbatim}

\item After installation potrace package:
\begin{verbatim}
# wget http://cekirdek.pardus.org.tr/~semen/dist/test/multimedia/converter/tepecik_01.pbm 
# potrace tepecik_01.pbm -o test.png
# gwenview test.png
\end{verbatim}

Observe a png file is generated and it is displayable.

\item After installation shntool package:
\begin{verbatim}
# wget http://cekirdek.pardus.org.tr/~semen/dist/test/multimedia/sound/sound/11k16bitpcm.wav
# wget http://cekirdek.pardus.org.tr/~semen/dist/test/multimedia/sound/sound/11k16bitpcm2.wav
# shncat 11k16bitpcm.wav
# shncmp 11k16bitpcm.wav 11k16bitpcm2.wav
\end{verbatim}
\item After installation shorten package:
\begin{verbatim}
 # wget http://cekirdek.pardus.org.tr/~semen/dist/test/multimedia/sound/sound/11k16bitpcm.wav
 # shorten 11k16bitpcm.wav
 # mplayer 11k16bitpcm.shn
\end{verbatim}

Observe a .shn file is generated and it is executable.
\item After installation transcode package:

\begin{itemize}
 \item Do the hardware-eng.pdf k3b test.
 \item Execute following commands:
\begin{verbatim}
# wget http://cekirdek.pardus.org.tr/~semen/dist/test/multimedia/video/cokluortam/
Lake_dance_XviD.AVI
# transcode -i Lake_dance_XviD.AVI -y xvid -o test.avi -k -z
# mplayer test.avi
\end{verbatim}
 Observe the test.avi works reserved.
\end{itemize}

\item After installation vcdimager package:

\begin{itemize}
 \item Do the hardware-eng.pdf k3b test.
 \item Execute following commands and onserve they work correctly:
\begin{verbatim}
# wget http://cekirdek.pardus.org.tr/~semen/dist/test/multimedia/video/cokluortam/DVD.mpg
# vcdimager DVD.mpg
# vcd-info -b videocd.bin
# vcdxgen DVD.mpg
# vcdxminfo -i DVD.mpg
\end{verbatim}

\end{itemize}
\item After installation vobcopy package:

\end{enumerate}
\section{Graphics sub component}
\begin{enumerate}
\item After installation jasper package:

\begin{verbatim}
 # wget http://cekirdek.pardus.org.tr/~semen/dist/test/multimedia/graphics/test_jasper.jpg
 # jiv test_jasper.jpg
 # jasper --input test_jasper.jpg --output test.jp2 --output-format jp2
 # jiv test.jp2
\end{verbatim}


 \item After installation gocr package:

Execute following commands and observe the application can scan characters and write on a file named test.
\begin{verbatim}
 # wget http://cekirdek.pardus.org.tr/~semen/dist/test/multimedia/graphics/font1.pbm.gz
 # gocr -i font1.pbm.gz -o test
 # vi test
\end{verbatim}

\item After installation graphviz package:

Execute following commands and observe they work correctly.
\begin{verbatim}
# wget http://cekirdek.pardus.org.tr/~semen/dist/test/multimedia/graphics/test_graphviz.mm
# mm2gv  test_graphviz.mm -o test.gv
# dotty test.gv
# gv2gxl test.gv -o test.gxl 
# gxl2dot test.gxl test.dot
# acyclic test.dot test_asyclic.dot
# lneato test.dot
# vimdot test.dot
\end{verbatim}



 \item Do only installation for gimp-data-extras package.
 \item After installation packages below, change your local language and open gimp at same directory on console and observe language is changed.

For change local language:
\begin{verbatim}
export LC_ALL= <lang_LANG>
\end{verbatim}

lang\_LANG is format, for example: pt-BT is pt\_BT.

After taht, execute gimp command and observe laguage is changed to which you decide.
\begin{verbatim}
gimp-i18n-es
gimp-i18n-sk
gimp-i18n-sl
gimp-i18n-sr
gimp-i18n-sr_Latn
gimp-i18n-sv
gimp-i18n-ta
gimp-i18n-th
gimp-i18n-tt
gimp-i18n-uk
gimp-i18n-vi
gimp-i18n-et
gimp-i18n-eu
gimp-i18n-fa
gimp-i18n-fi
gimp-i18n-fr
gimp-i18n-ga
gimp-i18n-gl
gimp-i18n-gu
gimp-i18n-he
gimp-i18n-hi
gimp-i18n-xh
gimp-i18n-yi
gimp-i18n-zh_CN
gimp-i18n-zh_HK
gimp-i18n-zh_TW
gimp-i18n-hr
gimp-i18n-hu
gimp-i18n-id
gimp-i18n-is
gimp-i18n-it
gimp-i18n-ja
gimp-i18n-ka
gimp-i18n-km
gimp-i18n-kn
gimp-i18n-ko
gimp-i18n-lt
gimp-i18n-lv
gimp-i18n-mk
gimp-i18n-ml
gimp-i18n-mr
gimp-i18n-ms
gimp-i18n-nb
gimp-i18n-ne
gimp-i18n-nl
gimp-i18n-nn
gimp-i18n-oc
gimp-i18n-or
gimp-i18n-pa
gimp-i18n-pl
gimp-i18n-pt
gimp-i18n-pt_BR
gimp-i18n-ro
gimp-i18n-ru
gimp-i18n-rw
gimp-i18n-si 
\end{verbatim}


 \item After installation GraphicsMagick package:

Do the office-eng.pdf koffice-krita test.

 \item After installation İmageJ package:

 Run the program from Kmenu and observe the file below is openable by following File $\rightarrow$ Open path.
  \begin{verbatim}
   # wget http://cekirdek.pardus.org.tr/~semen/dist/test/office/openoffice/test_oodraw.jpg
  \end{verbatim}

\item After installation autotrace package:

Execute following commands and observe they work correctly.
  \begin{verbatim}
   # wget http://cekirdek.pardus.org.tr/~semen/dist/test/multimedia/graphics/bmp_24.bmp
   # autotrace bmp_24.bmp -output-file test.eps  -output-format eps
   # gwenview test.eps 
  \end{verbatim}

 \item After installation gimp and gimp-devel packages:
  \begin{verbatim}
   http://cekirdek.pardus.org.tr/~semen/dist/test/multimedia/graphics/graphics.tar
  \end{verbatim}

 Open the files at link above with gimp and observe they open correctly.
\item After installation digikam package:

  \begin{verbatim}
   http://cekirdek.pardus.org.tr/~semen/dist/test/multimedia/graphics/graphics.tar
  \end{verbatim}

 Copy the files above to directory which is you selected for Digikam and observe it works correctly.
\item After installation imagemagick package:
  \begin{verbatim}
   # wget http://cekirdek.pardus.org.tr/~semen/dist/test/multimedia/graphics/graphics.tar
  \end{verbatim}

Observe the files above works correctly with following commands.
  \begin{verbatim}
   # animate test_animate.gif
   # diplay test.*
  \end{verbatim}
\item After installation tuxpaint, tuxpaint-stamps and tuxpaint-doc packages:
  \begin{itemize}
   \item  Run the program and make some tracks and save, observe it works correctly.
   \item Click the Stamps button and try to add a stamp from right side. Observe it can be added.
   \item Execute the following command and click the open button and observe picture which is you imported is displayed by application.
\begin{verbatim}
   # tuxpaint-import /usr/share/tuxpaint/stamps/vehicles/ship/walnutBoat.png
\end{verbatim} 
\end{itemize}

\item After installation inkscape package:

Open the file below with inkscape and change something on file. Observe that file works correctly.
\begin{verbatim}
# wget http://cekirdek.pardus.org.tr/~semen/dist/test/multimedia/graphics/drawing.svg 
\end{verbatim}

\item After installation asymptote package:

Observe that following commands work correctly:
\begin{verbatim}
# wget http://cekirdek.pardus.org.tr/~semen/dist/test/multimedia/graphics/test_asymptote.tex
# latex test_asymptote
# asy test_asymptote
# latex test_asymptote
# okular test_asymptote.dvi
\end{verbatim}
\item  After installation dcmtk package:

Observe that following commands work correctly:
\begin{verbatim}
# wget http://cekirdek.pardus.org.tr/~semen/dist/test/multimedia/graphics/test_dcmtk.dcm 
# dcmj2pnm test_dcmtk.dcm  test.png
# gwenview test.png
\end{verbatim}

\item After installation dcraw package:

Observe that following commands work correctly:
\begin{verbatim}
# wget http://cekirdek.pardus.org.tr/~semen/dist/test/multimedia/graphics/test_dcraw.jpg 
# dcparse test_dcraw.jpg 
\end{verbatim}

\end{enumerate}

\section{Editor sub component}

\begin{enumerate}

\item After installation lilypond package:

  Do multimedia-eng.pdf frescobaldi test.

\item After installation lilycomp package:
   
 Run the program on Kmenu and click the notes and observe you can see note codes.

\item After installation kino package:

  Run the program on Kmenu and open the file below and try to cut from some points with pushing trim button.
\begin{verbatim}
 # wget http://cekirdek.pardus.org.tr/~semen/dist/test/multimedia/editor/sample.dv
\end{verbatim}


\item After installation kid3 package:

Run the program on Kmenu and open the file below with following File $\rightarrow$ Open path and try edit tag part, observe that you can edit.

\begin{verbatim}
# wget http://cekirdek.pardus.org.tr/~semen/dist/test/multimedia/video/cokluortam/linux.mp3 
\end{verbatim}



 
 \item After installation blender package:

  Open the program on Kmenu and observe that desktop icon is not lost and work correctly. 

 \item After installation dvd-slideshow package:

Execute the following commands and observe a slideshow is created.
\begin{verbatim}
 # wget http://cekirdek.pardus.org.tr/~semen/dist/test/multimedia/editor/image.tar.gz
 # dir2slideshow -n test -s "slide test" image
 # dvd-slideshow image.txt
 # mplayer image.vob 
\end{verbatim}
\item After installation kdenlive package:

Open the Kdenlive and follow Projects $\rightarrow$ Add Clip add file below and run. Observe there is not a video and sound problem.
\begin{verbatim}
 # wget http://cekirdek.pardus.org.tr/~semen/dist/test/multimedia/video/cokluortam/DVD.mpg
\end{verbatim}

\item After installation kiconedit package:

Run the program from menu, follow File $\rightarrow$ Open and select a icon which is locate on /usr/kde/4/share/apps/amarok/icons/hicolor/16x16/actions/ and observe it opens correctly.

\item After installation dvdauthor package:

Do multimedia-eng.pdf dvd-slideshow and kdenlive tests.

 \item After installation avidemux-common package:

 Do multimedia-eng.pdf avidemux test.

 \item After installation avidemux and avidemux-qt packages:

\begin{verbatim}
 # wget http://cekirdek.pardus.org.tr/~semen/dist/test/multimedia/video/cokluortam/Lake_dance_XviD.AVI
 # wget http://cekirdek.pardus.org.tr/~semen/dist/test/multimedia/video/cokluortam/MPEG-1_with_
VCD_extensions.mpeg
\end{verbatim}
Open files with the program. Go $\rightarrow$ Play/Stop and observe there is not a video or sound problem.
\item After installation avidemux-cli package:

Observe that video.mpeg has been created correctly.
\begin{verbatim}
 # wget http://cekirdek.pardus.org.tr/~semen/dist/test/multimedia/video/cokluortam/Lake_dance_XviD.AVI
 # avidemux2_cli --force-alt-h264 --load "Lake_dance_XviD.AVI" --save "video.mpeg" 
--output-format MPEG --quit 
 # mplayer video.mpeg
\end{verbatim}

\end{enumerate}

\end{document}

