\documentclass[a4paper,10pt]{article}

\renewcommand{\labelenumi}{\arabic{enumi}.}
\renewcommand{\labelenumii}{\arabic{enumi}.\arabic{enumii}.}
\renewcommand{\labelenumiii}{\arabic{enumi}.\arabic{enumii}.\arabic{enumiii}.}
\renewcommand{\labelenumiv}{\arabic{enumi}.\arabic{enumii}.\arabic{enumiii}.\arabic{enumiv}.}


%opening
\title{Testing Guide for Pardus Alfa and Beta Versions}
\author{Semen Cirit}

\begin{document}

\maketitle

Attention!

    During the test you may change, remove, or create new disk partitions of your computer. 
    Before starting the tests you may want to back up your archives.
    
    
    If you have one more than test machine you have to send the test results for each machine.

    For each failed test result, you should report a bug to pardus bugzilla.

    Note: For each bug report please use the bug report template which is exist in \emph{efficient-bug-report.pdf} with related detailed descriptions and attachments. 

\begin{enumerate}
\item Preparation
  \begin{enumerate}
    \item Download the related ISO image.
    \item Check the data integrity with executing this command.
      \begin{verbatim}
       # sha1sum <iso_image>
      \end{verbatim}
    \item Burn the ISO image on mode DAO and with maximum 16x speed.
  \end{enumerate}
  \item Tests
    
 	!!!For some testcases you have to be root in order take the outputs from the console.
	
    Please apply the following tests cases in this order:
    \begin{itemize}
    \item YALI
    \item Kaptan
    \item Network Manager
    \item Package Manager
    \item History Manager
    \item Boot Manager
    \item Disk Manager
    \item User manager
    \item Service Manager
    \item Firewall Manager
    \end{itemize}
 \end{enumerate}
\end{document}
