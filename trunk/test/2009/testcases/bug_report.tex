\documentclass[a4paper,10pt]{article}

%opening
\title{Efficient Bug Reporting}
\author{Semen Cirit}

\begin{document}

\maketitle

\section{Efficient Bug Report Draft}

\begin{itemize}
  \item A smooth bug report is so valuable for its develpor or maintainer. Therefore, the first thing, you should read the informations of this webpage very carefully: 
  \begin{verbatim}
  http://bugs.pardus.org.tr/page.cgi?id=bug-writing.html#why 
  \end{verbatim}
  \item For the description part of bugzilla, a template is prepared. All Pardus testers will use this template to report bugs.
  In Description textbox

   Reproducible: (Always or arbitrary)
   
    Give a short description of the bug here.
   
   Steps to reproduce:
   \begin{enumerate}
    \item First step.
    \item Second step
    \item ...
   \end{enumerate}

   Actual results:
    Describe the actual results here. You can add the bug outputs.
   
   Expected results:
    Describe the expected according your computer hardware and system settings.
    
  \item An efficient bug reporting example:
  http://bugs.pardus.org.tr/show_bug.cgi?id=10043
\end{itemize}


What should be attached?
-------------------------------
For Xorg problems:
The outputs of these commands shoud be attached.
lspci -nn > lspci.txt
dmesg > dmesg.txt

If the computer or the keyboard was not lock, you can take the log from 
cat /var/log/Xorg.0
If not, restart your computer and open it in vesa mode and take the log from
cat /var/log/Xorg.conf.old

For all outputs, if X crashed, you can take the outputs of these command with the below procedure.
1) Plug an usb stick to the computer.
2) Mount it manually.
    Execute this command for mounting:
    mount /dev/<your_usb_stick_partition> /mnt/flash
3) Copy the outputs to /mnt/flash
     cp <output> /mnt/flash
4) Unmount it manually.
  umount /dev/<your_usb_stick_partition>
---------------------------

For Pardus specific applications:
The outputs of these commands should be attached for all below managers:
 tail /var/log/comar3/trace.log

 For network-manager:
  lspci > lspci.txt
  If the problem is about ethernet.
    ifconfig -a > ifconfig.txt

  If the problem is about ethernet.
    iwconfig > iwconfig.txt
  
  For disk-manager
    fdisk -l >fdisk.txt
    cat /etc/fstab

  For service-manager:
   service -N > service.txt

  For boot-manager
  cat /boot/grub/grub.conf
  
  user-manager

  firewall-manager
    service -N > service.txt
    iptables > iptables.txt

------------------------------------------------------------------------------------------------------
For camera devices
The outputs of these commands shoud be attached:
  lsusb > lsusb.txt
  test-webcam > webcame.txt

---------------------------------------------
For Audio devices
The outputs of these commands shoud be attached:
  cat /proc/asound/cards
  cat /proc/asound/*/codec*

---------------------------------------------------------------------------------------
And also if you solved the problem you can add the related patch to your bug.
-----------------------------------------------------------------------------------------------
And also one additional thing, if you don't know the erroneous product exactly, you can find it with this command
the variable can be any related file about the product. For example a bin file, a library file. 
# pisi sf `which <variable>`

For our example the effects can be configured from system settings. But there is no product about system settings so we can use the its binary file.
So the command 
pisi sf `which systemsettings`

And the output
/usr/kde/4/bin/systemsettings dosyası aranıyor
kdebase-workspace paketi içinde usr/kde/4/bin/systemsettings dosyası var

So this means, the bug is for the kdebase-workspace package.
---------------------------------------------------------------------------------

\end{document}
