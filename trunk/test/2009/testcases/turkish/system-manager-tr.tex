\documentclass[a4paper,10pt]{article}
\usepackage[turkish]{babel}
\usepackage[utf8]{inputenc}
\usepackage[left=1cm,top=2cm,right=2cm,bottom=1cm]{geometry}

\title{Sistem Yöneticisi Test Aşamaları}
\author{Semen Cirit}

\renewcommand{\labelenumi}{\arabic{enumi}.}
\renewcommand{\labelenumii}{\arabic{enumi}.\arabic{enumii}.}
\renewcommand{\labelenumiii}{\arabic{enumi}.\arabic{enumii}.\arabic{enumiii}.}
\renewcommand{\labelenumiv}{\arabic{enumi}.\arabic{enumii}.\arabic{enumiii}.\arabic{enumiv}.}

\begin{document}

\maketitle

\subsection*{Bu uygulamada yapılan tüm değişiklikler bilgisayarınızı yeniden başlattığınızda gözlemlenebilecektir!}
\begin{enumerate}
\item  Sistem dilini değiştirin. Bilgisayarınızı yeniden başlatın. CTRL+ALT+F1 tuşlarına aynı anda basın.

	Karşınıza çıkan bilgilerin değiştirmiş olduğunuz sistem dilinde olduğunu gözlemleyin.

\item  Konsol klavye düzenini değiştirin. Bilgisayarınızı yeniden başlatın. CTRL+ALT+F1 tuşlarına aynı anda basın.

	Karşınıza çıkan konsoldan, seçtiğiniz klavyenin özel tuşlarına basarak, doğru sonuç elde ettiğinizi gözlemleyin. (Örneğin klavye olarak fransızca seçmiş iseniz, 2 tuşuna bastığınızda ekranda é harfinin çıktığını gözlemlemelisiniz.)

\item  İlk başlatılacak servis olarak özel bir servis seçin. Bilgisayarınızı yeniden başlatın.

	Servis yöneticisinden bu servisin başlatılmış olduğunu gözlemleyin.

\item  İlk başlatılacak servis olarak özel bir servis seçin. Bilgisayarınızı yeniden başlatın.

	Servis yöneticisinden bu servisin başlatılmış olduğunu gözlemleyin.

\item Gerçek konsol sayısını değiştirin. Bilgisayarınızı yeniden başlatın.

	n konsol seçtiyseniz, n adet CTRL+ALT+F1,... ,  CTRL+ALT+Fn herbirinde konsol açıldığını gözlemleyin.
\end{enumerate}

\end{document}

