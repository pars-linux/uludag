\documentclass[a4paper,10pt]{article}
\usepackage[turkish]{babel}
\usepackage[utf8]{inputenc}
\usepackage[left=1cm,top=2cm,right=2cm,bottom=1cm]{geometry}

\title{Masaüstü Bileşeni Test Aşamaları}
\author{Semen Cirit}

\renewcommand{\labelenumi}{\arabic{enumi}.}
\renewcommand{\labelenumii}{\arabic{enumi}.\arabic{enumii}.}
\renewcommand{\labelenumiii}{\arabic{enumi}.\arabic{enumii}.\arabic{enumiii}.}
\renewcommand{\labelenumiv}{\arabic{enumi}.\arabic{enumii}.\arabic{enumiii}.\arabic{enumiv}.}

\begin{document}

\maketitle

\subsection*{Qt ve Qt4}

(Bu kısımda verilen paket adlarının qt ile başlayan bölümleri, 2008'de qt için qt, qt4 için qt4, 2009'da qt için qt3, qt4 için qt olacaktır.)
\begin{enumerate}
 \item Aşağıda bulunan paketler sadece kurulum testine tabidir.
\begin{verbatim}
 qt-doc
 qt-sql-ibase
 qt-sql-odbc
 qt-sql-postgresql
\end{verbatim}
 \item qt paketi kurulumu sonrası
\begin{verbatim}
 # mkdir test
 # cd test
 # wget http://cekirdek.pardus.org.tr/~semen/dist/test/desktop/toolkit/test.cpp
 # qmake-qt4 -project
 # qmake-qt4
 # make
 # ./test
\end{verbatim}

"Hello qt4!" yazan bir pencerenin açıldığını gözlemleyin.
\item qt-designer paketi kurulumu sonrası

Menu $\rightarrow$ Programlar $\rightarrow$ Geliştirme yolunu izleyerek sorunsuz bir şekilde açıldığını gözlemleyin.

\item qt-linguist paketi kurulumu sonrası

Menu $\rightarrow$ Programlar $\rightarrow$ Geliştirme yolunu izleyerek sorunsuz bir şekilde açıldığını gözlemleyin.

\item qt-sql-mysql, qt-sql-sqlite paketi kurulumu sonrası
\begin{verbatim}
 # mkdir test
 # cd test
 # wget http://cekirdek.pardus.org.tr/~semen/dist/test/desktop/toolkit/test-qt-sql-<ilgilidata_base>.cpp
 # qmake-qt -project
 # qmake-qt
\end{verbatim}
qmake komutundan sonra oluşan .pro dosyanıza QT += sql satırını eklemelisiniz. Daha sonra aşağıdaki komutları çalıştırın.
\begin{verbatim}
 # make
 # ./test
\end{verbatim}

Bağlamtının sorunsuz bir şekilde gerçekleştğini gözlemleyin.

\end{enumerate}

\end{document}

