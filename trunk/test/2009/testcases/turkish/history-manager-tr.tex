\documentclass[a4paper,10pt]{article}
\usepackage[left=1cm,top=2cm,right=2cm,bottom=1cm]{geometry}
\usepackage[turkish]{babel}
\usepackage[utf8]{inputenc}

\title{Geçmiş Yöneticisi Test Aşamaları}

\renewcommand{\labelenumi}{\arabic{enumi}.}
\renewcommand{\labelenumii}{\arabic{enumi}.\arabic{enumii}.}
\renewcommand{\labelenumiii}{\arabic{enumi}.\arabic{enumii}.\arabic{enumiii}.}
\renewcommand{\labelenumiv}{\arabic{enumi}.\arabic{enumii}.\arabic{enumiii}.\arabic{enumiv}.}

\begin{document}

\maketitle

\begin{enumerate}
\item Menü $\rightarrow$ Sistem Ayarların yolunu izleyerek geçmiş yöneticisini açmayı deneyin.

Sorunsuz bir şekilde açıldığını gözlemleyin.
\item Menü $\rightarrow$ Uygulamalar $\rightarrow$ Sistem yolunu izleyerek geçmiş yöneticisini açmayı deneyin.

Sorunsuz bir şekilde açıldığını gözlemleyin.
\item Sistem görüntüsü alma işlemi:
\begin{enumerate}
    \item Paket yöneticisinden bir paket kurun veya kaldırın.
    \item Geçmiş yöneticisindeki görüntü alma butonuna basın.
    \item Sistem görüntüsünü alın

   Görüntü aldıktan sonra, geçmiş yöneticisinin ekranı yenilediğini ve yeni değişiklikleri gösterdiğini gözlemleyin.
\end{enumerate}

\item Yeni yaptığınız işlemin tarihi için ayrıntıları göster butonuna basın.

    Aşağıdaki komutu çalıştırın:
\begin{verbatim}
    $ pisi hs
\end{verbatim} 

    Komut çıktısında, seçtiğiniz tarih ile ilgili listelenen değişikliklerin, geçmiş yöneticinde listelenenler ile aynı olduğunu gözlemleyin.

\item Yeni yaptığınız işlem için geri alma işlemi detaylarını göster butonuna basın. (Bir önceki tarihin geri alma işlemi detaylarında görüntülenecektir.)
    Aşağıdaki komutu çalıştırın:
\begin{verbatim}
     $ pisi hs
\end{verbatim} 
    Son değişiklikleri geçmiş yöneticisi ile ilgili tarih için gözlemleyin.


\item Geri alma işlemi detay butonuna basıldığında yapılacak işlemi olan bir tarih seçip, sistemi geçmiş bir tarihe getirmeyi deneyin. 

      Değişiklikler geri alındığında:
\begin{enumerate}
        \item İşlemden sonra geçmiş yöneticisi ekranın yenilendiğini gözlemleyin.
        \item İşlemi geri aldığınız tarih için işlemleri göster tuşuna basın.

	Bu işlemlerin işlem planından kaldırıldığını gözlemleyin.
     	\item Konsolda şu komutu yazın·
      \begin{verbatim}
	  $ pisi hs
      \end{verbatim} 
       Komut çıktısında, son yapılan değişiklik ile ilgili listelenen değişikliklerin, geçmiş yöneticinde listelenenler ile aynı olduğunu gözlemleyin.

\end{enumerate}

\item Herhangi bir işlem yapılmakta iken çıkan kimlik doğrulama penceresine iptal değiniz.

İptal işlemi sonucunda geçmiş yöneticisinin sorunsuz bir şekilde yapılmakta olan işlemden önceki duruma geçtiğini gözlemleyiniz.

\item Kullanıcı yöneticisi ile bir işlem yaptıktan sonra , kimlik doğrulamayı anımsa bölümünü işaretleyiniz.

Geçmiş yöneticisini kullanarak bir işlem yapınız.

Parola sorulmadan işlemin dorunsuz bir şekilde gerçekleştiğini gözlemleyiniz.
\end{enumerate}

\end{document}
