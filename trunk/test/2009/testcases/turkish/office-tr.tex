\documentclass[a4paper,10pt]{article}
\usepackage[turkish]{babel}
\usepackage[utf8]{inputenc}
\usepackage[left=1cm,top=2cm,right=2cm,bottom=1cm]{geometry}

\title{Ofis Bileşeni Test Aşamaları}
\author{Semen Cirit}

\renewcommand{\labelenumi}{\arabic{enumi}.}
\renewcommand{\labelenumii}{\arabic{enumi}.\arabic{enumii}.}
\renewcommand{\labelenumiii}{\arabic{enumi}.\arabic{enumii}.\arabic{enumiii}.}
\renewcommand{\labelenumiv}{\arabic{enumi}.\arabic{enumii}.\arabic{enumiii}.\arabic{enumiv}.}

\begin{document}

\maketitle
\subsection*{Dictionary}
\begin{enumerate}
 \item QstarDict dışındaki tüm sözlükler için ilgili text dosyasını aşağıda bulunan linkten indirebilirsiniz.
\begin{verbatim}
http://cekirdek.pardus.org.tr/~semen/dist/office/dictionary/
\end{verbatim}
İlgili sözlük için text dosyasını indirdikten sonra, bu dosyanın içinde, belirtilen dil ile ilgili, bir yanlış bir adette doğru olarak yazılmış kelime göreceksiniz.

Aşağıda verilen komut çıktısında .dic uzantılı dosyalar bulunmakta:
\begin{verbatim}
# pisi info -F <ilgili sözlüğün paket adı> 
\end{verbatim}

Bu dosya adlarını aşağıda bulunan çıktı için kullanacağız
\begin{verbatim}
#  enchant -d <dic uzantılı dosya adının uzantısız hali> <indirilen dosya> -a
\end{verbatim}

Bu çıktının verilen yanlış kelime ile ilgili alternatif doğru kelimeler verdiğini, doğru kelime ile ilgili de bir bilgi vermediğini gözlemleyin.

\begin{itemize}
 \item Örnek olarak: 
\begin{verbatim}
#  enchant -d en_US english.txt -a
\end{verbatim}

\end{itemize}
 \item Qstardict paketi için, contrib deposundan stardict-essential-turkish paketini kurun.

	Qstardict uygulmasının düzgün bir şekilde çalıştığını gözlemleyin.
\end{enumerate}

\subsection*{Docbook}
\begin{enumerate}
 \item SGMLSpm paketi sadece kurulum testine tabidir.
 \item asciidoc

\end{enumerate}


\end{document}

