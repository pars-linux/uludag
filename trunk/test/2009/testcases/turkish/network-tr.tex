\documentclass[a4paper,10pt]{article}
\usepackage[turkish]{babel}
\usepackage[utf8]{inputenc}
\usepackage[left=1cm,top=2cm,right=2cm,bottom=1cm]{geometry}

\title{Network Bileşeni Test Aşamaları}
\author{Semen Cirit}

\renewcommand{\labelenumi}{\arabic{enumi}.}
\renewcommand{\labelenumii}{\arabic{enumi}.\arabic{enumii}.}
\renewcommand{\labelenumiii}{\arabic{enumi}.\arabic{enumii}.\arabic{enumiii}.}
\renewcommand{\labelenumiv}{\arabic{enumi}.\arabic{enumii}.\arabic{enumiii}.\arabic{enumiv}.}

\begin{document}

\maketitle

\section{Chat alt Bileşeni}
\begin{enumerate}
\item choqok paketi kurulumu sonrası:

Uygulamayı açın ve bir twitter üyeliğiniz var ise, bu üyelik bilgilerinizi kaydedin ve sorunsuz bir şekilde twitter'a bağlanabildiğinizi gözlemleyin.	
\end{enumerate}

\end{document}

