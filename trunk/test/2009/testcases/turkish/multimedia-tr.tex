\documentclass[a4paper,10pt]{article}
\usepackage[turkish]{babel}
\usepackage[utf8]{inputenc}
\usepackage[left=1cm,top=2cm,right=2cm,bottom=1cm]{geometry}

\title{Multimedia Bileşeni Test Aşamaları}
\author{Semen Cirit}

\renewcommand{\labelenumi}{\arabic{enumi}.}
\renewcommand{\labelenumii}{\arabic{enumi}.\arabic{enumii}.}
\renewcommand{\labelenumiii}{\arabic{enumi}.\arabic{enumii}.\arabic{enumiii}.}
\renewcommand{\labelenumiv}{\arabic{enumi}.\arabic{enumii}.\arabic{enumiii}.\arabic{enumiv}.}

\begin{document}

\maketitle

\section{Sound alt Bileşeni}
\begin{enumerate}
\item frescobaldi paketi kurulumu sonrası:
 \begin{verbatim}
  # wget http://cekirdek.pardus.org.tr/~semen/dist/test/multimedia/sound/test_frescobaldi.ly
 \end{verbatim}
Yukarıda bulunan dosyayı frescobaldi ile açıp, uygulamanın sol tarafında bulunan Lilypond butonuna basın ve sorunsuz bir şekilde pdf dökümanının oluştuğunu gözlemleyin.
\item pulseaudio paketi kurulumu sonrası:
\begin{itemize}
 \item Sisteminizi tekrar başlatın ve açılış sesinin düzgün bir şekilde çıktığını gözlemleyin. 
 \item Amarok uygulamasını kurun ve aşağıda bulunan zip dosyasını indirdikten sonra seslerin düzgün çıktığını gözlemleyin.
 \begin{verbatim}
  # wget wget http://cekirdek.pardus.org.tr/~semen/dist/test/multimedia/sound/sound.tar
 \end{verbatim}
\end{itemize}
\item sox paketi kurulumu sonrası:

hardware-tr.pdf k3b testini gerçekleştiriniz.

\end{enumerate}

\section{Video alt Bileşeni}
\begin{enumerate}
 \item mplayer ve vlc paketleri kurulumu sonrası:
 \begin{verbatim}
  # wget http://cekirdek.pardus.org.tr/~semen/dist/test/multimedia/video/cokluortam.tar
 \end{verbatim}
Yukarıda bulunan dosyayı indirdikten sonra, uygulamayı her türlü dosya formatı ile çalıştırın ve sorunsuz bir şekilde çalıştığını gözlemleyin.
\end{enumerate}


\section{Converter alt Bileşeni}
\begin{enumerate}
 \item amrwb ve amrnb paketleri kurulumu sonrası:

 multimedia-tr.pdf sox ve mplayer testlerini gerçekleştiriniz.

\item ccd2iso paketi kurulumu sonrası:
\begin{verbatim}
 # wget http://cekirdek.pardus.org.tr/~semen/dist/test/multimedia/converter/default.img
 # ccd2iso default.img test.iso
\end{verbatim}

Sorunsuz bir şekilde iso dosyasının oluştuğunu gözlemleyin.

\end{enumerate}
\section{Graphics alt Bileşeni}
\begin{enumerate}
 \item gimp kurulumu sonrası:
  \begin{verbatim}
   http://cekirdek.pardus.org.tr/~semen/dist/test/multimedia/graphics/graphics.tar
  \end{verbatim}

 Yukarıda bulunan farklı formattaki dosyaları gimp ile açınız, sorunsuz bir şekilde açıldıklarını gözlemleyiniz.
\item digikam kurulumu sonrası:



  \begin{verbatim}
   http://cekirdek.pardus.org.tr/~semen/dist/test/multimedia/graphics/graphics.tar
  \end{verbatim}

 Digikam uygulaması için seçtiğiniz dosya dizinine yukarıda klasör içinde bulunan dosyaları kopyalayınız ve sorunsuz bir şekilde çalıştığını gözlemleyiniz.
\item imagemagick kurulumu sonrası:
  \begin{verbatim}
   http://cekirdek.pardus.org.tr/~semen/dist/test/multimedia/graphics/graphics.tar
  \end{verbatim}

Yukarıdaki bağlantıda bulunan dosyaların aşağıda bulunan komutlar ile düzgün çalıştığını gözlemleyiniz.
  \begin{verbatim}
   # animate test_animate.gif
   # diplay test.*
  \end{verbatim}

\end{enumerate}
\end{document}

