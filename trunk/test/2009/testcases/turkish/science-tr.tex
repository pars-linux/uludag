\documentclass[a4paper,10pt]{article}
\usepackage[turkish]{babel}
\usepackage[utf8]{inputenc}
\usepackage[left=1cm,top=1cm,right=2cm,bottom=2cm]{geometry}

\title{Science Bileşeni Test Aşamaları}
\author{Semen Cirit}

\renewcommand{\labelenumi}{\arabic{enumi}.}
\renewcommand{\labelenumii}{\arabic{enumi}.\arabic{enumii}.}
\renewcommand{\labelenumiii}{\arabic{enumi}.\arabic{enumii}.\arabic{enumiii}.}
\renewcommand{\labelenumiv}{\arabic{enumi}.\arabic{enumii}.\arabic{enumiii}.\arabic{enumiv}.}

\begin{document}

\maketitle

\section{Mathematics alt Bileşeni}
\begin{enumerate}
\item rkward paketi kurulumu sonrası:

Uygulamayı açın ve Uygulama paneli üzerinden Plots bölümünü tıklayın, Barplot'u seçin ve burada listelenen verilerden birini seçin, ekleyin ve onaylayın. 

Bu durumun sonunda ilgili grafiğin sorunsuz bir şekilde oluştuğunu gözlemleyin.

\item maxima paketi kurulumu sonrası:

Aşağıdaki komutların sorunsuz bir şekilde çalıştığını gözlemleyin:
\begin{verbatim}
 # maxima
 144*17 - 9;
 144^25;
\end{verbatim}

\item gfan paketi kurulum testine tabidir.

\end{enumerate}

\section{Robotics alt Bileşeni}
\begin{enumerate}
 \item opencv paketi kurulumu sonrası: (kamerası olanlar test edebilecektir.)

Resim çek butonuna basın sorunsuz bir şekilde ekranınyenilendiğini gözlemleyin
\begin{verbatim}
# wget http://svn.pardus.org.tr/projeler/facelock/pardus.py
# wget http://svn.pardus.org.tr/projeler/facelock/pardus.png
# python pardus.py
\end{verbatim}


\end{enumerate}

\end{document}

