\documentclass[a4paper,10pt]{article}
\usepackage[turkish]{babel}
\usepackage[utf8]{inputenc}
\usepackage[left=1cm,top=2cm,right=2cm,bottom=1cm]{geometry}

\title{Hardware Bileşeni Test Aşamaları}
\author{Semen Cirit}

\renewcommand{\labelenumi}{\arabic{enumi}.}
\renewcommand{\labelenumii}{\arabic{enumi}.\arabic{enumii}.}
\renewcommand{\labelenumiii}{\arabic{enumi}.\arabic{enumii}.\arabic{enumiii}.}
\renewcommand{\labelenumiv}{\arabic{enumi}.\arabic{enumii}.\arabic{enumiii}.\arabic{enumiv}.}

\begin{document}

\maketitle

\section{Scanner alt Bileşeni}
\begin{enumerate}
\item sane-backends paketi kurulumu sonrası:

Eğer bir tarayıcınız var ise bu paketi test edebilirsiniz!!
\begin{verbatim}
# sane-find-scanner
# scanimage 
\end{verbatim}

İlk komutun sorunsuz olarak tarayıcınızı bulduğunu gözlemleyin.
İkinci komutun ise düzgün bir şekilde tarama yaptığını ve bir .pnm uzantılı bir resim dosyası ürettiğini gözlemleyin.
\end{enumerate}

\section{Optical alt Bileşeni}
\begin{enumerate}
 \item k3b paketi kurulumu sonrası:
\begin{itemize}
\item Aşağıdaki bağlantıda bulunan iso'yu DVD ve CD'ye yazdırın. 
\begin{verbatim}
 # wget http://cekirdek.pardus.org.tr/~semen/dist/test/hardware/optical/boot.iso
\end{verbatim}

Bilgisayarınızı CD/DVD sürücünüzden başlatın ve iso'nun düzgün bir şekilde boot ettiğini gözlemleyin.

\item Aşağıdaki bağlantıda bulunan video ve müzikleri DVD ve CD'ye yazdırın. 
\begin{verbatim}
 # wget http://cekirdek.pardus.org.tr/~semen/dist/test/multimedia/sound/sound.tar
 # wget http://cekirdek.pardus.org.tr/~semen/dist/test/multimedia/video/cokluortam.tar
\end{verbatim}
Yazdırdığınız CD veya DVD'den video veya müziklerinizi açın, ses ve görüntünün sorunsuz bir şekilde olduğunu gözlemleyin.

\end{itemize}

\end{enumerate}

\end{document}

