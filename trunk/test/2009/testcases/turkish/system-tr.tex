\documentclass[a4paper,10pt]{article}
\usepackage[turkish]{babel}
\usepackage[utf8]{inputenc}
\usepackage[left=1cm,top=2cm,right=2cm,bottom=1cm]{geometry}

\title{System Bileşeni Test Aşamaları}
\author{Semen Cirit}

\renewcommand{\labelenumi}{\arabic{enumi}.}
\renewcommand{\labelenumii}{\arabic{enumi}.\arabic{enumii}.}
\renewcommand{\labelenumiii}{\arabic{enumi}.\arabic{enumii}.\arabic{enumiii}.}
\renewcommand{\labelenumiv}{\arabic{enumi}.\arabic{enumii}.\arabic{enumiii}.\arabic{enumiv}.}

\begin{document}

\maketitle

\section{Base alt Bileşeni}
\begin{enumerate}
\item mudur paketi kurulumu sonrası:

\begin{itemize}
  \item Makinenizi yeniden başlatın ve sistemin düzgün bir şekilde açıldığını  gözlemleyin.
 \item Ctrl+Alt+F1 tuşuna basıp sistem konsoluna geçin ve sistem dilinizin daha önceki dil ve kalvye düzeni ile aynı olduğunu gözlemleyin.
  \item /etc/mudur/ altında bulunan locale, language, keymap dosyalarının sistem dil ve klavye dilinize eskisi gibi uygun olduğunu gözlemleyin.
 \item Pisi komutlarının düzgün bir şekilde çalıştığını gözlemleyin. 

\end{itemize}
\item tiff paketi kurulumu sonrası:
\begin{verbatim}
 # wget http://cekirdek.pardus.org.tr/~semen/dist/test/desktop/kde/base/doga.tiff
 # wget http://cekirdek.pardus.org.tr/~semen/dist/test/desktop/kde/base/istanbul.tiff
\end{verbatim}

Resimlerin üzerine sağ tıklayarak gwenview, kolourPaint, gimp, showfoto ile açılabildiklerini gözlemleyin.
\item pisi paketi kurulumu sonrası:
\begin{itemize}
 \item package-manager-tr.pdf test aşamalarını gerçekleştiriniz.
 \item history-manager-tr.pdf test aşamalarını gerçekleştiriniz.
\end{itemize}
\item baselayout paketi kurulumu sonrası:
\begin{itemize}
 \item Bilgisayarınızı kapatın düzgün bir şekilde kapanabildiğini gözlemleyin.
 \item Bilgisayarınızı yeniden başlatın ve düzgün bir şekilde açılabildiğini gözlemleyin.
 \item 2009 için:
  \begin{verbatim}
   # pisi blame baselayout
  \end{verbatim}
      2008 için: http://svn.pardus.org.tr/pardus/2008/stable/system/base/baselayout/pspec.xml linkinde bulunan en son eklenmiş history tag'i yorumuna (comment) bakınız.

   Bu her iki durumda da yorumlarda belirtilmiş olan kullanıcının /etc/passwd dosyasına eklenmiş olduğunu gözlemleyin.
\end{itemize}


\end{enumerate}

\end{document}

