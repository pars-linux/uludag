\documentclass[a4paper,10pt]{article}
\usepackage[turkish]{babel}
\usepackage[utf8]{inputenc}
\usepackage[left=1cm,top=2cm,right=2cm,bottom=1cm]{geometry}

\title{Programming Bileşeni Test Aşamaları}
\author{Semen Cirit}

\renewcommand{\labelenumi}{\arabic{enumi}.}
\renewcommand{\labelenumii}{\arabic{enumi}.\arabic{enumii}.}
\renewcommand{\labelenumiii}{\arabic{enumi}.\arabic{enumii}.\arabic{enumiii}.}
\renewcommand{\labelenumiv}{\arabic{enumi}.\arabic{enumii}.\arabic{enumiii}.\arabic{enumiv}.}

\begin{document}

\maketitle
\section{Language alt bileşeni}
\subsection{Language alt bileşeni}
\begin{enumerate}
\item perl-IO-Socket-SSL paketi kurulumu sonrası:

Aşağıda bulunan dosyayı indirin ve açın.
\begin{verbatim}
# wget http://cekirdek.pardus.org.tr/~semen/dist/test/programming/language/perl/IO-Socket-SSL-1.26.tar.gz
\end{verbatim}

Konsoldan;
\begin{verbatim}
# cd IO-Socket-SSL-1.26/
# /usr/bin/perl5.10.0 "-MExtUtils::Command::MM" "-e" "test_harness(0,'blib/lib', 'blib/arch')" t/*.t
\end{verbatim}

Komutlarını çalıştırın ve testlerden "ok" sonuçlarının döndüğünü gözlemleyin.
\end{enumerate}


\end{document}

