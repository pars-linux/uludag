
\documentclass[a4paper,10pt]{article}
\usepackage[turkish]{babel}
\usepackage[utf8]{inputenc}
\usepackage[left=1cm,top=1cm,right=2cm,bottom=2cm]{geometry}

\title{Programming Bileşeni Test Aşamaları}
\author{Semen Cirit}

\renewcommand{\labelenumi}{\arabic{enumi}.}
\renewcommand{\labelenumii}{\arabic{enumi}.\arabic{enumii}.}
\renewcommand{\labelenumiii}{\arabic{enumi}.\arabic{enumii}.\arabic{enumiii}.}
\renewcommand{\labelenumiv}{\arabic{enumi}.\arabic{enumii}.\arabic{enumiii}.\arabic{enumiv}.}

\begin{document}

\maketitle
\section{Microcontroller alt bileşeni}
\begin{enumerate}
 \item Aşağıda bulunan paketler kurulum testine tabidir.
\begin{verbatim}
 avr-libc
 avrdude
 binutils-avr
 gcc-avr
\end{verbatim}

\end{enumerate}

\section{Tool alt bileşeni}
\begin{enumerate}
 \item fcgi sadece kurulum testine tabidir.
\end{enumerate}

\section{vcs alt bileşeni}
\begin{enumerate}
\item mod\_dav\_svn paketi sadece kurulum testine tabidir.
\item git paketi kurulumu sonrası:

Aşağıdaki komutları çalıştırın. Ve sorunsuz bir şekilde Git deposu oluşturduğunu ve klonlandığını gözlemleyin.
\begin{verbatim}
 # cd ~
 # mkdir test_git
 # cd test_git
 # git init
 # cd ..
 # git clone test_git test_clone
\end{verbatim}

\item subversion paketi kurulumu sonrası:

test dizininin sorunsuz bir şekilde eklenmiş olduğunu gözlemleyin:
\begin{verbatim}
  # svn co http://svn.pardus.org.tr/uludag/trunk/test/2009/testguide/turkish/
  # cd turkish
  # svn mkdir test
  # svn st
 \end{verbatim}
testfile dosyası içerisine bir kaç kelime yazın ve kaydedin. Yapılan değişiklik farkının alınabildiğini gözlemleyin:
\begin{verbatim}
  # vi testfile
  # svn add testfile
  # svn diff
 \end{verbatim}
\end{enumerate}

\section{Environment alt bileşeni}
\begin{enumerate}
 \item eric paketi kurulumu sonrası:
 
Aşağıda bulunan dosyayı eric uygulaması ile açın ve Start $\rightarrow$ Run Script yolunu izleyerek çalıştırın. 

Sorunsuz bir şekilde çalıştığını gözlemleyin.
\begin{verbatim}
 # wget http://cekirdek.pardus.org.tr/~semen/dist/test/programming/environment/test.py
\end{verbatim}
 \item Aşağıda bulunan paketlerin kurulumu sonrasında, yerel dilinizi değiştirip, konsoldan aynı dizinde bir open office uygulaması açın ve yardım dosyasının ilgili dilde olduğunu gözlemleyin.
\begin{verbatim}
eric-i18n-cs
eric-i18n-de
eric-i18n-es
eric-i18n-fr
eric-i18n-ru
eric-i18n-tr
 \end{verbatim}

Yerel dili değiştirmek için:
\begin{verbatim}
export LC_ALL= <lang_LANG>
\end{verbatim}

lang\_LANG şeklinde yazılmış olan, pt-BT için pt\_BT, diğer diller için örneğin de\_DE olacaktır.

Daha sonra bu çalıştırdığınız komut dizininde eric4 komutunu çalıştırın, paket eğer help ile ilgili ise help dosyasının, uygulama dili ise uygulamanın sorunsuz bir şekilde istenilen dilde açıldığını gözlemleyin.

\item ipython paketi kurulumu sonrası: 

Aşağıda bulunan komutları çalıştırdığınızda, bulunduğunuz dizinde test adında bir dosya oluştuğunu ve içerisinde "test ipython" yazdığını gözlemleyin:
\begin{verbatim}
 # ipython
 a = open("test", "a")
 a.write("test ipyton")
\end{verbatim}

\item drscheme paketi kurulumu sonrası:

Kmenüden uygulamayı açın ve sorunsuz bir şekilde çalıştığını gözlemleyin.

\item qt-creator paketi kurulumu sonrası:

Kmenüden uygulamayı açın ve sorunsuz bir şekilde açıldığını gözlemleyin.
\end{enumerate}


\section{Language alt bileşeni}
\subsection{Php alt bileşeni}
\begin{enumerate}
 \item php-cli ve php-common paketleri kurulumu sonrası:

Aşağıda bulunan komutları çalıştırdıktan sonra http://localhost/test.php adresine firefox ile girin ve php ile ilgili bilgilerin sayfalandığını gözlemleyin.

\begin{verbatim}
# cd /var/www/localhost/htdocs/
# wget http://cekirdek.pardus.org.tr/~semen/dist/test/programming/language/php/test.php 
\end{verbatim}

\end{enumerate}


\subsection{Perl alt bileşeni}
\begin{enumerate}
\item perl-IO-Socket-SSL paketi kurulumu sonrası:

Aşağıda bulunan dosyayı indirin ve açın.
\begin{verbatim}
# wget http://cekirdek.pardus.org.tr/~semen/dist/test/programming/language/perl/IO-Socket-SSL-1.26.tar.gz
\end{verbatim}

Konsoldan;
\begin{verbatim}
# cd IO-Socket-SSL-1.26/
# /usr/bin/perl5.10.0 "-MExtUtils::Command::MM" "-e" "test_harness(0,'blib/lib', 'blib/arch')" t/*.t
\end{verbatim}

Komutlarını çalıştırın ve testlerden "ok" sonuçlarının döndüğünü gözlemleyin.
\item perl-Compress-Zlib paketi kurulumu sonrası:

programming-tr.pdf git testini gerçekleştirin.

\item perl-Email-MIME-Encodings paketi kurulumu sonrası:
\begin{verbatim}
 # wget http://cekirdek.pardus.org.tr/~semen/dist/test/programming/language/perl/Email-MIME-Encodings.t
 # perl Email-MIME-Encodings.t
\end{verbatim}

Tüm sonuçların "ok" döndürdüğünü gözlemleyin. 

\item perl-Email-MIME-Encodings paketi kurulumu sonrası:
\begin{verbatim}
 # wget http://cekirdek.pardus.org.tr/~semen/dist/test/programming/language/perl/test_perl_Test_Simple.t
 # perl test_perl_Test_Simple.t
\end{verbatim}

Tüm sonuçların "ok" döndürdüğünü gözlemleyin. 


\end{enumerate}
\subsection{Python alt bileşeni}

\begin{enumerate}
\item PyX paketi kurulumu sonrası:

ipython paketini kurun ve aşağıda bulunan komutları çalıştırın:
\begin{verbatim}
 # ipython
 import pyx
\end{verbatim}

\item pyNotifier paketi kurulumu sonrası:

ipython paketini kurun ve aşağıda bulunan komutları çalıştırın:
\begin{verbatim}
 # ipython
 import pynotify
\end{verbatim}


\item httplib2 paketi kurulumu sonrası:
ipython paketini kurun ve aşağıda bulunan komutları çalıştırın:
\begin{verbatim}
 # ipython
 import httplib2
\end{verbatim}

 \item Django paketi kurulumu sonrası:
\begin{itemize}
 \item Aşağıda bulunan komutu çalıştırın:
 \begin{verbatim}
 # django-admin.py startproject test
 # cd test
 \end{verbatim}
 test adında bir dizinin oluştuğunu ve bu dizin altında aşağıda bulunan dosyaların oluştuğunu gözlemleyin.
  \begin{verbatim}
  __init__.py
  manage.py
  settings.py
  urls.py 
  \end{verbatim}
 \item Aşağıdaki komutu çalıştırın ve daha sonra firefox'tan http://localhost:8080/ adresini girin ve sunucuya bağlanabildiğinizi gözlemleyin
  \begin{verbatim}
  # python manage.py runserver 8080
  \end{verbatim}
 \item settings.py içerisinde DATABASE\_ENGINE DATABASE\_NAME değişkenlerini aşağıdaki database değişkenlerini atayın:
  \begin{verbatim}
   DATABASE_ENGINE = 'sqlite3'
   DATABASE_NAME = 'sqlite3_'   
  \end{verbatim}
\item Aşağıdaki komutu çalıştırın ve istemiş olduğu işlemleri sırasıyla gerçekleştirin ve sorunsuz bir şekilde Django onay sisteminin kurulduğunu gözlemleyin:
\begin{verbatim}
# python manage.py syncdb 
\end{verbatim}
\item Aşağıdaki komutu çalıştırın ve polls adında bir dizinin oluştuğunu gözlemleyin:
\begin{verbatim}
# python manage.py startapp polls 
\end{verbatim}
\item Ve dizin içerişinde aşağıdaki dosyaların oluştuğunu gözlemleyin:
\begin{verbatim}
 __init__.py
 models.py
 views.py
\end{verbatim}

\end{itemize}

\item python-memcached paketi kurulumu sonrası:

Apache sunucusunu servis yöneticisinden başlatın.

Aşağıdaki komutları çalıştırın ve sonucun "True" döndürdüğünü gözlemleyin.
\begin{verbatim}
 # wget http://cekirdek.pardus.org.tr/~semen/dist/test/programming/language/
python/test_python-memcache.py
 # python test_python-memcache.py
\end{verbatim}

\item pygtk paketi kurulumu sonrası: 

Aşağıda bulunan komutları çalıştırın ve sorunsuz bir şekilde bir pencerenin açıldığını gözlemleyin.

\begin{verbatim}
 # wget http://cekirdek.pardus.org.tr/~semen/dist/test/desktop/toolkit/test_pango.py
 # python test_pango.py
\end{verbatim}

\item mpmath paketi kurulumu sonrası:  

ipython paketini kurun ve aşağıda bulunan komutları çalıştırın:
\begin{verbatim}
 # ipython
 import mpmath
\end{verbatim}

\item mpmath paketi kurulumu sonrası:  

ipython paketini kurun ve aşağıda bulunan komutları çalıştırın:
\begin{verbatim}
 # ipython
 import mpmath
\end{verbatim}

\item python-M2Crypto paketi kurulumu sonrası:

ipython paketini kurun ve aşağıda bulunan komutları çalıştırın:
\begin{verbatim}
 # ipython
 import M2Crypto
\end{verbatim}

\item winpdb paketi kurulumu sonrası:

ipython paketini kurun ve aşağıda bulunan komutları çalıştırın:
\begin{verbatim}
 # ipython
 import winpdb
\end{verbatim}
(DeprecationWarning önemli değildir.)

\item cython paketi kurulumu sonrası:

ipython paketini kurun ve aşağıda bulunan komutları çalıştırın:
\begin{verbatim}
 # ipython
 import cython
\end{verbatim}

\item lxml paketi kurulumu sonrası:  

ipython paketini kurun ve aşağıda bulunan komutları çalıştırın:
\begin{verbatim}
 # ipython
 import lxm
\end{verbatim}
\item python-RuleDispatch paketi kurulumu sonrası:  

ipython paketini kurun ve aşağıda bulunan komutları çalıştırın:
\begin{verbatim}
 # ipython
 import dispatch
\end{verbatim}

\item python-nose paketi kurulumu sonrası:  

ipython paketini kurun ve aşağıda bulunan komutları çalıştırın:
\begin{verbatim}
 # ipython
 import nose
\end{verbatim}

\item PyICU paketi kurulumu sonrası:  

ipython paketini kurun ve aşağıda bulunan komutları çalıştırın:
\begin{verbatim}
 # ipython
 import PyICU
\end{verbatim}

\item python-simplejson paketi kurulumu sonrası:  

ipython paketini kurun ve aşağıda bulunan komutları çalıştırın:
\begin{verbatim}
 # ipython
 import simplejson
\end{verbatim}

\end{enumerate}

\subsection{Java alt bileşeni}
\begin{enumerate}
 \item Aşağıdaki paketlerin kurulumu sonrası:
\begin{verbatim}
 sun-jre
 sun-jdk
 sun-jdk-demo
 sun-jdk-samples
 sun-jdk-doc
\end{verbatim}

Aşağıda bulunan komutların düzgün bir şekilde çalıştığını gözlemleyin.
\begin{verbatim}
# java -version
# wget http://cekirdek.pardus.org.tr/~semen/dist/test/programming/language/java/test.java
# javac test.java
# java test
\end{verbatim}
\end{enumerate}
\subsection{Lisp alt bileşeni}
\begin{enumerate}
 \item clisp paketi kurulumu sonrası: (Warningleri önemsemeyiniz.)

Aşağıdaki komutların çalıştırın ve hata olmadığını gözlemleyin.
\begin{verbatim}
# wget http://cekirdek.pardus.org.tr/~semen/dist/test/programming/language/lisp/test_clisp.lisp 
# clisp -c test_clisp.lisp
\end{verbatim}

\end{enumerate}
\subsection{Dotnet alt bileşeni}
\begin{enumerate}

\item Aşağıda bulunan paketler sadece kurulum testine tabidir
\begin{verbatim}
taglib-sharp 
\end{verbatim}

 \item gmime, gmime-docs, gmime-sharp paketi kurulumu sonrası:

Aşağıda bulunan komutun jpeg dosyasını encode ettiğini gözlemleyin.
\begin{verbatim}
 # wget http://cekirdek.pardus.org.tr/~semen/dist/test/multimedia/graphics/test_dcraw.jpg
 # gmime-uuencode -m test_dcraw.jpg jpeg
\end{verbatim}


 \item mono paketi kurulumu sonrası:
Aşağıdaki komutların çalıştırın ve hata olmadığını gözlemleyin.
\begin{verbatim}
# wget http://cekirdek.pardus.org.tr/~semen/dist/test/programming/language/dotnet/test_mono.cs
# mcs test_mono.cs
# mopno test_mono.exe
\end{verbatim}
\end{enumerate}


\begin{itemize}
 \item R paketi kurulumu sonrası:

Aşağıda bulunan komutları çalıştırın ve bir grafiğin oluştuğunu gözlemleyin.
\begin{verbatim}
 # wget http://cekirdek.pardus.org.tr/~semen/dist/test/programming/language/test_R.R
 # R --vanilla --slave < test_R.R
\end{verbatim}
\item R-mathlib paketi kurulumu sonrası:

Aşağıda bulunan komutları çalıştırın ve sorunsuz bir şekilde çalıştıklarını gözlemleyin.	
\begin{verbatim}
 # wget http://cekirdek.pardus.org.tr/~semen/dist/test/programming/language/test_r-mathlib.c
 # gcc -o test_r-matlib test_r-matlib.c -lm -lRmath
\end{verbatim}
\end{itemize}
\end{document}

