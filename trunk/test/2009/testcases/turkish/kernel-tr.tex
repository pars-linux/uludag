\documentclass[a4paper,10pt]{article}
\usepackage[turkish]{babel}
\usepackage[utf8]{inputenc}
\usepackage[left=1cm,top=2cm,right=2cm,bottom=1cm]{geometry}


\renewcommand{\labelenumi}{\arabic{enumi}.}
\renewcommand{\labelenumii}{\arabic{enumi}.\arabic{enumii}.}
\renewcommand{\labelenumiii}{\arabic{enumi}.\arabic{enumii}.\arabic{enumiii}.}
\renewcommand{\labelenumiv}{\arabic{enumi}.\arabic{enumii}.\arabic{enumiii}.\arabic{enumiv}.}

\title{Kernel Test Aşamaları}
\author{Semen Cirit}

\begin{document}

\maketitle

\section{Default alt bileşeni}
\begin{enumerate}

 \item Kernel kurulumu sonrası:

\begin{itemize}
\item
Aşağıdaki komut çıktısı ile listelenen sisteminizde kurulu tüm paketlerin yeni kernele uygun versiyonlarını kurununuz.
\begin{verbatim}
 # pisi li -c kernel.default
\end{verbatim}

\item Bilgisayarınızı kapatınız ve düzgün bir şekilde kapanabildiğini gözlemleyiniz.
\item Açılış ekranın görüntüsünün (Arka tarafında Pardus logosu bulunan grub menüsü) Sorunsuz bir şekilde açıldığını gözlemleyiniz.
\item Bilgisayarınızın yeni kernel ile düzgün bir şekilde açılabildiğini gözlemleyiniz.

Bilgisayarınızı açtıktan sonra konsoldan aşağıda bulunan komutu çalıştırıp, kullandığınız kernelin grub menüsünde görüntülenen kernel ile aynı olduğunu gözlemleyiniz.
\begin{verbatim}
 # uname -r 
\end{verbatim}

\item Eğer dizüstü bilgisayar kullanıyorsanız kablonuzu çıkarıp taktığınızda uyarı verdiğini ve pil seviyesinin düzgün bir şekilde görüntülendiğini gözlemleyiniz.
\item USB bellek takınız ve algılandığını gözlemleyiniz.
\end{itemize}

\item module-alsa-driver ve module-alsa-driver-userspace paketleri kurulumu sonrası.

Bilgisayarınız yeniden başlatınız.
\begin{itemize}
\item Açılış sesinin düzgün bir şekilde takılmadan çalıştığını gözlemleyiniz.

\item Amarok ile aşağıdaki bağlantıda bulunan ses dosyalrından birkaçını deneyiniz. Sorunsuz bir şekilde çalıştıklarını gözlemleyiniz.
\begin{verbatim}
http://cekirdek.pardus.org.tr/~semen/dist/test/multimedia/sound/sound.tar 
\end{verbatim}
\end{itemize}

\end{enumerate}

\section{Pae alt bileşeni}

Bu bölümde bulunan paketleri Pentium 4 ve üzeri bilgisayara sahip olanlar test edebilecektir.
\begin{enumerate}

 \item Kernel-pae paketi kurulumu sonrası:

\begin{itemize}
\item
Aşağıdaki komut çıktısı ile listelenen sisteminizde kurulu tüm paketlerin yeni kernele uygun versiyonlarını kurununuz.
\begin{verbatim}
 # pisi li -c kernel.pae
\end{verbatim}

\item Bilgisayarınızı kapatınız ve düzgün bir şekilde kapanabildiğini gözlemleyiniz.
\item Açılış ekranın görüntüsünün (Arka tarafında Pardus logosu bulunan grub menüsü) Sorunsuz bir şekilde açıldığını gözlemleyiniz.
\item Bilgisayarınızın yeni kernel ile düzgün bir şekilde açılabildiğini gözlemleyiniz.

Bilgisayarınızı açtıktan sonra konsoldan aşağıda bulunan komutu çalıştırıp, kullandığınız kernelin grub menüsünde görüntülenen kernel ile aynı olduğunu gözlemleyiniz.
\begin{verbatim}
 # uname -r 
\end{verbatim}

\item Eğer dizüstü bilgisayar kullanıyorsanız kablonuzu çıkarıp taktığınızda uyarı verdiğini ve pil seviyesinin düzgün bir şekilde görüntülendiğini gözlemleyiniz.
\item USB bellek takınız ve algılandığını gözlemleyiniz.
\end{itemize}

\item module-pae-alsa-driver kurulumu sonrası.

Bilgisayarınız yeniden başlatınız.
\begin{itemize}
\item Açılış sesinin düzgün bir şekilde takılmadan çalıştığını gözlemleyiniz.

\item Amarok ile aşağıdaki bağlantıda bulunan ses dosyalrından birkaçını deneyiniz. Sorunsuz bir şekilde çalıştıklarını gözlemleyiniz.
\begin{verbatim}
http://cekirdek.pardus.org.tr/~semen/dist/test/multimedia/sound/sound.tar 
\end{verbatim}
\end{itemize}

\end{enumerate}

\end{document}

