\documentclass[a4paper,10pt]{article}
\usepackage[turkish]{babel}
\usepackage[utf8]{inputenc}
 \usepackage[left=1cm,top=1cm,right=2cm,bottom=2 cm]{geometry}

\title{Server Bileşeni Test Aşamaları}
\author{Semen Cirit}

\renewcommand{\labelenumi}{\arabic{enumi}.}
\renewcommand{\labelenumii}{\arabic{enumi}.\arabic{enumii}.}
\renewcommand{\labelenumiii}{\arabic{enumi}.\arabic{enumii}.\arabic{enumiii}.}
\renewcommand{\labelenumiv}{\arabic{enumi}.\arabic{enumii}.\arabic{enumiii}.\arabic{enumiv}.}

\begin{document}

\maketitle
\section{Auth alt bileşeni}
\begin{enumerate}
 \item ypserv paketi kurulumu sonrası:

Servis yöneticisinden ypserv başlatın, aşağıdaki komut ile başlatıldığına emin olun:
\begin{verbatim}
 # service ypserv status
\end{verbatim}

\item yp-tools paketi kurulumu sonrası:

Servis yöneticisinden ypserv başlatıldıktan sonra:
\begin{verbatim}
 # su -
 # domainname localdomain
 # domainname
 # nisdomainname
\end{verbatim}

\item ypbind paketi kurulumu sonrası:

Aşağıda bulunan dosyayı /etc altına kopylayın,
\begin{verbatim}
# wget http://cekirdek.pardus.org.tr/~semen/dist/test/server/auth/yp.conf 
# sudo cp yp.conf /etc/
# sudo domainname localdomain
\end{verbatim}

Servis yöneticisinden ypbind başlatın, aşağıdaki komut ile başlatıldığına emin olun:
\begin{verbatim}
 # service ypbind status
\end{verbatim}

\end{enumerate}

\section{Mta alt bileşeni}
\begin{enumerate}
 \item dovecot paketi kurulumu sonrası:

Servis yöneticisinde dovecot başlatılır.
Aşağıdaki komut ile başlatıldığını gözlemleyin:
\begin{verbatim}
 # service dovecot status
\end{verbatim}
Aşağıdaki komut ile dovecot kullanıcısı tarafından servisin başlatıldığını gözlemleyin.
\begin{verbatim}
# ps aux|grep dovecot 
\end{verbatim}

Aşağıda bulunan komutlar ile imap ve pop3 paketlerini kontrol edin:
\begin{verbatim}
# netstat -ln|grep 110
# netstat -ln|grep 143
\end{verbatim}

\end{enumerate}

\section{Web alt bileşeni}
\begin{enumerate}
\item Aşağıda bulunan paketler sadece kurulum testine tabidir:
\begin{verbatim}
mod_dav_svn 
\end{verbatim}

\item webalizer paketi kurulumu sonrası:

Service yöneticisinden apache servisini başlatın.

Aşağıda bulunan dosyayı /var/log/apache2 altına kopyalayın.
\begin{verbatim}
 # wget http://cekirdek.pardus.org.tr/~semen/dist/test/server/web/access_log
\end{verbatim}

Aşağıda bulunan komutu çalıştırın ve daha sonra http://localhost/webalizer/ bağlantısına firefox ile bağlanın. Ve apache kullanım grafiklerini gözlemleyin.
\begin{verbatim}
 # sudo webalizer
\end{verbatim}

\item apache paketi kurulumu sonrası:
\begin{itemize}
\item Servis yöneticisinden apache sunucusunu başlatın. Aşağıda bulunan komutu kullanarak sunucunun başlatılmıl olduğunu gözlemleyin.
\begin{verbatim}
# service list
\end{verbatim}
\item Firefox üzerinden http://localhost adresine bağlanın ve sorunsuz bir şekilde bağlanabildiğinizi gözlemleyin. 
\item Aşağıdaki komutu çalıştırın ve komut çıktısında `Syntax Ok` aldığınzı gözlemleyin.
\begin{verbatim}
# apachectl -M 
\end{verbatim}

\end{itemize}

\item mod\_php paketi kurulumu sonrası:

util-tr.pdf phpmyadmin testini gerçekleştiriniz.

\end{enumerate}

\section{Database alt bileşeni}
\begin{enumerate}
 \item Aşağıda bulunan paketlerin kurulumu sonrası:
\begin{verbatim}
 postgresql-doc
 postgresql-lib
 postgresql-pl
 postgresql-server
\end{verbatim}

Servis yöneticisinden postgreql sunucusunu başlatın. Aşağıdaki komut ile sunucunun başlatılmış olduğunu gözlemleyin.
\begin{verbatim}
 # service postgresql-server status
\end{verbatim}

Süreçlerin postgres kullanıcısı ile çalıştığını gözlemleyin:
\begin{verbatim}
# ps aux|grep postgres 
\end{verbatim}

Aşağıdaki komutu kullanarak sql komut satırına geçin ve ikinci satırdaki sql komutunu çalıştırın:
\begin{verbatim}
# psql -h localhost -d postgres -U postgres
# select * from information_schema.tables ;
\end{verbatim}

İşlemlerin sorunsuz olarak gerçekleştiğini gözlemleyin.

 \item firebird-superserver ve firebird-client paketleri kurulumu sonrası:

Bilgisayarınızı yeniden başlatın,

Servis yöneticisinden firebird-superserver'ı başlatın.

Aşağıda bulunan komutları sırası ile çalıştırın ve sorunsuz bir şekilde çalıştığını gözlemleyin:
\begin{verbatim}
# cd /opt/firebird/examples/empbuild
# isql (2008 için)
# fb_isql (2009 için)

SQL> CONNECT employee.fdb user sysdba password masterkey;
SQL> show tables;
SQL> select *from COUNTRY
\end{verbatim}

 \item mysql-client, mysql-server, mysql-lib paketleri kurulumu sonrası:
\begin{itemize}
 \item Servise yöneticisinden Mysql'i başlatın ve  aşağıda bulunan komutu kullanarak başlatılmış olduğundan emin olun:

\begin{verbatim}
 # service list
\end{verbatim}
 \item desktop-tr.pdf qt-sql-mysql testini gerçekleştirin.

\end{itemize}

\item mysql-man-pages paketi kurulumu sonrası:

Aşağıda bulunan komutun man sayfasını düzgün açtığından emin olun.
\begin{verbatim}
# man myisampack 
\end{verbatim}

\end{enumerate}

\section{Diğerleri}

\begin{itemize}

 \item dhcp paketi kurulumu sonrası:

Ağ yöneticisinden dhcp kullanarak bir ağa bağlanmayı deneyin. Daha sonra konsoldan aşağıda bulunan komutu çalıştırın ve ağa bağlı olduğununuzu gözlemleyin.
\begin{verbatim}
# ping 4.2.2.1
\end{verbatim}

\item bind ve bind-tools paketleri kurulumu sonrası:
\begin{verbatim}
# dig www.google.com
\end{verbatim}
Yukarıda bulunan komutun düzgün bir şekilde dns sunucuları listelediğini gözlemleyin.
\item samba paketi kurulumu sonrası:

Servis yöneticisinden samba servisini başlatın.

Aşağıda bulunan komut ile servisin başlatıldığını gözlemleyin.
\begin{verbatim}
 # service samba status 
\end{verbatim}

 Aşağıda bulunan komutun sorunsuz çalıştığını gözlemleyin:
\begin{verbatim}
# sudo testparm /etc/samba/smb.conf
\end{verbatim}

/etc/samba/smb.conf dosyasına aşağıda bulunanları ekleyin.
\begin{verbatim}
[public]
   path = /tmp
   public = yes
   only guest = yes
   writable = yes
   printable = no
\end{verbatim}

Servis yöneticisinden yeniden başlatın. 

Aşağıdaki komutların hatasız bir şekilde çalıştığını gözlemleyin.
\begin{verbatim}
smbclient //localhost/public 
ls
\end{verbatim}

\end{itemize}


\end{document}

