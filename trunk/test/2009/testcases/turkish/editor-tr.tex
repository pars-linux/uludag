\documentclass[a4paper,10pt]{article}
\usepackage[turkish]{babel}
\usepackage[utf8]{inputenc}
\usepackage[left=1cm,top=2cm,right=2cm,bottom=1cm]{geometry}

\title{Editör Bileşeni Test Aşamaları}
\author{Semen Cirit}

\renewcommand{\labelenumi}{\arabic{enumi}.}
\renewcommand{\labelenumii}{\arabic{enumi}.\arabic{enumii}.}
\renewcommand{\labelenumiii}{\arabic{enumi}.\arabic{enumii}.\arabic{enumiii}.}
\renewcommand{\labelenumiv}{\arabic{enumi}.\arabic{enumii}.\arabic{enumiii}.\arabic{enumiv}.}

\begin{document}

\maketitle

\subsection*{Ed}

Aşağıda örnek bir ed  oturumu bulunmaktadır. Burada yazanları ed komutu çalıştırıldıktan sonra uygulayarak, ed test edilebilir.

\begin{verbatim}
a
ed standart bir text editorudur.
Bu ikinci satirdir.
.
2i

.
%l
ed standart bir text editorudur.$
$
Bu ikinci satirdir.$
3s/ikinci/ucuncu/
,l
ed standart bir text editorudur.$
$
Bu ucuncu satirdir.$
w text.txt
54
q
\end{verbatim}
Oluşan text.txt dosyasında aşağıdaki şekilde bir bilgi yazdığını gözlemleyin.
\begin{verbatim}
ed standart bir text editorudur.

Bu ucuncu satirdir.
\end{verbatim}


\subsection*{Emacs}
\begin{enumerate}
\item Emacs paketi kurulumu sonrası:

Menü $\rightarrow$ Uygulamalar $\rightarrow$ Yardımcı Programlar bölümünden Emacs'i açın.

Sorunsuz bir şekilde açılabildiğini gözlemleyin.

\item color-theme paketi kurulumu sonrası:


Aşağıdaki komutu kullanarak, ev dizininizin altına .emacs dosyasını kaydedin.
\begin{verbatim}
# wget http://cekirdek.pardus.org.tr/~semen/dist/test/editor/.emacs 
\end{verbatim}

Emacs uygulamasını açtıktan sonra, emacs komut satırından aşağıdaki komutu çalıştırın.
\begin{verbatim}
 M-x color-theme-aalto-dark
\end{verbatim}

Sorunsuz bir şekilde temanın değiştiğini gözlemleyin.

\item emacs-muse paketi kurulumu sonrası:

Aşağıdaki komutu kullanarak, ev dizininizin altına .emacs dosyasını kaydedin.
\begin{verbatim}
# wget http://cekirdek.pardus.org.tr/~semen/dist/test/editor/.emacs 
\end{verbatim}

Emacs uygulamasını açtıktan sonra, aşağıdaki komutu tab tuşuna basarak ilgili muse seçeneklerinin geldiğini gözlemleyin
\begin{verbatim}
 M-x muse-
\end{verbatim}


\end{enumerate}


\subsection*{Joe}

Paketini kurduktan sonra, konsoldan aşağıdaki komutu çalıştırın.

\begin{verbatim}
 # joe ornek.txt
\end{verbatim}

Aşağıda bulunan bilgileri joe ile ne yazın.
\begin{verbatim}
Bu bir joe örnek testidir.
Bu ikinci satırdır.
\end{verbatim}

Aşağıda bulunan komutları kullanarak dosyayı kaydedin ve çıkın.
\begin{verbatim}
C-kd tuşu dosyayı kaydetmek için.
C-c dosyadan çıkmak için
\end{verbatim}


Sorunsuz bir şekilde dosyanın kaydedilmiş olduğunu, aşağıdaki komutu kullanarak gözlemleyin.
\begin{verbatim}
# joe ornek.txt 
\end{verbatim}

\subsection*{Vi}

\begin{enumerate}

\item Vi paketi

Paketini kurduktan sonra, konsoldan aşağıdaki komutu çalıştırın.

\begin{verbatim}
 # vi ornek.txt
\end{verbatim}

Aşağıda bulunan bilgileri joe ile ne yazın.
\begin{verbatim}
Bu bir vi örnek testidir.
Bu ikinci satırdır.
\end{verbatim}

Aşağıda bulunan komutları, vi editörü komut satırını kullanarak, dosyayı kaydedin ve çıkın.
\begin{verbatim}
:w tuşu dosyayı kaydetmek için.
:q dosyadan çıkmak için
\end{verbatim}

Sorunsuz bir şekilde dosyanın kaydedilmiş olduğunu, aşağıdaki komutu kullanarak gözlemleyin.
\begin{verbatim}
# vi ornek.txt 
\end{verbatim}

\item vim-colorschemes paketi

Aşağıda bulunanan dosyayı açın.
\begin{verbatim}
# vi ornek.py
\end{verbatim}

Daha sonra vi komut satırından 
\begin{verbatim}
:colorscheme Blacksea
\end{verbatim}

Renk temasının değiştiğini gözlemleyin.

\item emacs-muse paketi kurulduktan sonra:


\end{enumerate}

\end{document}

