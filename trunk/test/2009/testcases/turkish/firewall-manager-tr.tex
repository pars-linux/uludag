\documentclass[a4paper,10pt]{article}

\title{Firewall Manager Test Cases}

\renewcommand{\labelenumi}{\arabic{enumi}.}
\renewcommand{\labelenumii}{\arabic{enumi}.\arabic{enumii}.}
\renewcommand{\labelenumiii}{\arabic{enumi}.\arabic{enumii}.\arabic{enumiii}.}
\renewcommand{\labelenumiv}{\arabic{enumi}.\arabic{enumii}.\arabic{enumiii}.\arabic{enumiv}.}

\begin{document}

\maketitle

\begin{enumerate}
    \item Sistem Ayarları'ndan Güvenlik Duvarı Yöneticisi'ni açmayı deneyin.
    \item Kmenu'den Güvenlik Duvarı Yöneticisi'ni açmayı deneyin.
    \item Güvenlik Duvarı servisini Güvenlik Duvarı Yöneticisindeki başlat/durdur butonunu kullanarak açıp kapatmay deneyin.

    Şu komutu çalıştırın:

\begin{verbatim}
    # service list
\end{verbatim} 

    iptables servisinin kapalı olduğunu doğrulayın.

    \item Try to start service from the start/stop button of firewall manager.

        Şu komutu çalıştırın:

\begin{verbatim}
    # service list
\end{verbatim}

       iptables servisinin açık olduğunu doğrulayın.

\item Push the configuration button and try these actions for options: Block incoming connections or Block outgoing connections.
    \begin{enumerate}
        \item Port ekleme
        \item Port silme
        \item Port açmak
        \item Port kapatmak
    \end{enumerate}

\textbf{Note:} Bu komutun her durum için çıktısını doğrulayın.

Şu komutu çalıştırın:

\begin{verbatim}
    # iptables --list-rules
\end{verbatim} 

\item Gelen bağlantıları engelleme:

\begin{enumerate}
     \item Güvenlik duvarı Yöneticisinden port ekledikten sonra.
        \begin{enumerate}
        \item Kısıtlı içerik paylaşımını aktifleştirin
            	Yukarıdaki komuutun çıktısının şunları içerdiğini doğrulayın: 
\begin{verbatim}
-A PARDUS-IN-MOD-SERVING -p tcp -m multiport --dports <addedPORT> \
    -j ACCEPT
-A PARDUS-IN-MOD-SERVING -p tcp -m multiport --dports 0:1024 \
    -m tcp --tcp-flags FIN,SYN,RST,ACK SYN -j REJECT --reject-with \
    icmp-port-unreachable
-A PARDUS-IN-MOD-SERVING -p udp -m multiport --dports 0:1024 \
    -j REJECT --reject-with icmp-port-unreachable
\end{verbatim} 
        \item Gelen bağlantıları engellemeyi pasifleştirin

              Yukarıdaki satırların aşağıdaki komutun çıktısından silindiğini doğrulayın.
        \end{enumerate}
    \item Port eklemeden

        Yukarıdaki komutun şunları içerdiğini doğrulayın:

\begin{verbatim}
-A PARDUS-IN-MOD-SERVING -p tcp -m multiport --dports 0:1024 \
    -m tcp --tcp-flags FIN,SYN,RST,ACK SYN -j REJECT --reject-with \
    icmp-port-unreachable
-A PARDUS-IN-MOD-SERVING -p udp -m multiport --dports 0:1024 \
    -j REJECT --reject-with icmp-port-unreachable
\end{verbatim} 


\item Internet paylaşımı
    \begin{enumerate}
    \item Internet paylaşımını aktifleştirin
    	(Dahili iki ethernet kartınız veya fazladan harici bir ethernet kartınız varsa bu bölümü test edebilirsiniz, yoksa bu adımı geçin.
        \begin{enumerate}

        \item İnternet ve ev ağınız için farklı köprüler seçin.
            	Yukarıdaki komutun çıktısının şunları içerdiğini doğrulayın:
\begin{verbatim}
-A PARDUS-FW-MOD-SHARING -i <input> -o <output> -m state \
    --state ESTABLISHED,RELATED -j ACCEPT
-A PARDUS-FW-MOD-SHARING -i <output> -o <input> -j ACCEPT
-t nat -A PARDUS-POST-MOD-SHARING -o <input> -j MASQUERADE
\end{verbatim} 
        \item Aynı değerleri verin

    	Yukarıdaki satırlaırn aşağıdaki komutun çıktısından silindiğini doğrulayın.
        \end{enumerate}
    \end{enumerate}

\item Giden bağlantılar engelle 
\begin{enumerate}
    \item Güvenlik Duvarı Yöneticisinden bir port ekledikten sonra.
        \begin{enumerate}
        \item Giden bağlantıları engellemeyi aktifleştirin

            Bu portun eklendiğini doğrulayın

            Komutun çıktısı aşağıdaki gibi olmalı:

\begin{verbatim}
-A PARDUS-FW-MOD-BLOCK -p tcp -m multiport --dports <addedPORT> \
    -j DROP
-A PARDUS-OUT-MOD-BLOCK -p tcp -m multiport --dports <addedPORT> \
    -j DROP
\end{verbatim} 

        \item Giden bağlantıları engellemeyi iptal edin·

              Yukarıdaki satırlaırn aşağıdaki komutun çıktısından silindiğini doğrulayın.
        \end{enumerate}
    \end{enumerate}
\end{enumerate}
\item Testlerin pratik bölümü

	Her iki bilgisayarda güvenlik duvarını aktileştirin.	

    	openssh servisi kapalı ise, servis yöneticisinden başlatın .
\begin{enumerate}
    \item Gelen Bağlantıları Engelle: 

	
	
    \begin{enumerate}
        \item Gelen Bağlantıları Engellemeyi kapatın

		(Sabit ip'niz varsa ya da aynı ağda iki makine varsa, bu adımı test edin, yoksa atlayın.)

              Diğer bilgisyardan kendi bilgisyarınıza uzaktan bağlantı oluşturmayı deneyin.

              Şu komutu çalıştırın:
		\begin{verbatim}
		# ssh <your_computer_name>@<static_ip>
		\end{verbatim} 
              Bağlantının kabul edildiğini doğrulaın.

        \item Bilinen bir portu eklemeyi deneyin ve gelen bağlanıları engellmeyi aktifleştirin.
            Bu portun ilgili servisini servis yöneticisinden açın.

        \begin{enumerate}
            \item Başka bir bilgisyardan kendi bilgisayarınıza uzaktan bağlanmayı deneyin.
	\begin{verbatim}
	# ssh <your_computer_name>@<static_ip>
	\end{verbatim}
	Bağlantının geri çevrildiğini gözlemleyin.

            \item Başka bir bilgisyardan eklediğiniz portu kullanarak kendi bilgisyarınıza uzaktan bağlanmayı deneyin.
	\begin{verbatim}
	# ssh -p <port> <your_computer_name>@<static_ip>
	\end{verbatim}
                 Bağlantının kabul edildiğini gözlemleyin.
        \end{enumerate}
    \end{enumerate}
    \item İnternet paylaşımı için: 

	(Dahili iki ethernet kartınız veya fazladan harici bir ethernet kartınız varsa bu bölümü test edebilirsiniz, yoksa bu adımı geçin.)
        \begin{enumerate}
        \item Bilgisyarınıza fazladan harici veya dahili ethernet kartınızı takın. Kendi bilgisayarınıza ve diğer bilgisayara ethernet kablonuzu takın. Eğer diğer bilgisayarın internet erişimi varsa durdurun. )

        \item Güvenlik Duvarı Yöneticisinden internet paylaşımını aktifleştirin.

        \item Birinci ethernet kartınızı internete köprü için, ikincisini ev ağınıza köprü için seçin.

              Now try to connect to connect internet over your computer and observe it.
        \end{enumerate}
    \item Giden bağlantıları engelleme:

	(Sabit ip'niz varsa ya da aynı ağda iki makine varsa, bu adımı test edin, yoksa atlayın)
        \begin{enumerate}
        \item erişim kısıtlamasını pasifleştirin
            \begin{enumerate}
            \item Kendi bilgisayarıızdan bilinen bir portu kullanarak uzaktaki bilgisyara uzaktan bağlantı oluşturmayı deneyin.
                Uzaktaki bilgisayar statik ip'ye sahipse uzaktan bağlantı için onu kullanabilirsiniz. 

                Şu komutu çalıştırın:
		\begin{verbatim}
		# ssh -p <port> <remote_computer_name>@<static_ip>
		\end{verbatim} 
                Bağlantının kabul edildiğini gözlemleyin.

            \item Bilinen bir portu ekleyin ve giden bağlantıları engellemeyi aktifleştirin.
            	
                  Kendi bilgisayarınızdan eklediğiniz portu kullanarak diğer bilgisayara uazaktan bağlantı oluşturmayı deneyin.
\begin{verbatim}
    # ssh -p <port> <remote_computer_name>@<static_ip>
\end{verbatim}

                    Bağlantının geri çevrildiğini gözlemleyin. 
            \end{enumerate}
        \end{enumerate}
    \end{enumerate}
\end{enumerate}

\end{document}


