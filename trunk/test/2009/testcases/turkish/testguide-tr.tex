\documentclass[a4paper,10pt]{article}
\usepackage[turkish]{babel}
\usepackage[utf8]{inputenc}
\usepackage[left=1cm,top=1cm,right=2cm,bottom=1cm]{geometry}
\renewcommand{\labelenumi}{\arabic{enumi}.}
\renewcommand{\labelenumii}{\arabic{enumi}.\arabic{enumii}.}
\renewcommand{\labelenumiii}{\arabic{enumi}.\arabic{enumii}.\arabic{enumiii}.}
\renewcommand{\labelenumiv}{\arabic{enumi}.\arabic{enumii}.\arabic{enumiii}.\arabic{enumiv}.}


%opening
\title{Pardus Alfa ve Beta Sürümleri İçin Test Kılavuzu}
\author{Semen Cirit}
\begin{document}

\maketitle

Dikkat!

    Test esnasında bilgisayarınızdaki disk bölümlerini değiştirebilir, silebilir ya da yeni bir tane oluşturabilirsiniz. Bu sebeple testlere başlamadan önce bilgilerinizi yedeklemeniz önerilir.
    
    Birden fazla test makinanız varsa her makina için ayrı test sonucu göndermelisiniz.

    Başarısız olan her test sonucu için pardus hata takip sistemine hata girmelisiniz.

    Not: Her hata girişi için lütfen bug\_report-tr.pdf'te ayrıntılı açıklamalar ve ekler ile yer alan hata raporlama şablonunu kullanınız.

\begin{enumerate}
\item Hazırlık
  \begin{enumerate}
    \item Gerekli ISO kalıbını indirin.
    \item Kalıptaki veriyi aşağıdaki komutu çalıştırarak doğrulayın.
      \begin{verbatim}
       # sha1sum <iso_image>
      \end{verbatim}
    \item ISO kalıbını DAO modunda ve en fazla 16x hızında yazdırın.
  \end{enumerate}
  \item Testler
    
 	!!!Bazı testlerde konsoldan çıktı alabilmek için root olmanız gerekmektedir.
	
    Lütfen http://svn.pardus.org.tr/uludag/trunk/test/2009/testguide/turkish/ bağlantısı altında bulunan ilgili test aşamaları dökümanlarını kullanarak aşağıdaki testleri sırasıyla uygulayın:
    \begin{itemize}
    \item YALI
    \item Kaptan
    \item Ağ Yöneticisi
    \item Paket Yöneticisi
    \item Geçmiş Yöneticisi
    \item Açılış Yöneticisi
    \item Disk Yöneticisi
    \item Kullanıcı Yöneticisi
    \item Servis Yöneticisi
    \item Güvenlik Duvarı Yöneticisi
    \item Ekran Ayarları
    \end{itemize}
 \end{enumerate}
\end{document}
