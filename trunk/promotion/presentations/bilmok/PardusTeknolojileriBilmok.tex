\documentclass{beamer}

\usepackage[utf8]{inputenc}
\usepackage[turkish]{babel}

\usepackage{beamerthemesplit}
\usepackage[amssymb,cdot]{SIunits}

\title{Pardus Teknolojileri ve Pardus 2009}
\author[Pardus Geliştiricileri]{bilgi@pardus.org.tr}
\date{Bilmök ODTÜ, Ankara\\ 28 Şubat, 2009}
\institute{
  Ulusal Elektronik ve Kriptoloji Araştırma Enstitüsü\\
  TÜBİTAK}

\setbeamertemplate{navigation symbols}{}
\usetheme{Warsaw}

%images
\pgfdeclareimage[width=8cm]{2009}{2009}
\pgfdeclareimage[width=9cm]{2009tech}{2009tech}
\pgfdeclareimage[width=7cm]{bilesen}{bilesen}
\pgfdeclareimage[width=2cm]{pardus-logo}{logo}
\logo{\pgfuseimage{pardus-logo}}

\begin{document}

\frame{\titlepage}

\section{Pardus Teknolojileri}

\subsection{Pardus Bileşenleri}
\frame
{
    \frametitle{Pardus Bileşenleri}
    \begin{center}{\pgfuseimage{bilesen}}\end{center}
}

\subsection{Çomar}
\frame{
    \frametitle{Çomar}
       \only<1>{
       \begin{itemize}
       \item Yeni bir yaklaşım
       \item \emph{Herkes onu arıyor!}
       \item Sorunun doğru tarifi $\to$ Doğru çözüm
       \end{itemize}
       }

       \only<2>{
       \begin{itemize}
       \item Yapılandırma arayüzleri görev tabanlı olmalı
       \item Gündelik işler için komut satırı gerekmemeli
       \item \emph{Herkes belge okumuyor}
       \item Güvenlik seviyeleri atomik olarak belirlenebilmeli (PolicyKit)
       \item Bilgisayar kendi işini kendi yapmalı
       \item Uygulamalar bir arada çalışabilmeli
       \end{itemize}
       }
       \only<3>{
          \begin{itemize}
          \item Basit ve hızlı, profil tabanlı ağ yapılandırması
          \item Otomatik grafik arayüz yapılandırması
          \item Açılış sistemi; \textbf{hızlı}, kolay
          \item Kullanıcı ve kullanıcı hakları yönetimi
          \item Yazılım kurulum ve güncelleme (PiSi ile birlikte)
          \item Yazılım kurulum ve güncelleme (uzaktaki bir makine için)
          \item Uzaktan yönetim (Lider)
          \end{itemize}
       }

}

\subsection{PiSi}
\frame{
  \frametitle{PiSi}
  \only<1>{
   \begin{itemize}
   \item \emph{Packages Installed Successfully as Intended}
   \item "Paket yönetimi" yeni değil
   \item Doğru yapılması gereken bir işi doğru yap! \\ Gerekirse yeniden yap !
   \item Paket yönetimi dağıtım için \textbf{çok önemli}
   \item Parçaları bir araya getiren sistem
   \end{itemize}
  }
  \only<2>{
   \begin{itemize}
    \item Hızlı, basit, mantıklı
    \item Paket yapmak, paket bakımı yapmak çok kolay
    \item Benzerlerinden kat ve kat daha ufak bir kod, fazlaca özellik
   \end{itemize}
  }
}

\subsection{Yalı}
\frame{
  \frametitle{Yalı}
   \begin{itemize}
   \item Sade, basit : görev temelli, insan odaklı
   \item PiSi ve Çomar altyapısı
   \item Ufak ve hızlı
   \item \emph{Herkes için kolay ve hızlı bir kurulum deneyimi} 
   \end{itemize}
}

\section{Pardus 2009}

\subsection{Yeni Teknolojiler}
\frame
{
    \frametitle{Yeni Teknolojiler}
    \begin{center}{\pgfuseimage{2009}}\end{center}
}

\subsection{Kde 4 Teknolojileri}
\frame
{
    \frametitle{Kde 4 Teknolojileri}
    \begin{itemize}
       \item Plasma  : Masaüstüne yeni bir bakış
       \item Phonon  : Multimedya uygulamaları için ortak bir altyapı
       \item Akonadi : Kişisel veriler tek bir çatı altında toplansın
       \item Solid   : Donanım bilgisine kolayca ulaşmanın yolu
    \end{itemize}

}

\subsection{Pardus 2009 Teknolojileri}
\frame
{
    \frametitle{Pardus 2009 Teknolojileri}
    \begin{center}{\pgfuseimage{2009tech}}\end{center}
}

%\section{The End}
\frame
{
	\frametitle{Teşekkürler}
	\begin{itemize}
        \item Sorular ?
        \item Bilgi için www.pardus.org.tr, www.ozgurlukicin.com
        \item Geliştirici adayları developer.pardus.org.tr
	\end{itemize}

}

\end{document}
