\documentclass{beamer}

% preamble
\usepackage[utf8]{inputenc}
\usepackage[turkish]{babel}

\title{Pardus}
\author{T. Barış Metin, A. Murat Eren, Gürer Özen}
\institute{
  Ulusal Dağıtım Projesi \\
  Ulusal Elektronik ve Kriptoloji Enstitüsü\\
  TÜBİTAK}

\usetheme{Warsaw}

%images
\pgfdeclareimage[width=5cm]{pardus-logo}{pardus-logo}
\pgfdeclareimage[height=3cm]{ok}{ok}
\pgfdeclareimage[height=3.5cm]{evrim}{evrim}
\pgfdeclareimage[height=4.5cm]{ekip}{ekip}
\pgfdeclareimage[height=4.5cm]{yollar}{yollar}
\pgfdeclareimage[width=4cm]{sanci}{sanci}
\pgfdeclareimage[height=4.5cm]{zemberek}{zemberek}
\pgfdeclareimage[width=4cm]{pisi}{pisi}
\pgfdeclareimage[width=4cm]{farkli}{farkli}
\pgfdeclareimage[height=4cm]{yali}{yali}
\pgfdeclareimage[height=3cm]{hedef}{hedef}
\pgfdeclareimage[width=5cm]{wantyou}{wantyou}

% document
\begin{document}

\frame{\titlepage}


% contents
\frame{
  \frametitle{İçerik}
  \begin{columns}

    \begin{column}{5cm}
      \tableofcontents
    \end{column}

    \begin{column}{5cm}
      \pgfuseimage{pardus-logo}
    \end{column}

  \end{columns}
}


\section{Tarihçe}

\frame<beamer>{\tableofcontents[current]}

\subsection{Neden?}
\frame{
  \frametitle{Neden?}
  \only<1>{
    \begin{block}{Ana sözleşme}
    Pardus, UEKAE tarafından, bilişim okur-yazarlığına sahip bilgisayar kullanıcılarının temel masaüstü ihtiyaçlarını hedefleyerek; mevcut Linux dağıtımlarının üstün taraflarını kavram, mimari ya da kod olarak kullanan; otonom sisteme evrilebilecek bir yapılandırma çerçevesi ve araçları ile kurulum, yapılandırma ve kullanım kolaylığı sağlamak üzere geliştirilen bir GNU/Linux dağıtımıdır.
    \end{block}

    \center{\pgfuseimage{ok}}

   }
   \only<2>{
       \begin{itemize}
	\item Ulusal bağımsızlık, güvenlik ve tasarruf amacıyla, kritik uygulamaların üzerinde çalışabileceği, açık ve standart bir veri yapısını destekleyen, güvenlik izlemesine imkan verecek şekilde kaynak kodu açık olan ve finansal yük oluşturmadan yaygınlaştırılabilecek bir işletim sistemi
	\item Türkiye’nin bilgi teknolojileri konusundaki etkinliğinin katma değerli projelere yöneltilmesi, araştırma ve geliştirme ağırlıklı yüksek teknoloji üretimi
	\item Bir yandan öncülü ve bir yandan da ürünü olarak yerel bilgi birikiminin, gerek teknolojik alanda ve gerekse iş süreçleri düzeyinde, sağlanması zorunluluğu
	\item Ülke gereklerine bağlı olarak teknolojik gelişmenin yönünü belirlemek, farklı alanların ağırlığını değiştirmek ve dolayısıyla söz konusu işletim sisteminin yol haritasına hakim olmak
       \end{itemize}
   }
   \only<3>{
	\begin{itemize}
	\item Tam Türkçe desteğini, hem karakter yapısının Türkçe’ye uygun (UTF-8 uyumluluğu), hem de kullanıcıya görünen tüm mesaj ve belgelerin Türkçe olması yoluyla sağlaması.
	\item Mevcut Linux dağıtımlarından ve rakip diğer işletim sistemlerinden daha kolay kurulabilen ve kullanılabilen bir işletim sistemi olması.
	\item Araç temelli ve teknoloji merkezli bir tasarım yerine görev temelli ve insan merkezli bir yaklaşımla ve esnekliği ve yüksek performansı sağlayabilecek modüler bir yapıda tasarlanması. 
	\end{itemize}
   }
}

\subsection{Nasıl?}
\frame{
  \frametitle{1.0'a doğru}
  \only<1>{
	\begin{itemize}
	\item \textbf{2003}: Kavramsal hazırlık. "Neden?", "Nasıl?", "Kim ile?" sorularına yanıtlar
	\item \textbf{2003 sonu}: \emph{"Dağıtım yacağız"}
	\item \textbf{2004 ilk çeyrek}: Geliştirici ekibin toplanması
	\item \textbf{2004}: Teknik analiz ve alt yapının hazırlanması
	\item \textbf{2005 ilk çeyrek}: Pardus Çalışan CD (1 Şubat)
	\item \textbf{2005 sonu}: Pardus 1.0 (26 Aralık)
	\end{itemize}

    \center{\pgfuseimage{evrim}}

  }
}

\subsection{Geliştirici Ekip}
\frame{
  \frametitle{Geliştiriciler}

  \center{\pgfuseimage{ekip}}

  \only<1>{
   \begin{itemize}
   \item Çekirdek geliştiriciler
   \item Katkıcılar
   \end{itemize}
  }
  \only<2>{
   \begin{itemize}
   \item Özgür yazılım geliştiricileri
   \item Bilinen simalar
   \item Aktif destekçiler
   \end{itemize}
  }
}

\subsection{Nereden Başlasak?}
\frame{
  \frametitle{Nereden Başlasak?}

   \center{\pgfuseimage{yollar}}

   \only<1>{
    \begin{itemize}
    \item Dağıtım \emph{"amorf"} bir yapı
    \item Doğru sorunları bulmak
    \item Doğru çözümler üretmek
    \item Mevcut bilgi birikimini doğru kullanabilmek
    \end{itemize}
   }
   \only<2>{
    \begin{itemize}
    \item Önce, kısa bir süre için ikiye ayrıldık
	\begin{itemize}
	\item Harcıalem Dağıtım
	\item Dürtücü Teknolojiler
	\end{itemize}
    \item Hep bir aradaydık, \emph{"birleştirmek"} zor olmadı
    \item Olabildiğince çok kaynağı \emph{"dışarıdan"} kullandık
    \end{itemize}
   }
}


\subsection{Açık Proje}
\frame{
  \frametitle{Ortak Aklı Oluşturmak}
  \begin{columns}

  \begin{column}{6cm}
    \begin{itemize}
    \item Katkıcılar ile büyüyen bir proje
    \item Bilgiyi dışarıya aktarmak
      \begin{itemize}
      \item Belgeler
      \item Açık tasarım
      \item Kaynak kodlar her zaman erişilebilir
      \item Tartışma ortamları; listeler, toplantılar, ...
      \end{itemize}
    \only<2>{
    \item \textbf{Sancılı bir iş!}
    }
    \end{itemize}
  \end{column}

  \begin{column}{4cm}
   \only<2>{
    \pgfuseimage{sanci}
   }
  \end{column}

  \end{columns}

}


% ikinci bölüm: pardus çözümleri
\section{Pardus Çözümleri}

\frame<beamer>{\tableofcontents[current]}

\subsection{Yerel çözümler}
\frame{
  \frametitle{Türkçe kullanmak istiyoruz}
   \begin{itemize}
   \item Yerelleştirme çalışmaları
   \item Bireysel çabalar
   \item Sesimizi, istediğimiz kadar, duyurmak için yeterli olmamışlardı
   \item Sorunları buluyor ve çözüm üretiyoruz
   \item Çözümlerimizi herkes ile paylaşıyoruz
   \end{itemize}
}

\frame{
  \frametitle{İmla denetimi}
   \begin{itemize}
   \item Masaüstünde, her yerde Türkçe imla denetimi.
   \item Artık hayal değil ;)
   \end{itemize}

  \center{\pgfuseimage{zemberek}}

}

\subsection{PiSi}
\frame{
  \frametitle{PiSi}

  \begin{columns}

  \begin{column}{6cm}

  \only<1>{
   \begin{itemize}
   \item \emph{Packages Installed Successfully as Intended}
   \item "Paket yönetimi" yeni değil
   \item Doğru yapılması gereken bir işi doğru yap!
   \item Paket yönetimi dağıtım için \textbf{çok önemli}
   \item Parçaları bir araya getiren sistem
   \end{itemize}
  }
  \only<2>{
   \begin{itemize}
   \item Daha kolay paket \emph{yönetimi}
     \begin{itemize}
     \item Geniş özellik kümesi, sorunsuz kurulum ve güncelleme
     \item Grafiksel, tam bir yönetim arayüzü
     \end{itemize}
   \item Daha kolay paket \emph{üretimi}
     \begin{itemize}
     \item Standartlara uygun, basit paket tanımları
     \item Genişletilebilir üzerine kolayca yeni uygulamalar eklenebilir bir sistem
     \end{itemize}
   \end{itemize}
  }
  \only<3>{
   \begin{itemize}
   \item Yüksek seviyeli ve düşük seviyeli paket yönetim özellikleri
   \item Kaynak tabanlı ve ikili paket sistemlerinin iyi huyları
   \item Paketler XML dosyaları ve basit python betikleri ile ifade ediliyor
   \item Paketler bileşen ve kategoriler ile düzenleniyor
   \item Python ile yazılmış
   \item İkili paketler PKZIP arşivleri
   \item Hızlı
   \end{itemize}
  }

  \end{column}

  \begin{column}{4cm}
    \pgfuseimage{pisi}
  \end{column}

  \end{columns}

}


\subsection{ÇOMAR}
\frame{
  \frametitle{ÇOMAR}

   \center{\pgfuseimage{farkli}}

   \only<1>{
   \begin{itemize}
   \item Yeni bir yaklaşım
   \item \emph{Herkes onu arıyor!}
   \item Sorunun doğru tarifi $\to$ Doğru çözüm
   \end{itemize}
   }

   \only<2>{
   \begin{itemize}
   \item Yapılandırma arayüzleri görev tabanlı olmalı
   \item Gündelik işler için komut satırı gerekmemeli
   \item \emph{Herkes belge okumuyor}
   \item Bilgisayar kendi işini kendi yapmalı
   \item Uygulamalar bir arada çalışabilmeli
   \end{itemize}
   }

   \only<3>{
    \begin{itemize}
    \item Şu anda neler yapabiliyor
      \begin{itemize}
      \item Basit ve hızlı, profil tabanlı ağ yapılandırması
      \item Otomatik grafik arayüz yapılandırması
      \item Açılış sistemi; \textbf{hızlı}, kolay
      \end{itemize}
    \item Önümüzdekiler
      \begin{itemize}
       \item Kullanıcı yönetimi
       \item Yazılım kurulum ve güncelleme (PiSi ile birlikte)
       \item Sunucu yönetimi
       \item Uzaktan yönetim
      \end{itemize}
    \end{itemize}
   }
}



\subsection{YALI}
\frame{
  \frametitle{YALI}
   \begin{itemize}
   \item Genel geçer bir çözüm
   \item Zorunluluk
   \item İyileştirmeler olabilir
   \item \emph{Herkes için kolay ve hızlı bir kurulum deneyimi} 
   \end{itemize}

   \center{\pgfuseimage{yali}}

}


%% 3. bölüm: Pardus kim için
\section{Pardus: Kim için?}

\frame<beamer>{\tableofcontents[current]}


\subsection{Hedef Kitle}
\frame{
  \frametitle{Hedef Kitle}

  \only<1>{
  \begin{block}{Ana Sözleşme'den...}
    .... bilişim okur-yazarlığına sahip bilgisayar kullanıcılarının temel masaüstü ihtiyaçlarını hedefleyerek ...
  \end{block}
  }
  
  \only<2>{
   \begin{itemize}
   \item Aslında herkes için...
     \begin{itemize}
     \item Yeni bilgisayar kullanıcıları
     \item Mevcut Windows kullanıcıları
     \item Bir şekilde Linux ile tanışmış, fakat tatmin olmamış olanlar
     \item Linux uzmanları, geliştiriciler
     \item ISV'ler
     \end{itemize}
   \end{itemize}
  }

  \center{\pgfuseimage{hedef}}
}


\section{Ne Yapabilirim?}

\frame<beamer>{\tableofcontents[current]}

\subsection{Herkese Açık}
\frame{
  \frametitle{Herkese Açık}

   \begin{columns}

   \begin{column}{5cm}
     \pgfuseimage{wantyou}
   \end{column}

   \begin{column}{5cm}
   \begin{itemize}
   \item Pardus açık bir proje
   \item Gerçekten \textbf{açık}
      \begin{itemize}
      \item Tasarım belgeleri
      \item Kaynak kodlar
      \item Hata takip sistemi
      \item Tartışma listeleri
      \item ...
      \end{itemize}
   \item \textbf{Herkes destek olabilir!}
   \end{itemize}
   \end{column}

   \end{columns}
}


\subsection{Ne bilmek gerekiyor?}
\frame{
  \frametitle{Ne bilmek gerekiyor?}

  \begin{block}{Ne yapmak istiyorsunuz?}
   Ne yapmak istediğinize bağlı olarak, bilmeniz gerekenler değişiyor.
  \end{block}

}


\frame{
  \frametitle{Pardus'u Tanıtmak istiyorum}
  \begin{itemize}
  \item Pardus'un hedeflerini ve tarihçesini öğren
  \item Pardus projelerini öğren, bilgi sahibi ol
  \item Daha önceki tanıtım sunumlarına göz at
  \item Pardus Gönüllüsü Programı'na başvur
  \end{itemize}
}

\frame{
  \frametitle{Pardus'u test etmek istiyorum}
   \begin{itemize}
   \item Hata takip sistemine üye ol ve kullanmayı öğren
   \item Yayınlanan test sürümlerinden haberdar olmak için tartışma listelerini takip et
   \item Pardus test paket deposunu kullan, hataları bildir
   \end{itemize}
}

\frame{
  \frametitle{Yerelleştirme çalışmalarına destek olmak istiyorum}
   \begin{itemize}
   \item Yerelleştirme gruplarına katıl
   \item Pardus "Türkçe" listesine üye ol
   \item Test ederek sistemdeki yerelleştirme sorunlarını bul, raporla/düzelt
   \end{itemize}
}


\frame{
  \frametitle{Paket geliştirmek istiyorum}
  \begin{itemize}
  \item PiSi'nin kullandığı teknolojileri öğren; Python, XML
  \item PiSi mimari belgesini oku
  \item Merhaba PiSi belgesini oku
  \item Action API belgesine göz at
  \item Subversion kullanmayı öğren
  \item Depodaki paket örneklerine göz at
  \item Pardus "paketler" listesine üye ol
  \item Depo politikası belgesini oku
  \end{itemize}
}

\frame{
  \frametitle{Pardus projelerini/teknolojilerini geliştirmek istiyorum}
  \begin{itemize}
  \item Yeni geliştirici belgesini oku
  \item Hata takip sistemini kullanmayı öğren
  \item Subversion kullanmayı öğren
  \item "uludag-commits" ve "paketler-commits" listelerine üye ol
  \item Dişine göre bir proje bulmak için projeleri incele
  \item "kalite" listesine üye ol
  \item Destek bekleyen projeleri ara, sor
  \item Açık hatalardan başla
  \end{itemize}
}


\frame{
  \frametitle{Projemi Pardus'a eklemek istiyorum}
  \begin{itemize}
  \item Tartışma listelerine üye ol
  \item Projeni tanıt ve nasıl faydalı olabileceğini anlat
  \item Hem kullanıcıları hem de geliştiricleri ikna et
  \end{itemize}

}

\section{Son}
%end presentation
\frame{
  \frametitle{Bitti}
  \begin{centering}
    \Huge{\alert{Sorular, Öneriler, Sohbet}} \newline
  \end{centering}

  \begin{block}{İLETİŞİM}
    \begin{itemize}
    \item E-Posta: bilgi@pardus.org.tr
    \item Web: http://www.pardus.org.tr
    \end{itemize}
  \end{block}
}

\end{document}





