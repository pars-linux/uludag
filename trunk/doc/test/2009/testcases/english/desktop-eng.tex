\documentclass[a4paper,10pt]{article}
\usepackage[turkish]{babel}
\usepackage[utf8]{inputenc}
\usepackage[left=1cm,top=1cm,right=2cm,bottom=2cm]{geometry}

\title{Desktop Testcases}
\author{Semen Cirit}

\renewcommand{\labelenumi}{\arabic{enumi}.}
\renewcommand{\labelenumii}{\arabic{enumi}.\arabic{enumii}.}
\renewcommand{\labelenumiii}{\arabic{enumi}.\arabic{enumii}.\arabic{enumiii}.}
\renewcommand{\labelenumiv}{\arabic{enumi}.\arabic{enumii}.\arabic{enumiii}.\arabic{enumiv}.}

\begin{document}

\maketitle
\section{Gnome Sub Compponent}
\begin{itemize}
 \item The following packages subject to installation test:
\begin{verbatim}
 libgnomecanvasmm
\end{verbatim}

\end{itemize}

\section{Look and Feel sub component}
\begin{itemize}
\item The following packages subject to installation test:
\begin{verbatim}
 icon-theme-oxygen-svg
\end{verbatim}
\item After installation icon-theme-oxygen-index and icon-theme-oxygen-png packages:

Restart your computer and observe that the icons changed according to related icon theme.

 \item The following packages will be tested in the same way:
\begin{verbatim}
cursor-theme-oxygen-black
cursor-theme-oxygen-black-big
cursor-theme-oxygen-blue
cursor-theme-oxygen-blue-big
cursor-theme-oxygen-white
cursor-theme-oxygen-white-big
cursor-theme-oxygen-yellow
cursor-theme-oxygen-yellow-big
cursor-theme-oxygen-zion
cursor-theme-oxygen-zion-big
\end{verbatim}

Observe that the related cursor theme is listed on System Settings $\rightarrow$  Keyboard and Mouse $\rightarrow$ Cursor window. Change your cursor theme and observe that it is changed without any problem.
\end{itemize}


\section{Font sub component}
\begin{itemize}
 \item After fontforge package installation:

Install aquafont package, open fontforge application from kmenu and Follow /usr/share/fonts/aquafont/aquafont.ttf path and double click over listed aquafont.ttf file and observe that it opens a window about the font's character correctly.
\item after liberation-fonts package installation:

Open Open Office Writer and observe that LiberationMono, LiberationSans, LiberationSerif added.

\item After installation ecofont package:

Open openoffice writer and observe that Sproanq eco sans font is added to font list.
\item After installation gbfed package:

Open the application from kmenu and observe that it can be opened without any problem.
\end{itemize}

\section{Misc sub component}
\begin{itemize}
\item The following packages subject to installation test:
\begin{verbatim}
 iTest
\end{verbatim}
\item After google-gadgets, google-gadgets-qt and google-gadgets-gtk packages installation:

Run the application following Applications $\rightarrow$ Internet and observe that widget added without effects break.

\item after basket package installation:

observe that the application runs correctly from kmenu.

\end{itemize}

\section{Toolkit Bileşeni}
\subsection*{Qt ve Qt4}

(At this section, given package names' starting with qt parts will be qt for qt, q4 for qt4 on 2008; and qt3 for qt, qt for qt4.)
\begin{enumerate}
\item The following packages subject to installation test:
\begin{verbatim}
 qt-doc
 qt-sql-ibase
 qt-sql-odbc
 qt-sql-postgresql
\end{verbatim}
 \item after qt package installation:

\begin{verbatim}
 # sudo pisi it -c system.devel
 # mkdir test
 # cd test
 # wget http://cekirdek.pardus.org.tr/~semen/dist/test/desktop/toolkit/test.cpp
 # qmake-qt4 -project
 # qmake-qt4
 # make
 # ./test
\end{verbatim}

observe that it opened a window which written "Hello world!".
\item after qt-designer package installation:

observe that it opened correctly following Menu $\rightarrow$ Applications $\rightarrow$ Development path.

\item after qt-linguist package installation:

observe that it opened correctly following Menu $\rightarrow$ Applications $\rightarrow$ Development path.

\item after qt-sql-mysql, qt-sql-sqlite packages installation:

Start Mysql from service manger for qt-sql-mysql.
\begin{verbatim}
 # mkdir test
 # cd test
 # wget http://cekirdek.pardus.org.tr/~semen/dist/test/desktop/ 
toolkit/test-qt-sql-<related_database>.cpp
 # qmake-qt4 -project
 # qmake-qt4	
\end{verbatim}
you must add QT += sql line to the .pro file which created after qmake command. then run below commands.
\begin{verbatim}
 # make
 # ./test
\end{verbatim}

observe that connection is succesful.

\end{enumerate}

\subsection*{Gtk}
\begin{enumerate}
\item The following packages subject to installation test:
\begin{verbatim}
gtk+extra
\end{verbatim}
\item After installation libglademm package:

Do multimedia-tr.pdf pavucontrol test.
 \item after gtk2, gtk2-docs and gtk2-demo  packages installation: 

observe that below command opens a gui for demo codes:
\begin{verbatim}
 # gtk-demo
\end{verbatim}

from list at the gui:

run double clicking Drawing Area, Clipboard, Color Selector sections and observe that they run correctly.

\item after gtk2 package installation: 
\begin{itemize}
 \item do multimedia-eng.pdf avidemux test.
\item do desktop-eng.pdf gtk2-demo test.
\end{itemize}
\item after pango and pango-docs package installation: 
\begin{itemize}
 \item do progrmming-eng.pdf pygtk test.
 \item do multimedia-eng.pdf inkscape test.
\end{itemize}

\end{enumerate}

\section{Kde4 sub component}
\subsection*{Base sub component}
all kde4 packages includes 4 version number on 2008, they doesn't include that version number on 2009.
\begin{enumerate} 
\item The following packages subject to installation test:
\begin{verbatim}
kdelibs-devel
kdelibs-apidox 
\end{verbatim}
\item After installation kdebase-emoticons package:

Follow System Settings $\rightarrow$ Appareance $\rightarrow$ Emoticons path and observe that a emoticon is added for kde4.
\item After installation kdebase-sound package:

Restart the system and observe that the start music is run without any problem.

Burn a CD or DVD and observe that the k3b music run without any problem.

\item After installation kdebase-runtime and kdebase-runtime-doc packages:

Do network-tr.pdf choqok test.

Run the below command and observe that it executed without any problem.
\begin{verbatim}
 # nepomukserver
\end{verbatim}

 \item after kdelibs and kdelibs-devel packages installation:
\begin{itemize}
 \item install choqok package: 

open application and if you have a twitter account, save your account information and observe that you connect twitter succesfully.
 \item install kdegraphics package: (for 2008, kdegraphics4)
\begin{verbatim}
 # wget http://cekirdek.pardus.org.tr/~semen/dist/test/desktop/kde/base/circus-bw_hats.jpg
 # wget http://cekirdek.pardus.org.tr/~semen/dist/test/desktop/kde/base/tepecik_01.png
\end{verbatim}
observe that above files opened with okuler and gwenview.
\item install amarok package: (for 2008, amarok-kde4) 

\begin{verbatim}
/usr/kde/4/share/sounds/k3b_error1.wav
/usr/kde/4/share/sounds/KDE-Im-Irc-Event.ogg
\end{verbatim}

observe that the files opened with amarok succesfully.

\item install yakuake package: (for 2008, yakuake4)

observe that opened yakuake when you press F12 key.
\end{itemize}
\item after kdebase-workspace and kdebase-workspace-doc packages installation:
\begin{itemize}

 \item turn off and restart your computer, and observe that closed and opened kde succesfully.

 \item observe that below commands run correctly:
\begin{verbatim}
# plasmoidviewer nm-applet 
# klipper
# krunner
# kfontview
\end{verbatim}

\end{itemize}

\item after kdebase-wallpapers package installation:

right click desktop, select change wallpaper from appearence settings.observe that Red Leaf and Vector Sunset walpapers added.

\item after kdm package installation:

restart your computers. Açılışta, çıkışta ve kullanıcı değiştirirken çıkan grafiksel giriş ekranınının düzgün bir şekilde açıldığını gözlemleyin.

\item after kdeplasma-addons package installation:

right click desktop and unlock plasmoid lock.

right click panel, select add plasmoid and try to add lancolet as plasmoid. observe that it added and run correctly.

try LCD Weather Station, Twitter Microblogging, RSSNOW, Blue Marble plasmoids same way,too.

\item after kdegraphics package installation:

observe that below files opened with gwenview correctly.
\begin{verbatim} 
 # wget http://cekirdek.pardus.org.tr/~semen/dist/test/office/openoffice/test_oodraw.jpg
 # wget http://cekirdek.pardus.org.tr/~semen/dist/test/office/openoffice/test_oodraw.mng
 # wget http://cekirdek.pardus.org.tr/~semen/dist/test/office/openoffice/test_oodraw.png
 # wget http://cekirdek.pardus.org.tr/~semen/dist/test/office/openoffice/test_oodraw.ps
 # wget http://cekirdek.pardus.org.tr/~semen/dist/test/office/openoffice/test_oodraw.tif
 # wget http://cekirdek.pardus.org.tr/~semen/dist/test/office/openoffice/test_oodraw.xcf
 # wget http://cekirdek.pardus.org.tr/~semen/dist/test/office/openoffice/test_openoffice-extension-pdfimport.pdf
 # gwenview
\end{verbatim}
observe that below files opened with okular correctly.:
\begin{verbatim} 
 # wget http://cekirdek.pardus.org.tr/~semen/dist/test/office/postscript/test_ghostscript.dvi
 # wget http://cekirdek.pardus.org.tr/~semen/dist/test/office/openoffice/test_openoffice-extension-pdfimport.pdf
 # wget http://cekirdek.pardus.org.tr/~semen/dist/test/office/openoffice/test_oodraw.ps
 # okular
 \end{verbatim}

observe that below applications run correctly.
\begin{verbatim}
 # kcolorchooser
 # kruller
 # ksnapshot
\end{verbatim}


\end{enumerate}

\subsection*{Addon sub component}
\begin{enumerate}
\item After installation QtCurve-KDE4 package:

Follow system settings $\rightarrow$ Appereance $\rightarrow$ Style path, and select QtCurve style andobserve that the style changed without any problem.

 \item After installation plasmoid-translatoid package:

Right click on the panel and select Add Widgets. Observe that Translatoid is added to the list.

Open that widget and observe that it can be translate a word from turkish to english.

 \item after kshutdown package:

run below command and observe that opened turn off window.
\begin{verbatim}
 # kshutdown 
\end{verbatim}

 \item after kaptan package installation:

  do http://svn.pardus.org.tr/uludag/trunk/test/2009/testguide/turkish/alfabeta/kaptan-eng.pdf test.

\end{enumerate}

\end{document}

