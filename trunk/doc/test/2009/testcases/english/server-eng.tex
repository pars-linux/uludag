\documentclass[a4paper,10pt]{article}
\usepackage[turkish]{babel}
\usepackage[utf8]{inputenc}
 \usepackage[left=1cm,top=1cm,right=2cm,bottom=2 cm]{geometry}

\title{Server Test Cases}
\author{Semen Cirit}

\renewcommand{\labelenumi}{\arabic{enumi}.}
\renewcommand{\labelenumii}{\arabic{enumi}.\arabic{enumii}.}
\renewcommand{\labelenumiii}{\arabic{enumi}.\arabic{enumii}.\arabic{enumiii}.}
\renewcommand{\labelenumiv}{\arabic{enumi}.\arabic{enumii}.\arabic{enumiii}.\arabic{enumiv}.}

\begin{document}

\maketitle
\section{Auth sub component}
\begin{enumerate}
 \item After installation ypserv package:

Start ypserv service from service manager. In order to be sure run the following command:
\begin{verbatim}
 # service ypserv status
\end{verbatim}

\item After installation yp-tools package:

After starting the ypserv service from service manager run the following commands and observe they run without any problem:
\begin{verbatim}
 # su -
 # domainname localdomain
 # domainname
 # nisdomainname
\end{verbatim}

\item After installation ypbind package:

Copy the above file under /etc directory:
\begin{verbatim}
# wget http://cekirdek.pardus.org.tr/~semen/dist/test/server/auth/yp.conf 
# sudo cp yp.conf /etc/
# sudo domainname localdomain
\end{verbatim}

Start ypbind service from service manager. In order to be sure run the following command:
\begin{verbatim}
 # service ypbind status
\end{verbatim}

\end{enumerate}

\section{Mta sub component }
\begin{enumerate}
 \item After installation dovecot package:

Start dovecot service from service manager.  In order to be sure run the following command:
\begin{verbatim}
 # service dovecot status
\end{verbatim}
 With the following command observe that the service started from the devecot user.
\begin{verbatim}
# ps aux|grep dovecot 
\end{verbatim}

Control the imap ve pop3 packages with the following commands:
\begin{verbatim}
# netstat -ln|grep 110
# netstat -ln|grep 143
\end{verbatim}

\end{enumerate}

\section{Web sub component}
\begin{enumerate}
\item Following packages subject to installation test:
\begin{verbatim}
mod_dav_svn 
\end{verbatim}

\item After installation webalizer package:

Start apache service from service manager:

Copy the below file under /var/log/apache2 directory:
\begin{verbatim}
 # wget http://cekirdek.pardus.org.tr/~semen/dist/test/server/web/access_log
\end{verbatim}

Run the following command and open http://localhost/webalizer/ page with firefox. Observe the apache usage graphics.
\begin{verbatim}
 # sudo webalizer
\end{verbatim}

\item After installation apache package:
\begin{itemize}
\item Start apache server from service-manager. Execute the following command and observe the service had started.
\begin{verbatim}
# service list
\end{verbatim}
\item Connect http://localhost on firefox and observe you connected correctly.
\item Execute following command and observe output is `Syntax Ok`.
\begin{verbatim}
# apachectl -M 
\end{verbatim}

\end{itemize}

\item After installation mod\_php package:

\begin{itemize}
 \item Install phpmyadmin which is in Contrib repository.
 \item Start Apache and mysql servers from service-manager.
 \item Enter http://localhost/phpmyadmin/ using firefox. (type root for username, password will be null)
 \item Observe Mysql header page is opened correctly.
\end{itemize}


\end{enumerate}

\section{Database sub component}
\begin{enumerate}
 \item After installation packages below:
\begin{verbatim}
 postgresql-doc
 postgresql-lib
 postgresql-pl
 postgresql-server
\end{verbatim}

Start postgregl server from service-manager. Observe with command below that server has been started.
\begin{verbatim}
 # service postgresql-server status
\end{verbatim}

Observe that processes work with postgres user:
\begin{verbatim}
# ps aux|grep postgres 
\end{verbatim}

Enter sql command line with using following command and execute the sql command at second line below:
\begin{verbatim}
# psql -h localhost -d postgres -U postgres
# select * from information_schema.tables ;
\end{verbatim}

Observe processes are worked correctly.

 \item After installation firebird-superserver and firebird-client packages:

Restart your computer,

Start firebird-superserver from service-manager.

Execute following commands in order and observe they are working:
\begin{verbatim}
# cd /opt/firebird/examples/empbuild
# isql (2008 için)
# fb_isql (2009 için)

SQL> CONNECT employee.fdb user sysdba password masterkey;
SQL> show tables;
SQL> select *from COUNTRY
\end{verbatim}

 \item After installation mysql-client, mysql-server, mysql-lib packages:
\begin{itemize}
 \item Start Mysql from service-manager, execute the following command and be sure the service is started.:

\begin{verbatim}
 # service list
\end{verbatim}
 \item Do desktop-eng.pdf qt-sql-mysql test.

\end{itemize}

\item After installation mysql-man-pages package:

Execute the following command and observe man page is opened correctly.
\begin{verbatim}
# man myisampack 
\end{verbatim}

\end{enumerate}

\section{Others}
\begin{itemize}
\item After installation samba package:
\begin{itemize}

\item Start samba service from service manager. In order to be sure run the following command:

\begin{verbatim}
 # service samba status 
\end{verbatim}

\item Observe that the following command runs without any problem.
\begin{verbatim}
# sudo testparm /etc/samba/smb.conf
\end{verbatim}

\item Add the following information to /etc/samba/smb.conf file:
\begin{verbatim}
[public]
   path = /tmp
   public = yes
   only guest = yes
   writable = yes
   printable = no
\end{verbatim}

\item Restart the samba service from service manager.

\item Observe that the following commands run without any problem.
\begin{verbatim}
smbclient //localhost/public 
ls
\end{verbatim}

\end{itemize}
 \item After installation dhcp package:

Try connect a network by using dhcp from network-manager. After that, execute the following command and observe you have connected the network.
\begin{verbatim}
# ping 4.2.2.1
\end{verbatim}

\item After installation bind and bind-tools packages:
\begin{verbatim}
# dig www.google.com
\end{verbatim}
Execute the command and observe dns servers are listed correctly.

\end{itemize}


\end{document}

