\documentclass[a4paper,10pt]{article}
\usepackage[turkish]{babel}
\usepackage[utf8]{inputenc}
\usepackage[left=1cm,top=1cm,right=2cm,bottom=2cm]{geometry}

\title{Network Bileşeni Test Aşamaları}
\author{Semen Cirit}

\renewcommand{\labelenumi}{\arabic{enumi}.}
\renewcommand{\labelenumii}{\arabic{enumi}.\arabic{enumii}.}
\renewcommand{\labelenumiii}{\arabic{enumi}.\arabic{enumii}.\arabic{enumiii}.}
\renewcommand{\labelenumiv}{\arabic{enumi}.\arabic{enumii}.\arabic{enumiii}.\arabic{enumiv}.}

\begin{document}

\maketitle
\section{Filter alt Bileşeni}
\begin{enumerate}
\item sabishape paketi kurulumu sonrası:

Aşağıda bulunan komutların sorunsuz bir şekilde çalıştığını gözlemleyin.
\begin{verbatim}
 su -
 sabishape start
 sabishape stop
\end{verbatim}

 \item iproute2 paketi kurulumu sonrası:

Aşağıda bulunan komutun ağ arayüzleri ile ilgili bilgileri listelediğini gözlemleyin. 
\begin{verbatim}
 ifstat
\end{verbatim}
\item conntrack-tools paketi kurulumu sonrası:

Servis yöneticisinden conntrack\_tools servisini başlatın.

Aşağıda bulunan komut ile servisin başlatıldığından emin olun.
\begin{verbatim}
 service conntrack_tools status
\end{verbatim}
\item iptables paketi kurulumu sonrası:

Servis yöneticisinden iptables servisini başlatın.

Aşağıda bulunan komut ile servisin başlatıldığından emin olun.
\begin{verbatim}
 service  iptables status
\end{verbatim}
\end{enumerate}

\section{analyzer alt Bileşeni}
\begin{enumerate}
\item Aşağıda bulunan paketler sadece kurulum testine tabidir.
\begin{verbatim}
 aircrack-ng
 hydra
 ncrack
 
\end{verbatim}
\item libsmi paketi kurulumu sonrası:

network-tr.pdf wireshark testini gerçekleştiriniz.

\item zenmap paketi kurulumu sonrası:

Kmenu'den uygulamayı açın ve sorunsuz bi şekilde açıldığını gözlemleyin.

\item nikto paketi kurulumu sonrası:

Aşağıda bulunan komutun hata vermeden çalıştığını gözlemleyin.
\begin{verbatim}
nikto -h 192.168.0.1 -p 443 -ssl 
\end{verbatim}


\item avahi paketi kurulumu sonrası:

hardware-tr.pdf cups testini gerçekleştirin.

\item nmap paketi kurulumu sonrası:

Aşağıda bulunan komutun verilen adresleri taradığını gözlemleyin.
\begin{verbatim}
 nmap -v -sP 192.168.0.0/16 10.0.0.0/8
\end{verbatim}

 
\item whois paketi kurulumu sonrası:
Aşağıda bulunan komutun www.google.com adresinin tüm etki alanlarının listelendiğini gözlemleyin.
\begin{verbatim}
 whois www.google.com
\end{verbatim}


\item bind-tools paketi kurulumu sonrası:

\begin{verbatim}
# dig www.google.com
\end{verbatim}
Yukarıda bulunan komutun düzgün bir şekilde dns sunucuları listelediğini gözlemleyin.

\item netcat paketi kurulumu sonrası:

Aşağıda bulunan komutu çalıştırın ve komutun sorunsuz bir şekilde ilgili portu dinlediğini gözlemleyin.
\begin{verbatim}
 nc -l 3333 
\end{verbatim}

\item rrdtool paketi kurulumu sonrası:

Aşağıda bulunan komutların sorunsuz çalıştığını gözlemleyin.
\begin{verbatim}
rrdtool create target.rrd --start 1023654125 --step 300 DS:mem:GAUGE:600:0:671744  
	RRA:AVERAGE:0.5:12:24 RRA:AVERAGE:0.5:288:31 
rrdtool info target.rrd
\end{verbatim}


\item arptools paketi kurulumu sonrası:

Aşağıda bulunan komutu çalıştırın ve ağın izlendiğini gözlemleyin.
\begin{verbatim}
su -
arpdiscover 192.168.3.235 5
\end{verbatim}

\item zniper paketi kurulumu sonrası:

Aşağıda bulunan komutu çalıştırın ve ağın izlendiğini gözlemleyin.
\begin{verbatim}
sudo zniper
\end{verbatim}

\item ngrep paketi kurulumu sonrası:

Aşağıda bulunan komutu çalıştırın ve ağın izlendiğini gözlemleyin.
\begin{verbatim}
sudo ngrep
\end{verbatim}


\item iftop paketi kurulumu sonrası:
Aşağıda bulunan komutu çalıştırın ve ağın izlendiğini gözlemleyin.
\begin{verbatim}
sudo iftop 
\end{verbatim}

 \item hping paketi kurulumu sonrası:

Aşağıda bulunan komutu çalıştırın ve paketin example.com adresini gözderildiğini gözlemleyin.
\begin{verbatim}
sudo hping example.com -S -V 
\end{verbatim}

 \item dsniff paketi kurulumu sonrası:

Aşağıda bulunan komutun ağı izlediğini gözlemleyin.
\begin{verbatim}
sudo dsniff
\end{verbatim}

\item etherape paketi kurulumu sonrası:

Kmenuden uygulamayı açın ve ağdaki ip'leri izlediğini gözlemleyin.

\item ettercape paketi kurulumu sonrası:

kmenuden uygulamayı açın ve sniff $\rightarrow$ unified Sniff yolunu izleyin ve ağı izlemeye başladığını gözlemeyin.
\end{enumerate}

\section{p2p alt Bileşeni}
\begin{enumerate}
 \item Aşağıda bulunan paketler sadece kurulum testine tabidir.

\begin{verbatim}
gift
gift-ares
gift-fasttrack
gift-gnutella
gift-openft
giftcurs
\end{verbatim}

\item ktorrent paketi kurulumu sonrası:

Aşağıda bulunan bağlantıyı Firefox bağlantı bölümüne kopyalayın ve ENTER'a basın. Ktorrent uygulamasının açıldığını gözlemleyin.
\begin{verbatim}
 http://www.torrentdownloads.net/download/879161/Pardus+2009+International+iso
\end{verbatim}

 \item linuxdcpp paketi kurulumu sonrası:
Tercihler bölümünden bir rumuz giriniz. 

Uygulamayı açın kullanıcısı fazla olan public hublardan birine bağlanmaya çalışın ve bağlanabildiğinizi gözlemleyin.
\end{enumerate}

\section{Download alt Bileşeni}
\begin{enumerate}
\item wget paketi kurulumu sonrası:

Aşağıda bulunan komutun sorunsuz çalıştığını gözlemleyin.
\begin{verbatim}
 wget http://cekirdek.pardus.org.tr/~semen/dist/test/hardware/optical/boot.iso
\end{verbatim}


\item aria2 paketi kurulumu sonrası:

Aşağıda bulunan komutun boot.iso'yu sorunsuz bir şekilde indirdiğini gözlemleyin.
\begin{verbatim}
 aria2c http://cekirdek.pardus.org.tr/~semen/dist/test/hardware/optical/boot.iso
\end{verbatim}

 \item youtube-dl paketi kurulumu sonrası:

Aşağıdaki komutu çalıştırdığınızda .flv uzantılı bir dosyanın indirildiğini gözlemleyin.
\begin{verbatim}
# http://www.youtube.com/watch?v=iLNngHQ_lj0
\end{verbatim}

\end{enumerate}

\section{Chat alt Bileşeni}
\begin{enumerate}

\item Aşağıda bulunan paketler sadece kurulum testine tabidir:

\begin{verbatim}
 telepathy-mission-control
 amsn-plugins
 konversation-docs 
\end{verbatim}

\item weechat paketi kurulumu sonrası:

Konsole'a "weechat" komutunu yazın ve sorunsuz bir şekilde chat arayüzünğn açıldığını gözlemleyin.

\item kvirc paketi kurulumu sonrası:

Kmenu'den uygulamayı açın ve sorunsuz çalıştığını gözlemleyin.

\item quassel ve qassel-data paketleri kurulumu sonrası:

quassel uygulamasını başlatın ve Başlat butonuna basın, uygun kanalı seçerek bağlanmaya çalışın ve bağlanabildiğinizi gözlemleyin.

\item quasselclient ve quasselcore paketleri kurulumu sonrası:

Servis yöneticisinden "quasselcore" servisini başlatın başlatılabildiğini gözlemleyin.

"quasselclient" uygulamasını açın ve "Bağlan" butonuna basın, açılacak olan pencerede ekle butonuna basın ve sunucu adı olarak "localhost" diyerek "tamam" butonuna basın.

Bağlantının şağlanmış olduğunu gözlemleyin.

\item bitlbee paketi kurulumu sonrası:

Aşağıda bulunan dosyada "disable" değişkenine "no" değerini atayın. Daha sonra service yöneticisinden xinetd servisini başlatın. 
\begin{verbatim}
 su -
 vi /etc/xinetd.d/bitlbee
\end{verbatim}

Konversation uygulamasını çalıştırın, Dosya $\rightarrow$ Çabuk bağlan yolunu izleyerek, Sunucu bilgisayar kısmına "localhost" yazın ve "Bağlan" butonuna basın. Uygulamanın alt bölümünde bulunan kannallarda "bitlbee" kanalının açılmış olduğunu gözlemleyin.

\item skype paketi kurulumu sonrası:

Eğer bir skype hesabınız mevcut ise, bu hesap ile skype açın ve test araması yapın sorunsuz bir şekilde yapabildiğinizi gözlemleyin.
\item amsn-skin-dark-matter paketi kurulumu sonrası:

Kmenüden amsn uygulamasını açın ve Hesap $\rightarrow$ Kabuk Seç yolunu izleyerek "Dark Matter" seçin ve kabuğun değiştiğini gözlemleyin.

\item amsn-skin-oxygen paketi kurulumu sonrası:

Kmenüden amsn uygulamasını açın ve Hesap $\rightarrow$ Kabuk Seç yolunu izleyerek Oxygen" seçin ve kabuğun değiştiğini gözlemleyin.

\item amsn paketi kurulumu sonrası:

Kmenüden uygulamayı açın ve msn ayarlarınızı yapınız. Sorunsuz bir şekilde beğlanabildiğinizi gözlemleyin.

\item psi ve psimedia paketi kurulumu sonrası:

Kmenüden uygulamayı açın ve eğer bir jabber hesabınız var ise bu bu hesabı ekleyin ve bağlantının şağlanabildiğini gözlemleyin.

\item choqok paketi kurulumu sonrası:

Uygulamayı açın ve bir twitter üyeliğiniz var ise, bu üyelik bilgilerinizi kaydedin ve sorunsuz bir şekilde twitter'a bağlanabildiğinizi gözlemleyin.	
\item konversation paketi kurulumu sonrası:

Uygulamayı çalıştırın ve öntanımlı kanallara bağlandığınızı gözlemleyin.

\item pidgin paketi kurulumu sonrası:
Uygulamayı açın ve bir üyelik için bilgilerinizi kaydedin ve sorunsuz bir şekilde bağlanabildiğinizi gözlemleyin.

\end{enumerate}
\section{Web alt Bileşeni}
\begin{enumerate}
\item Aşağıda bulunan paketler sadece kurulum testine tabidir.
\begin{verbatim}
firefox-devel 
\end{verbatim}

\item links paketi kurulumu sonrası:

  Aşağıda bulunan komutun google ana sayfasını konsolda açtığını gözlemleyin.
	\begin{verbatim}
	 links www.google.com
	\end{verbatim}

\item bilbo paketi kurulumu sonrası:

Uygulamayı menüden açın ve yeni bir günlük başlığı ve konusu girin ve local olarak kaydedin. 

Eğer bir bloğunuz var ise Blog $\rightarrow$ Blog ekle kısmından bloğunuzu ekleyip hazırladığınız bloğu gönderebildiğinizi gözlemleyin.
\item firefox, rekonq, chromium-browser veya  arora paketleri kurulumu sonrası:
\begin{itemize}
 \item Aşağıda bulunan sayfanın ilgili butonlarına basıldığında sorunsuz bir şekilde çalıştığını gözlemleyin.
	\begin{verbatim}
	 http://www.croczilla.com/~alex/old-site/dom2.xml
	\end{verbatim}
 \item (Firefox için)Home dizininiz altında bulunan .firefox dizininin baştan oluşturulmadığını gözlemleyin.
	
	Bookmarklarınızın kaybolmadığını gözlemleyin.
	
 	Daha önce açmış olduğunuz sayfaların yeniden yüklendiğinde açıldığını gözlemleyin.

\item Aşağıda bulunan siteyi açın ve videoyu tam ekran olarak oynatmaya çalışın, sesinin ve görüntüsünün sorunsuz bir şekilde olduğunu gözlemleyin.
	\begin{verbatim}
	http://www.dailymotion.com/video/x3akre_loreena-mckennitt-all-souls-night-l 
	\end{verbatim}
\item \begin{verbatim}http://cekirdek.pardus.org.tr/~semen/dist/test/office/openoffice/\end{verbatim} dizini altında bulunan dökümanlardan birini indirmeye çalışın kaydetme penceresinin açıldığını gözlemleyin.

Bu dosyayı indirin ve indirme penceresinin düzgün bir şekilde açıldığını gözlemleyin.
\item \begin{verbatim}http://cekirdek.pardus.org.tr/~semen/dist/test/multimedia/video/cokluortam/\end{verbatim} dizini altında bulunan videolardan birkaçını çalıştırın ve firefox üzerinden çalışabildiğini gözlemleyin.

\end{itemize}

\end{enumerate}

\section{Share alt Bileşeni}
\begin{enumerate}
 \item Aşağıda bulunan paketler sadece kurulum testine tabidir.
\begin{verbatim}
 smbldap-tools
\end{verbatim}

\item smb4k paketi kurulumu sonrası.

Uygulamanın sorunsuz bir şekilde açıldığını gözlemleyin.

\end{enumerate}


\section{Monitor alt Bileşeni}
\begin{enumerate}
\item knemo paketi kurulumu sonrası:

Uygulamayı Kmenü'den açın ve sistem çekmesine uygulamanın  eklendiğini gözlemleyin. 
 \item net-snmp paketi kurulumu sonrası:

Servis yöneticisinden net\_snp servisini başlatın ve aşağıda bulunan komut ile başlatıldığını gözlemleyin. Uygulamanın üzerine tıklayın ve açılam pencereden trafik seçeneğini seçin, sorusuz bir şekilde paket trafiğinin izlenebildiğini gözlemleyin.

\begin{verbatim}
 service net_snmp status
\end{verbatim}

\item net-snmptrap paketi kurulumu sonrası:

Servis yöneticisinden net\_snp servisini başlatın ve aşağıda bulunan komut ile başlatıldığını gözlemleyin.
\begin{verbatim}
 service net_snmptrap status
\end{verbatim}



 \item wireshark paketi kurulumu sonrası:
	Wireshark uygulamasını açın, interface listesinden eth0 seçin ve bu interface ile ilgili paketleri listelendiğini gözlemleyin.
\end{enumerate}

\section{Mail alt Bileşeni}
\begin{enumerate}
 \item thunderbird ve sylpheed paketleri kurulumu sonrası:
\begin{itemize}
\item Bir e-posta hesabı oluşturuyoruz.
\item Mail alabildiğimizi gözlemliyoruz.
\item Mail gönderebildiğimizi gözlemliyoruz.
\item Filtre oluşturabildiğimizi gözlemliyoruz.
\item Eğer daha önce thunderbird kullanıyor isek, ev dizini altında .thunderbird dizininin silinmediğini gözlemliyoruz.
\end{itemize}
\item pyzor paketi kurulumu sonrası:

(Warninglari önemsemeyiniz.)

Aşağıda bulunan komutun "public.pyzor.org:24441 (200, OK)" olarak döndüğünü gözlemleyin.
\begin{verbatim}
 pyzor ping
\end{verbatim}


 \item spamassassin, spamd, bogofilter paketleri kurulumu sonrası:
\begin{enumerate}
	\item İlgili paketi kurduktan sonra:
	\item Menüden uygulamalar $\rightarrow$ Kmail'i açın ve spam filtresini aktifleştirmek için:
		
	\begin{enumerate}
		\item Kmail menü çubuğundan Araçlar $\rightarrow$  Spam engelleme sihirbazı yolunu izleyin.
		\item İndirdiğiniz ilgili spam filtresini seçin.
		\item Bir postaya sağ tıklayın ve combobox'tan Filtreyi uygula $\rightarrow$ Filtreyi çöp posta olarak sınıflandır yolunu izleyin.

		Bu spam'in ilgili spam klasörüne gittiğini gözlemleyin.( Default spam klasörü eğer değitirmediyseniz çöp klasörü olacaktır.)

		\item Aşağıdaki linkten gtube.txt'yi indirin: 
		\begin{verbatim}
 		wget http://www.anta.net/irt/gtube.txt
		\end{verbatim}
		\item  Konsoldan komutu çalıştırın:
		\begin{verbatim}
 		cat  gtube.txt | spamc 
		\end{verbatim}
		
		Bu komut size içerisinde şifrelenmiş bir satır içeren buna benzer bir çıktı gönderecek:
		
		\emph{If your spam filter supports it, the GTUBE provides a test by which you
	    	can verify that the filter is installed correctly and is detecting incoming
    		spam. You can send yourself a test mail containing the following string of
    		characters (in upper case and with no white spaces and line breaks):}
		\begin{verbatim}
 		XJS*C4JDBQADN1.NSBN3*2IDNEN*GTUBE-STANDARD-ANTI-UBE-TEST-EMAIL*C.34X
		\end{verbatim}
    		\emph{You should send this test mail from an account outside of your network.}

		\item Daha sonra bu ilgili şifrelenmiş kısmı kopyalayıp mail olarak kendinize gönderin.
		
		Bu mailin direk olarak ilgili spam klasörüne gittiğini gözlemleyin.
	\end{enumerate} 
\end{enumerate} 
\end{enumerate}
\section{Plugin alt Bileşeni}
\begin{enumerate}
\item flashplugin paketi kurulumu sonrası:

Firefox'u açın ve aşağıdaki bağlantıda bulunan videonun düzgün bir şekilde çalıştığını gözlemeyin.

Tam ekran olabiliyor mu? Ses sorunu var mı?
\begin{verbatim}
http://www.dailymotion.com/relevance/search/lorena+mckennit/video/xd9s3_princesse
-mononoke-studioslorenna 
\end{verbatim}

\item gecko-mediaplayer paketi kurulumu sonrası:
\begin{itemize}
  \item Firefox $\rightarrow$ Düzen $\rightarrow$ Seçenekler $\rightarrow$ Eklentileri Yönet $\rightarrow$ Yan Uygulamalar bölümünde gecko-mediaplayer eklentisinin eklenmiş olduğunu ve etkin olduğunu gözlemleyin.
  \item Aşağıda bulunan uzantıdaki videoları firefox üzerinden açınız. Ve düzgün bir şekilde çalıştıklarını gözlemleyiniz.
  \begin{verbatim}
  http://cekirdek.pardus.org.tr/~semen/dist/test/multimedia/video/cokluortam/  
  \end{verbatim}
\end{itemize}
\end{enumerate}

\section{Ftp alt Bileşeni}
\begin{enumerate}
\item netkit-ftp paketi kurulumu sonrası:

Aşağıda bulunan komutun sorunsuz bir şekilde parola sorduğunu gözlemleyin. 
\begin{verbatim}
 ftp ftp.waytec.com
\end{verbatim}

 \item filezilla paketi kurulumu sonrası:

Uygulamayı açın, sunucu kısmına "ftp.pardus.org.tr" yazın ve "Hızlı bağlan" butonuna basın. "Uzaktaki site" penceresinde "ftp.pardus.org.tr" altında bulunan dizinlerin listelendiğini gözlemleyin.

 \item lftp paketi kurulumu sonrası:
\begin{verbatim}
  lftp http://ftp.pardus.org.tr/pub/
  ls 
  cd pardus
\end{verbatim}
Yukarıda bulunan komutların sorunsuz bir şekilde çalıştığını gözlemleyin.
 \item ncftp paketi kurulumu sonrası:

Aşağıda bulunan komutların sorunsuz çalıştığını gözlemleyin.
\begin{verbatim}
ncftp ftp://ftp.freebsd.org/pub/FreeBSD 
ls
cd misc
\end{verbatim}



\end{enumerate}

\section{Connection alt Bileşeni}
\begin{enumerate}
\item netcf paketi kurulumu sonrası:

 "ncftool" komutunu yazın ve yeni bir komut satırı açıldığını gözlemleyin.

\item rfkill paketi kurulumu sonrası:

Aşağıda bulunan komutun kablosuz aygıtları listelediğini gözlemleyin.
\begin{verbatim}
 frkill list
\end{verbatim}


\item kvpnc paketi kurulumu sonrası:

Kmenüden uygulamanın düzgün çalıştığını gözlemleyin.
\item wireless-regdb paketi kurulumu sonrası:

networ-tr.pdf crda testini gerçekleştirin.
\item crda paketi kurulumu sonrası:

Eğer bir edimax kablosuz aygıtınız var ise aşağıda bulunan komutu çalıştırın.
\begin{verbatim}
 su -
 setregdomain TR
\end{verbatim}

\item ifplugd paketi kurulumu sonrası:

Ağ yöneticisinin kablolu kablosuz tüm ağları listeleyip, otomatik olarak ağları etkinleştirebildiğini ve kapatabildiğini gözlemleyiniz.

 \item Aşağıda bulunan paketler kurulum testine tabidir.
\begin{verbatim}
iw
wireless-regdb
mobile-broadband-provider-info
\end{verbatim}

\end{enumerate}

\section{Fax alt Bileşeni}
\begin{enumerate}
 \item efax-gtk paketi kurulumu sonrası:

Kmenu'den efax-gtk uygulamasını açın ve sorunsuz bir şekilde açıldığını gözlemleyin.
\end{enumerate}

\section{Remoteshell alt Bileşeni}
\begin{enumerate}
 \item Aşağıda bulunan paketler sadece kurulum testine tabidir.

\begin{verbatim}
italc-client
italc-master
libitalc
\end{verbatim}

\end{enumerate}
\end{document}

