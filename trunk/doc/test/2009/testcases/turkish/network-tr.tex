\documentclass[a4paper,10pt]{article}
\usepackage[turkish]{babel}
\usepackage[utf8]{inputenc}
\usepackage[left=1cm,top=1cm,right=2cm,bottom=2cm]{geometry}

\title{Network Bileşeni Test Aşamaları}
\author{Semen Cirit}

\renewcommand{\labelenumi}{\arabic{enumi}.}
\renewcommand{\labelenumii}{\arabic{enumi}.\arabic{enumii}.}
\renewcommand{\labelenumiii}{\arabic{enumi}.\arabic{enumii}.\arabic{enumiii}.}
\renewcommand{\labelenumiv}{\arabic{enumi}.\arabic{enumii}.\arabic{enumiii}.\arabic{enumiv}.}

\begin{document}

\maketitle
\section{analyzer alt Bileşeni}
\begin{enumerate}
\item aşağıda bulunan paketler sadece kurulum testine tabidir.
\begin{verbatim}
 aircrack-ng
\end{verbatim}
\item netcat paketi kurulumu sonrası:

Aşağıda bulunan komutu çalıştırın ve komutun sorunsuz bir şekilde ilgili portu dinlediğini gözlemleyin.
\begin{verbatim}
 nc -l 3333 
\end{verbatim}

\item rrdtool paketi kurulumu sonrası:

Aşağıda bulunan komutların sorunsuz çalıştığını gözlemleyin.
\begin{verbatim}
rrdtool create target.rrd --start 1023654125 --step 300 DS:mem:GAUGE:600:0:671744  
	RRA:AVERAGE:0.5:12:24 RRA:AVERAGE:0.5:288:31 
rrdtool info target.rrd
\end{verbatim}


\item arptools paketi kurulumu sonrası:

Aşağıda bulunan komutu çalıştırın ve ağın izlendiğini gözlemleyin.
\begin{verbatim}
su -
arpdiscover 192.168.3.235 5
\end{verbatim}

\item zniper paketi kurulumu sonrası:

Aşağıda bulunan komutu çalıştırın ve ağın izlendiğini gözlemleyin.
\begin{verbatim}
sudo zniper
\end{verbatim}

\item ngrep paketi kurulumu sonrası:

Aşağıda bulunan komutu çalıştırın ve ağın izlendiğini gözlemleyin.
\begin{verbatim}
sudo ngrep
\end{verbatim}


\item iftop paketi kurulumu sonrası:
Aşağıda bulunan komutu çalıştırın ve ağın izlendiğini gözlemleyin.
\begin{verbatim}
sudo iftop 
\end{verbatim}

 \item hping paketi kurulumu sonrası:

Aşağıda bulunan komutu çalıştırın ve paketin example.com adresini gözderildiğini gözlemleyin.
\begin{verbatim}
sudo hping example.com -S -V 
\end{verbatim}


 \item dsniff paketi kurulumu sonrası:

Aşağıda bulunan komutun ağı izlediğini gözlemleyin.
\begin{verbatim}
 # dsniff
\end{verbatim}

\item etherape paketi kurulumu sonrası:

Kmenuden uygulamayı açın ve ağdaki ipleri izlediğini gözlemleyin.

\item ettercape paketi kurulumu sonrası:

kmenuden uygulamayı açın ve sniff $\rightarrow$ unified Sniff yolunu izleyin ve ağı izlemeye başladığını gözlemeyin.
\end{enumerate}

\section{p2p alt Bileşeni}
\begin{enumerate}
 \item Aşağıda bulunan paketler sadece kurulum testine tabidir.

\begin{verbatim}
gift
gift-ares
gift-fasttrack
gift-gnutella
gift-openft
giftcurs
\end{verbatim}

 \item linuxdcpp paketi kurulumu sonrası:
Tercihler bölümünden bir rumuz giriniz. 

Uygulamayı açın kullanıcısı fazla olan public hublardan birine bağlanmaya çalışın ve bağlanabildiğinizi gözlemleyin.
\end{enumerate}

\section{Download alt Bileşeni}
\begin{enumerate}
\item wget paketi kurulumu sonrası:

Aşağıda bulunan komutun sorunsuz çalıştığını gözlemleyin.
\begin{verbatim}
 wget http://cekirdek.pardus.org.tr/~semen/dist/test/hardware/optical/boot.iso
\end{verbatim}


\item aria2 paketi kurulumu sonrası:

Aşağıda bulunan komutun boot.iso'yu sorunsuz bir şekilde indirdiğini gözlemleyin.
\begin{verbatim}
 aria2c http://cekirdek.pardus.org.tr/~semen/dist/test/hardware/optical/boot.iso
\end{verbatim}

 \item youtube-dl paketi kurulumu sonrası:

Aşağıdaki komutu çalıştırdığınızda .flv uzantılı bir dosyanın indirildiğini gözlemleyin.
\begin{verbatim}
# http://www.youtube.com/watch?v=iLNngHQ_lj0
\end{verbatim}

\end{enumerate}

\section{Chat alt Bileşeni}
\begin{enumerate}
\item psi ve psimedia paketi kurulumu sonrası:

Kmenüden uygulamayı açın ve eğer bir jabber hesabınız var ise bu bu hesabı ekleyin ve bağlantının şağlanabildiğini gözlemleyin.

\item choqok paketi kurulumu sonrası:

Uygulamayı açın ve bir twitter üyeliğiniz var ise, bu üyelik bilgilerinizi kaydedin ve sorunsuz bir şekilde twitter'a bağlanabildiğinizi gözlemleyin.	
\item konversation ve konversation-docs paketleri kurulumu sonrası:

Uygulamayı çalıştırın ve öntanımlı kanallara bağlandığınızı gözlemleyin.

\item pidgin paketi kurulumu sonrası:
Uygulamayı açın ve bir üyelik için bilgilerinizi kaydedin ve sorunsuz bir şekilde bağlanabildiğinizi gözlemleyin.

\end{enumerate}
\section{Web alt Bileşeni}
\begin{enumerate}
\item Aşağıda bulunan paketler sadece kurulum testine tabidir.
\begin{verbatim}
firefox-devel 
\end{verbatim}


\item links paketi kurulumu sonrası:

  Aşağıda bulunan komutun google ana sayfasını konsolda açtığını gözlemleyin.
	\begin{verbatim}
	 # links www.google.com
	\end{verbatim}

\item bilbo paketi kurulumu sonrası:

Uygulamayı menüden açın ve yeni bir günlük başlığı ve konusu girin ve local olarak kaydedin. 

Eğer bir bloğunuz var ise Blog $\rightarrow$ Blog ekle kısmından bloğunuzu ekleyip hazırladığınız bloğu gönderebildiğinizi gözlemleyin.
\item firefox ve arora paketi kurulumu sonrası:
\begin{itemize}
 \item Aşağıda bulunan sayfanın ilgili butonlarına basıldığında sorunsuz bir şekilde çalıştığını gözlemleyin.
	\begin{verbatim}
	 http://www.croczilla.com/~alex/old-site/dom2.xml
	\end{verbatim}
 \item (Firefox için)Home dizininiz altında bulunan .firefox dizininin baştan oluşturulmadığını gözlemleyin.
	
	Bookmarklarınızın kaybolmadığını gözlemleyin.
	
 	Daha önce açmış olduğunuz sayfaların yeniden yüklendiğinde açıldığını gözlemleyin.

\item Aşağıda bulunan siteyi açın ve videoyu tam ekran olarak oynatmaya çalışın, sesinin ve görüntüsünün sorunsuz bir şekilde olduğunu gözlemleyin.
	\begin{verbatim}
	http://www.dailymotion.com/video/x3akre_loreena-mckennitt-all-souls-night-l 
	\end{verbatim}
\item \begin{verbatim}http://cekirdek.pardus.org.tr/~semen/dist/test/office/openoffice/\end{verbatim} dizini altında bulunan dökümanlardan birini indirmeye çalışın kaydetme penceresinin açıldığını gözlemleyin.

Bu dosyayı indirin ve indirme penceresinin düzgün bir şekilde açıldığını gözlemleyin.
\item \begin{verbatim}http://cekirdek.pardus.org.tr/~semen/dist/test/multimedia/video/cokluortam/\end{verbatim} dizini altında bulunan videolardan birkaçını çalıştırın ve firefox üzerinden çalışabildiğini gözlemleyin.

\end{itemize}

\end{enumerate}

\section{Share alt Bileşeni}
\begin{enumerate}
 \item Aşağıda bulunan paketler sadece kurulum testine tabidir.
\begin{verbatim}
 smbldap-tools
\end{verbatim}

\end{enumerate}

\section{Monitor alt Bileşeni}
\begin{enumerate}
 \item wireshark paketi kurulumu sonrası:
	Wireshark uygulamasını açın, interface listesinden eth0 seçin ve bu interface ile ilgili paketleri listelendiğini gözlemleyin.
\end{enumerate}

\section{Mail alt Bileşeni}
\begin{enumerate}
 \item thunderbird ve sylpheed paketleri kurulumu sonrası:
\begin{itemize}
\item Bir e-posta hesabı oluşturuyoruz.
\item Mail alabildiğimizi gözlemliyoruz.
\item Mail gönderebildiğimizi gözlemliyoruz.
\item Filtre oluşturabildiğimizi gözlemliyoruz.
\item Eğer daha önce thunderbird kullanıyor isek, ev dizini altında .thunderbird dizininin silinmediğini gözlemliyoruz.
\end{itemize}
 \item spamassassin ve spamd paketleri kurulumu sonrası:
\begin{enumerate}
	\item İlgili paketi kurduktan sonra:
	\item Menüden uygulamalar $\rightarrow$ Kmail'i açın ve spam filtresini aktifleştirmek için:
		
	\begin{enumerate}
		\item Kmail menü çubuğundan Araçlar $\rightarrow$  Spam engelleme sihirbazı yolunu izleyin.
		\item İndirdiğiniz ilgili spam filtresini seçin.
		\item Bir postaya sağ tıklayın ve combobox'tan Filtreyi uygula $\rightarrow$ Filtreyi çöp posta olarak sınıflandır yolunu izleyin.
		Bu spam'in ilgili spam klasörüne gittiğini gözlemleyin.( Default spam klasörü eğer değitirmediyseniz çöp klasörü olacaktır.)

		\item Aşağıdaki linkten gtube.txt'yi indirin: 
		\begin{verbatim}
 		http://spamassassin.apache.org/gtube/
		\end{verbatim}
		\item  Konsoldan komutu çalıştırın:
		\begin{verbatim}
 		cat  gtube.txt | spamc 
		\end{verbatim}
		
		Bu komut size içerisinde şifrelenmiş bir satır içeren buna benzer bir çıktı gönderecek:
		
		\emph{If your spam filter supports it, the GTUBE provides a test by which you
	    	can verify that the filter is installed correctly and is detecting incoming
    		spam. You can send yourself a test mail containing the following string of
    		characters (in upper case and with no white spaces and line breaks):}
		\begin{verbatim}
 		XJS*C4JDBQADN1.NSBN3*2IDNEN*GTUBE-STANDARD-ANTI-UBE-TEST-EMAIL*C.34X
		\end{verbatim}
    		\emph{You should send this test mail from an account outside of your network.}

		\item Daha sonra bu ilgili şifrelenmiş kısmı kopyalayıp mail olarak kendinize gönderin.
		
		Bu mailin direk olarak ilgili spam klasörüne gittiğini gözlemleyin.
	\end{enumerate} 
\end{enumerate} 
\end{enumerate}
\section{Plugin alt Bileşeni}
\begin{enumerate}
\item flashplugin paketi kurulumu sonrası:

Opera paketini kurun ve aşağıdaki bağlantıda bulunan videonun düzgün bir şekilde çalıştığını gözlemeyin.

Tam ekran olabiliyor mu? Ses sorunu var mı?
\begin{verbatim}
http://www.dailymotion.com/relevance/search/lorena+mckennit/video/xd9s3_princesse
-mononoke-studioslorenna 
\end{verbatim}
\item gecko-mediaplayer paketi kurulumu sonrası:
\begin{itemize}
  \item Firefox $\rightarrow$ Düzen $\rightarrow$ Seçenekler $\rightarrow$ Eklentileri Yönet $\rightarrow$ Yan Uygulamalar bölümünde gecko-mediaplayer eklentisinin eklenmiş olduğunu gözlemleyin.
  \item Aşağıda bulunan uzantıdaki videoları firefox üzerinden açınız. Ve düzgün bir şekilde çalıştıklarını gözlemleyiniz.
  \begin{verbatim}
  http://cekirdek.pardus.org.tr/~semen/dist/test/multimedia/video/cokluortam/  
  \end{verbatim}
\end{itemize}
\end{enumerate}

\section{Ftp alt Bileşeni}
\begin{enumerate}
 \item lftp paketi kurulumu sonrası:
\begin{verbatim}
 # lftp http://ftp.pardus.org.tr/pub/
 # ls 
 # cd pardus
\end{verbatim}
Yukarıda bulunan komutların sorunsuz bir şekilde çalıştığını gözlemleyin.

\end{enumerate}

\section{Connection alt Bileşeni}
\begin{enumerate}
\item kvpnc paketi kurulumu sonrası:

Kmenüden uygulamanın düzgün çalıştığını gözlemleyin.
\item wireless-regdb paketi kurulumu sonrası:

networ-tr.pdf crda testini gerçekleştirin.
\item crda paketi kurulumu sonrası:

Eğer bir edimax kablosuz aygıtınız var ise aşağıda bulunan komutu çalıştırın.
\begin{verbatim}
 su -
 setregdomain TR
\end{verbatim}

\item ifplugd paketi kurulumu sonrası:

Ağ yöneticisinin kablolu kablosuz tüm ağları listeleyip, otomatik olarak ağları etkinleştirebildiğini ve kapatabildiğini gözlemleyiniz.

 \item Aşağıda bulunan paketler kurulum testine tabidir.
\begin{verbatim}
iw
wireless-regdb
mobile-broadband-provider-info
\end{verbatim}

\end{enumerate}


\end{document}

