\documentclass[a4paper,10pt]{article}
\usepackage[turkish]{babel}
\usepackage[utf8]{inputenc}
\usepackage[left=1cm,top=1cm,right=2cm,bottom=2cm]{geometry}

\title{Library Test Aşamaları}
\author{Semen Cirit}

\renewcommand{\labelenumi}{\arabic{enumi}.}
\renewcommand{\labelenumii}{\arabic{enumi}.\arabic{enumii}.}
\renewcommand{\labelenumiii}{\arabic{enumi}.\arabic{enumii}.\arabic{enumiii}.}
\renewcommand{\labelenumiv}{\arabic{enumi}.\arabic{enumii}.\arabic{enumiii}.\arabic{enumiv}.}

\begin{document}

\maketitle
\begin{enumerate}
\item Aşağıda bulunan paketler sadece kurulum testine tabidir.
\begin{verbatim}
yazpp
xmlsec
libftdi
libp11
libuuid
cddlib
atlas
blas
lapack
libcap-ng
libcap-ng-utils
libtasn1
netcdf
neon
avce00
e00compr
webkit-gtk
webkit-gtk-devel
facile
qca2
qca2-apidocs
qca2-ossl
omniORB
libunistring
FusionSound
DevIL
obexftp
FreeImage
redland-mysql
redland-postgresql
log4c
matio
lensfun
zziplib
gmm
libraw1394
gsm
attica
libiodbc
libffado
gl2ps
linbox
linbox-doc
libytnef
gtkglarea
lablgtk2
egenix-mx-base-doc
libmms
libmpdclient
telepathy-farsight
telepathy-farsight-docs
telepathy-gabble
telepathy-haze
telepathy-sofiasip
telepathy-sunshine
telepathy-butterfly
telepathy-glib
telepathy-glib-docs
libpqxx
libpqxx-docs
xulrunner-devel
givaro
givaro-doc
CImg
CImg-docs
libdumbtts-devel
papyon
libktorrent-devel
gtkglext
gtkglext-devel
libyaml
libyaml-devel
dotconf-devel
yajl
yajl-devel
phoebe-lib-devel
geos
libmtp
libmtp-devel
frei0r-plugins
pigment-devel
gnet
gnet-devel
gnet-docs,
libgs-gnome
libgsf-docs
ETL
dxflib
google-sparsehash
libbinio
libbinio-devel
libconfuse
libconfuse-devel
libmcs
libmcs-devel
libmowgli
libmowgli-devel
libtommath
libtommath-devel
libtommath-docs
loki
loki-devel
loki-docs
onig
onig-devel
redland-bindings-common
xylib
xylib-devel
liblastfm-devel
liblastfm_fingerprint
libssh
\end{verbatim}

\item libprojectM paketi kurulumu sonrası:

multimedia-tr.pdf vlc testini gerçekleştiriniz.

\item liblastfm paketi kurulumu sonrası:

amarok uygulamasını açın ve sol tarafta bulunan "internet"  bölümünü seçin e lastfm'in listelendiğini gözlemleyin.

\item valgrind paketi kurulumu sonrası:

Aşağıdaki komutun sorunsuz bir şekilde çalıştığını gözlemleyin.

\begin{verbatim}
 valgrind /usr/bin/link
\end{verbatim}

\item libgsf paketi kurulumu sonrası:

office-tr.pdf wv testini gerçekleştiriniz.

\item pigment paketi kurulumu sonrası:

programming-tr.pdf pigment-python testini gerçekleştiriniz.

\item glpng paketi kurulumu sonrası:

game-tr.pdf chromium-bsu testini gerçekleştiriniz.

\item hdf5 paketi kurulumu sonrası:

programming-trçpdf python-pytables testini gerçekleştiriniz.

\item phoebe-lib paketi kurulumu sonrası:

science-tr.pdf phoebe-gui testini gerçekleştiriniz.

\item dotconf paketi kurulumu sonrası:

programmming-tr.pdf python-speech-dispatcher testini gerçekleştiriniz.

\item akonadi paketi kurulumu sonrası:

desktop-tr.pdf kdepim testini gerçekleştiriniz.

\item libdlna paketi kurulumu sonrası:

multimedia-tr.pdf ushare tesitini gerçekleştiriniz.

\item libdumbtts paketi kurulumu sonrası:

programming-tr.pdf python-speech-dispatcher testini gerçekleştirin.

\item xine-lib paketi kurulumu sonrası:

Sistem ayarları $\rightarrow$ Çokluortam $\rightarrow$ Arka uç $\rightarrow$ xine seçin. 

Aşağıda bulunan dosyaları ve dragon player ile çalıştırın ve sorunsuz bir şekilde çalıştıklarını gözlemleyin.

\begin{verbatim}
 wget http://cekirdek.pardus.org.tr/~semen/dist/test/multimedia/video/cokluortam/DVD.mpg
 wget http://cekirdek.pardus.org.tr/~semen/dist/test/multimedia/video/cokluortam/Lake_dance_XviD.AVI
\end{verbatim}


\item glew paketi kurulumu sonrası:

game-tr.pdf supertux testini gerçekleştirin.

\item imlib2 paketi kurulumu sonrası:

multimedia-tr.pdf ffmpeg testini gerçekleştirin.

\item libktorrent paketi kurulumu sonrası:

network-tr.pdf ktorrent testini gerçekleştiriniz.

\item gnonlin paketi kurulumu sonrası:

multimedia-tr.pdf pitivi testini gerçekleştirin.

\item OpenSceneGraph paketi kurulumu sonrası:

game-tr.pdf flightgear testini gerçekleştirin.

\item egenix-mx-base paketi kurulumu sonrası:

ipython pakteni kurun ve aşğıda bulunan komutların sorusuz olarak çalıştırğını gözlemleyin.

\begin{verbatim}
 ipython
 import mx
\end{verbatim}


\item pari paketi kurulumu sonrası:

Aşağıda bulunan komutun $2*2$ sonucunu bulduğunu gözlemleyin.
\begin{verbatim}
 gp 
 2*2
\end{verbatim}

\item libpano13 paketi kurulumu sonrası:
  
multimedia-tr.pdf hugin testini gerçekleştirin.

\item libzip paketi kurulumu sonrası:

Aşağıda bulunan zipcmp komutunun düzgün çalıştığını gözlemleyin.
\begin{verbatim}
 mkdir test1
 zip test1.zip test1
 mkdir test2
 zip test2.zip test2
 zipcmp test1.zip test2.zip
\end{verbatim}

\item limprojectM paketi kurulumu sonrası:
  
multimedia-tr.pdf qmmp testini gerçekleştirin.

\item libnids paketi kurulumu sonrası:

network-tr.pdf dsniff testini gerçekleştiriniz.

\item ktoblzcheck paketi kurulumu sonrası:
ipython paketini kurun ve aşağıda bulunan komutları çalıştırın:
\begin{verbatim}
 ipython
 import ktoblzcheck
\end{verbatim}


\item gobject-introspection paketi kurulumu sonrası:
ipython paketini kurun ve aşağıda bulunan komutları çalıştırın:
\begin{verbatim}
 ipython
 import giscanner
\end{verbatim}


\item libnl paketi kurulumu sonrası:

hardwar-tr.pdf bluez testini gerçekleştirin.
\item protobuf paketi kurulumu sonrası:

ipython paketini kurun ve aşağıda bulunan komutları çalıştırın:
\begin{verbatim}
 ipython
 import google
\end{verbatim}
\item tunepimp paketi kurulumu sonrası:

multimedia-tr.pdf kid3 testini gerçekleştirin.

\item libmp4v2 paketi kurulumu sonrası:

Aşağıda bulunan komutun mp4 dosyası ile ilgili özellikleri listelediğini gözlemleyin.
\begin{verbatim}
wget http://cekirdek.pardus.org.tr/~semen/dist/test/multimedia/sound/sound/sample_mpeg4.mp4
mp4info sample_mpeg4.mp4 
\end{verbatim}

\item libcaca paketi kurulumu sonrası:

multimedia-tr.pdf mplayer testini gerçekleştirin.

Aşağıda bulunan komutun sorunsuz çalıştığını gözlemleyin.
\begin{verbatim}
 cacademo
\end{verbatim}

\item opencore-amr paketi kurulumu sonrası:

multimedia-tr.pdf mplayer testini gerçekleştirin.

\item libass paketi kurulumu sonrası:

multimedia-tr.pdf mplayer testini gerçekleştirin.

\item minixml paketi kurulumu sonrası:
game-tr.pdf dreamchess testini gerçekleştiriniz.

\item libmcrypt paketi kurulumu sonrası:

util-tr.pdf mcrypt testini gerçekleştiriniz.

\item libnfnetlink ve libnetfilter\_conntrack paketleri kurulumu sonrası:

network-tr.pdf conntrack-tools testini gerçekleştiriniz.

\item simgear paketi kurulumu sonrası:
game-tr.pdf flightgear testini gerçekleştiriniz.

\item libwww paketi kurulumu sonrası:

util-tr.pdf ntp-client testini gerçekleştiriniz.

\item physfs paketi kurulumu sonrası:

game-tr.pdf supertux paketi kurulumu sonrası.

\item libqjson paketi kurulumu sonrası:

desktop-tr.pdf plasmois-translatoid testini gerçekleştirin.

\item gupnp, gupnp-igd, gssdp paketleri kurulumu sonrası:

network-tr.pdf amsn testini gerçekleştiriniz.

\item farsight2 paketi kurulumu sonrası:

network-tr.pdf amsn testini gerçekleştiriniz.

network-tr.pdf pidgin testini gerçekleştiriniz.

\item gnutls paketi kurulumu sonrası:

network-tr.pdf pidgin testini gerçekleştiriniz.

network-tr.pdf wireshark testini gerçekleştiriniz.

\item avahi-glib, avahi-qt paketleri kurulumu sonrası:

network-tr.pdf pidgin testini gerçekleştiriniz.

\item phonon, phonon-gstreamer, phonon-xine, gst-plugins-good paketleri kurulumu sonrası:

office-tr.pdf koffice-kpresenter testini gerçekleştiriniz.

Sistem Ayarları $\rightarrow$ Çokluortam $\rightarrow$ Arka uç bölümünden ilk önce xine daha sonra gstreamer seçerek aşağıda bulunan testi gerçekleştiriniz.

desktop-tr.pdf kdegames testini gerçekleştirin.
\item ode paketi kurulumu sonrası.

game-tr.pdf stormbaancoureur testini gerçekleştiriniz.

\item libiphone, ifuse, libplist, libusb1, usbmuxd, libimobiledevice paketleri kurulumu sonrası:

Bir iphone cihazınız var ise bunu sisteminize takın ve sistemin algılayıp, cihazınızı mount ettiğini gözlemleyin.

\item libmp3splt paketi kurulumu sonrası:

multimedia-tr.pdf mp3splt testini gerçekleştirin.

\item gd paketi kurulumu sonrası:

Aşağıda bulunan komutun sorunsuz çalıştığını gözlemleyiniz.

\begin{verbatim}
 wget http://cekirdek.pardus.org.tr/~semen/dist/test/library/gdtest.png
 pngtogd  gdtest.png test.gd2
\end{verbatim}

\item taglib ve taglib-extras paketleri kurulumu sonrası:

multimedia-tr.pdf kid3 testini gerçekleştirin.

\item libgphoto2 ve libgphoto2-doc paketleri kurulumu sonrası:

multimedia-tr.pdf digikam testini gerçekleştirin.

\item libimobiledevicebtheora paketi kurulumu sonrası:

multimedia-tr.pdf k3b testini gerçekleştirin.

\item libquicktime paketi kurulumu sonrası:

multimedia-tr.pdf transcode testini gerçekleştirin.

\item libdvdread paketi kurulumu sonrası:

hardware-tr.pdf k3b DVD testini gerçekleştirin.

\item libdvdnav paketi kurulumu sonrası:

multimedia-tr.pdf vlc testini gerçekleştirin.

\item qimageblitz paketi kurulumu sonrası:

blitztest uygulaması ile aşağıda bulunan resmi açın ve effect testinin sorunsuz bir şekilde gerçekleştiğini gözlemleyin.
\begin{verbatim}
  wget http://cekirdek.pardus.org.tr/~semen/dist/test/office/openoffice/test_oodraw.png
  blitztest
\end{verbatim}


\item tre paketi kurulumu sonrası:

multimedia-tr.pdf streamripper testini gerçekleştiriniz.

\item liboil paketi kurulumu sonrası:

library-tr.pdf gst-plugins-ugly testini gerçekleştirin.

\item gstreamer, soundtouch, libkate, gst-plugin-ugly, gst-plugin-bad, gst-plugin-base ve gst-ffmeg paketleri kurulumu sonrası:

Sistem ayarları $\rightarrow$ Çokluortam $\rightarrow$ Arka uç $\rightarrow$ gstreamer seçin. 

dragon player ve kaffein ile aşağıda bulunan dosyaları çalıştırın ve sesin düzgün bir şekilde çıktığını gözlemleyin.
\begin{verbatim}
  wget http://cekirdek.pardus.org.tr/~semen/dist/test/multimedia/video/cokluortam/niceday.asf
  wget http://cekirdek.pardus.org.tr/~semen/dist/test/multimedia/video/cokluortam/MPEG-1_with_VCD_extensions.mpeg
  wget http://cekirdek.pardus.org.tr/~semen/dist/test/multimedia/video/cokluortam/Lake_dance_XviD.AVI
\end{verbatim}

\item libkate, gst-plugin-bad paketleri kurulumu sonrası:

Sistem ayarları $\rightarrow$ Çokluortam $\rightarrow$ Arka uç $\rightarrow$ gstreamer seçin. 

amsn paketini kurun ve msn seslerinin sorunsuz bir şekilde çalıştığını gözlemleyin.


\item exempi ve yaz paketi kurulumu sonrası:

office-tr.pdf tellico paketi testini gerçekleştiriniz.

\item eet ve eina paketi kurulumu sonrası
\begin{itemize}
 \item [2008 için] edb paketini kurun. Aşağıda buluna komutların sorunsuz bir şekilde çalıştığını ve sonuç olarak "default" diye bir çıktı döndürdüğünü gözlemleyin.
\begin{verbatim}
  wget http://cekirdek.pardus.org.tr/~semen/dist/test/library/test_edb
  chmod 755 test_edb
  ./test_edb
  edb_ed test.db get /foo/theme str
\end{verbatim}
\item [2009 için] library-tr.pdf qedje paketi testini gerçekleştiriniz.
\end{itemize}

\item qedje kurulumu sonrası:

Aşağıdaki komutu çalıştırdığınızda hata vermeden çalıştığını gözlemleyiniz.
\begin{verbatim}
 qedje_viewer
\end{verbatim}

\item geoip paketi kurulumu sonrası:
\begin{verbatim}
 geoiplookup www.google.com 
\end{verbatim}
"GeoIP Country Edition: US, United States" gibi bir sonuç döndürdüğünü gözlemleyin.

\item nss ve nspr paketleri kurulumu sonrası:

network-tr.pdf firefox ve office-tr.pdf openoffice testlerini gerçekleştiriniz.

\item openexr paketi kurulumu sonrası:
\begin{itemize}
 \item multimedia-tr.pdf gimp, digikam ve imagemagick testlerini geçekleştiriniz.
 \item Aşağıdaki bağlantıda bulunan resim dosyalarının gwenview ile sorunsuz bir şekilde açıldığını gözlemleyin.
  \begin{verbatim}
   http://cekirdek.pardus.org.tr/~semen/dist/test/multimedia/graphics/graphics.tar
  \end{verbatim}
\end{itemize}
\item iksemel paketi kurulumu sonrası:
\begin{verbatim}
wget http://cekirdek.pardus.org.tr/~semen/dist/test/library/component.xml
ikslint -s component.xml
iksperf -a component.xml 
\end{verbatim}

Yukarıda bulunan komutların düzgün çalıştığını gözlemleyin.

\item xulrunner paketi kurulumu sonrası:
\begin{itemize}
\item office-tr.pdf openoffice testlerini gerçekleştirin.
\item network-tr.pdf firefox testlerini gerçekleştirin.
\item network-tr.pdf gecko-mediaplayer testlerini gerçekleştirin.
\item multimedia-tr.pdf vlc-firefox testlerini gerçekleştirin.
\end{itemize}

\item libvorbis paketi kurulumu sonrası:

multimedia-tr.pdf vorbis-tools testini gerçekleştirin.

\item xerces-c paketi kurulumu sonrası:

Aşağıda bulunan dosyaları aynı dizin içerisine indirin.
\begin{verbatim}
wget http://cekirdek.pardus.org.tr/~semen/dist/test/library/shiporder.xml
wget http://cekirdek.pardus.org.tr/~semen/dist/test/library/shiporder.xsd
wget http://cekirdek.pardus.org.tr/~semen/dist/test/library/test-xerces-c.sh
\end{verbatim}

./test-xerces-c.sh dosyasını çalıştırın ve çıktısında hata çıktısı olup olmadığını gözlemleyin:
\begin{verbatim}
chmod 755 test-xerces-c.sh
./test-xerces-c.sh | less
\end{verbatim}

\item apr ve apr-util paketi kurulumu sonrası:

programming-tr.pdf subversion testini gerçekleştirin.

\item portaudio paketi kurulumu sonrası:

multimedia-tr.pdf qpitch testini gerçekleştirin.

\item libv4l paketi kurulumu sonrası: (Kamerası olanlar test edebilecektir.)


multimedia-tr.pdf vlc testini gerçekleştirin.

\item qca2 ve qca2-apidocs paketi kurulumu sonrası:
\begin{itemize}
 \item Aşağıda bulunan komutu çalıştırın:
\begin{verbatim}
qcatool2 plugins 
\end{verbatim}
\item network-tr.pdf konversation testini gerçekleştirin.
\end{itemize}

\item icu4c paketi kurulumu sonrası:

programming-tr.pdf PyICU testini gerçekleştirin.

\item raptor ve redland paketleri kurulumu sonrası:

office-tr.pdf openoffice yazıcı testini gerçekleştirin.

desktop-tr.pdf kdegraphics testini gerçekleştirin.

\item libnice paketi kurulumu sonrası:

network-tr.pdf pidgin testtini gerçekleştiriniz.

\item libtdb paketi kurulumu sonrası:

samba paketini kurunuz. Servis yöneticisinden samba servisini başlatınız.

Aşağıda bulunan komutların sorunsuz çalıştığını gözlemleyiniz.
\begin{verbatim}
  tdbtool
  create test
  open test 
  insert testkey testdata
  show testkey
\end{verbatim}
\item qzion paketi kurulumu sonrası:

library-tr.pdf qedje testini gerçekleştiriniz.

\end{enumerate}

\end{document}

