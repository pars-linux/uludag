\documentclass[a4paper,10pt]{article}
\usepackage[turkish]{babel}
\usepackage[utf8]{inputenc}
\usepackage[left=1cm,top=1cm,right=2cm,bottom=2cm]{geometry}


\renewcommand{\labelenumi}{\arabic{enumi}.}
\renewcommand{\labelenumii}{\arabic{enumi}.\arabic{enumii}.}
\renewcommand{\labelenumiii}{\arabic{enumi}.\arabic{enumii}.\arabic{enumiii}.}
\renewcommand{\labelenumiv}{\arabic{enumi}.\arabic{enumii}.\arabic{enumiii}.\arabic{enumiv}.}

\title{Kernel Test Aşamaları}
\author{Semen Cirit}

\begin{document}

\maketitle

\section{Default alt bileşeni}

\begin{enumerate}
\item Aşağıda bulunan paketler kurulum testine tabidir:
\begin{verbatim}
 module-vloopback
 module-vloopback-userspace
\end{verbatim}

\item module-ipheth-driver ve module-ipheth-driver-userspace paketleri kurulumu sonrası:

Eğer iphone 3G veya 3Gs cihazınız var ise bu testi gerçekleştirebilirsiniz.

iphone'u usb kablosu ile bağladıktan sonra ağ yöneticisinden iphone'u bir ağ olarak seçin ve internete bağlanmayı deneyin, sorunsuz bir şekilde bağlanabildiğinizi gözlemleyin.

\item perf paketi kurulumu sonrası:
\begin{verbatim}
uname -r  
\end{verbatim}

Yukarıdaki komut sonucunda dönen kernel versiyon numarasını aşağıda bulunan komutun versiyon değeri yerine ekleyiniz.

Versiyon değeri örneğin 2.6.31.11-130-pae için:  2.6.31.11

Aşağıda bulunan komutun kernel fonksiyon bilgilerini listelediğini gözlemleyin.
\begin{verbatim}
sudo  perf.kernel-versiyon top 
\end{verbatim} 


\item kernel, kernel-firmware, kernel-headers, kernel-module-headers kurulumu sonrası:

\begin{itemize}
\item Eğer e-posta ile gönderilen listede kernel modülleri de mevcut ise aşağıda bulunan işlemi yapmalısınız ve ilgili paketlerin test aşamalarını gerçekleştirmelisiniz.

Aşağıdaki komut çıktısı ile listelenen sisteminizde kurulu tüm paketlerin yeni kernele uygun versiyonlarını kurununuz.
\begin{verbatim}
 # pisi li -c kernel.default
\end{verbatim}

\item Bilgisayarınızı kapatınız ve düzgün bir şekilde kapanabildiğini gözlemleyiniz.
\item Açılış ekranın görüntüsünün (Arka tarafında Pardus logosu bulunan grub menüsü) Sorunsuz bir şekilde açıldığını gözlemleyiniz.
\item Bilgisayarınızın yeni kernel ile düzgün bir şekilde açılabildiğini gözlemleyiniz.

Bilgisayarınızı açtıktan sonra konsoldan aşağıda bulunan komutu çalıştırıp, kullandığınız kernelin grub menüsünde görüntülenen kernel ile aynı olduğunu gözlemleyiniz.
\begin{verbatim}
 # uname -r 
\end{verbatim}

\item Eğer dizüstü bilgisayar kullanıyorsanız kablonuzu çıkarıp taktığınızda uyarı verdiğini ve pil seviyesinin düzgün bir şekilde görüntülendiğini gözlemleyiniz.
\item USB bellek takınız ve algılandığını gözlemleyiniz.
\end{itemize}

\item module-alsa-driver, module-alsa-driver-userspace, module-rt-alsa-driver kurulumu sonrası.

Bilgisayarınız yeniden başlatınız.
\begin{itemize}
\item Açılış sesinin düzgün bir şekilde takılmadan çalıştığını gözlemleyiniz.

\item Amarok ile aşağıdaki bağlantıda bulunan ses dosyalarından birkaçını deneyiniz. Sorunsuz bir şekilde çalıştıklarını gözlemleyiniz.
\begin{verbatim}
http://cekirdek.pardus.org.tr/~semen/dist/test/multimedia/sound/sound.tar 
\end{verbatim}

\end{itemize}

\item module-broadcom-wl ve module-broadcom-wl-userspace kurulumu sonrası: 

(Eğer boradcom kablosuz kartınız var ise test edebilirsiniz, kartınızın ne olduğuna bakabilmek için gerekli komut 'lspci')

Bilgisayarınız yeniden başlatınız.

Ağ yöneticisi ile ağa bağlanabildiğinizi gözlemleyiniz.

Aşağıdaki komutu kullanarak ağa bağlı olduğunuzu gözlemleyiniz.
\begin{verbatim}
# ping 4.2.2.1 
\end{verbatim}

\item module-fglrx, module-fglrx-userspace  kurulumu sonrası: 

(Eğer ATI ekran kartınız var ise test edebilirsiniz, kartınızın ne olduğuna bakabilmek için gerekli komut lspci)

Bilgisayarınız yeniden başlatınız. x11-tr.tex zorg test aşamalarını gerçekleştiriniz.

\item module-gspca ve module-gspca-userspace kurulumu sonrası:

Eğer bağlı veya entegre bir kameranız var ise aşağıdaki komutu kullanarak kameranızın çalıştığını gözlemleyiniz.
\begin{verbatim}
# test-webcam
\end{verbatim}

\item module-kvm ve module-kvm-userspace kurulumu sonrası:

Aşağıda bulunan komutu çalıştırarak, bilgisayarınızın sanallaştırma destekleyip desteklemediğine bakınız. Eğer destekliyor ise testi gerçekleştirebilirsiniz. 
\begin{verbatim}
# egrep '^flags.*(vmx|svm)' /proc/cpuinfo	
\end{verbatim}

Aşağıda bulunan komutu çalıştırın:

\begin{verbatim}
 sudo modprobe kvm
\end{verbatim}

Aşağıda bulunan komutun bir çıktı döndürdüğünü gözlemleyin.
\begin{verbatim}
# lsmod | grep kvm
\end{verbatim}

\item module-kqemu ve module-kqemu-userspace kurulumu sonrası:

Aşağıda bulunan komutu çalıştırarak, bilgisayarınızın sanallaştırma destekleyip desteklemediğine bakınız. Eğer destekliyor ise testi gerçekleştirebilirsiniz. 
\begin{verbatim}
# egrep '^flags.*(vmx|svm)' /proc/cpuinfo
\end{verbatim}

Aşağıda bulunan komutun bir çıktı döndürdüğünü gözlemleyiniz.
\begin{verbatim}
# lsmod | grep kqemu
\end{verbatim}

\item module-lirc ve module-lirc-userspace kurulumu sonrası:

(Eğer bilgisayarınız infrared destekliyor ise test edebilirsiniz.)

Aşağıda bulunan komutun bir çıktı döndürdüğünü gözlemleyiniz.
\begin{verbatim}
# lsmod | grep lirc
\end{verbatim}

Aşağıda bulunan komutaları çalıştırın ve daha sonra uzaktan kumanda düğmelerinize basın ve kumandanın tanınmış olduğunu gözlemleyin.
\begin{verbatim}
# modprobe lirc_gpio
# lircd
# irw  
\end{verbatim}
\item module-ltmodem ve module-ltmodem-userspace kurulumu sonrası:

(Eğer bilgisayarınızda entegre bir Lucent Apollo (ISA) ve Mars (PCI) yonga setinde bir modem bulunuyor ise test edebilirsiniz.)

Aşağıda bulunan komutun "ltmodem" içeren bir çıktı döndürdüğünü gözlemleyin.
\begin{verbatim}
 lsmod | grep ltmodem
\end{verbatim}

\item module-slmodem ve module-slmodem-userspace kurulumu sonrası:

(Eğer bilgisayarınızda Smart Link Modem sürücüsü bulunuyor ise test edebilirsiniz.)

Aşağıda bulunan komutun "ltmodem" içeren bir çıktı döndürdüğünü gözlemleyin.
\begin{verbatim}
 lsmod | grep slmodem
\end{verbatim}


\item Aşağıda bulunan paketlerin kurulumu sonrası:
\begin{verbatim}
 module-microdia
 module-microdia-userspace
 module-ov51x-jpeg
 module-ov51x-jpeg-userspace
 module-r5u870
 module-r5u870-userspace
 module-syntekdriver
 module-syntekdriver-userspace
 module-uvcvideo
 module-uvcvideo-userspace
\end{verbatim}

Eğer açıklamalarda bahsedilen sürücülerde kameraya sahip iseniz testleri gerçekleştirebilirsiniz.
Paketin açıklamasına ve kameranızın sürücüsüne bakabilmeniz için gerekli komutlar:
\begin{verbatim}
 # pisi info <paketadı>
 # lsusb
\end{verbatim}

Aşağıdaki komutu kullanarak kameranızın çalıştığını gözlemleyiniz.
\begin{verbatim}
# test-webcam
\end{verbatim}

\item module-ndiswrapper ve module-ndiswrapper-userspace kurulumu sonrası:

(Eğer kablosuz kartınız için windows xp sürücüleri kullanıyorsanız. Kartınız henüz linux'u desteklemiyor ise.)

\begin{itemize}
 \item Aşağıda bulunan komutun bir çıktı döndürdüğünü gözlemleyiniz.
 \begin{verbatim}
 # lsmod | grep ltmodem
 \end{verbatim}
\item Ağ yöneticisinden ağa bağlanabildiğinizi gözlemleyiniz.
\end{itemize}

\item Aşağıda bulunan paketlerin kurulumu sonrası:
 \begin{verbatim}
 module-nvidia-current
 module-nvidia-current-userspace
 module-nvidia173
 module-nvidia173-userspace
 module-nvidia71
 module-nvidia71-userspace
 module-nvidia96
 module-nvidia96-userspace
\end{verbatim}
(Eğer NVIDIA ekran kartınız var ise test edebilirsiniz, kartınızın ne olduğuna bakabilmek için gerekli komut lspci)

Bilgisayarınız yeniden başlatınız. x11-tr.tex altında bulunan test aşamalarını gerçekleştiriniz.

\item module-pabook ve module-omnibook-userspace  kurulumu sonrası: 

Eğer HP Omnibook/Pavillon, Toshiba Satellite (with Phoenix BIOS) dizüstü bilgisayarınız var ise test edebilirsiniz. 

Çeşitli fonksiyon ve multimedya kısayol tuşlarının çalıştığını gözlemleyiniz.

\item module-ungrab-winmodem ve module-ungrab-winmodem-userspace kurulumu sonrası: 

Eğer smartlink mode kullanıyorsanız test edebilirsiniz.

Ağ yöneticisinden ağa bağlanabildiğinizi gözlemleyiniz.

\item  Aşağıdaki paketlerin kurulumu sonrası: 
\begin{verbatim}
 module-virtualbox
 module-virtualbox-userspace
 module-virtualbox-guest
 module-virtualbox-guest-userspace
\end{verbatim}

VirtualBox paketini kurup sorunsuz bir şekilde çalıştığını gözlemleyiniz.

\end{enumerate}

\section{Pae alt bileşeni}

Eğer aşağıda bulunan çıktı pae işaretli bir liste döndürüyor ise bu bölümü test edebilirsiniz.
\begin{verbatim}
cat /proc/cpuinfo | grep pae 
\end{verbatim}

\begin{enumerate}

 \item kernel-pae paketi kurulumu sonrası:

\begin{itemize}
\item
Aşağıdaki komut çıktısı ile listelenen sisteminizde kurulu tüm paketlerin yeni kernele uygun versiyonlarını kurununuz.
\begin{verbatim}
pisi li -c kernel.pae
\end{verbatim}

Eğer daha önce kernel pae kullanmadıysanız yukarıda bulunan komut çıktısı boş dönecektir. Bu durumda aşağıdaki komut çıktısında listelenen paketlerin pae'ye özgü olanlarını sisteminize kurup bu paketler ile ilgili testleri gerçekleştirebilirsiniz. Bu listeye 2.2 bölümünden ulaşabilirsiniz.

\begin{verbatim}
 pisi li -c kernel.default
\end{verbatim}


\item Bilgisayarınızı kapatınız ve düzgün bir şekilde kapanabildiğini gözlemleyiniz.
\item Açılış ekranın görüntüsünün (Arka tarafında Pardus logosu bulunan grub menüsü) Sorunsuz bir şekilde açıldığını gözlemleyiniz.
\item Bilgisayarınızın yeni kernel ile düzgün bir şekilde açılabildiğini gözlemleyiniz.

Bilgisayarınızı açtıktan sonra konsoldan aşağıda bulunan komutu çalıştırıp, kullandığınız kernelin grub menüsünde görüntülenen kernel ile aynı olduğunu gözlemleyiniz.
\begin{verbatim}
 uname -r 
\end{verbatim}

\item Eğer dizüstü bilgisayar kullanıyorsanız kablonuzu çıkarıp taktığınızda uyarı verdiğini ve pil seviyesinin düzgün bir şekilde görüntülendiğini gözlemleyiniz.
\item USB bellek takınız ve algılandığını gözlemleyiniz.
\end{itemize}

\item Aşağıda bulunan paketleri Default bileşeni altında bulunan benzer paketlerin test aşamalarını kullanarak test edebilirsiniz.

\begin{verbatim}
2008 için
virtualbox-modules
virtualbox-guest-modules
virtualbox
virtualbox-guest-utils

2009 için
kernel-module-headers-pae
module-pae-alsa-driver
module-pae-virtualbox-guest
module-pae-virtualbox
module-pae-broadcom-wl
module-pae-lirc
module-pae-nvidia173
module-pae-omnibook
module-pae-ungrab-winmodem
module-pae-gspca
module-pae-ltmodem
module-pae-nvidia71
module-pae-ov51x-jpeg
module-pae-uvcvideo
module-pae-kqemu
module-pae-microdia
module-pae-nvidia96
module-pae-r5u870
module-pae-kvm
module-pae-ndiswrapper
module-pae-nvidia-current
module-pae-syntekdriver   
module-pae-vloopback
module-pae-slmodem
\end{verbatim}
\end{enumerate}
\end{document}

