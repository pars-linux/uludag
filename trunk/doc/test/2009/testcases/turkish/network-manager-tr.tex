\documentclass[a4paper,10pt]{article}
\usepackage[left=1cm,top=2cm,right=2cm,bottom=1cm]{geometry}
\usepackage[turkish]{babel}
\usepackage[utf8]{inputenc}

\title{Ağ Yöneticisi Test Aşamaları}	
\author{Semen Cirit}
\renewcommand{\labelenumi}{\arabic{enumi}.}
\renewcommand{\labelenumii}{\arabic{enumi}.\arabic{enumii}.}
\renewcommand{\labelenumiii}{\arabic{enumi}.\arabic{enumii}.\arabic{enumiii}.}
\renewcommand{\labelenumiv}{\arabic{enumi}.\arabic{enumii}.\arabic{enumiii}.\arabic{enumiv}.}

\begin{document}

\maketitle

\begin{enumerate}
  \item Profil Yarat butonuna tıklayın.
    \begin{enumerate}
     \subsection*{Ethernet veya Kablosuz ağlar için:}
	
      \item Yeni profil oluşturmayı deneyin.
	\begin{enumerate}
	  \item  Hiçbir isim vermeden profili kaydetmeyi deneyin

		Profil ismi hakkında uyarı verildiğini ve yeni profil oluşturmanın engellendiğini gözlemleyin.
	  \item Yeni isim ekleyerek bu profili oluşturmayı tekrar deneyin.

		Problemsiz bir şekilde eklendiğini gözlemleyin.
	\end{enumerate}
      
     \subsection*{Kablosuz Ağlar için:}
     \item Kablosuz ağ için tarama yapmayı deneyin
      
	Kablosuz ağ bulunabildiğini onaylayın.
     \item Parolayı gizle özelliğini kapatın ve parola girin.
        
	Parolanın açık bir şekilde yazılabildiğini gözlemleyin.
      \item Bu kablosuz ağ parolasını kaydetmeyi deneyin.
	
      	Durumu gözlemleyin.
     
     \end{enumerate}
     \item Bağlantıları düzenleme (Her profilin sağ kısmında bulunan bağlantı düzenle butonuna tıklayın.)
    
      \begin{enumerate}
      \item Aygıtların açılır listesinde ethernet veya kablosuz aygıtınızın listede olduğundan emin olun.

	  Bu aygıtları gözlemlemek için şu komutu çalıştırabilirsiniz:
	  \begin{verbatim}
	  network devices
	  \end{verbatim}
	  Komut çıktısıyla açılır listedeki aygıtları karşılaştırın ve aynı olduklarını gözlemleyin.

	  \item Profil ismini değiştirmeyi deneyin
	    \begin{verbatim}
	     network connections
	    \end{verbatim}
	    Ve ismin değiştiğini gözlemleyin.
   
   	  \item Ağ Ayarları ve İsim sunucuları

	    IP adreslerini ve ağ maskelerini listelemek için bu komutu kullanın:
		  \begin{verbatim}
		    # ifconfig
		  \end{verbatim}
	   Bu çıktıda bizim ilgilenecek olduğumz değişkenler:
		   \begin{verbatim}
		    inet addr: değişkeni IP adresimiz
		    Mask: ağ maskemiz
		  \end{verbatim}
	   Ağ geçitlerinin listesi için bu komutu çalıştırın:
	    	   \begin{verbatim}
		     # route -n 
		  \end{verbatim}
	   Ağ geçidinin, 0.0.0.0'a eşit olduğu değer Genmask'i bizim öntanımlı ağ geçidimiz.

	Yukarıdaki bilgiler eşliğinde aşağıdaki aşamaları gerçekleştireceğiz.
	      \begin{enumerate} 
	      \item DHCP Kullan'ı seçin
		\begin{itemize}
		  \item İsim sunucularından öntanımlı veya otomatiği seçip bağlanmaya çalışın.
				      
		   Yukarıdaki iki komutu çalıştırın ve çıktılarından seçili aygıtın otomatik IP adresi, ağ maskesi ve ağ geçidi numarası aldığını gözlemleyin.

		  \item Adresi elle giri seçin.

		   Uygun bir IP adresi ve ağ maskesi girin ve bağlanmaya çalışın.
		
		   Yukarıdaki ilk komutu çalıştırın ve seçili aygıtın sizin verdiğiniz IP adresini ve ağ maskesini aldığını gözlemleyin.
  
		  \item Öntanımlı ağ geçidini kullanı seçin

			Uygun bir ağ geçidi seçin ve bağlanmaya çalışın.
		     
		      Yukarıdaki ikinci komutu çalıştırın ve seçili aygıtın sizin verdiğiniz ağ geçidini aldığını gözlemleyin.
		\end{itemize}
	      \end{enumerate}
	  
	
      \end{enumerate}

	\item Bağlantı için gerekli işlemleri gerçekleştirin ve uygula butonuna basın ve sonra aşağıdaki komutun çıktısını gözlemleyin.
		\begin{verbatim}
		 ping 4.2.2.1
		\end{verbatim}

	Eğer internet bağlantınız varsa aşağıdaki gibi sürekli akacak olan bir çıktı almalısınız:
	\begin{verbatim}
	PING 4.2.2.1 (4.2.2.1) 56(84) bytes of data.
	64 bytes from 4.2.2.1: icmp_seq=1 ttl=244 time=74.8 ms
	64 bytes from 4.2.2.1: icmp_seq=2 ttl=244 time=83.9 ms
      	\end{verbatim}

      \item Profili kaldırmayı deneyin. ((Her profilin sağ tarafında bulunan kaldır butonuna tıklayın.) )
      
	    Şu komutu çalıştırın:
	    \begin{verbatim}
	     network connections
	    \end{verbatim}
	    ve bu komutun çıktısından kaldırılmış olduğunu gözlemleyin.

	\item Herhangi bir işlem yapılmakta iken çıkan kimlik doğrulama penceresine iptal değiniz.

	İptal işlemi sonucunda ağ yöneticisinin sorunsuz bir şekilde yapılmakta olan işlemden önceki duruma geçtiğini gözlemleyiniz.

	\item Herhangi bir işlem yapılmakta iken kimlik doğrulamayı anımsayı seçin 
	
	Ağ yöneticisini kullanarak ağa bağlanmayı deneyin.
	
	Sorunsuz bir şekilde parola sormadan ağa bağlanabildiğinizi gözlemleyin.
	\item Profilleri listele bölümü (Sağ tarafta bulunan bölüm)
	\begin{itemize}
	 \item Kablosuz ağları seçin.
	
	 Var olan kablosuz ağları sorunsuz bir şekilde listeleyebildiğini gözlemleyin.
	
 	\item Kullanılabilir profilleri seçin.
	
	Kullanılabilir profilleri sorunsuz bir şekilde listeleyebildiğini gözlemleyin.
	
	\item Ethernet ağlarını seçin.
	
	 Var olan ethernet ağlarını sorunsuz bir şekilde listeleyebildiğini gözlemleyin.
	
	\end{itemize}
	\item Ethernet bağlantınız var iken ethernet kablonuzu çekin.

	Ait olduğu profil için kablo yada aygıt bağlı değil uyarısını verdiğini gözlemleyin.
	\item Harici kablosuz aygıt kullanıyorsanız ve bu aygıt ile bağlı durumda iseniz, aygıtı çekin.

	Ait olduğu profil için kablo yada aygıt bağlı değil uyarısını verdiğini gözlemleyin.

	\item Kablosuz bir ağa bağlı durumda iken, modemin kapatın.

	Ağ yöneticisindeki ilgili profilin, 180 saniye içerisinde, ağ bağlantısını sonlandırdığını gözlemleyin.

	\item Modem kapalı durumda iken ağa bağlanmaya çalışın.

	 180 saniye içerisinde, bağlantının sağlnamadığı ile ilgili bir uyarı verdiğini gözlemleyin.

\end{enumerate}
\end{document}
