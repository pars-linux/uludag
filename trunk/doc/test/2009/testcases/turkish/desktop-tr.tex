\documentclass[a4paper,10pt]{article}
\usepackage[turkish]{babel}
\usepackage[utf8]{inputenc}
\usepackage[left=1cm,top=1cm,right=2cm,bottom=2cm]{geometry}

\title{Masaüstü Bileşeni Test Aşamaları}
\author{Semen Cirit}

\renewcommand{\labelenumi}{\arabic{enumi}.}
\renewcommand{\labelenumii}{\arabic{enumi}.\arabic{enumii}.}
\renewcommand{\labelenumiii}{\arabic{enumi}.\arabic{enumii}.\arabic{enumiii}.}
\renewcommand{\labelenumiv}{\arabic{enumi}.\arabic{enumii}.\arabic{enumiii}.\arabic{enumiv}.}

\begin{document}

\maketitle
\section{Gnome alt Bileşeni}
\begin{itemize}
 \item Aşağıda bulunan paketler sadece kurulum testine tabidir.
\begin{verbatim}
 libgnomecanvasmm
 gnome-keyring
 gnome-keyring-docs
 gnome-doc-utils
 gnome-common
 libgnome-docs
 libgweather
 libgweather-docs
\end{verbatim}

\item libgnome paketi kurulumu sonrası:

multimedia-tr.pdf padevchooser testini gerçekleştiriniz.


\item gnome-desktop paketi kurulumu sonrası:

multimedia-tr.pdf padevchooser testini gerçekleştiriniz.

\item gvfs-common paketi kurulumu sonrası:

Aşağıda bulunan komutların sorunsuz çalıştığını gözlemleyin.
\begin{verbatim}
 gvfs-mkdir test
 gvfs-ls
 gvfs-info test
\end{verbatim}


\item Aşağıda bulunan paketler aynı şekilde test edilecektir.

\begin{verbatim}
gvfs
gvfs-archive
gvfs-fuse
gvfs-gphoto2
gvfs-obexftp
gvfs-smb
\end{verbatim}

desktop-tr.pdf gvfs-common testini gerçekleştirin.
\end{itemize}

\section{
 and Feel alt Bileşeni}
\begin{itemize}
\item Aşağıda bulunan paketler sadece kurulum testine tabidir.
\begin{verbatim}
 icon-theme-oxygen-svg
 icon-theme-milky-svg
 icon-theme-milky-index
\end{verbatim}

\item icon-theme-milky-png paketi kurulumu sonrası:

Sistem Ayarları $\rightarrow$ Görünüm $\rightarrow$ Simgeler bölümünden milky simge setini seçin ve uygula deyin, sorunsuz bir şekilde milky simge setinin uygulandığını gözlemleyin.

Eğer daha önceden Milky simge seti kurulu ise güncellemelerden sonra sorunsuz olarak açıdığını gözlemleyin.

\item gnome-icon-theme paketi kurulumu sonrası:

Sistem Ayarları $\rightarrow$ Görünüm $\rightarrow$ Simgeler bölümünden GNOME simge setini seçin ve uygula deyin, sorunsuz bir şekilde gnome simge setinin uygulandığını gözlemleyin.

\item artwork-pardus-release paketi kurulumu sonrası:

Masaüstüne sağ tıklayın, görünüm ayarlarını seçin. Resim bölümünden Anıt Kabir veya Lily duvar kağıtlarından birini seçin ve değişiklikleri uygulayın. Değişimin sorunsuz bir şekilde gerçekleştirğini gözlemleyin.

\item icon-theme-oxygen-index ve icon-theme-oxygen-png paketleri kurulumu sonrası: 

Bilgisayarınızı yeniden başlatın ve ikonların düzgün bir şekilde görüntülendiğni gözlemleyin.

 \item Aşağıdaki paketler aynı şekilde test edilecektir.
\begin{verbatim}
cursor-theme-oxygen-black
cursor-theme-oxygen-black-big
cursor-theme-oxygen-blue
cursor-theme-oxygen-blue-big
cursor-theme-oxygen-white
cursor-theme-oxygen-white-big
cursor-theme-oxygen-yellow
cursor-theme-oxygen-yellow-big
cursor-theme-oxygen-zion
cursor-theme-oxygen-zion-big
\end{verbatim}

Sistem ayarları $\rightarrow$  Klavye ve Fare $\rightarrow$ Fare $\rightarrow$ İmleç teması altında listelendiğini gözlemleyin. İmleç temasını değiştirin ve sorunsuz bir şekilde değiştiğini gözlemleyin.

\end{itemize}


\section{Font alt Bileşeni}
\begin{itemize}
\item  terminus-font paketi kurulumu sonrası:

Konsole uygulamasını açın, uygulama üzerine sağ tıklayın, geçerli profili düzenle $\rightarrow$ görünüm $\rightarrow$ Yazı tipini düzenle seçeneklerini izleyin, ve terminus fontunu seçin ve fontun değiştiğini gözlemleyin.

\item  droid-fonts paketi kurulumu sonrası:

Open ofis yazıcı açın ve Droid fontlarının ekli olduğunu gözlemleyin.


\item  urw-fonts paketi kurulumu sonrası:

Open ofis yazıcı açın ve URW fontunun ekli olduğunu gözlemleyin.


\item  dejavu-fonts paketi kurulumu sonrası:

Open ofis yazıcı açın ve Dejavu fontlarının ekli olduğunu gözlemleyin.
 \item fontforge paketi kurulumu sonrası:

aquafont paketini kurun, kmenü'den fontforge uygulamasını açın ve /usr/share/fonts/aquafont/aquafont.ttf uzantısını seçin ve listelenen aquafont.ttf  dosyası üzerine çift tıklayın ve sorunsuz bir şekilde bu fontun karakterleri ile ilgili bir pencere açıldıpğını gözlemleyin.
\item liberation-fonts paketi kurulumu sonrası:

Open ofis yazıcı açın ve LiberationMono, LiberationSans, LiberationSerif fontlarının ekli olduğunu gözlemleyin.

\item ecofont paketi kurulumu sonrası:

Open ofis yazıcı açın ve Sproanq eco sans fontunun ekli olduğunu gözlemleyin.
\item gbfed paketi kurulumu sonrası:

Kmenüden uygulamayı açın ve sorunsuz bir şekilde açılabildiğini gözlemleyin.
\end{itemize}

\section{Misc alt Bileşeni}
\begin{itemize}
\item Aşağıda bulunan paketler sadece kurulum testine tabidir:
\begin{verbatim}
 iTest
 shared-mime-info
 shared-desktop-ontologies
 strigi-devel
\end{verbatim}

\item strigi paketi kurulumu sonrası: 

Sistem ayarları $\rightarrow$ Gelişmiş $\rightarrow$ Masaüstü araması $\rightarrow$  Strigi masaüstü indeksleyiciyi etkinleştir yolunu izleyin ve uygula butonuna basın. Sorunsuz bir şekilde etkinleştiğini gözlemleyin.

\item xdg-user-dirs paketi kurulumu sonrası:

Kullanıcı yöneticisi ile yeni bir kullanıcı yaratın ve daha sonra bu kullanıcı ile sistemi açın ve kullanıcı ev dizini altında sistemin dilinde "Belgeler", "İndirilenler", "Masaüstü" ve "Belgeler" altında da "Müzik", "Resimler", "Videolar" dizinlerinin oluştuğunu gözlemleyin.
\item krename paketi kurulumu sonrası:

Kmenüden uygulamayı açın, aşağıda bulunan resim dosyalarını Ekle butonuna basarak ekleyin. Daha sonra Dosya adı tab'ına gidip Örnek düşen kutusundan sayıyı seçin ve bitir butonuna basın ve sorunsuz bir şekilde yeni ismlendirmelerin yapıldığını gözlemleyin.
\begin{verbatim}
 wget http://cekirdek.pardus.org.tr/~semen/dist/test/desktop/kde/base/circus-bw_hats.jpg
 wget http://cekirdek.pardus.org.tr/~semen/dist/test/desktop/kde/base/tepecik_01.png
\end{verbatim}

\item google-gadgets, google-gadgets-qt  paketleri kurulumu sonrası:

Uygulamalar $\rightarrow$ İnternet yolunu izleyerek uygulamanızı çalıştırın ve programcığın efektler bozulmadan eklendiğini gözlemleyin.

\item google-gadgets-gtk paketi kurulumu sonrası:

Aşağıda bulunan komutu çalıştırdığınızda, google programcığının eklendiğini gözlemleyin.
\begin{verbatim}
 ggl-gtk 
\end{verbatim}

\item basket paketi kurulumu sonrası:

Uygulamanın kemenüden gdüzgün bir şekilde çalıştığını gözlemleyin.

\end{itemize}

\section{Toolkit Bileşeni}
\subsection*{Qt ve Qt4}

(Bu kısımda verilen paket adlarının qt ile başlayan bölümleri, 2008'de qt için qt, qt4 için qt4, 2009'da qt için qt3, qt4 için qt olacaktır.)
\begin{enumerate}
 \item Aşağıda bulunan paketler sadece kurulum testine tabidir.
\begin{verbatim}
 qt-doc
 qt-sql-ibase
 qt-sql-odbc
 qt-sql-postgresql
 qwt-doc
 qwtplot3d-doc
 qwtplot3d
\end{verbatim}

\item qwt ve qwt-qt-designer paketleri kurulumu sonrası:

qt-designer uygulamasını açın ve sol tarfta bulunan parçacık kutusuna "Qwt widgets" bölümünün eklendiğini gözlemleyin.

\item PyQwt paketi kurulumu sonrası

Aşağıda bulunan komutların sorunsuz çalıştığını gözlemleyin.
\begin{verbatim}
 ipython
 import PyQt4
\end{verbatim}


\item gtk-qt-engine paketi kurulumu sonrası

desktop-tr.pdf gtk2-demo testini gerçekleştirin.

gnome-mplayer uygulamasını açın ve görsel olarak sorunsuz açıldığını gözlemleyin.

 \item qt paketi kurulumu sonrası

\begin{verbatim}
 # sudo pisi it -c system.devel
 # mkdir test
 # cd test
 # wget http://cekirdek.pardus.org.tr/~semen/dist/test/desktop/toolkit/test.cpp
 # qmake-qt4 -project
 # qmake-qt4
 # make
 # ./test
\end{verbatim}

"Hello world!" yazan bir pencerenin açıldığını gözlemleyin.
\item qt-designer paketi kurulumu sonrası

Menu $\rightarrow$ Programlar $\rightarrow$ Geliştirme yolunu izleyerek sorunsuz bir şekilde açıldığını gözlemleyin.

\item qt-linguist paketi kurulumu sonrası

Menu $\rightarrow$ Programlar $\rightarrow$ Geliştirme yolunu izleyerek sorunsuz bir şekilde açıldığını gözlemleyin.

\item qt-sql-mysql, qt-sql-sqlite paketi kurulumu sonrası

qt-sql-mysql için Mysql'i servis yöneticisinden başaltınız.
\begin{verbatim}
 sudo pisi it -c system.devel
 mkdir test
 cd test
 wget http://cekirdek.pardus.org.tr/~semen/dist/test/desktop/toolkit/test-qt-sql-<ilgili_veritabanı>.cpp
 qmake-qt4 -project
 qmake-qt4	
\end{verbatim}
qmake komutundan sonra oluşan .pro dosyanıza QT += sql satırını eklemelisiniz. Daha sonra aşağıdaki komutları çalıştırın.
\begin{verbatim}
 # make
 # ./test
\end{verbatim}

Bağlamtının sorunsuz bir şekilde gerçekleştğini gözlemleyin.

\end{enumerate}

\subsection*{Gtk}
\begin{enumerate}
\item Aşağıda bulunan paketler sadece kurulum testine tabidir:
\begin{verbatim}
gtk+extra
gtk-doc
pango-docs
atk
gtk2-docs
gtksourceview
gtksourceview-docs
\end{verbatim}

\item gtkspell paketi kurulumu sonrası:

  Pidgin uygulamasından mesaj gönderme penceresini açın ve yanlış türkçe bir kelime girin, yanlış girdiğiniz kelimenin altının kırmızı olarak çizildiğini ve üzerine sağ tıkladığınızdada doğru olabilecek elternatifleri listelediğini gözlemleyin.

\item gtkhtml3 paketi kurulumu sonrası:

Aşağıda bulunan komutun html editor açtığını gözlemleyin.
\begin{verbatim}
 gtkhtml-editor-test
\end{verbatim}

\item QtCurve-Gtk2 paketi kurulumu sonrası:

Firefox uygulamasının buton sitilleri ile paket yöneticisinin aynı işe yarayan butonlarının sitil olarak da aynı olduğunu gözlemleyin.

Örneğin: Yenileme (Refresh) butonu...
\item libglademm paketi kurulumu sonrası:

multimedia-tr.pdf pavucontrol testini gerçekleştiriniz.
 
 \item cairo,  gtk2-demo paketleri kurulumu sonrası: 

Aşağıda bulunan komutun demo kodlar için bir gui açtığını gözlemleyin:
\begin{verbatim}
gtk-demo
\end{verbatim}

Bu gui'de bulunan listeden:

Drawing Area, Clipboard, Color Selector bölümlerine çift tıklayarak çalıştırın ve sorunsuz bir şekilde çalıştıklarını gözlemleyin.

\item gtk2 paketi kurulumu sonrası: 
\begin{itemize}
 \item multimedia-tr.pdf avidemux testini gerçekleştirin
\item desktop-tr.pdf gtk2-demo testini gerçekleştirin.
\end{itemize}

\item pango paketi kurulumu sonrası: 
\begin{itemize}
 \item progrmming-tr.pdf pygtk testini gerçekleştirin.
 \item multimedia-tr.pdf inkscape testini gerçekleştirin.
\end{itemize}

\end{enumerate}
\subsection*{Others}
\begin{enumerate}
\item wxGTK ve wxGTK-devel paketleri kurulumu sonrası:

science-tr.pdf wxMaxima testini gerçekleştirin.

 \item newt paketi kurulumu sonrası:
\begin{itemize}
 \item ipython paketini kurun ve aşağıda bulunan komutların sorunsuz çalıştığını gözlemleyin:
\begin{verbatim}
 # ipython
 # import snack
\end{verbatim}
  \item hardware-tr.pdf partimage testini gerçekleştirin.

\end{itemize}
 


\end{enumerate}

\section{Kde3 alt Bileşeni}
\subsection*{Base alt bileşeni}
\begin{enumerate}
\item kdebase-beagle paketi kurulumu sonrası:

Kmenu üzerine sağ tıklayın ve panelin kilidini kaldırın ve kickoff menüye geç seçeneğini seçin. Sorunsuz olarak bu menü stiline geçtiğini ve uygulama arama, açma işlemlerini gerçekleştirebildiğini gözlemleyin.

\item arts paketi kurulumu sonrası:

Bilgisayarınızı yeniden başlatın ve kde başlatma sesinin sorunsuz çalıştığını gözlemleyin.

k3b ile bir cd yazdırın bitiminde bitiş sesinin sorunsuz çaldığını gözlemleyin.

 \item kdebase paketi kurulumu sonrası:
Kmenu'den Programlar $\rightarrow$ Yardımcı Programlar $\rightarrow$ Düzenleyiciler $\rightarrow$ Kwrite çalıştırın ve sorunsuz çalıştığını gözlemleyin.

Kmenu'den Programlar $\rightarrow$ İnternet $\rightarrow$ Konqueror çalıştırın ve düzgün çalıştığını gözlemleyin.

Kmenu'den Programlar $\rightarrow$ Sistem $\rightarrow$ Konsole çalıştırın ve düzgün çalıştığını gözlemleyin.

Aşağıda bulunan komutu çalıştırın ve klipper'ı başlatın sorunsuz bir şekilde başladığını gözlemleyin.

\begin{verbatim}
 klipper
\end{verbatim}

\end{enumerate}

\section{Kde4 alt Bileşeni}

\subsection*{l10n alt bileşeni}

\begin{enumerate} 
 \item Aşağıda bulunan paketler için; 

Sistem Ayarları $\rightarrow$ Ülke, Bölge ve Dil yolunu izleyin ve paketle ilgili dili seçerek dilin değiştiğini gözlemleyin. 

\begin{verbatim}
 kde-l10n-ar
kde-l10n-bg
kde-l10n-bn_IN
kde-l10n-ca
kde-l10n-ca-doc
kde-l10n-cs
kde-l10n-csb
kde-l10n-da
kde-l10n-de
kde-l10n-de-doc
kde-l10n-el
kde-l10n-en_GB
kde-l10n-es
kde-l10n-es-doc
kde-l10n-et
kde-l10n-eu
kde-l10n-fi
kde-l10n-fr
kde-l10n-fr-doc
kde-l10n-ga
kde-l10n-gl
kde-l10n-he
kde-l10n-hi
kde-l10n-hu
kde-l10n-is
kde-l10n-it
kde-l10n-it-doc
kde-l10n-ja
kde-l10n-kk
kde-l10n-km
kde-l10n-ko
kde-l10n-ku
kde-l10n-lt
kde-l10n-lv
kde-l10n-mk
kde-l10n-ml
kde-l10n-nb
kde-l10n-nds
kde-l10n-nl
kde-l10n-nl-doc
kde-l10n-nn
kde-l10n-pa
kde-l10n-pl
kde-l10n-pl-doc
kde-l10n-pt
kde-l10n-pt_BR
kde-l10n-pt_BR-doc
kde-l10n-ro
kde-l10n-ru
kde-l10n-sk
kde-l10n-sl
kde-l10n-sr
kde-l10n-sv
kde-l10n-sv-doc
kde-l10n-tg
kde-l10n-th
kde-l10n-tr
kde-l10n-tr-doc
kde-l10n-uk
kde-l10n-wa
kde-l10n-zh_CN
kde-l10n-zh_TW
\end{verbatim}

\end{enumerate}


\subsection*{Admin alt bileşeni}
\begin{enumerate} 
\item display-settings paketi kurulumu sonrası:
  
Eğer yedek bir monitörünüz veya projektörünüz var ise test edebilirsiniz.

Uygulamayı açın ve ilgili çıkışların algılandığını ve ilgili ekranlara görüntünün uygun şekilde geldiğini gözlemleyin.

\item plasmoid-network-applet paketi kurulumu sonrası:

Ağ yöneticisi programcığının masaüstünde sorunsuz bir şekilde bağlantı ile ilgili bilgileri yansıttığını gözlemleyin.

\begin{itemize}
\item Sistem ayarlarından ekran ayarlarını seçin ve aygıtlar bölümüne tıklayın, tıkladığınızda ekran kartınızın modelinin gösterildiğini gözlemleyin.

\item Eğer ekran kartınıza ait uygun sürücü kurulu değil ise uygulamanın paket yöneticisinden ilgili sürücünün kurulması gerektiğine dair uyarı verdiğini gözlemleyin.

\item Eğer birden fazla monitörünüz veya projeksiyon aletiniz var ise makinanınızı be aygıtlara bağlayın.Daha sonra Ekranlar seçeneğini seçin. Açılan pencerede masaüstünü tüm çıkışlara yay kutusunu işaretleyip çıkışları algıla butonuna basın. İki çıkışında algılandığını ve görüntünün iki ekrandada çıktığını gözlemleyin. 
\end{itemize}

 \item history-manager paketi kurulumu sonrası:

 http://svn.pardus.org.tr/uludag/trunk/doc/test/2009/testguide/turkish/alfabeta/history-manager-tr.pdf  testini gerçekleştiriniz.

\item package-manager paketi kurulumu sonrası:

http://svn.pardus.org.tr/uludag/trunk/doc/test/2009/testguide/turkish/alfabeta/package-manager-tr.pdf  testini gerçekleştiriniz.

\item service-manager paketi kurulumu sonrası:

http://svn.pardus.org.tr/uludag/trunk/doc/test/2009/testguide/turkish/alfabeta/service-manager-tr.pdf  testini gerçekleştiriniz.

\end{enumerate}

\subsection*{Base alt bileşeni}

\begin{enumerate} 
\item  Aşağıda bulunan paketler sadece kurulum testine tabidir.
\begin{verbatim}
kdelibs-devel
kdelibs-experimental
kdelibs-apidox 
kdebindings
kdepimlibs-devel
kdepimlibs
kdevplatform-devel
\end{verbatim}
\item kdevplatform paketi kurulumu sonrası:

programming-tr.pdf  kdevelop testini gerçekleştirin.

\item kdewebdev paketi kurulumu sonrası:
\begin{itemize}
 \item quanta uygulamasının sorunsuz çalıştırğını gözlemleyin
\end{itemize}

\item kdeutils paketi kurulumu sonrası:

\begin{itemize}
 \item ark uygulamasının sorunsuz çalıştırğını gözlemleyin

Aşağıda bulunan sıkıştırılmış dosyanın ark ile açılabildiğini gözlemleyin.
\begin{verbatim}
 wget http://cekirdek.pardus.org.tr/~semen/dist/test/multimedia/video/cokluortam.tar
\end{verbatim}

 \item Superkaramba uygulamasını açın ve bu uygulama ile masaüstüne yeni bir programcık ekleyin. Sorunsuz bir şekilde ekleyebildiğinizi gözlemleyin.
\end{itemize}

\item kdesdk paketi kurulumu sonrası:
\begin{itemize}
 \item kate uygulamasının sorunsuz çalıştırğını gözlemleyin.
 \item umbrello uygulamasının sorunsuz çalıştığını gözlemleyin.
\end{itemize}

\item kdepim paketi kurulumu sonrası:
\begin{itemize}
 \item Korganizer uygulamasını açın. Olay, Yapılacaklar Öğesi, Günlük ayarlarını yapın ve sorunsuz bir şekilde yapabildiğinizi gözlemleyin.
 \item Knotes ayarlarını yapın ve sorunsuz bir şekilde yapabildiğinizi gözlemleyin.
 \item Akregator bir rss adresi ekleyin ve rss'leri çekebildiğini gözlemleyin.
\end{itemize}

\item kdenetwork paketi kurulumu sonrası: 

Kopete uygulamasının sorunsuz çalıştığını gözlemleyin.

\item kdeedu paketi kurulumu sonrası: 
\begin{itemize}
 \item kstars uygulamasını Kmenu'den açın ve sorunsuz bir şekilde çalıştığını gözlemleyin. 
 \item kalgebra uygulamasını Kmenu'den açın ve sorunsuz bir şekilde çalıştığını gözlemleyin. 
 \item kig uygulamasını Kmenu'den açın ve sorunsuz bir şekilde çalıştığını gözlemleyin. 
\end{itemize}

\item kdeedu-marble paketi kurulumu sonrası: 

marble uygulamasını Kmenu'den açın ve sorunsuz bir şekilde çalıştığını gözlemleyin. 
\item kdeartwork-wallpapers paketi kurulumu sonrası: 

Masaüstüne sağ tıklayın ve Masaüstü ayarlarını seçin. Duvarkağıdı olarak Çiçek Damlalarını seçin ve duvar kağıdının değiştiğini gözlemleyin.

\item kdeartwork-styles paketi kurulumu sonrası: 

Masaüstüne sağ tıklayın ve Masaüstü ayarlarını seçin. Masaüstü temalarından Heron'u seçin ve temanın değiştiğini gözlemleyin.

\item kdeartwork-sounds paketi kurulumu sonrası:

Oturumu kapatın ve açılış sesinin düzgün çalıştığını gözlemleyin.

\item kdeartwork-screensavers paketi kurulumu sonrası:

Sistem Ayarları $\rightarrow$ Masaüstü $\rightarrow$ Ekran Koruyucu bölümünden asciiquarium seçeneğini seçin ve dene butonuna basın ve ekran koruyucusunun düzgün bir şekilde göründüğünü gözlemleyin.

\item kdeartwork-icons paketi kurulumu sonrası:

Sistem Ayarları $\rightarrow$ Görünüm $\rightarrow$ Simgeler bölümünden Siyah-Beyaz seçeneğini seçin ve uygula butonuna basın. Ve kde ikonlarınızın siyah-beyaz olduğunu gözlemleyin.

\item kdeartwork-emoticons paketi kurulumu sonrası:

Sistem Ayarları $\rightarrow$ Görünüm $\rightarrow$ Duygu Simgeleri bölümünden Kmess-Cartoon seçeneğini seçin ve uygula butonuna basın. Kopete uygulamasını açın ve duygu simgelerinizin değiştiğini gözlemleyin.

\item kdeartwork-colorscheme paketi kurulumu sonrası:

Sistem Ayarları $\rightarrow$ Görünüm $\rightarrow$ Renkler bölümünden Desert seçeneğini seçin ve uygula butonuna basın ve kde renginin kırmızı olduğunu gözlemleyin.

\item kdeadmin paketi kurulumu sonrası:
Aşağıda bulunan komutu çalıştırın ve kde sistem çağrılarının listelendiğini gözlemleyin.
\begin{verbatim}
 ksystemlog
\end{verbatim}

\item PyKDE paketi kurulumu sonrası:

Aşağıda bulunan betiği çalıştırın ve renk paletinini ayarlanabildiğini gözlemleyin.
\begin{verbatim}
cd /usr/kde/4/share/apps/pykde4/examples/kdeuiExamples/
python kcolordialog.py
\end{verbatim}

\item kdeaccessibility paketi kurulumu sonrası:

kmag uygulamasını çalıştırın ve açık olan ekranı büyüttüğünü gözlemleyin.

\item kdemultimedia paketi kurulumu sonrası:
\begin{itemize}
 \item dragon uygulaması için:
Aşağıda bulunan dosyaların uygulama ile düzgün çalıştığını gözlemleyin.
\begin{verbatim}
 wget http://cekirdek.pardus.org.tr/~semen/dist/test/multimedia/video/cokluortam.tar
\end{verbatim}

\item juk uygulaması için:
Aşağıda bulunan dosyaların uygulama ile düzgün çalıştığını gözlemleyin.
\begin{verbatim}
 wget http://cekirdek.pardus.org.tr/~semen/dist/test/multimedia/sound/sound.tar
\end{verbatim}

\item kmix uygulamasının ses ayarlarını düzgün yapabildiğini gözlemleyin.

\end{itemize}


\item kdegames paketi kurulumu sonrası:

Kmenüden ktuberlink ve kapman oyunlarını açın ve ilgili seslerin çıktığını ve sorunsuz çalıştıklarını gözlemleyin.

\item pardus-default-settings paketi kurulumu sonrası:

Pardusun ön tanımlı ayarları olan milky ikon temasının, açılış ekranının, kullanıcı giriş ekranının düzgün bir şekilde açıldığını gözlemleyin.

\item kbd paketi kurulumu sonrası:

Makinenizi yeniden başlatın, klavyenizin dilinin ve fonksiyon tuşlarının düzgün olduğunu gözlemleyiniz. 
\item kdebase-emoticons paketi kurulumu sonrası:

Sistem ayarları $\rightarrow$ Görünüm $\rightarrow$ Emoticon yolunu izleyin ve kde4 için emoticon eklendiğini gözlemleyin.
\item kdebase-sound paketi kurulumu sonrası:

Sistemi yeniden başlatın bitiş ve başlangıç seslerinin çıktığını gözlemleyin.

K3b ile bir cd yazdırın ve sesin cd yazdırma bitiş muziğinin çıktığını gözlemleyin.
\item kdebase-runtime ve kdebase-runtime-doc paketleri kurulumu sonrası:

network-tr.pdf choqok testini gerçekleştirin

Aşağıda bulunan komutu çalıştırın ve sorunsuz bir şekilde çalıştırğını gözlemleyin:
\begin{verbatim}
 # nepomukserver
\end{verbatim}

 \item kdelibs paketi kurulumu sonrası:
\begin{itemize}
 \item network-tr.pdf choqok testini gerçekleştirin


 \item kdegraphics paketini kurun:
\begin{verbatim}	
 # wget http://cekirdek.pardus.org.tr/~semen/dist/test/desktop/kde/base/circus-bw_hats.jpg
 # wget http://cekirdek.pardus.org.tr/~semen/dist/test/desktop/kde/base/tepecik_01.png
\end{verbatim}
Yukarıda bulunan dosyaların (2009 okular) (2008 kpdf) ve gwenview ile açıldığını gözlemleyin.
\item amarok paketini kurun:

x kde4 için 4, kde3 için 3.5 olacak.
\begin{verbatim}
/usr/kde/x/share/sounds/k3b_error1.wav
/usr/kde/x/share/sounds/KDE-Im-Irc-Event.ogg
\end{verbatim}

Dosyalarının düzgün bir şekilde amarok ile açıldığını gözlemleyin.

\item yakuake paketini kurun:

F12 tuşuna basıldığında sorunsuz bir şekilde yakuake'nin açıldığını gözlemleyin.
\end{itemize}
\item kdebase-workspace ve kdebase-workspace-doc paketi kurulumu sonrası:
\begin{itemize}

 \item Bilgisayarınızı kapatın ve yeniden başlatın ve düzgün bir şekilde kde'yi kapattığını ve başlattığını gözlemleyin.

 \item Aşağıda bulunan komutların düzgün bir şekilde çalıştığını gözlemleyin:
\begin{verbatim}
# plasmoidviewer nm-applet 
# klipper
# krunner
# kfontview
\end{verbatim}

\end{itemize}

\item kdebase-wallpapers paketi kurulumu sonrası:

masaüstüne sağ tıklayıp, görünüm ayarlarından duvar kağıdını değiştiri seçiniz. Red Leaf ve Vector Sunset duvar kağıtlarının eklenmiş olduğunu gözlemleyiniz.

\item kdm paketi kurulumu sonrası:

Bilgisayarınızı yeniden başlatın. Açılışta, çıkışta ve kullanıcı değiştirirken çıkan grafiksel giriş ekranınının düzgün bir şekilde açıldığını gözlemleyin.

\item kdeplasma-addons paketi kurulumu sonrası:

Masaüstüne sağ tıklayarak programcık kilidini açın.

Panel Üzerine sağ tıklayarak, programcık ekleyi seçin ve lancelot'u programcık olarak eklemeye çalışın. Düzgün bir şekilde eklendiğini ve çalıştığını gözlemleyin.

Aynı şekilde LCD Weather Station, Twitter Microblogging, RSSNOW, Blue Marble programcıklarını da deneyiniz.

\item kdegraphics paketi kurulumu sonrası:

Aşağıda bulunan dosyaların gwenview  ile çalıştıklarını gözlemleyin.  
\begin{verbatim} 
 # wget http://cekirdek.pardus.org.tr/~semen/dist/test/office/openoffice/test_oodraw.jpg
 # wget http://cekirdek.pardus.org.tr/~semen/dist/test/office/openoffice/test_oodraw.mng
 # wget http://cekirdek.pardus.org.tr/~semen/dist/test/office/openoffice/test_oodraw.png
 # wget http://cekirdek.pardus.org.tr/~semen/dist/test/office/openoffice/test_oodraw.ps
 # wget http://cekirdek.pardus.org.tr/~semen/dist/test/office/openoffice/test_oodraw.tif
 # wget http://cekirdek.pardus.org.tr/~semen/dist/test/office/openoffice/test_oodraw.xcf
 # wget http://cekirdek.pardus.org.tr/~semen/dist/test/office/openoffice/test_openoffice-extension-pdfimport.pdf
 # gwenview
\end{verbatim}
Aşapıda bulunan dosyaların okular ile düzgün çalıştığını gözlemleyin:
\begin{verbatim} 
 # wget http://cekirdek.pardus.org.tr/~semen/dist/test/office/postscript/test_ghostscript.dvi
 # wget http://cekirdek.pardus.org.tr/~semen/dist/test/office/openoffice/test_openoffice-extension-pdfimport.pdf
 # wget http://cekirdek.pardus.org.tr/~semen/dist/test/office/openoffice/test_oodraw.ps
 # okular
 \end{verbatim}

Aşağıda bulunan ugulamaların düzgün çalıştığını gözlemleyin.
\begin{verbatim}
 # kcolorchooser
 # kruller
 # ksnapshot
\end{verbatim}


\end{enumerate}

\subsection*{Addon alt bileşeni}

\begin{enumerate}

\item  kdm-pardus-theme, kdm paketleri kurulumu sonrası:

Bilgisayarınız açılırken kullanıcı giriş ekranının sorunsuz açıldığını ve login olabildiğinizi gözlemleyin.

\item  ksplash-pardus-theme paketi kurulumu sonrası:

Bilgisayarınız açılırken KDE açılış ekranının sorunsuz bir şekilde açıldığını gözlemleyin.
 \item  QtCurve-KDE4 paketi kurulumu sonrası:

Sistem ayarları $\rightarrow$ Görünüm $\rightarrow$ Biçim yolunu izleyerek QtCurve stilini seçiniz ve stilin sorunsuz bir şekilde değiştiğini gözlemleyiniz.

Yeni nir kullanıcı açınız ve türkçe olan sisteminizin uygulamalarının türkçe olduğundan emin olunuz. (gwenview, service manager, kaptan)

\item plasmoid-daisy paketi kurulumu sonrası:

Masaüstüne sağ tıklayıp programcık ekleyi seçin. Daisy'un eklenmiş olduğunu gözlemleyin.

Panele eklenen boşluk üzerine sağ tıklayın ve panel ayarlarını seçin. Daha sonra plasmoid üzerinde çıkan ok üzerine tıklayın ve daisy ayarlarının açıldığını gözlemleyin.

 \item plasmoid-adjustable-clock paketi kurulumu sonrası:

Masaüstüne sağ tıklayıp programcık ekleyi seçin. Adjustable Clock'un eklenmiş olduğunu gözlemleyin.


 \item plasmoid-translatoid paketi kurulumu sonrası:

Masaüstüne sağ tıklayıp programcık ekleyi seçin. Translatoid'in eklenmiş olduğunu gözlemleyin.

Programcığı açın ve türkçeden ingilizceye bir çeviri yapmayı deneyin.
 \item kshutdown paketi kurulumu sonrası:

Aşağıda bulunan komutu çalıştırın ve kapatma penceresinin açıldığını gözlemleyin.
\begin{verbatim}
 # kshutdown 
\end{verbatim}

 \item kaptan paketi kurulumu sonrası:

  http://svn.pardus.org.tr/uludag/trunk/doc/test/2009/testguide/turkish/alfabeta/kaptan-tr.pdf testini gerçekleştirin.

\end{enumerate}

\end{document}

