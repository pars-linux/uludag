\documentclass[a4paper,10pt]{article}
\usepackage[turkish]{babel}
\usepackage[utf8]{inputenc}
\usepackage[left=1cm,top=1cm,right=2cm,bottom=2cm]{geometry}

\title{System Bileşeni Test Aşamaları}
\author{Semen Cirit}

\renewcommand{\labelenumi}{\arabic{enumi}.}
\renewcommand{\labelenumii}{\arabic{enumi}.\arabic{enumii}.}
\renewcommand{\labelenumiii}{\arabic{enumi}.\arabic{enumii}.\arabic{enumiii}.}
\renewcommand{\labelenumiv}{\arabic{enumi}.\arabic{enumii}.\arabic{enumiii}.\arabic{enumiv}.}

\begin{document}

\maketitle
\section{Auth alt Bileşeni}
\begin{enumerate}
  \item Aşağıda bulunan paketler kurulum testine tabidir.
\begin{verbatim}
 pam_krb5
 pam_ldap
 nss_ldap
\end{verbatim}

\end{enumerate}

\section{Boot alt Bileşeni}
\begin{enumerate}
 \item Aşağıda bulunan paketler kurulum testine tabidir.
\begin{verbatim}
 gfxboot
 gfxtheme-pardus-install
 syslinux
 bootchart
\end{verbatim}

\item gfxtheme-base ve gfxtheme-pardus-boot paketleri kurulumu sonrası.

Bilgisayarınızı yeniden başlatın ve açılış ekranının düzgün bir şekilde görüntülendiğini gözlemleyin.
\end{enumerate}

\section{Devel alt Bileşeni}
\begin{enumerate}
 \item Aşağıda bulunan paketler sadece kurulum testine tabidir.
\begin{verbatim}
 xtrans
 automake
 dietlibc 
 quilt
 gcc
 libgfortran
 libgomp
 libobjc
 nasm
 patch
 util-macros
 xorg-proto
\end{verbatim}
 \item icecream ve icecream scheduler paketleri kurulumu sonrası:

Servis yöneticisinden icecream servisini başlatın.

Servisin başlatılmış olduğunu aşağıdaki komutu çalıştırarak emin olun.
 \begin{verbatim}
   # service icercream status
 \end{verbatim}

\end{enumerate}


\section{Base alt Bileşeni}

\begin{enumerate}
\item Aşağıda bulunan paketler kurulum testine tabidir.
\begin{verbatim}
 libgcc
 libXau
 hdparm
 popt
 libssh2
 dbus-glib
\end{verbatim}

\item diffutils paketi kurulumu sonrası:

Aşağıda bulunan komutların sorunsuz bir şekilde çalıştığını gözlemleyin.
\begin{verbatim}
 wget http://cekirdek.pardus.org.tr/~semen/dist/test/system/base/test-diffutils
 wget http://cekirdek.pardus.org.tr/~semen/dist/test/system/base/test-diffutils2
 touch test
 diff3  test-diffutils test-diffutils2 test 
\end{verbatim}

diff3 
\item dbus-python paketi kurulumu sonrası:

Aşağıda bulunan komutların sorunsuz çalıştığını gözlemleyin.
\begin{verbatim}
 ipython
 import dbus
\end{verbatim}


\item comar paketi kurulumu sonrası:

desktop-tr.pdf service-manager ve package-manager testlerini gerçekleştiriniz.
\item python-fchksum paketi kurulumu sonrası:

Aşağıda bulunan komutları çalıştırın ve "Test python-fchksum"'ın crc32'sinin alınabildiğini gözlemleyin.
\begin{verbatim}
 ipython
 import binascii
 binascii.crc32("Test python-fchksum")
\end{verbatim}

\item flex paketi kurulumu sonrası:

m4 ve yacc paketlerini kurun.
Aşağıdaki komutları çalıştırın ve sorunsuz bir şekilde çalıştıklarını gözlemleyin.
\begin{verbatim}
 wget http://cekirdek.pardus.org.tr/~semen/dist/test/system/base/flex-test.l
 flex flex-test.l
 gcc lex.yy.c -o flex-test -lfl 
 ./flex-test
 stop
 start
\end{verbatim}


\item sqlite paketi kurulumu sonrası:
programming-tr.pdf pysqlite testini gerçekleştiriniz.

\item urlgrabber paketi kurulumu sonrası:

ipython paketini kurun ve aşağıda bulunan komutları çalıştırın:
\begin{verbatim}
 # ipython
 import urlgrabber
\end{verbatim}

\item pycurl paketi kurulumu sonrası:

ipython paketini kurun ve aşağıda bulunan komutları çalıştırın:
\begin{verbatim}
 # ipython
 import curl
\end{verbatim}

\item freetype paketi kurulumu sonrası: 

Bilgisayarınızı yeniden başlatın ve ilk açılış ekranında fontların düzgün olduğunu gözlemleyin.
\item texinfo paketi kurulumu sonrası:
\begin{verbatim}
 wget http://svn.pardus.org.tr/uludag/trunk/doc/test/2009/testcases/turkish/editor-tr.tex
 texi2dvi editor-tr.tex
 okular editor-tr.dvi
\end{verbatim}


\item pcmciautils paketi kurulumu sonrası:

Eğer takılı bir PCIA kartınız bulunmakta ise, aşağıda bulunan komutun bu kartı listelediğini gözlemleyiniz.
\begin{verbatim}
 su -
 lspcmcia
\end{verbatim}



\item man-pages paketi kurulumu sonrası:

Aşağıda bulunan sayfaların sorunsuz bir şekilde çalıştırklarını gözlemleyin.
\begin{verbatim}
 man tcp
 man getspent
\end{verbatim}


\item less paketi kurulumu sonrası:

Aşağıda bulunan komutun sorunsuz çalıştığını gözlemleyin.
\begin{verbatim}
pisi info -F python | less 
\end{verbatim}


\item bootsplash-theme-pardus paketi kurulumu sonrası:

Bilgisayarınızı yeniden başlatın ve açılış ekranının sorunsuz bir şekilde açıldığını gözlemleyiniz.

\item coreutils paketi kurulumu sonrası:

Aşğıda bulunan komutların sorunsuz bir şekilde çalıştığını gözlemleyin.
\begin{verbatim}
 # date
 # pwd
 # uname -a
\end{verbatim}


\item python paketi kurulumu sonrası:
\begin{itemize}
 \item system-tr.pdf pisi testini gerçekleştiriniz.
 \item programming-tr.pdf sympy testini gerçekleştiriniz.
 \item programming-tr.pdf python-iptables testini gerçekleştiriniz.
\end{itemize}


\item openssl paketi kurulumu sonrası:

Aşağıda bulunan komutların sorunsuz bir şekilde çalıştığını gözlemleyin:
\begin{verbatim}
  # openssl ciphers -v 
  # openssl ciphers -v -tls1
  # openssl speed
  # openssl req  -new -newkey rsa:1024 -nodes -keyout mykey.pem -out myreq.pem
  // verify signature
  # openssl req -in myreq.pem -noout -verify -key mykey.pem
  // check info
  # openssl req -in myreq.pem -noout -text

\end{verbatim}


\item usbutils paketi kurulumu sonrası:

Aşağıda bulunan komutların düzgün çalıştığını gözlemleyin:
\begin{verbatim}
# usb-devices
# lsusb
\end{verbatim}


\item udev paketi kurulumu sonrası:

Bilgisayarınızı yeniden başlatın ve düzgün bir şekilde açıldığını gözlemleyin. 

Ses, kamera, mount, görüntü gibi işlemlerin düzgün çalıştığını gözlemleyin.
\item Aşağıda bulunan paketler kurulum testine tabidir.
\begin{verbatim}
e2fsprogs 
\end{verbatim}
\item module-init-tools paketi kurulumu sonrası:

Bilgisayarınızı yeniden başlatınız:
Ses, kablosuz ağ, bluetooth, kamera gibi şeylerin çalıştığını gözlemleyiniz.

Aşağıda bulunan komutların sorunsuz olarak çalıştığını gözlemleyiniz.
\begin{verbatim}
 # lsmod
 # modinfo ahci
\end{verbatim}


\item dnsmasq paketi kurulumu sonrası:

\begin{itemize}
 \item Service yöneticinizden dnsmasq servisini başlatın.
 \item Ağ yöneticisinden bağlantınızı durdurun ve yeniden başlatın.
 \item Aşağıdaki komutu çalıştırın ve sorunsuz bir şekilde sorfu zamanını döndürdüğünü gözlemleyin.
\begin{verbatim}
 # dig http://archlinux.org | grep "Query time"
\end{verbatim}

\end{itemize}


\item file paketi kurulumu sonrası:

\begin{verbatim}
 # wget http://cekirdek.pardus.org.tr/~semen/dist/test/office/openoffice/test_oodraw.mng
 # wget http://cekirdek.pardus.org.tr/~semen/dist/test/office/openoffice/test_oodraw.odg
 # wget http://cekirdek.pardus.org.tr/~semen/dist/test/office/openoffice/test_oodraw.jpg
 # wget http://cekirdek.pardus.org.tr/~semen/dist/test/office/openoffice/test_oodraw.gif
 # wget http://cekirdek.pardus.org.tr/~semen/dist/test/office/openoffice/test_oodraw.png
 # wget http://cekirdek.pardus.org.tr/~semen/dist/test/office/openoffice/test_oodraw.tif
 # wget http://cekirdek.pardus.org.tr/~semen/dist/test/office/openoffice/test_oowriter.txt
 # wget http://cekirdek.pardus.org.tr/~semen/dist/test/office/openoffice/test_oodraw.ps
 # wget http://cekirdek.pardus.org.tr/~semen/dist/test/office/openoffice/
   test_openoffice-extension-pdfimport.pdf
\end{verbatim}

Yukarıda bulunan dosyaları aşağıda bulunan komut ile çalıştırın, dosya formatlarını düzgün bir şekilde bulduğunu gözlemleyin.
\begin{verbatim}
 # file <dosya adı>
\end{verbatim}

\item mudur paketi kurulumu sonrası:

\begin{itemize}
  \item Makinenizi yeniden başlatın ve sistemin düzgün bir şekilde açıldığını  gözlemleyin.
 \item Ctrl+Alt+F1 tuşuna basıp sistem konsoluna geçin ve sistem dilinizin daha önceki dil ve kalvye düzeni ile aynı olduğunu gözlemleyin.
  \item /etc/mudur/ altında bulunan locale, language, keymap dosyalarının sistem dil ve klavye dilinize eskisi gibi uygun olduğunu gözlemleyin.
 \item Pisi komutlarının düzgün bir şekilde çalıştığını gözlemleyin. 

\end{itemize}
\item tiff paketi kurulumu sonrası:
\begin{verbatim}
 # wget http://cekirdek.pardus.org.tr/~semen/dist/test/desktop/kde/base/doga.tiff
 # wget http://cekirdek.pardus.org.tr/~semen/dist/test/desktop/kde/base/istanbul.tiff
\end{verbatim}

Resimlerin üzerine sağ tıklayarak gwenview, kolourPaint, gimp, showfoto ile açılabildiklerini gözlemleyin.
\item pisi paketi kurulumu sonrası:
\begin{itemize}
 \item package-manager-tr.pdf test aşamalarını gerçekleştiriniz.
 \item history-manager-tr.pdf test aşamalarını gerçekleştiriniz.
\end{itemize}
\item baselayout paketi kurulumu sonrası:
\begin{itemize}
 \item Bilgisayarınızı kapatın düzgün bir şekilde kapanabildiğini gözlemleyin.
 \item Bilgisayarınızı yeniden başlatın ve düzgün bir şekilde açılabildiğini gözlemleyin.
 \item 2009 için:
  \begin{verbatim}
   # pisi blame baselayout
  \end{verbatim}
      2008 için: http://svn.pardus.org.tr/pardus/2008/stable/system/base/baselayout/pspec.xml linkinde bulunan en son eklenmiş history tag'i yorumuna (comment) bakınız.

   Bu her iki durumda da yorumlarda belirtilmiş olan kullanıcının /etc/passwd dosyasına eklenmiş olduğunu gözlemleyin.

\end{itemize}
\item libxml2 paketi kurulumu sonrası:

\begin{itemize}
\item Aşağıda bulunan komutun bir katalog oluşturduğunu gözlemleyin.
  \begin{verbatim}
   #  xmlcatalog --create
  \end{verbatim}
\item multimedia-tr.pdf avidemux-qt, avidemux ve inkscape testlerini gerçekleştirin.
\end{itemize}
\item curl paketi kurulumu sonrası:

\begin{itemize}
\item http://pardus.org.tr adresinin içeriğinin hata alınmadan çıktı olarak alındığını gözlemleyin.
\begin{verbatim}
# wget http://cekirdek.pardus.org.tr/~semen/dist/test/system/base/test_curl.php
# php test_curl.php
\end{verbatim}

\item network-tr.pdf sylpheed testini geçekleştirin.
\end{itemize}

\item glib2 paketi kurulumu sonrası:
\begin{itemize}
 \item system-tr.pdf openssh testini gerçekleştirin.
\item system-tr.pdf getext testini gerçekleştirin.
\end{itemize}

\item gettext paketi kurulumu sonrası:

Aşağıda bulunan komutun "network-manager.pot" dosyasını oluşturduğunu gözlemleyin.
\begin{verbatim}
 # svn co http://svn.pardus.org.tr/uludag/trunk/kde4/network-manager/
 # cd network-manager/manager
 # rm -rf po
 # mkdir po
 # xgettext setup.py 
 # ./setup.py update_messages
\end{verbatim}


\item openssh paketi kurulumu sonrası:

Servis yöneticisinden openssh'ı çalıştırın ve aşağıda bulunan komutun çıktısında openssh'ın çalıştığını gözlemleyin:
\begin{verbatim}
 # service list 
\end{verbatim}

\item mkinitramfs paketi kurulumu sonrası:

Bilgisayarınızı yeniden başlatın. Düzgün bir şekilde açıldığını gözlemleyin.

/boot/ dizini altında kullanmakta olduğunuz kernelin initramfs dosyasının var olduğunu gözlemleyin.

\item pardus-python paketi kurulum testine tabidir.

\end{enumerate}
\section{Service alt Bileşeni}
\begin{enumerate}
\item omniORB paketi kurulumu sonrası:

Servis yöneticisinden omniORB başlatın, aşağıdaki komut ile başlatıldığına emin olun:
\begin{verbatim}
 # service omniORB status
\end{verbatim}

 \item memcached paketi kurulumu sonrası:

  programming-tr.pdf python-memcached testini gerçekleştirin. 
\end{enumerate}



\end{document}

