 \documentclass[a4paper,10pt]{article}
\usepackage[turkish]{babel}
\usepackage[utf8]{inputenc}
\usepackage[left=1cm,top=1cm,right=2cm,bottom=2cm]{geometry}

\title{System Bileşeni Test Aşamaları}
\author{Semen Cirit}

\renewcommand{\labelenumi}{\arabic{enumi}.}
\renewcommand{\labelenumii}{\arabic{enumi}.\arabic{enumii}.}
\renewcommand{\labelenumiii}{\arabic{enumi}.\arabic{enumii}.\arabic{enumiii}.}
\renewcommand{\labelenumiv}{\arabic{enumi}.\arabic{enumii}.\arabic{enumiii}.\arabic{enumiv}.}

\begin{document}

\maketitle
\section{Auth alt Bileşeni}
\begin{enumerate}
  \item Aşağıda bulunan paketler kurulum testine tabidir.
\begin{verbatim}
 pam_krb5
 pam_ldap
 nss_ldap
 pam_pkcs11
\end{verbatim}

\end{enumerate}

\section{Boot alt Bileşeni}
\begin{enumerate}
 \item Aşağıda bulunan paketler kurulum testine tabidir.
\begin{verbatim}
 gfxboot
 gfxtheme-pardus-install
 syslinux
 bootchart
\end{verbatim}

\item gfxtheme-base ve gfxtheme-pardus-boot paketleri kurulumu sonrası.

Bilgisayarınızı yeniden başlatın ve açılış ekranının düzgün bir şekilde görüntülendiğini gözlemleyin.
\end{enumerate}

\section{Devel alt Bileşeni}
\begin{enumerate}
 \item Aşağıda bulunan paketler sadece kurulum testine tabidir.
\begin{verbatim}
 xtrans
 dietlibc 
 quilt
 gcc
 libgfortran
 libgomp
 libobjc
 nasm
 yasm
 patch
 util-macros
 xorg-proto
 libtool
 perl-XML-Parser
 cmake
 intltool
 scons
 autoconf
 pkgconfig
\end{verbatim}
\item automake paketi kurulumu sonrası:

Aşağıda bulunan paketlerin sorunsuz derlendiğini gözlemleyin.
\begin{verbatim}
 sudo pisi it subversion
 2008 için
 svn co http://svn.pardus.org.tr/pardus/2008/devel/applications/archive
 cd applications/xdelta
 sudo pisi  bi pspec.xml -vd

 2009 için
 svn co http://svn.pardus.org.tr/pardus/2008/devel/hardware/library/
 cd library/libgpod
 sudo pisi  bi pspec.xml -vd

\end{verbatim}

 \item icecream ve icecream scheduler paketleri kurulumu sonrası:

Servis yöneticisinden icecream servisini başlatın.

Servisin başlatılmış olduğunu aşağıdaki komutu çalıştırarak emin olun.
 \begin{verbatim}
   # service icercream status
 \end{verbatim}

\end{enumerate}


\section{Base alt Bileşeni}

\begin{enumerate}
\item Aşağıda bulunan paketler kurulum testine tabidir.
\begin{verbatim}
 libgcc
 libXau
 hdparm
 popt
 libssh2
 dbus-glib 
 libtool-ltdl 
 memtest 
 zlib
 dbus
 pam
\end{verbatim}
\item shadow paketi kurulumu sonrası:

Aşağıda bulunan komutun, root yetkisine geçirdiğini gözlemleyin.
\begin{verbatim}
 su -
\end{verbatim}


\item bzip2 paketi kurulumu sonrası:

Aşağıda bulunan komutun sorunsuz bir şekilde dosyaları sıkıştırdığını gözlemleyin:

\begin{verbatim}
 touch file1
 bzip2 -v file1
\end{verbatim}

\item grep paketi kurulumu sonrası:

Aşağıda bulunan komutun sorunsuz çalıştığını gözlemleyin.

\begin{verbatim}
 ps aux | grep firefox
\end{verbatim}

\item disktype paketi kurulumu sonrası:

Aşağıda bulunan komutun sorunsuz çalıştığını gözlemleyin.

\begin{verbatim}
 disktype /dev/sda1
\end{verbatim}

\item iputils paketi kurulumu sonrası:

Aşağıda bulunan komut ile ağ paketlerinin gidip gelebildiğini gözlemleyin, çıkmak için CTRL+C tuşlarına basın.
\begin{verbatim}
 ping 4.2.2.1
\end{verbatim}

\item cpio paketi kurulumu sonrası:

Aşağıda bulunan komutun sorunsuz bir şekilde çalıştığını gözlemleyin.
\begin{verbatim}
 su -
 find . -depth -print | cpio -pvd /mnt/backup
\end{verbatim}


\item tar paketi kurulumu sonrası:

Aşağıda bulunan tar komutlarının sorunsuz bir şekilde sıkıştırma ve açma işlemlerini yaptığını gözlemleyin.
\begin{verbatim}
 touch test-tar
 tar -cvf test-tar.tar.gz test-tar 
 tar -xvf test-tar.tar.gz
\end{verbatim}


\item sysklogd paketi kurulumu sonrası:
Servis yöneticisi ile bu servisin çalışmakta olduğunu gözlemleyin.

\item grub paketi kurulumu sonrası:

Bilgisayarınızı başlatığınızda makinanızda bulunan sistemleri listeleyen pencerenin sorunsuz bir şekilde sistemlerinizi listelediğini gözlemleyin. 

\item libpng paketi kurulumu sonrası:

multimedia-tr.pdf inkscape testini gerçekleştiriniz.

\item gzip paketi kurulumu sonrası:
\begin{verbatim}
 touch test-gzip
 gzip -9 test-gzip
 gunzip test-gzip 
\end{verbatim}


\item perl paketi kurulumu sonrası:

Aşağıda bulunan komutun sorunsuz çalıştığını gözlemleyin.
\begin{verbatim}
 wget http://cekirdek.pardus.org.tr/~semen/dist/test/system/base/test_perl.pl
 perl test_perl.pl
\end{verbatim}

\item net-tools paketi kurulumu sonrası:

Aşağıda bulunan komutların sorunsuz olarak çalıştığını gözlemleyin.
\begin{verbatim}
 ifconfig
 hostname
\end{verbatim}


\item wireless-tools paketi kurulumu sonrası:

Eğer bir kablosuz kartınızın bulunduğu makinanız var ise test edebilirsiniz!

Aşağıda bulunan komutun bağli olduğunuz bağlantı noktası bilgilerini listelediğini gözlemleyin.
\begin{verbatim}
 iwlist scanning
\end{verbatim}


\item util-linux paketi kurulumu sonrası:
\begin{verbatim}
 dmesg
 mount 
 blkid
 fdisk
\end{verbatim}

\item nano paketi kurulumu sonrası:

Aşağıda bulunan dosyanın nano ile açılabildiğini gözlemleyin daha sonra bir kaç değişiklik yapın ve CTRL+X ve CTRL+Y tuşuna basın ve kaydedin. 

Tekrar nano ile açın ve değişikliklerin yapılmış olduğunu ve sorunsuz açıldığını gözlemleyin.
\begin{verbatim}
wget http://cekirdek.pardus.org.tr/~semen/dist/test/system/base/test-nano.txt 
nano test-nano.txt
\end{verbatim}


\item timezone paketi kurulumu sonrası:

Panel'de bulunan takviminizin zaman dilimini değiştirin, ve sorunsuz bir şekilde değiştiğini gözlemleyin.

Örneğin Pasifik Line adaları...

\item kbd paketi kurulumu sonrası:

Aşağıda bulunan betiğin "Test encountered 0 errors." çıktısını döndürdüğünü gözlemleyin.
\begin{verbatim}
 wget http://cekirdek.pardus.org.tr/~semen/dist/test/system/base/test_keymap.sh
 chmod 755 test_keymap.sh
 ./test_keymap.sh
\end{verbatim}



\item glibc-zoneinfo paketi kurulumu sonrası:
Panel'de bulunan takviminizin zaman dilimini değiştirin, ve sorunsuz bir şekilde değiştiğini gözlemleyin.

\item glibc paketi kurulumu sonrası:

Screen paketini kurun ve aşağıda bulunan komutun, başlı kütüphaneleri listelediğini gözlemleyin.
\begin{verbatim}
ldd /usr/bin/screen 
\end{verbatim}

\item expat paketi kurulumu sonrası:

Aşağıdaki konutun hiç bir çıktı göndermediğini gözlemleyiniz.
\begin{verbatim}
 wget http://cekirdek.pardus.org.tr/~semen/dist/test/system/base/test-expat.xml
 xmlwf -c test-expat.xml
\end{verbatim}

\item diffutils paketi kurulumu sonrası:

Aşağıda bulunan komutların sorunsuz bir şekilde çalıştığını gözlemleyin.
\begin{verbatim}
 wget http://cekirdek.pardus.org.tr/~semen/dist/test/system/base/test-diffutils
 wget http://cekirdek.pardus.org.tr/~semen/dist/test/system/base/test-diffutils2
 touch test
 diff3  test-diffutils test-diffutils2 test 
\end{verbatim}

diff3 
\item dbus-python paketi kurulumu sonrası:

Aşağıda bulunan komutların sorunsuz çalıştığını gözlemleyin.
\begin{verbatim}
 ipython
 import dbus
\end{verbatim}


\item comar ve comar-api paketleri kurulumu sonrası:

desktop-tr.pdf service-manager ve package-manager testlerini gerçekleştiriniz.
\item python-fchksum paketi kurulumu sonrası:

Aşağıda bulunan komutları çalıştırın ve "Test python-fchksum"'ın crc32'sinin alınabildiğini gözlemleyin.
\begin{verbatim}
 ipython
 import binascii
 binascii.crc32("Test python-fchksum")
\end{verbatim}

\item flex paketi kurulumu sonrası:

m4 ve yacc paketlerini kurun.
Aşağıdaki komutları çalıştırın ve sorunsuz bir şekilde çalıştıklarını gözlemleyin.
\begin{verbatim}
 wget http://cekirdek.pardus.org.tr/~semen/dist/test/system/base/flex-test.l
 flex flex-test.l
 gcc lex.yy.c -o flex-test -lfl 
 ./flex-test
 stop
 start
\end{verbatim}


\item sqlite paketi kurulumu sonrası:
programming-tr.pdf pysqlite testini gerçekleştiriniz.

\item urlgrabber paketi kurulumu sonrası:

ipython paketini kurun ve aşağıda bulunan komutları çalıştırın:
\begin{verbatim}
 ipython
 import urlgrabber
\end{verbatim}

\item pycurl paketi kurulumu sonrası:

ipython paketini kurun ve aşağıda bulunan komutları çalıştırın:
\begin{verbatim}
 ipython
 import curl
\end{verbatim}

\item freetype paketi kurulumu sonrası: 

\begin{itemize}
 \item Bilgisayarınızı yeniden başlatın ve sisteminizde bulunan fontların sorunsuz bir şekilde açıldığını gözlemleyin.
 \item Aşağıda bulunan dosyayı indirin ve üzerine tıklayarak açın, tüm karakterlerin düzgün bir şekilde çıktığını gözlemleyin.

  \begin{verbatim}
  wget http://cekirdek.pardus.org.tr/~semen/dist/test/system/base/vollkorn.otf
  \end{verbatim}
\end{itemize}


\item texinfo paketi kurulumu sonrası:
\begin{verbatim}
 wget http://svn.pardus.org.tr/uludag/trunk/doc/test/2009/testcases/turkish/editor-tr.tex
 texi2dvi editor-tr.tex
 okular editor-tr.dvi
\end{verbatim}


\item pcmciautils paketi kurulumu sonrası:

Eğer takılı bir PCIA kartınız bulunmakta ise, aşağıda bulunan komutun bu kartı listelediğini gözlemleyiniz.
\begin{verbatim}
 su -
 lspcmcia
\end{verbatim}



\item man ve man-pages paketleri kurulumu sonrası:

Aşağıda bulunan sayfaların sorunsuz bir şekilde çalıştırklarını gözlemleyin.
\begin{verbatim}
 man tcp
 man getspent
\end{verbatim}


\item less paketi kurulumu sonrası:

Aşağıda bulunan komutun sorunsuz çalıştığını gözlemleyin.
\begin{verbatim}
pisi info -F python | less 
\end{verbatim}


\item bootsplash-theme-pardus paketi kurulumu sonrası:

Bilgisayarınızı yeniden başlatın ve açılış ekranının sorunsuz bir şekilde açıldığını gözlemleyiniz.

\item coreutils paketi kurulumu sonrası:

Aşağıda bulunan komutların sorunsuz bir şekilde çalıştığını gözlemleyin.
\begin{verbatim}
  date
  pwd
  uname -a
\end{verbatim}


\item python paketi kurulumu sonrası:
\begin{itemize}
 \item system-tr.pdf pisi testini gerçekleştiriniz.
 \item programming-tr.pdf sympy testini gerçekleştiriniz.
 \item programming-tr.pdf python-iptables testini gerçekleştiriniz.
\end{itemize}


\item openssl paketi kurulumu sonrası:

Aşağıda bulunan komutların sorunsuz bir şekilde çalıştığını gözlemleyin:
\begin{verbatim}
   openssl ciphers -v 
   openssl ciphers -v -tls1
   openssl speed
   openssl req  -new -newkey rsa:1024 -nodes -keyout mykey.pem -out myreq.pem
  // verify signature
   openssl req -in myreq.pem -noout -verify -key mykey.pem
  // check info
   openssl req -in myreq.pem -noout -text

\end{verbatim}


\item usbutils paketi kurulumu sonrası:

Aşağıda bulunan komutların düzgün çalıştığını gözlemleyin:
\begin{verbatim}
 usb-devices
 lsusb
\end{verbatim}


\item udev paketi kurulumu sonrası:

Bilgisayarınızı yeniden başlatın ve düzgün bir şekilde açıldığını gözlemleyin. 

Ses, kamera, mount, görüntü gibi işlemlerin düzgün çalıştığını gözlemleyin.
\item Aşağıda bulunan paketler kurulum testine tabidir.
\begin{verbatim}
e2fsprogs 
\end{verbatim}
\item module-init-tools paketi kurulumu sonrası:

Bilgisayarınızı yeniden başlatınız:
Ses, kablosuz ağ, bluetooth, kamera gibi şeylerin çalıştığını gözlemleyiniz.

Aşağıda bulunan komutların sorunsuz olarak çalıştığını gözlemleyiniz.
\begin{verbatim}
 # lsmod
 # modinfo ahci
\end{verbatim}


\item dnsmasq paketi kurulumu sonrası:

\begin{itemize}
 \item Service yöneticinizden dnsmasq servisini başlatın.
 \item Ağ yöneticisinden bağlantınızı durdurun ve yeniden başlatın.
 \item Aşağıdaki komutu çalıştırın ve sorunsuz bir şekilde sorgu zamanını döndürdüğünü gözlemleyin.
\begin{verbatim}
 dig http://archlinux.org | grep "Query time"
\end{verbatim}

\end{itemize}


\item file paketi kurulumu sonrası:

\begin{verbatim}
  wget http://cekirdek.pardus.org.tr/~semen/dist/test/office/openoffice/test_oodraw.mng
  wget http://cekirdek.pardus.org.tr/~semen/dist/test/office/openoffice/test_oodraw.odg
  wget http://cekirdek.pardus.org.tr/~semen/dist/test/office/openoffice/test_oodraw.jpg
  wget http://cekirdek.pardus.org.tr/~semen/dist/test/office/openoffice/test_oodraw.gif
  wget http://cekirdek.pardus.org.tr/~semen/dist/test/office/openoffice/test_oodraw.png
  wget http://cekirdek.pardus.org.tr/~semen/dist/test/office/openoffice/test_oodraw.tif
  wget http://cekirdek.pardus.org.tr/~semen/dist/test/office/openoffice/test_oowriter.txt
  wget http://cekirdek.pardus.org.tr/~semen/dist/test/office/openoffice/test_oodraw.ps
  wget http://cekirdek.pardus.org.tr/~semen/dist/test/office/openoffice/
   test_openoffice-extension-pdfimport.pdf
\end{verbatim}

Yukarıda bulunan dosyaları aşağıda bulunan komut ile çalıştırın, dosya formatlarını düzgün bir şekilde bulduğunu gözlemleyin.
\begin{verbatim}
  file <dosya adı>
\end{verbatim}

\item mudur paketi kurulumu sonrası:

\begin{itemize}
  \item Makinenizi yeniden başlatın ve sistemin düzgün bir şekilde açıldığını  gözlemleyin.
 \item Ctrl+Alt+F1 tuşuna basıp sistem konsoluna geçin ve sistem dilinizin daha önceki dil ve klavye düzeni ile aynı olduğunu gözlemleyin.
  \item /etc/mudur/ altında bulunan locale, language, keymap dosyalarının sistem dil ve klavye dilinize eskisi gibi uygun olduğunu gözlemleyin.
 \item Pisi komutlarının düzgün bir şekilde çalıştığını gözlemleyin. 

\end{itemize}
\item tiff ve tiff-devel paketi kurulumu sonrası:
\begin{verbatim}
  wget http://cekirdek.pardus.org.tr/~semen/dist/test/desktop/kde/base/doga.tiff
  wget http://cekirdek.pardus.org.tr/~semen/dist/test/desktop/kde/base/istanbul.tiff
\end{verbatim}

Resimlerin üzerine sağ tıklayarak gwenview, kolourPaint, gimp, showfoto ile açılabildiklerini gözlemleyin.

\item pisi paketi kurulumu sonrası:
\begin{itemize}
 \item package-manager-tr.pdf test aşamalarını gerçekleştiriniz.
 \item history-manager-tr.pdf test aşamalarını gerçekleştiriniz.
\end{itemize}

\item baselayout paketi kurulumu sonrası:
\begin{itemize}
 \item Bilgisayarınızı kapatın düzgün bir şekilde kapanabildiğini gözlemleyin.
 \item Bilgisayarınızı yeniden başlatın ve düzgün bir şekilde açılabildiğini gözlemleyin.
 \item 2009 için:
  \begin{verbatim}
  pisi blame baselayout
  \end{verbatim}
      2008 için: http://svn.pardus.org.tr/pardus/2008/stable/system/base/baselayout/pspec.xml linkinde bulunan en son eklenmiş history tag'i yorumuna (comment) bakınız.

   Bu her iki durumda da yorumlarda belirtilmiş olan kullanıcının (user) /etc/passwd dosyasına eklenmiş olduğunu gözlemleyin.

\end{itemize}
\item libxml2 paketi kurulumu sonrası:

\begin{itemize}
\item Aşağıda bulunan komutun bir katalog oluşturduğunu gözlemleyin.
  \begin{verbatim}
   #  xmlcatalog --create
  \end{verbatim}
\item multimedia-tr.pdf avidemux-qt, avidemux ve inkscape testlerini gerçekleştirin.
\end{itemize}
\item curl paketi kurulumu sonrası:

\begin{itemize}
\item http://packages.pardus.org.tr/search/pardus-2009/libkate/ içeriğinin sorunsuz bir şekilde listelendiğini gözlemleyin.
\begin{verbatim}
curl http://packages.pardus.org.tr/search/pardus-2009/libkate/
\end{verbatim}

\item network-tr.pdf sylpheed testini geçekleştirin.
\end{itemize}

\item glib2 paketi kurulumu sonrası:
\begin{itemize}
 \item system-tr.pdf openssh testini gerçekleştirin.
\item system-tr.pdf gettext testini gerçekleştirin.
\end{itemize}

\item gettext paketi kurulumu sonrası:

Aşağıda bulunan komutun po dizini altında "network-manager.pot" dosyasını oluşturduğunu gözlemleyin.
\begin{verbatim}
 # svn co http://svn.pardus.org.tr/uludag/trunk/kde/network-manager/
 # cd network-manager/manager
 # rm -rf po
 # mkdir po
 # xgettext setup.py 
 # ./setup.py update_messages
\end{verbatim}


\item openssh paketi kurulumu sonrası:

Servis yöneticisinden openssh'ı çalıştırın ve aşağıda bulunan komutun çıktısında openssh'ın çalıştığını gözlemleyin:
\begin{verbatim}
 # service list 
\end{verbatim}

\item mkinitramfs paketi kurulumu sonrası:

Bilgisayarınızı yeniden başlatın. Düzgün bir şekilde açıldığını gözlemleyin.

/boot/ dizini altında kullanmakta olduğunuz kernel'in initramfs dosyasının var olduğunu gözlemleyin.

\item pardus-python paketi kurulum testine tabidir.

\end{enumerate}
\section{Service alt Bileşeni}
\begin{enumerate}
\item omniORB paketi kurulumu sonrası:

Servis yöneticisinden omniORB başlatın, aşağıdaki komut ile başlatıldığına emin olun:
\begin{verbatim}
 # service omniORB status
\end{verbatim}

 \item memcached paketi kurulumu sonrası:

  programming-tr.pdf python-memcached testini gerçekleştirin. 
\end{enumerate}

\section{Locale alt Bileşeni}
\begin{enumerate}
 \item Aşağıda bulunan paketler için:

multimedia-tr.pdf gimp-i18n-<lang> testlerini gerçekleştiriniz. Sadece ilgili dili bulunanları test etmeniz yeterli olacaktır.

(2009 için)Sistem yöneticisinden sistem dilini değiştirin (değiştirdiğiniz dil için ilgili glibc locale paketini kurduğunuza emin oldun), sisteminizi yeniden başlatın ve CTRL+ALT+F1 tuşuna basın ve sistem konsolunun ilgili dilde açıldığını gözlemleyin.

(2008 için) Aşağıda bulunan dosyada sistem dilini değiştirin ve bilgisayarınızı yeniden başlatın ve CTRL+ALT+F1 tuşuna basın ve sistem konsolunun ilgili dilde açıldığını gözlemleyin.
\begin{verbatim}
 su -
 vi /etc/conf.d/mudur
\end{verbatim}

\begin{verbatim}
    glibc-locales-aa
    glibc-locales-af
    glibc-locales-am
    glibc-locales-an
    glibc-locales-ar
    glibc-locales-as
    glibc-locales-ast
    glibc-locales-az
    glibc-locales-be
    glibc-locales-ber
    glibc-locales-bg
    glibc-locales-bn
    glibc-locales-bo
    glibc-locales-br
    glibc-locales-bs
    glibc-locales-byn
    glibc-locales-ca
    glibc-locales-crh
    glibc-locales-cs
    glibc-locales-csb
    glibc-locales-cy
    glibc-locales-da
    glibc-locales-de
    glibc-locales-dz
    glibc-locales-el
    glibc-locales-en
    glibc-locales-es
    glibc-locales-et
    glibc-locales-eu
    glibc-locales-fa
    glibc-locales-fi
    glibc-locales-fil
    glibc-locales-fo
    glibc-locales-fr
    glibc-locales-fur
    glibc-locales-fy
    glibc-locales-ga
    glibc-locales-gd
    glibc-locales-gez
    glibc-locales-gl
    glibc-locales-gu
    glibc-locales-gv
    glibc-locales-ha
    glibc-locales-he
    glibc-locales-hi
    glibc-locales-hne
    glibc-locales-hr
    glibc-locales-hsb
    glibc-locales-ht
    glibc-locales-hu
    glibc-locales-hy
    glibc-locales-id
    glibc-locales-ig
    glibc-locales-ik
    glibc-locales-is
    glibc-locales-it
    glibc-locales-iu
    glibc-locales-iw
    glibc-locales-ja
    glibc-locales-ka
    glibc-locales-kk
    glibc-locales-kl
    glibc-locales-km
    glibc-locales-kn
    glibc-locales-ko
    glibc-locales-ks
    glibc-locales-ku
    glibc-locales-kw
    glibc-locales-ky
    glibc-locales-lg
    glibc-locales-li
    glibc-locales-lo
    glibc-locales-lt
    glibc-locales-lv
    glibc-locales-mai
    glibc-locales-mg
    glibc-locales-mi
    glibc-locales-mk
    glibc-locales-ml
    glibc-locales-mn
    glibc-locales-mr
    glibc-locales-ms
    glibc-locales-mt
    glibc-locales-nb
    glibc-locales-nds
    glibc-locales-ne
    glibc-locales-nl
    glibc-locales-nn
    glibc-locales-nr
    glibc-locales-nso
    glibc-locales-oc
    glibc-locales-om
    glibc-locales-or
    glibc-locales-pa
    glibc-locales-pap
    glibc-locales-pl
    glibc-locales-pt
    glibc-locales-ro
    glibc-locales-ru
    glibc-locales-rw
    glibc-locales-sa
    glibc-locales-sc
    glibc-locales-sd
    glibc-locales-se
    glibc-locales-shs
    glibc-locales-si
    glibc-locales-sid
    glibc-locales-sk
    glibc-locales-sl
    glibc-locales-so
    glibc-locales-sq
    glibc-locales-sr
    glibc-locales-ss
    glibc-locales-st
    glibc-locales-sv
    glibc-locales-ta
    glibc-locales-te
    glibc-locales-tg
    glibc-locales-th
    glibc-locales-ti
    glibc-locales-tig
    glibc-locales-tk
    glibc-locales-tl
    glibc-locales-tn
    glibc-locales-ts
    glibc-locales-tt
    glibc-locales-ug
    glibc-locales-uk
    glibc-locales-ur
    glibc-locales-uz
    glibc-locales-ve
    glibc-locales-vi
    glibc-locales-wa
    glibc-locales-wo
    glibc-locales-xh
    glibc-locales-yi
    glibc-locales-yo
    glibc-locales-zh
    glibc-locales-zu
\end{verbatim}
\end{enumerate}

\section{Doc alt Bileşeni}

Aşağıda bulunan paketler sadece kurulum testine tabidir.
\begin{verbatim}
 glibc-doc
 tiff-docs
\end{verbatim}


\end{document}

