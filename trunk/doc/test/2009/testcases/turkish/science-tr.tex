\documentclass[a4paper,10pt]{article}
\usepackage[turkish]{babel}
\usepackage[utf8]{inputenc}
\usepackage[left=1cm,top=1cm,right=2cm,bottom=2cm]{geometry}

\title{Science Bileşeni Test Aşamaları}
\author{Semen Cirit}

\renewcommand{\labelenumi}{\arabic{enumi}.}
\renewcommand{\labelenumii}{\arabic{enumi}.\arabic{enumii}.}
\renewcommand{\labelenumiii}{\arabic{enumi}.\arabic{enumii}.\arabic{enumiii}.}
\renewcommand{\labelenumiv}{\arabic{enumi}.\arabic{enumii}.\arabic{enumiii}.\arabic{enumiv}.}

\begin{document}

\maketitle


\section{Plotting alt Bileşeni}
\begin{enumerate}
 \item gdpc paketi kurulumu sonrası:

Kmenu'den uygulamayı açın ve /usr/share/doc/gdpc/liquid.test dosyasını uygulama ile açın ve tamam tuşuna basın. Sorunsuz bir şekilde noktalı şekillerin oluştuğunu gözlemleyin.

\item rlplot paketi kurulumu sonrası:
Kmenu'den uygulamayı açın ve aşağıda bulunan dosyayı açın. Daha sonra grafikler bölümünden grafik yaratı seçin. Çizgi grafiğini yaratın, sorunsuz bir şekilde grafiğin oluştuğunu gözlemleyin.

\begin{verbatim}
http://cekirdek.pardus.org.tr/~semen/dist/test/science/test.rlw 
\end{verbatim}
\item veusz paketi kurulumu sonrası:
ipython paketini kurun ve aşağıda bulunan komutları çalıştırın:
\begin{verbatim}
 ipython
 import veusz
\end{verbatim}
\end{enumerate}



\section{Biology alt Bileşeni}
\begin{enumerate}

\item Aşağıda bulunan paketler sadece kurulum testine tabidir.
\begin{verbatim}
 biopython-doc
 clustalw
\end{verbatim}

\item biopython paketi kurulumu sonrası:
ipython paketini kurun ve aşağıda bulunan komutları çalıştırın:
\begin{verbatim}
 ipython
 import Bio
\end{verbatim}

\item clustalx paketi kurulumu sonrası:

Kmenüden uygulamayı açın ve sorunsuz bir şekilde açıldığını gözlemleyin.

\item hmmer, hmmer-doc paketleri kurulumu sonrası:

Aşağıda bulunan komutun seorunsuz bir çıktı ürettiğini gözlemleyin.
\begin{verbatim}
 cd /usr/share/doc/hmmer/tutorial
 sudo hmmbuild globins4.hmm  globins4.sto
\end{verbatim}

\end{enumerate}

\section{Electronics alt Bileşeni}
\begin{enumerate}
 \item Aşağıda bulunan paketler sadece kurulum testine tabidir.
\begin{verbatim}
 gpsim
\end{verbatim}
\item pcb paketi kurulumu sonrası:

Kmenu'den uygulamanın sorunsuz olarak açıldığını gözlemleyin.
\end{enumerate}


\section{Gis alt Bileşeni}
\begin{enumerate}
\item Aşağıda bulunan paketler sadece kurulum testine tabidir.
\begin{verbatim}
ogdi
proj
\end{verbatim}
\item geotiff-viewer ve libgeotiff paketleri kurulumu sonrası:

GeoTIFF uygulamasını açın ve sorunsuz bir şekilde açıldığını gözlemleyin.

\item gpsd ve libgps paketleri kurulumu sonrası:

Eğer gps'iniz var ise bu testi gerçekleştirebilirsiniz.

Gps'inizi takın. Servis yöneticisinden gpsd servisini başlatın.

Aşağıda bulunan komut ile servisin başlatıldığını gözlemleyin.
\begin{verbatim}
 service gpsd status
\end{verbatim}


\end{enumerate}


\section{Astronomy alt Bileşeni}
\begin{enumerate}
\item xplanet paketi kurulumu sonrası:

Aşağıda bulunan komutun uzay ile ilgili bir resim ürettiğini gözlemleyin.
\begin{verbatim}
wget ftp://ssd.jpl.nasa.gov/pub/eph/export/unix/unxp1900.405
xplanet -ephemeris unxp1900.405 -light_time -origin uranus -body sun -radius 30 -date 19140503.093000  
\end{verbatim}

 \item stellarium paketi kurulumu sonrası:

Kmenüden uygulamayı açın ve düzgün bir şekilde açılabildiğini gözlemleyin. 
\end{enumerate}

\section{Chemistry alt Bileşeni}
\begin{enumerate}
 \item avogadropaketi kurulumu sonrası:

Kmenu'den uygulamayı açın, çıkan ekranda bir molekül oluşturmaya ve kaydetmeye çalışın. Sorunsuz bir şekilde molekulün kaydedildiğini ve tekrar açılabildiğini gözlemleyin.

 \item openbabel paketi kurulumu sonrası:

Aşağıda bulunan komutları çalıştırın ve test.smi dosyasında "c1cccc(c1C(=O)O)OC(=O)C·C9H8O4" formülünün yazılmış olduğunu gözlemleyin.
\begin{verbatim}
 # babel -H sdf
 # wget http://cekirdek.pardus.org.tr/~semen/dist/test/science/aspirin.sdf
 # babel -isdf 'aspirin.sdf' -osmi 'test.smi'
 # vi test.smi
\end{verbatim}

\end{enumerate}

\section{Mathematics alt Bileşeni}
\begin{enumerate}

\item Aşağıda bulunan paketler sadece kurulum testine tabidir.

\begin{verbatim}
gnuplot-doc
pspp-docs
\end{verbatim}

\item gnuplot paketi kurulumu sonrası:

\begin{verbatim}
 gnuplot
 plot sin(x)
\end{verbatim}

\item pspp paketi kurulumu sonrası:

Kmenu'den uygulamanın sorunsuz açıldığını gözlemleyin.
\item octave paketi kurulumu sonrası:

Aşağıda bulunan komutların düzgün bir şekilde çalıştığını gözlemleyin.
\begin{verbatim}
 # octave 
 octave:1> A = [ 1, 1, 2; 3, 5, 8; 13, 21, 34 ]
\end{verbatim}

\item FreeMat paketi kurulumu sonrası:

Kmenüden uygulamayı açın ve ve aşağıda bulunan satırı kopyalayıp ENTER tuşuna basın ve sorunsuz olarak diziyi listelediğini gözlemleyin.
\begin{verbatim}
A = [ 1, 1, 2; 3, 5, 8; 13, 21, 34 ]
\end{verbatim}



\item rkward paketi kurulumu sonrası:

Uygulamayı açın ve Uygulama paneli üzerinden Plots bölümünü tıklayın, Barplot'u seçin ve burada listelenen verilerden birini seçin, ekleyin ve onaylayın. 

Bu durumun sonunda ilgili grafiğin sorunsuz bir şekilde oluştuğunu gözlemleyin.

\item wxMaxima paketi kurulumu sonrası:

Kmenüden uygulamanın sorunsuz olarak açılabildiğini gözlemleyin.

Birkaç matematiksel işlem yapın ve sorunsuz bir şekilde yapılabildiğini gözlemleyin.
\item maxima paketi kurulumu sonrası:

Aşağıdaki komutların sorunsuz bir şekilde çalıştığını gözlemleyin:
\begin{verbatim}
 # maxima
 144*17 - 9;
 144^25;
\end{verbatim}
\item lpsolve paketi kurulumu sonrası:

Aşağıda bulunan komutun sorunsuz çalıştığını gözlemleyin.

\begin{verbatim}
wget http://cekirdek.pardus.org.tr/~semen/dist/test/science/test_lpsolve
lp_solve  < test_lpsolve
\end{verbatim}


\item Aşağıda bulunan paketler kurulum testine tabidir.
\begin{verbatim}
gfan 
\end{verbatim}

\end{enumerate}

\section{Robotics alt Bileşeni}
\begin{enumerate}
 \item opencv paketi kurulumu sonrası: (kamerası olanlar test edebilecektir.)

Resim çek butonuna basın sorunsuz bir şekilde ekranınyenilendiğini gözlemleyin
\begin{verbatim}
# wget http://svn.pardus.org.tr/projeler/facelock/pardus.py
# wget http://svn.pardus.org.tr/projeler/facelock/pardus.png
# python pardus.py
\end{verbatim}


\end{enumerate}

\end{document}

