\documentclass[a4]{article}

\usepackage[utf-8]{inputenc}
\usepackage[turkish]{babel}
%\usepackage{fullpage}

\title{Pardus Sürüm Yönetimi}
\author{T. Barış Metin}



\begin{document}

\maketitle

Yazılım projelerinde \emph{sürüm}, en basit hali ile yazılımın
dağıtılabilir veya satılabilir aşaması olarak tarif edilebilir. Her
yazılım geliştirme projesinde olduğu gibi kullanılabilir bir sürüm
çıkartmak, özgür yazılım projelerinde de asıl amaçtır. Projeye yapılan
katkının çok büyük bir oranı kullanıcılara sunulacak sürüme yönelik
bir çalışmadır.

Özgür yazılım projeleri gelişim modellerinde çeşitlilik gösteriyor
olsalar da, rutin sayılabilecek, \emph{planla/belgele $\to$ geliştir
  $\to$ yayınla} sürecini uygularlar.

Pardus, bir özgür yazılım projesi olarak, özgür yazılım geliştirme
etiklerine uygun olarak geliştirilir. Açık geliştirme modeli
benimsenerek, proje ilk andan itibaren tasarım ve geliştirme sürecini
proje dışındaki katkıcılar ile paylaşmıştır. Tasarlanan teknolojiler
öncelikli olarak belgelenerek, daha sonra geliştirilen yazılımların
kaynak kodları sunularak, oluşturulan bilgi birikiminin paylaşılması
amaçlanmıştır.

Bir özgür yazılım projesi olmanın verdiği avantaj ile Pardus, diğer
özgür yazılım projelerindeki gelişmeleri yakından takip etmekte ve
kullanmaktadır. Bu sayede teknolojik bilgi birikimini bünyesine dahil
etmektedir. Kullanılan yazılımlarda bulunan ve giderilen hatalar,
özgür yazılım etiğine uygun bir şekilde, yazılımların ana
geliştiricileri ile paylaşılır.

Justin R. Erenkrantz ``Açık Kaynak Kodlu Projelerde Sürüm
Yönetimi'' isimli makalesinde, açık kaynak kodlu projeleri süreç ve
kullanılan araçlar açısından inceleyerek, sürümü etkileyen bileşenleri
aşağıdaki gibi sınıflandırmıştır.

\begin{description}
\item[Sürüm Yetkilisi] \emph{Sürüm Yöneticisi} olarak da adlandırılan
  sürüm yetkilisi, geliştiriciler içerisinden bir grup veya
  belirlenmiş tek bir kişi olabilir. Pardus'da sürüm yöneticisi
  çekirdek geliştirme ekibinden bir kişi olarak belirlenmiştir.

  Pardus, sürüm yöneticisi dışında, gelişim sürecinde yol gösterici
  görevi olan, TÜBİTAK/UEKAE çalışanı bir \emph{çekirdek ekibe}
  sahiptir.
\item[Sürüm Numaraları] Geliştirme paralel olarak işletilen birkaç
  dalda birlikte yapılabilir. Bu durumda kullanıcılara sunulacak
  kararlı geliştirme sürümünün diğerlerinden ayırılması genellikle
  tercih edilen bir yöntemdir. Pardus iki ana paket deposu üzerinde
  çalışmaktadır: \emph{kararlı} ve \emph{geliştirme} depoları. Yeni
  özellikler ilk olarak geliştirme deposunda uygulanır ve daha sonra
  kararlı depoya aktarılır.
\item[Sürüm Öncesi Testler] Sürüm genel kullanım için yayınlanmadan
  önce belirli bir grup tarafından test edilmesi genellikle kullanılan
  bir yöntemdir. Pardus kararlı sürümünü yayınlamadan önce sürüm
  öncesi testleri birden fazla aşamada işletmiştir. İlk olarak
  hazırlanan \emph{Çalışan CD} sürümü, kullanıcıların Pardus'u
  sistemlerine kurmadan denemelerine olanak vermiştir. Geliştirme
  sürecinin çeşitli aşamalarında, yalnızca geliştiriciler için
  hazırlanan \emph{kök dosya sistemleri} sunulmuştur. Kararlı sürüme
  yaklaşıldığında ise doğal süreç olarak kabul edilen,
  \emph{alpha(ilk deneme sürümü), beta(deneme sürümü) ve RC(sürüm
    adayı)} sürümleri testçilere ve geliştiricilere sunularak kararlı
  sürüm öncesi testlerin yapılmasına olanak verilmiştir.

  İlk kararlı sürümün yayınlanmış olduğu bu aşamadan sonra, sürüm
  öncesi testler belirlenecek zaman aralıkları ile yayınlanacak
  \emph{alpha, beta ve RC} sürümleri ile yapılacaktır.
\item[Sürümün Onaylanması] Sürüm Yetkilisi ile doğrudan ilişkili olan
  sürümün onaylanması aşaması, sürümün yayınlanmak için gerekli
  nitelikleri barındırıp barındırmadığının analizinin yapıldığı
  aşamadır. Pardus içerisinde sürüm çekirdek geliştirme ekibi
  tarafından onaylanır. Çekirdek geliştirme ekibi sürüm öncesi
  testlerinden de aldıkları veriler ile sürümü onaylarlar.
\item[Dağıtım Yöntemi] Merkezi veya dağıtık olabilecek dağıtım
  yöntemi, ürünün hangi kanallar ile kullanıcılara dağıtılacağını
  belirler. Pardus, Internet üzerinden yayınlanan sürümü ile, dağıtık
  olarak kabul edilecek ``yansılar'' yöntemini kullanmaktadır.
\end{description}

Sürümü etkileyen yukarıdaki bileşenlerin yanın da ``sürümün
desteklenmesi'' de sürümün başarısı için önemli bir etkendir. Sürüm
sonrasında karşılaşılan hataların takibi ve çözümü bu konuda rol
oynar. Pardus projesinde, planlama ve geliştirme aşamasında olduğu
gibi, hataların takibinde de açık bir model
izlenmektedir. Internet'ten ulaşılabilir olan hata takip sistemi genel
kullanıma açıktır. Mevcut hataların sorgulanması ve yeni hataların
eklenmesi mümkündür.

Hatalara getirilen çözümler Pardus paket depolarına yansır. Pardus ile
birlikte dağıtılan yazılımları barındıran paket depoları,
kullanıcıların sistemlerini hatalarından arındırılmış paketler ile
güncellemelerine olanak verir.

Sonuç olarak, Pardus özgür yazılım geliştirme modelini uygulayarak,
hedef kitlesine ve amaçlarına uygun bir işletim sistemi sürümü
oluşturmaktadır.

\paragraph{Kaynaklar}
\begin{enumerate}
\item Erenkrantz, J.R. ``Release Management Within Open Source
  Projects'', 2003.
\end{enumerate}


\end{document}
