\documentclass{beamer}

\usepackage[utf8]{inputenc}
\usepackage[turkish]{babel}

\usepackage{beamerthemesplit}
\usepackage[amssymb,cdot]{SIunits}

\title{Nasıl Pardus Geliştiricisi Olunur ?}
\author[Onur Küçük \& Semen Cirit]{Onur Küçük \texttt{onur@pardus.org.tr}\\
        Semen Cirit \texttt{semen@pardus.org.tr}}
\date{BİLMÖK\\ 28 Şubat 2009}
\institute{
  Ulusal Elektronik ve Kriptoloji Araştırma Enstitüsü\\
  TÜBİTAK}

\setbeamertemplate{navigation symbols}{}
\usetheme{Darmstadt}

%images
\pgfdeclareimage[width=2cm]{pardus-logo}{logo}
\logo{\pgfuseimage{pardus-logo}}

\begin{document}

\frame{\titlepage}

\section{Pardus Geliştiricisi Olmak}

\subsection{Açıkkodlu Projelerde Geliştirici Olmak}
\frame
{
    \frametitle{Açıkkodlu Projelerde Geliştirici Olmak}
    \begin{itemize}
        \item Süreci öğrenmek
        \item İletişim
        \item Koda hakim olmak
        \item Yamalama
    \end{itemize}
}

\subsection{Pardus Geliştiricisi Kimdir ?}
\frame
{
    \frametitle{Pardus Geliştiricisi Kimdir ?}
    \begin{itemize}
        \item Paketler
        \item Pardus araçları
        \item Hata çözmek
        \item Yeni teknolojiler
        \item İletişim
        \item Anageliştiricilerle halay çekmek
    \end{itemize}
}

\subsection{Sürüm Takip Sistemi}
\frame
{
    \frametitle{Sürüm Takip Sistemi}
    \begin{itemize}
        \item Sürüm takip sistemi
        \item Yazılım projeleri için olmazsa olmaz
        \item Merkezi depodan süreç takibi
        \item Gelişim takibi, geriye dönüş, istatistik
        \item Geliştiriciler arası kaynak kod paylaşımı
        \item Subversion ile sürüm takibi
            \begin{itemize}
                \item Trunk
                \item Branch
                \item Tag
            \end{itemize}
    \end{itemize}
}

\subsection{Pardus SVN}
\frame
{
    \frametitle{Pardus SVN}
    \begin{itemize}
        \item http://svn.pardus.org.tr
        \item uludag/trunk \textbf{2,248,171 satır}
        \item pardus-devel \textbf{2100 kaynak paket}
        \item contrib-devel \textbf{1008 kaynak paket}
        \item projeler \textbf{442,231 satır}
        \item playground
    \end{itemize}
}

\subsection{Programlama Dilleri}
\frame
{
    \frametitle{Programlama Dilleri}
    \begin{itemize}
        \item \textbf{Python}
        \item \textbf{C}
        \item \textbf{C++}
        \item \textbf{Shell}
        \item \textbf{ASM}
        \item \textbf{Java}
        \item \textbf{Javascript}
        \item \textbf{Perl}
        \item \textbf{Ruby}
        \item \textbf{Lisp}
        \item \textbf{Dotnet}
        \item \textbf{...}
    \end{itemize}
}

\subsection{Hata Takip Sistemi}
\frame
{
    \frametitle{Hata Takip Sistemi}
    \begin{itemize}
        \item \textbf{Bugzilla}
        \item Hata bildirimi
        \item Yeni paket istekleri
        \item Yeni özellik istek ve önerileri
        \item Hataların çözülmesi ve kapatılması
    \end{itemize}
}


\subsection{E-Posta Listeleri}
\frame
{
    \frametitle{E-Posta Listeleri}
    \begin{itemize}
        \item Bilgilendirici Listeler
            \begin{itemize}
                \item paketler-commits
                \item contrib-commits
                \item uludag-commits
                \item projeler-commits
                \item bugzilla
                \item buildfarm
            \end{itemize}
        \item  Tartışma Listeleri
            \begin{itemize}
                \item gelistirici
                \item pardus-devel
                \item paketler
                \item stable
            \end{itemize}
        \item pardus-kullanicilari
        \item http://liste.pardus.org.tr/mailman/listinfo/
    \end{itemize}
}

\subsection{Faydalı Adresler}
\frame
{
    \frametitle{Faydalı Adresler}
    \begin{itemize}
        \item http://www.pardus-wiki.org
        \item http://www.pardus.org.tr/projeler/
        \item http://hata.pardus.org.tr
        \item http://bugs.pardus.org.tr
        \item http://tr.pardus-wiki.org/Pardus:Yeni.geliştirici.kılavuzu
    \end{itemize}
}


\subsection{Geliştirici Adayları}
\frame
{
    \frametitle{Geliştirici Adayları}
    \begin{itemize}
        \item Açıkkodlu projelerin işleyişini öğrenin
        \item Listelere üye olun
        \item Hata çözün
        \item Yeni geliştirici klavuzu
    \end{itemize}
}


\subsection{Sorular}
\frame
{
    \frametitle{Sorular}
    \begin{itemize}
        \item Sorular ?
    \end{itemize}
}

\end{document}
