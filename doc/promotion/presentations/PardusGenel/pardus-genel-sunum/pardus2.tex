\documentclass{beamer}

% preamble
\usepackage[utf8]{inputenc}
\usepackage[turkish]{babel}

\title{Pardus}
\author{Erkan Tekman, T. Barış Metin}
\institute{
  Ulusal Dağıtım Projesi \\
  Ulusal Elektronik ve Kriptoloji Enstitüsü\\
  TÜBİTAK}

\usetheme{Warsaw}

%images
\pgfdeclareimage[width=5cm]{pardus-logo}{pardus-logo}
\pgfdeclareimage[height=3.5cm]{evrim}{evrim}
\pgfdeclareimage[height=4.5cm]{ekip}{ekip}
\pgfdeclareimage[height=4.5cm]{yollar}{yollar}
\pgfdeclareimage[width=4cm]{sanci}{sanci}
\pgfdeclareimage[height=4.5cm]{zemberek}{zemberek}
\pgfdeclareimage[width=4cm]{pisi}{pisi}
\pgfdeclareimage[width=4cm]{farkli}{farkli}
\pgfdeclareimage[height=4cm]{yali}{yali}
\pgfdeclareimage[height=3cm]{hedef}{hedef}
\pgfdeclareimage[width=5cm]{wantyou}{wantyou}

% document
\begin{document}

\frame{\titlepage}


% contents
\frame{
  \frametitle{İçerik}
  \begin{columns}

    \begin{column}{5cm}
      \tableofcontents
    \end{column}

    \begin{column}{5cm}
      \pgfuseimage{pardus-logo}
    \end{column}

  \end{columns}
}


\section{Tarihçe}

\subsection{İsim}
\frame{
  \frametitle{Panthera Pardus Tulliana}
  \begin{itemize}
  \item Panthera Pardus Tulliana
  \item Anadolu Parsı
  \item Ege ve Batı Akdeniz, Doğu Akdeniz ve Doğu Karadeniz bölgelerinin dağlık ve ormanlık alanlarında yaşamış
  \item Doğal yaşam alanları ve av kaynaklarının azalması sonucu insanların yaşadığı yerlere yaklaşmış ve avlanmış
  \item Son kayıt: 17 Ocak 1954 - Beypazarı
  \item Bugün Türkiye'de 10-15 Anadolu Parsı kaldığı da öne sürülmekte
  \item Varlığını kanıtlamak ve koruma altına almak için doğa gönüllülerinin çabaları aralıksız sürmekte
  \end{itemize}
}

\subsection{Neden?}
\frame{
  \frametitle{Neden?}
  \only<1>{
    \begin{block}{Ana sözleşme}
    Pardus, UEKAE tarafından, bilişim okur-yazarlığına sahip bilgisayar kullanıcılarının temel masaüstü ihtiyaçlarını hedefleyerek; mevcut Linux dağıtımlarının üstün taraflarını kavram, mimari ya da kod olarak kullanan; otonom sisteme evrilebilecek bir yapılandırma çerçevesi ve araçları ile kurulum, yapılandırma ve kullanım kolaylığı sağlamak üzere geliştirilen bir GNU/Linux dağıtımıdır.
    \end{block}

   }
   \only<2>{
       \begin{itemize}
	\item Ulusal bağımsızlık, güvenlik ve tasarruf amacıyla, kritik uygulamaların üzerinde çalışabileceği, açık ve standart bir veri yapısını destekleyen, güvenlik izlemesine imkan verecek şekilde kaynak kodu açık olan ve finansal yük oluşturmadan yaygınlaştırılabilecek bir işletim sistemi
	\item Türkiye’nin bilgi teknolojileri konusundaki etkinliğinin katma değerli projelere yöneltilmesi, araştırma ve geliştirme ağırlıklı yüksek teknoloji üretimi
	\item Bir yandan öncülü ve bir yandan da ürünü olarak yerel bilgi birikiminin, gerek teknolojik alanda ve gerekse iş süreçleri düzeyinde, sağlanması zorunluluğu
	\item Ülke gereklerine bağlı olarak teknolojik gelişmenin yönünü belirlemek, farklı alanların ağırlığını değiştirmek ve dolayısıyla söz konusu işletim sisteminin yol haritasına hakim olmak
       \end{itemize}
   }
   \only<3>{
	\begin{itemize}
	\item Tam Türkçe desteğini, hem karakter yapısının Türkçe’ye uygun (UTF-8 uyumluluğu), hem de kullanıcıya görünen tüm mesaj ve belgelerin Türkçe olması yoluyla sağlaması.
	\item Mevcut Linux dağıtımlarından ve rakip diğer işletim sistemlerinden daha kolay kurulabilen ve kullanılabilen bir işletim sistemi olması.
	\item Araç temelli ve teknoloji merkezli bir tasarım yerine görev temelli ve insan merkezli bir yaklaşımla ve esnekliği ve yüksek performansı sağlayabilecek modüler bir yapıda tasarlanması. 
	\end{itemize}
   }
}


\subsection{Nasıl?}
\frame{
  \frametitle{1.0'a doğru}
  \only<1>{
	\begin{itemize}
	\item \textbf{2003}: Kavramsal hazırlık. "Neden?", "Nasıl?", "Kim ile?" sorularına yanıtlar
	\item \textbf{2003 sonu}: \emph{"Dağıtım yapacağız"}
	\item \textbf{2004 ilk çeyrek}: Geliştirici ekibin toplanması
	\item \textbf{2004}: Teknik analiz ve alt yapının hazırlanması
	\item \textbf{2005 ilk çeyrek}: Pardus Çalışan CD (1 Şubat)
	\item \textbf{2005 sonu}: Pardus 1.0 (26 Aralık)
	\end{itemize}

    \center{\pgfuseimage{evrim}}

  }
}


\section{Pardus için 10 Neden}
\subsection{10 Neden}
\frame{
  \frametitle{Çünkü Pardus ...}

  \only<1>{
    \begin{block}{... Özgürdür}
    Lisansı sizi kısıtlamaz, üreticiyi sizden korumak için değil sizin haklarınızı korumak için tasarlanmıştır. Genel Kamu Lisansı (GPL) ile dağıtılır.
    \end{block}
  }

  \only<2>{
    \begin{block}{... Türkçe sever}
    Çekinmeden Türk alfabesinin tüm harflerini kullanabilirsiniz. Yazım ve sözlük denetimi yapar, sizi güzel bir Türkçe kullanmaya teşvik eder.
    \end{block}
  }

  \only<3>{
    \begin{block}{... Virüslere geçit vermez}
    İnternet'ten gelen her dosyayı kontrolden geçirmekle, ya da virüs bulaşmış bilgisayarınızı temizlemekle zaman kaybetmezsiniz.
    \end{block}
  }

  \only<4>{
    \begin{block}{... Hızlı kurulur}
    30 dakikada yüklenir. Tek kurulum işlemiyle bilgisayarınıza ofis yazılımı, İnternet gezgini, sohbet programı gibi gerekli bütün programlar da yüklenir.
    \end{block}
  }

  \only<5>{
    \begin{block}{... Kolay kullanılır}
    Grafik arayüzleri, menüleri, ikonları ile aklınıza ve güdülerinize hitap eder. Kullanmak için bilgisayar öğrenmek, ikinci bir dil bilmek ya da uzun eğitimlerden geçmek gerekmez.
    \end{block}
  }

  \only<6>{
    \begin{block}{... Her şey dahildir}
    Bir masaüstü kullanıcısının gereksinim duyacağı her türlü yazılım Kurulum CD'si içinde mevcuttur. İnternet araçları, ofis paketi, her türlü resim, müzik, film için oynatıcı ve düzenleyiciler, oyunlar, aklınıza ne gelirse...
    \end{block}
  }

  \only<7>{
    \begin{block}{... Özelleştirilebilir}
    Sistemi, kendi beğeninize göre özgürce değiştirebilirsiniz. Hem görünüş, hem de davranış açısından. Tek sınır, hayal gücünüzdür.
    \end{block}
  }

  \only<8>{
    \begin{block}{... Şeffaftır}
    Kaynak kodlarını kendi ihtiyaçlarınıza göre değiştirebilir, kendi dağıtımınızı üretebilirsiniz. Ne yaptığını ve ne yapmadığını kaynak koduna bakarak bilirsiniz.
    \end{block}
  }

  \only<9>{
    \begin{block}{... Çok dil bilir}
    Ayrı ayrı CD'ler ile tekrar yükleme yapmadan, iki dokunuşla Türkçe'den İngilizce'ye dönüşür. Herhangi başka bir dilin desteğini eklemek de son derece kolaydır.
    \end{block}
  }

  \only<10>{
    \begin{block}{... Eğlencelidir}
    Kaptan Masaüstü, PiSi ve ÇOMAR ile bilgisayar kullanmanın keyfini, tam Türkçe desteği ile bilgisayarınızı kendi dilinizde kullanmanın kolaylığını yaşarsınız. 
    \end{block}
  }
}


\section{Özgür Yazılım}
\subsection{Genel Kamu Lisansı}
\frame{
  \frametitle{Genel Kamu Lisansı}
  \begin{block}{Lisans}
   GPL $\to$ General Public License
  \end{block}
  \begin{itemize}
  \item \textbf{Özgürlük 0} Çalıştırma
  \item \textbf{Özgürlük 1} Çoğaltma ve Dağıtma
  \item \textbf{Özgürlük 2} İnceleme ve Değiştirme
  \item \textbf{Özgürlük 3} Değiştirdiğini yeniden dağıtma
  \end{itemize}
}




% ikinci bölüm: pardus çözümleri
\section{Pardus Çözümleri}
\subsection{Altyapı Çözümleri}
\frame{
  \frametitle{Pardus Teknolojileri}
   \begin{itemize}
    \item Yapılandırma sistemi ÇOMAR: Görev merkezli yaklaşım ile kolay yapılandırma.
    \item Paket yöneticisi PiSi: Hızlı ve kolay yazılım yönetimi.
    \item İmla denetleyicisi Zemberek: Tüm masaüstünde Türkçe imla denetimi.
    \item Kurlum yazılımı YALI: Hızlı ve kolay sistem kurulumu.
    \item Türkçe ile sorunsuz Pardus paketleri.
    \item Kolay yapılandırma arayüzleri.
   \end{itemize}
}

\subsection{Yazılım Çözümleri}
\frame{
  \frametitle{Yazılımlar}
   \begin{itemize}
    
  \only<1>{
    \item Web tarayıcıları
    \item E-Posta İstemcileri
    \item Hızlı mesajlaşma istemcileri
    \item Dosya transferi uygulaması
    \item Web sayfaları düzenleyicisi
    \item Dosya paylaşımı yazılımları
  }
  \only<2>{
    \item Ajanda
    \item Adres defteri
    \item Ofis yazılımları
      \begin{itemize}
        \item Kelime İşlemci
        \item Hesap Tablosu
        \item Sunum Yazılımı
        \item Çizim Yazılımı
        \item Veritabanı
      \end{itemize}
  }
  \only<3>{
    \item Müzik çalıcıları, ses kaydedici
    \item Video oynatıcıları
    \item Resim işleme yazılımları
    \item Dosya yöneticisi
    \item Dosya sıkıştırma açma yazılımları
    \item Yerel ağ bağlantı, dosya paylaşımı yazılımları...
    \item ... ve diğerleri ...
  }
   \end{itemize}
}


%% 3. bölüm: Pardus kim için
\section{Pardus: Kim için?}

\subsection{Hedef Kitle}
\frame{
  \frametitle{Hedef Kitle}

  \only<1>{
  \begin{block}{Ana Sözleşme'den...}
    .... bilişim okur-yazarlığına sahip bilgisayar kullanıcılarının temel masaüstü ihtiyaçlarını hedefleyerek ...
  \end{block}
  }
  
  \only<2>{
   \begin{itemize}
   \item Aslında herkes için...
     \begin{itemize}
     \item Yeni bilgisayar kullanıcıları
     \item Mevcut Windows kullanıcıları
     \item Bir şekilde Linux ile tanışmış, fakat tatmin olmamış olanlar
     \item Linux uzmanları, geliştiriciler
     \item ISV'ler
     \end{itemize}
   \end{itemize}
  }

  \center{\pgfuseimage{hedef}}
}



\section{Son}
%end presentation
\frame{
  \frametitle{Bitti}
  \begin{centering}
    \Huge{\alert{Sorular, Öneriler, Sohbet}} \newline
  \end{centering}

  \begin{block}{İLETİŞİM}
    \begin{itemize}
    \item E-Posta: bilgi@pardus.org.tr
    \item Web: http://www.pardus.org.tr
    \end{itemize}
  \end{block}
}

\end{document}





