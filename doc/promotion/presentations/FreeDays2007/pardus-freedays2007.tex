\documentclass{beamer}

% preamble
\usepackage[utf8]{inputenc}
\usepackage[turkish]{babel}

\title{Pardus}
\author{Ekin Meroğlu, ekin@pardus.org.tr}
\institute{
  Ulusal Dağıtım Projesi \\
  Ulusal Elektronik ve Kriptoloji Enstitüsü\\
  TÜBİTAK}

\usetheme{Warsaw}

%images
\pgfdeclareimage[width=5cm]{pardus-logo}{pardus-logo}
\pgfdeclareimage[height=3cm]{ok}{ok}
\pgfdeclareimage[height=4.5cm]{yollar}{yollar}
\pgfdeclareimage[height=4.5cm]{zemberek}{zemberek}
\pgfdeclareimage[width=4cm]{pisi}{pisi}
\pgfdeclareimage[width=4cm]{farkli}{farkli}
\pgfdeclareimage[height=4cm]{yali}{yali}
\pgfdeclareimage[height=4cm]{pardusman}{pardusman}
\pgfdeclareimage[height=4cm]{kaptan}{kaptan}
\pgfdeclareimage[height=4cm]{tasma}{tasma}
\pgfdeclareimage[height=3cm]{hedef}{hedef}
\pgfdeclareimage[width=5cm]{wantyou}{wantyou}

% document
\begin{document}

\frame{\titlepage}


\section{Tarihçe}

\subsection{Neden?}
\frame{
  \frametitle{Neden?}
  \only<1>{
    \begin{block}{Ana sözleşme}
    Pardus, UEKAE tarafından, bilişim okur-yazarlığına sahip bilgisayar kullanıcılarının temel masaüstü ihtiyaçlarını hedefleyerek; mevcut Linux dağıtımlarının üstün taraflarını kavram, mimari ya da kod olarak kullanan; otonom sisteme evrilebilecek bir yapılandırma çerçevesi ve araçları ile kurulum, yapılandırma ve kullanım kolaylığı sağlamak üzere geliştirilen bir GNU/Linux dağıtımıdır.
    \end{block}

    \center{\pgfuseimage{ok}}

   }
   \only<2>{
       \begin{itemize}
	\item Ulusal bağımsızlık, güvenlik ve tasarruf amacıyla, kritik uygulamaların üzerinde çalışabileceği, açık ve standart bir veri yapısını destekleyen, güvenlik izlemesine imkan verecek şekilde kaynak kodu açık olan ve finansal yük oluşturmadan yaygınlaştırılabilecek bir işletim sistemi
	\item Türkiye’nin bilgi teknolojileri konusundaki etkinliğinin katma değerli projelere yöneltilmesi, araştırma ve geliştirme ağırlıklı yüksek teknoloji üretimi
	\item Bir yandan öncülü ve bir yandan da ürünü olarak yerel bilgi birikiminin, gerek teknolojik alanda ve gerekse iş süreçleri düzeyinde, sağlanması zorunluluğu
	\item Ülke gereklerine bağlı olarak teknolojik gelişmenin yönünü belirlemek, farklı alanların ağırlığını değiştirmek, söz konusu işletim sisteminin yol haritasına hakim olmak
       \end{itemize}
   }
   \only<3>{
	\begin{itemize}
	\item Tam Türkçe desteğini, hem karakter yapısının Türkçe’ye uygun (UTF-8 uyumluluğu), hem de kullanıcıya görünen tüm mesaj ve belgelerin Türkçe olması yoluyla sağlaması.
	\item Mevcut Linux dağıtımlarından ve rakip diğer işletim sistemlerinden daha kolay kurulabilen ve kullanılabilen bir işletim sistemi olması.
	\item Araç temelli ve teknoloji merkezli bir tasarım yerine görev temelli ve insan merkezli bir yaklaşımla ve esnekliği ve yüksek performansı sağlayabilecek modüler bir yapıda tasarlanması. 
	\end{itemize}
   }
}

\subsection{Nasıl?}
\frame{
  \frametitle{Pardus 2007'ye doğru}
  \only<1>{
	\begin{itemize}
	\item \textbf{2003}: Kavramsal hazırlık. "Neden?", "Nasıl?", "Kim ile?" sorularına yanıtlar
	\item \textbf{2003 sonu}: \emph{"Dağıtım yacağız"}
	\item \textbf{2004 ilk çeyrek}: Geliştirici ekibin toplanması
	\item \textbf{2004}: Teknik analiz ve alt yapının hazırlanması
	\item \textbf{2005 ilk çeyrek}: Pardus Çalışan CD (1 Şubat)
	\item \textbf{2005 sonu}: Pardus 1.0 (26 Aralık)
	\item \textbf{2006 sonu}: Pardus 2007 (18 Aralık)
	\item \textbf{Yakında}: Pardus 2007.1 
	\item \textbf{Uzakta}: Pardus 2008 
	\end{itemize}

  }
}

\subsection{Hedef Kitle}
\frame{
  \frametitle{Hedef Kitle}

  \only<1>{
  \begin{block}{Ana Sözleşme'den...}
    .... bilişim okur-yazarlığına sahip bilgisayar kullanıcılarının temel masaüstü ihtiyaçlarını hedefleyerek ...
  \end{block}
  }
  
  \only<2>{
   \begin{itemize}
   \item Aslında herkes için...
     \begin{itemize}
     \item Yeni bilgisayar kullanıcıları
     \item Mevcut Windows kullanıcıları
     \item Bir şekilde Linux ile tanışmış, fakat tatmin olmamış olanlar
     \item Linux uzmanları, geliştiriciler
     \item Uygulama geliştiricileri, entegratörler.
     \end{itemize}
   \end{itemize}
  }

  \center{\pgfuseimage{hedef}}
}

\subsection{Geliştirici Ekip}
\frame{
  \frametitle{Geliştiriciler}

  \only<1>{
   \begin{itemize}
   \item Çekirdek geliştiriciler
        \begin{itemize}
        \item TUBİTAK / UEKAE bünyesinde çalışan tam / yarı zamanlı geliştiriciler 
   	\end{itemize}
   \item Katkıcılar
   	\begin{itemize}
        \item Paketçiler
	\item Pardus proje katkıcıları
	\item Yerelleştirme katkıcıları
	\item Topluluk projelerinin yürütücüleri
   	\end{itemize}
   \item Kullanıcılar
   	\begin{itemize}
   	\item Pardus'u tanıtmayı görev edinmiş,
        \item Hata bulan, çözümü için uğraşan,
	\item İyileştirme önerileri ile gelen,
	\item Sadece şikayet eden :-) kullanıcılar.
   	\end{itemize}	
   \end{itemize}
  }
  \only<2>{
   \begin{itemize}
   \item Özgür yazılım geliştiricileri
   	\begin{itemize}
        \item Pardus'un yararlandığı/desteklediği özgür yazılım projelerinin geliştiricileri
	\item Tüm yerelleştirme gönüllüleri
   	\end{itemize}
   \item Bilinen simalar, bilinen mecralar
	\begin{itemize}
        \item Pardus deneyen, krtitik yazan, tanıtan, kullanan ÖY simaları
	\item Belgelerde ve tartışmalarda Pardus atıfları
   	\item Pardus çözümlerinin tanınması
   	\end{itemize}
   \item Aktif destekçiler
	\begin{itemize}
        \item Pardus temelli gönüllü projeler
	\item Tanıtım kampanyaları
   	\end{itemize}
   \end{itemize}
  }
}

\subsection{Nereden Başlasak?}
\frame{
  \frametitle{Nereden Başlasak?}

   \center{\pgfuseimage{yollar}}
    \only<1>{
    \begin{itemize}
    \item Özgür yazılım üretim süreci ile,
    \item Bir özgür yazılım ekosistemi kurmak ve
    \item Bilgi birikimi oluşturmak.
    \item Bu süreçte ürünümüzü ortaya çıkarmak
    \end{itemize}
   }
   \only<2>{
    \begin{itemize}
    \item Dağıtım \emph{"amorf"} bir yapı
    \item Doğru sorunları bulmak
    \item Doğru çözümler üretmek
    \item Mevcut bilgi birikimini doğru kullanabilmek
    \end{itemize}
   }
   \only<3>{
    \begin{itemize}
    \item Önce, kısa bir süre için ikiye ayrıldık
	\begin{itemize}
	\item Harcıalem Dağıtım
	\item Dürtücü Teknolojiler
	\end{itemize}
    \item Olabildiğince çok kaynağı \emph{"dışarıdan"} kullandık
    \end{itemize}
   }
   \only<4>{
    \begin{itemize}
    \item Pardus 1.0 sonrası en temel sorunlara odaklandık
	\begin{itemize}
	\item Açılış altsistemi, servis yönetimi 
	\item Yönetim olabildiğince otomatik
	\item Kullanıcılar için yönetim arayüzleri
	\end{itemize}
    \end{itemize}
   }
   \only<5>{
    \begin{itemize}
    \item Yeni projelere giriştik
	\begin{itemize}
	\item MÜDÜR, daha çok görevde ÇOMAR  
        \end{itemize}
    \item Yeni teknolojileri araştırdık.
        \begin{itemize}
	\item Güncel masaüstü teknolojileri entegre (hal, dbus, ntfs-3g...) 
        \end{itemize}
    \end{itemize}
   }
}

\subsection{Açık Proje}
\frame{
  \frametitle{Ortak Aklı Oluşturmak}
    \begin{itemize}
    \item Katkıcılar ile büyüyen bir proje
    \item Bilgiyi dışarıya aktarmak
      \begin{itemize}
      \item Belgeler
      \item Açık tasarım
      \item Kaynak kodlar her zaman erişilebilir
      \item Tartışma ortamları; listeler, toplantılar, hata takip sistemi
      \item Proje geliştirme süreci her türlü katkıya açık 
      \end{itemize}
    \item \textbf{Sancılı bir iş!}
    \end{itemize}
}

% ikinci bölüm: pardus çözümleri
\section{Pardus Çözümleri}

\subsection{Yerel çözümler}
\frame{
  \frametitle{Türkçe kullanmak istiyoruz}
   \begin{itemize}
   \item Yerelleştirme çalışmaları
	\begin{itemize}
	\item Uygulamalar Türkçe konuşmalı
	\item Uygulamalar Türkçe ile sorunsuz çalışmalı.
        \end{itemize}
   \item Bireysel çabalar
   \item Sesimizi, istediğimiz kadar, duyurmak için yeterli olmamışlardı
   \item Sorunları buluyor ve çözüm üretiyoruz
   \item Çözümlerimizi herkes ile paylaşıyoruz
	\begin{itemize}
	\item Çözüm ana geliştiriciye iletiliyor
	\item Çözüm Pardus tarafında sunuluyor
        \end{itemize}
   \end{itemize}
}

\frame{
  \frametitle{İmla denetimi}
   \begin{itemize}
   \item Masaüstünde, her uygulamada Türkçe imla denetimi
   \item Türkçe yazabilmek, yazılanı okuyabilmek
   \item ı-I, i-İ harflerinin dönüşümleri yüzünden uygulama hatalarına takılmak
   \item Artık hayal değil ;)
   \end{itemize}

  \center{\pgfuseimage{zemberek}}

}

\subsection{PiSi}
\frame{
  \frametitle{PiSi}

  \begin{columns}

  \begin{column}{6cm}

  \only<1>{
   \begin{itemize}
   \item \emph{Packages Installed Successfully as Intended}
   \item "Paket yönetimi" yeni değil
   \item Doğru yapılması gereken bir işi doğru yap! \\ Gerekirse yeniden yap !
   \item Paket yönetimi dağıtım için \textbf{çok önemli}
   \item Parçaları bir araya getiren sistem
   \end{itemize}
  }
  \only<2>{
   \begin{itemize}
   \item Daha kolay paket \emph{yönetimi}
     \begin{itemize}
     \item Geniş özellik kümesi, sorunsuz kurulum ve güncelleme
     \item Grafiksel, tam bir yönetim arayüzü
     \end{itemize}
   \item Daha kolay paket \emph{üretimi}
     \begin{itemize}
     \item Standartlara uygun, basit paket tanımları
     \item Genişletilebilir üzerine kolayca yeni uygulamalar eklenebilir bir sistem
     \item Geliştiriciler zamanlarını paket \emph{yapmaya} değil, paket sorunlarını \emph{sorunlarını} çözmeye harcamalı
     \end{itemize}
   \end{itemize}
  }
  \only<3>{
   \begin{itemize}
   \item Yüksek seviyeli ve düşük seviyeli paket yönetim özellikleri
   \item Kaynak tabanlı ve ikili paket sistemlerinin iyi huyları
   \item Paketler XML dosyaları ve basit python betikleri ile ifade ediliyor
   \item Paketler bileşen ve kategoriler ile düzenleniyor
   \item Python ile yazılmış
   \item İkili paketler ZIP arşivi içinde LZMA sıkıştırması ile \emph{çok küçük}
   
   \end{itemize}
  }
   \only<4>{
   \begin{itemize}
    \item Hızlı, basit, mantıklı
    \item Paket yapmak, paket bakımı yapmak çok kolay
    \item Benzerlerinden kat ve kat daha ufak bir kod, fazlaca özellik
   \end{itemize}
  }
  \end{column}

  \begin{column}{4cm}
    \pgfuseimage{pisi}
  \end{column}

  \end{columns}

}


\subsection{ÇOMAR}
\frame{
  \frametitle{ÇOMAR}
  \begin{columns}

  \begin{column}{6cm}

   \only<1>{
   \begin{itemize}
   \item Yeni bir yaklaşım
   \item \emph{Herkes onu arıyor!}
   \item Sorunun doğru tarifi $\to$ Doğru çözüm
   \end{itemize}
   }

   \only<2>{
   \begin{itemize}
   \item Yapılandırma arayüzleri görev tabanlı olmalı
   \item Gündelik işler için komut satırı gerekmemeli
   \item \emph{Herkes belge okumuyor}
   \item Bilgisayar kendi işini kendi yapmalı
   \item Uygulamalar bir arada çalışabilmeli
   \end{itemize}
   }

   \only<3>{
    \begin{itemize}
    \item Şu anda neler yapabiliyor
      \begin{itemize}
      \item Basit ve hızlı, profil tabanlı ağ yapılandırması
      \item Otomatik grafik arayüz yapılandırması
      \item Açılış sistemi; \textbf{hızlı}, kolay
      \item Kullanıcı ve kullanıcı hakları yönetimi
      \item Yazılım kurulum ve güncelleme (PiSi ile birlikte)
      \end{itemize}
    \item Önümüzdekiler
      \begin{itemize}
       \item Sunucu yönetimi
       \item Uzaktan yönetim
       \item Güç Yönetimi
      \end{itemize}
    \end{itemize}
   }
   \end{column}

   \begin{column}{4cm}
    \pgfuseimage{farkli}
   \end{column}

  \end{columns}  
}

\subsection{YALI}
\frame{
  \frametitle{YALI}
   \center{\pgfuseimage{yali}}
   \begin{itemize}
   \item Genel geçer bir çözüm
   \item Sade, basit : görev temelli, insan odaklı
   \item PiSi ve Çomar altyapısı
   \item Ufak ve hızlı 
   \item \emph{Herkes için kolay ve hızlı bir kurulum deneyimi} 
   \end{itemize}
}

\subsection{Kaptan -Tasma}
\frame{
    \frametitle{Kaptan}
        \center{\pgfuseimage{kaptan}}
        \begin{itemize}
            \item Kullanıcının ilk gördüğü uygulama
            \item Kullanıcının masaüstü görüntüsünü, ağ ayarlarını, paket yöneticisi ile ilgili ayarları yaptığı uygulama
            \item Basit, sade ve amaca yönelik
        \end{itemize}
}

\frame{
    \frametitle{TASMA}
        \center{\pgfuseimage{tasma}}
        \begin{itemize}
            \item Kullanıcının sisteme dair ayarları değiştirebildiği uygulama
            \item Bir kısmını KDE araçları diğer kısmını Pardus yapılandırma araçları oluşturuyor
            \item Basit, sade ve amaca yönelik
        \end{itemize}
}

\subsection{Sürüm : Pardus 2007}
\frame{
    \frametitle{2007 Yenilikleri : Kullanıcının Hayatını Kolaylaştıralım}
        \begin{itemize}
            \item 70 dil desteği (Türkçe, İngilizce, Almanca, Hollandaca ve İspanyolca kurulum)
            \item AppArmor güvenlik yazılımı
            \item Gelişmiş donanım tanıma sistemi (Müdür)
            \item Gelişmiş güç yönetimi (Suspend/Hibernate v.b.)
            \item Gelişmiş X yapılandırma aracı (Zorg)
            \item Güncel paketler/teknolojiler [bkz: AIGLX :)]
            \item Masaüstü arama yazılımları (Beagle/Kerry)
            \item Ntfs-3g ile NTFS disklere yazma desteği
            \item Onlarca yeni oyun (Çok güzeller :P)
            \item Suse'nin gelişmiş Kickoff menüsü
            \item Xen, Qemu ve VirtualBox gibi sanal makina yazılımları
        \end{itemize}
}

\frame{
    \frametitle{Sürüm Sonrası : Biraz da istatistik}
        \begin{itemize}
            \item Pardus 2007 çıktıktan sonraki bir ayda : 
            \begin{itemize}
            	\item 207.617 kişi web sayfalarımızı gezdi
		\item 30.009 kişi Google'dan pardus diye aratarak,
	        \item 812 kişi ise “pardüs” diye aratarak bize gelmiş
            \end{itemize}	
            \item Yine bu zaman aralığında : 
            \begin{itemize}
            	\item Pardus sunucularından 80.000 üzerinde
	    	\item ÇOMÜ yansıdan 13.166
	    	\item Ulakbim yansıdan 6.540
	    	\item BitTorrent aracılığıyla 1.300 kişi Pardus 2007'yi indirdi.
	    	\item Chip dergisi 55.000, PC Magazine 30.000, PC Net 35.000 adet dağıttı, EMO üyelerine 10.000 adet Pardus 2007 hediye etti.
		\item Bir ayda 32.369 kişi PiSi aracılığıyla sunucularımızdan yazılım güncelledi
	    \end{itemize}
        \end{itemize}
}
%% 3. bölüm: Pardus Geliştirme Süreci
\section{Geliştirme Süreci}

\subsection{Topluluk ile Geliştirmek}
\frame{
  \frametitle{Sanal Ofis : www.pardus.org.tr}
   \only<1>{
   \begin{itemize}
   \item E-posta listeleri.\\ \emph{http://liste.pardus.org.tr}
       \begin{itemize}
       \item \emph{gelistirici}, \emph{*-commits}, \emph{duyuru}, \emph{security} e-posta listeleri
       \item \emph{pardus-kullanıcıları} e-posta listesi - son derece aktif.
       \item Projelerin tartışma listeleri
       \end{itemize}
   \item Hata takip sistemi \\ \emph{http://hata.pardus.org.tr}
       \begin{itemize}
       \item Hem Pardus projeleri hem paketler deposundaki paketler için hata takip sistemi
       \item Kullanıcı için daha ağrısız bir hata bildirme uygulaması yolda..
       \item Yeni paket, iyileştirme ve yerelleştirme istekleri de bu sistemde.
       \end{itemize}      
   \end{itemize}
   }
   \only<2>{
   \begin{itemize}
   \item Wiki \\ \emph{http://tr.pardus-wiki.org}
       \begin{itemize}
       \item Tüm Pardus belgeleri Wiki'de. 
       \item Tüm geliştirici ve katkıcılar belgelerini wiki'de geliştiriyor  
       \item Ortak bir bilgi birikimi ilk amaç.
       \end{itemize}
   \item Topluluk projeleri, topluluk portalı.
       \begin{itemize}
       \item Tüm gönüllü topluluk projeleri için teknik altyapı ve ortak çatı sağlamak amaçlı.
       \item Yakın gelecekte hizmete girecek.
       \end{itemize}
   \end{itemize}
   }
}

\subsection{Kaynak Paketler}
\frame{
  \frametitle{Tüm Geliştirme Gözönünde}	
   \begin{itemize}
   \item SVN sürüm kontrol sistemi ile merkezi kod deposu
   \item Proje kapsamındaki her türlü kod, belge, sunum depolarda
       \begin{itemize}
       \item Projenin tüm geçmişi svn günlüklerinde.
       \item Birlikte geliştirmek için uygun ortam.
       \end{itemize}
   \item Pardus paketleri için \emph{devel}, \emph{2007} ve \emph{contrib} depoları \\
   \emph{https://svn.pardus.org.tr/pardus/}
   \item Pardus projeleri için \emph{uludag} deposu \\
   \emph{https://svn.pardus.org.tr/uludag/}
	\begin{itemize}
	\item Web gösterimi http://svn.pardus.org.tr
	\item Tüm depolar commits e-posta listeleri
        \item Tartışmalar için geliştirici e-posta listesi
   	\end{itemize}
   \end{itemize}
}

\subsection{İkili Paketler}
\frame{
  \frametitle{Anında Sonuç : Derleme Çiftliği, İkili Paket Depoları}
   \only<1>{
   \begin{itemize}
   \item İkili paket oluşturma : Otomatik ve çabuk, hatasız.
       \begin{itemize}
       \item Şu anda \emph{2007} deposu
       \item Yakın gelecekte \emph{devel} ve \emph{contrib} depoları 
       \end{itemize}
   \item SVN ile entegre, benzerlerinden çok çok kısa bir Python uygulaması.
   \item Yüksek işlemci gücü, otomatize süreç, geliştiriciye anında ikili paket. 
   \item İkili paket depoları 
       \begin{itemize}
       \item Pardus-2007 : Pardus 2007 için kararlı paketler
       \item Pardus-2007-test : Pardus 2007 için test edilen paketler 
       \end{itemize}
   \end{itemize}
    }
   \only<2>{
    \begin{itemize}
    \item Pardus-2007 kararlı depo,
	\begin{itemize}
    	\item Şu an için resmi depoda 1322 kaynak paket,
    	\item bu paketlerden üretilmiş 1538 ikili paket içeriyor.	
	\end{itemize}
    \item Büyük oranda katkıcıların bakımını yaptığı contrib deposu,
    	\begin{itemize}
    	\item Şu anda depoda 456 kaynak paket,	
    	\item bu paketlerden üretilmiş 473 ikili paket içeriyor.	
	\end{itemize}
    \item Tüm depo ve paketçi bilgileri için \\
    \emph{http://paketler.pardus.org.tr/}
    \end{itemize}
    }
   }

\subsection{Son Ürün : Sürüm}
\frame{
  \frametitle{Anında Ürün : Pardusman ile Çok Kolay !}
   \begin{columns}

   \begin{column}{6cm}
   \begin{itemize}
   \item Sürüm oluşturma / yayınlama sürecini büyük oranda kısaltıyor 
   \item SVN, PiSi ve ÇOMAR altyapısı ile amaca yönelik dağıtım sürümleri hazırlıyor.
   \item Hem Kurulan CD, hem Çalışan CD : isterseniz çalışan USB disk, isterseniz Çalışan DVD !
   \item Zamanımızı CD hazırlayarak değil, test ederek kullanıyoruz.
   \end{itemize}
   \end{column}

   \begin{column}{4cm}
    \pgfuseimage{pardusman}
   \end{column}

  \end{columns}  
}

\section{Ne Yapabilirim?}

\subsection{Herkese Açık}
\frame{
  \frametitle{Herkese Açık}

   \begin{columns}

   \begin{column}{5cm}
     \pgfuseimage{wantyou}
   \end{column}

   \begin{column}{5cm}
   \begin{itemize}
   \item Pardus açık bir proje
   \item Gerçekten \textbf{açık}
      \begin{itemize}
      \item Tasarım belgeleri
      \item Kaynak kodlar
      \item Hata takip sistemi
      \item Tartışma listeleri
      \item ...
      \end{itemize}
   \item \textbf{Herkes destek olabilir!}
   \end{itemize}
   \end{column}

   \end{columns}
}


\subsection{Ne bilmek gerekiyor?}
\frame{
  \frametitle{Ne bilmek gerekiyor?}

  \begin{block}{Ne yapmak istiyorsunuz?}
   Ne yapmak istediğinize bağlı olarak, bilmeniz gerekenler değişiyor.
  \end{block}

}


\frame{
  \frametitle{Pardus'u Tanıtmak istiyorum}
  \begin{itemize}
  \item Pardus'un hedeflerini ve tarihçesini öğrenin
  \item Pardus projelerini öğren, bilgi sahibi olun
  \item Daha önceki tanıtım sunumlarına göz atın
  \end{itemize}
}

\frame{
  \frametitle{Pardus'u test etmek istiyorum}
   \begin{itemize}
   \item Hata takip sistemine üye olun ve kullanmayı öğrenin
   \item Yayınlanan test sürümlerinden haberdar olmak için tartışma listelerini takip edin
   \item Pardus test paket deposunu kullanın, hataları bildirin
   \end{itemize}
}

\frame{
  \frametitle{Yerelleştirme çalışmalarına destek olmak istiyorum}
   \begin{itemize}
   \item Yerelleştirme gruplarına katılın
   \item Pardus "Türkçe" listesine üye olun
   \item Test ederek sistemdeki yerelleştirme sorunlarını bulun, raporlayın/düzeltin
   \end{itemize}
}


\frame{
  \frametitle{Paket geliştirmek istiyorum}
  \begin{itemize}
  \item PiSi'nin kullandığı teknolojileri öğrenin; Python, XML
  \item PiSi mimari belgesini okuyun
  \item Merhaba PiSi belgesini okuyun
  \item Action API belgesine göz atın
  \item Subversion kullanmayı öğrenin
  \item Depodaki paket örneklerine göz atın
  \item Pardus "paketler" listesine üye olun
  \item Depo politikası belgesini okuyun
  \end{itemize}
}

\frame{
  \frametitle{Pardus projelerini/teknolojilerini geliştirmek istiyorum}
  \begin{itemize}
  \item Yeni geliştirici belgesini okuyun
  \item Hata takip sistemini kullanmayı öğrenin
  \item Subversion kullanmayı öğrenin
  \item "uludag-commits" ve "paketler-commits" listelerine üye olun
  \item Dişine göre bir proje bulmak için projeleri inceleyin
  \item Destek bekleyen projeleri arayıp sorun
  \item Açık hatalardan başlamak iyi bir fikirdir
  \end{itemize}
}


\frame{
  \frametitle{Projemi Pardus'a eklemek istiyorum}
  \begin{itemize}
  \item Tartışma listelerine üye olun
  \item Projenizi tanıtın ve nasıl faydalı olabileceğini anlatın 
  \item Hem kullanıcıları hem de geliştiricleri ikna edin
  \item Kolay gelsin :-P
  \end{itemize}

}

\section{Son}
%end presentation
\frame{
  \frametitle{Bitti}
  \begin{centering}
    \Huge{\alert{Sorular, Öneriler, Sohbet}} \newline
  \end{centering}

    \begin{block}{İLETİŞİM}
            \begin{itemize}
                \item E-Posta: bilgi@pardus.org.tr
                \item Web: http://www.pardus.org.tr
                \item E-Posta Listeleri: http://liste.uludag.org.tr/mailman/listinfo
                \item SVN Depoları: http://svn.pardus.org.tr/
            \end{itemize}
        \end{block}
}
\end{document}