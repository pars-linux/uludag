\documentclass[a4paper,10pt]{article}
\usepackage[turkish]{babel}
\usepackage[utf8]{inputenc}
\usepackage[left=1cm,top=1cm,right=2cm,bottom=2cm]{geometry}

\title{Hardware Component Test Cases}
\author{Semen Cirit}

\renewcommand{\labelenumi}{\arabic{enumi}.}
\renewcommand{\labelenumii}{\arabic{enumi}.\arabic{enumii}.}
\renewcommand{\labelenumiii}{\arabic{enumi}.\arabic{enumii}.\arabic{enumiii}.}
\renewcommand{\labelenumiv}{\arabic{enumi}.\arabic{enumii}.\arabic{enumiii}.\arabic{enumiv}.}

\begin{document}

\maketitle
\section{Firmware sub component}
\begin{enumerate}
 \item Following packages subject to installation test:
\begin{verbatim}
ql2x00-firmware
\end{verbatim}
\end{enumerate}

\section{Printer sub component}
\begin{enumerate}
 \item Following packages subject to installation test:
\begin{verbatim}
gutenprint 
\end{verbatim}


\end{enumerate}

\section{Smart Card sub component}
\begin{enumerate}
 \item Following packages subject to installation test:
\begin{verbatim}
ccid 
mozilla-opensc-signer
openct
opensc
pcsc-lite
\end{verbatim}


\end{enumerate}

\section{Info sub package}
\begin{enumerate}
 \item After installation x86info package:

Observe that the below command list the x86 base CPUs.
\begin{verbatim}
 # x86info
\end{verbatim}


 \item After installation gkrellm package:

Start the gkrellm service from service manager. In order to be sure, type the below command.

\begin{verbatim}
 service gkrellm status
\end{verbatim}

\end{enumerate}

\section{Powermanagement sub component}
\begin{enumerate}
 \item After installation lm\_sensors package:

Do hardware-eng.pdf gkrellm test.

Observe that the below command list information about cpu fan, power fan, voltage etc.
\begin{verbatim}
 # sensors 
\end{verbatim}

\end{enumerate}

\section{Scanner sub component}
\begin{enumerate}
\item After installation sane-backends package:

(If you have a scanner you can test this package.)
\begin{verbatim}
# sane-find-scanner
# scanimage 
\end{verbatim}

Observe that the first command list the available scanners.
Observe that the second command make the scan operation and create a .pnm file.
\end{enumerate}

\section{Optical sub component}
\begin{enumerate}
 \item After installation cdrtools package:

Burn the below files as an audio cd with k3b. Observe that they are burned without any problem.
 \begin{verbatim}
  # wget http://cekirdek.pardus.org.tr/~semen/dist/test/multimedia/sound/sound/11k16bitpcm.wav
  # wget http://cekirdek.pardus.org.tr/~semen/dist/test/multimedia/sound/sound/music.mp3
  # wget http://cekirdek.pardus.org.tr/~semen/dist/test/multimedia/video/cokluortam/linux.ogg
 \end{verbatim}

 \item After installation cdparanoia package:

Burn an audio CD with using the below files with k3b. Observe that they are burned without any problem.
 \begin{verbatim}
  # wget http://cekirdek.pardus.org.tr/~semen/dist/test/multimedia/sound/sound/11k16bitpcm.wav
  # wget http://cekirdek.pardus.org.tr/~semen/dist/test/multimedia/sound/sound/music.mp3
  # wget http://cekirdek.pardus.org.tr/~semen/dist/test/multimedia/video/cokluortam/linux.ogg
 \end{verbatim}

Mount the audio cd and run the following commands.
\begin{verbatim}
 # cdparanoia -vsQ
 # cdparanoia -B
\end{verbatim}

\item After installation k3b package
\begin{itemize}
\item Burn a DVD and CD with using below files.
\begin{verbatim}
 # wget http://cekirdek.pardus.org.tr/~semen/dist/test/hardware/optical/boot.iso
\end{verbatim}

Run this CD/DVD and observe that ISO is booted.

\item Burn an audio DVD and CD with using below files. 
\begin{verbatim}
 # wget http://cekirdek.pardus.org.tr/~semen/dist/test/multimedia/sound/sound.tar
 # wget http://cekirdek.pardus.org.tr/~semen/dist/test/multimedia/video/cokluortam.tar
\end{verbatim}

Run this file and try to listen them with amarok.

\end{itemize}

\end{enumerate}
\section{Emulator sub component}
\begin{enumerate}
 \item After installation zsnes package:

Open this nintendo game with this application and observe that it runs without any problem.
\begin{verbatim}
# wget http://cekirdek.pardus.org.tr/~semen/dist/test/hardware/emulator/10%20Yard%20Fight%20(A&S%20NES%20Hack).smc 
\end{verbatim}

 \item After installation wine and wine-devel packages:

Download the below game and install it with wine.  
\begin{verbatim}
# wget http://cekirdek.pardus.org.tr/~semen/dist/test/hardware/emulator/egyptianball_setup.exe
# wine  egyptianball_setup.exe
\end{verbatim}
\begin{itemize}
 \item Observe that the game install without any problem.
 \item Follow KMenu $\rightarrow$ Wine $\rightarrow$ Programs $\rightarrow$ MyPlayCity.com $\rightarrow$ Egyptian Ball path and open the game and observe taht it runs without any problem.
\end{itemize}

\end{enumerate}

\section{Virtalization alt Bileşeni}
\begin{enumerate}
 \item After installation libvirt package:

Do hardware-eng.pdf virtualbox test.
 \item After installation virtualbox, virtualbox-guestadditions and virt-wrapper packages:
 
Run Virtualbox application and create a new virtual system. Observe that the above system run the below bootable iso without any problem.
\begin{verbatim}
# wget http://cekirdek.pardus.org.tr/~gokmen/Pardus_Minimal_2009.iso
\end{verbatim}

\end{enumerate}

\section{Misc sub component}
\begin{enumerate}
 \item Following packages subject to installation test: 
\begin{verbatim}
acpi4asus 
\end{verbatim}

\end{enumerate}

\section{Bluetooth sub component}
\begin{enumerate}

 \item Following packages subject to installation test: 
\begin{verbatim}
  obexd
\end{verbatim}

 \item After installation cwiid package:
 (İf you have a Nintendo video game remote controller, you can test this package.)

Open wiimote application and try to activate your remote controller.
 \item After installation blueman and kdebluetooth packages:

(If your mobile phone has a blue tooth and you also have a bluetooth device, you can test this package.) 

Open the application and observe that the available bluetooth devices listed on the application screen.

\end{enumerate}

\section{Mobile sub component}
\begin{enumerate}
 \item Following packages subject to installation test: 
\begin{verbatim}
 gobi_loader
 usb-modeswitch
\end{verbatim}

\end{enumerate}

\section{Disk alt Bileşeni}
\begin{enumerate}
 \item After installation partimage package:

Observe that the below command execute without any problem.
\begin{verbatim}
 # su -
 # partimage
\end{verbatim}


 \item After installation mountmanager package:

Open the application then plug a usb stick and observe that this device is added to mount manager list.

 \item After installation partitionmanager package:

open Partitionmanager from kmenu and try to resize a partition and observe that the partition is resized without any problem.

( If you want you can use a usb stick for this operation.)

\item After installation reiserfsprogs package:

Plug a usb stick and execute the below commands using the mount point of usb stick. Observe that they run without any problem.

\begin{verbatim}
 # mkreiserfs /dev/<device-path> -f
 # reiserfsck /dev/<device-path>
\end{verbatim}

\end{enumerate}
\section{Sound sub component}
\begin{enumerate}
 \item After installation alsa-driver, alsa-headers, alsa-lib, alsa-plugins, alsa-plugins-pulseaudio, alsa-tools, alsa-utils pacakges:
\begin{itemize}
 \item Restart your computer and observe that the kde starting sound runs without any problem.
 \item Run the below file amarok or mplayer and observe that it runs without any problem.
\begin{verbatim}
# wget http://cekirdek.pardus.org.tr/~semen/dist/test/multimedia/sound/sound/music.mp3
# mplayer music.mp3
\end{verbatim}

\item After installation alsa-tools-gui package:

If you have EchoAudio, Envy24, Hammerfall HDSP or RMedigicontrol audio devices you can test this package.

Echomixer application for EchoAudio device.

Envy24control application Envy24 device.

HDSPconf and HDSPmixer applications for Hammerfall HDSP device.

Rmedigicontrol application for RMedigicontrol device.
\end{itemize}

\end{enumerate}
\section{Mobile sub component}
\begin{enumerate}
 \item After installation libopensync-plugin-google-calendar package:

(If you have an gmail account, you can test this package.)
\begin{itemize}
 \item Install msynctool package.
 \item Activate your calender from http://www.google.com/calendar/feeds/.
 \item Run the below commands and type the necessary information about your account for third and forth command.

\begin{verbatim}
 msynctool --addgroup GoogleCalendar
 msynctool --addmember GoogleCalendar google-calendar
 msynctool --configure GoogleCalendar 1
 msynctool --configure GoogleCalendar 2
 msynctool --sync GoogleCalendar	
\end{verbatim}

\end{itemize}

Observe that the calenders syncronized without any problem.
\end{enumerate}


\end{document}

