\documentclass[a4paper,10pt]{article}
\usepackage[turkish]{babel}
\usepackage[utf8]{inputenc}
\usepackage[left=1cm,top=1cm,right=2cm,bottom=2cm]{geometry}

\title{Util Bileşeni Test Aşamaları}
\author{Semen Cirit}

\renewcommand{\labelenumi}{\arabic{enumi}.}
\renewcommand{\labelenumii}{\arabic{enumi}.\arabic{enumii}.}
\renewcommand{\labelenumiii}{\arabic{enumi}.\arabic{enumii}.\arabic{enumiii}.}
\renewcommand{\labelenumiv}{\arabic{enumi}.\arabic{enumii}.\arabic{enumiii}.\arabic{enumiv}.}

\begin{document}

\maketitle
\section{Benchmark alt bileşeni}
\begin{enumerate}
 \item phoronix-test-suite paketi kurulumu sonası:

Aşağıda bulunan komutu çalıştırın, bir pencere açıldığını gözlemleyin.
  \begin{verbatim}
   phoronix-test-suite gui
  \end{verbatim}

Testler bölümünden "SuperTuxCard" seçin ve install butonuna basın. 

\end{enumerate}


\section{Archive alt bileşeni}
\begin{enumerate}
\item Aşağıda bulunan paketler sadece kurulum testine tabidir.
\begin{verbatim}
 unshield
 minizip
\end{verbatim}
\item rdiff-backup paketi kurulumu sonrası:

Aşağıda bulunan komutların sorunsuz çalıştığını gözlemleyin:
\begin{verbatim}
 mkdir test
 rdiff-backup test backup
 rdiff-backup-statistics backup
\end{verbatim}

\item rpm2targz paketi kurulumu sonrası:

Aşağıda bulunan komutun sorunsuz bir şekilde rpm'i targz2ye dönüştürdüğünü gözlemleyin.
\begin{verbatim}
 wget http://cekirdek.pardus.org.tr/~semen/dist/test/util/inkscape-0.48.0-2pclos2010.src.rpm
 rpm2targz inkscape-0.48.0-2pclos2010.src.rpm
\end{verbatim}


\item pbzip2 paketi kurulumu sonrası:

Aşağıda bulunan komutun bz2 uzantılı dosyayı oluşturduğunu gözlemleyin:

\begin{verbatim}
 wget http://cekirdek.pardus.org.tr/~semen/dist/test/multimedia/video/cokluortam.tar
 pbzip2 -b15qk cokluortam.tar
\end{verbatim}

\item cabextract paketi kurulumu sonrası:

hardware-tr.pdf wine testini gerçekleştiriniz.

\item xz paketi kurulumu sonrası:

Aşağıda bulunan komutların sorunsuz bir şekilde dosyayı skıştırıp açtığını gözlemleyin.
\begin{verbatim}
 mkidir test
 xz test
 unxz test.xz
\end{verbatim}


\item rar paketi kurulumu sonrası:

Aşağıda bulunan komutun sorunsuz bir şekilde test.rar dosyasını oluşturduğunu gözlemleyin.
\begin{verbatim}
 mkdir test
 rar a test.rar -ri1 -rv10 -v51200 -m5 test
\end{verbatim}


 \item lrzip paketi kurulumu sonrası:

Aşağıda bulunan komutların dosyayı önce sıkıştırıp sonra açtığını gözlemleyin.
\begin{verbatim}
  wget http://cekirdek.pardus.org.tr/~semen/dist/test/util/test_lrzip
  lrzip test_lrzip
  mkdir test
  lrzip -d test_lrzip.lrz -O test/
\end{verbatim}


\end{enumerate}


\section{Crypt alt bileşeni}
\begin{enumerate}
\item gnupg paketi kurulumu sonrası:

Aşağıda bulunan komutun sorunsuz bir şekilde anahtar ürettiğini gözlemleyin.
\begin{verbatim}
gpg --gen-key
\end{verbatim}

\item johntheripper paketi kurulumu sonrası:

Aşağıda bulunan komutların sorunsuz bir şekilde çalıştığını gözlemleyin.
\begin{verbatim}
 sudo unshadow /etc/passwd /etc/shadow > mypasswd
 sudo john -si mypasswd 
\end{verbatim}

 \item mcrypt paketi kurulumu sonrası:

Aşağıda bulunan komutların sorunsuz çalıştığını gözlemleyin.
\begin{verbatim}
  mcrypt --list
  wget http://cekirdek.pardus.org.tr/~semen/dist/test/util/test_mcrypt
  mcrypt -a blowfish test_mcrypt
  mcrypt -d test_mcrypt.nc
  vi test_mcrypt
\end{verbatim}


\end{enumerate}


\section{Antivirus alt bileşeni}
\begin{enumerate}
\item clamtk paketi kurulumu sonrası:

Kmenüden uygulamayı açın, daha sonra Ev butonuna basıp Ev dizini altında antivirus için tarama yapın.
Sorunsuz bir şekilde yapılabildiğini gözlemleyin.

\item Klamav paketi kurulumu sonrası:

Aşağıda bulunan dosyayı belirli bir dizin içerisine kopyalayın. 
\begin{verbatim}
 wget https://secure.eicar.org/eicar.com.txt
\end{verbatim}

Klamav uygulamasını kullanarak, bu dizini virüs taramasından geçirin, ve klamavın bu virüsü bulup karantinaya almak itediğini gözlemleyin.

\item clamav paketi kurulumu sonrası:

Aşağıda bulunan dosyayı belirli bir dizin içerisine kopyalayın. 
\begin{verbatim}
 wget https://secure.eicar.org/eicar.com.txt
 clamscan eicar.com.txt
\end{verbatim}

Yukarıda verilen ikinci komutun, virüsü tespit edebildiğini gözlemleyin.

\end{enumerate}
\section{Admin alt bileşeni}
\begin{enumerate}
\item sudo paketi kurulumu sonrası:

Aşağıda bulunan komutun sorunsuz çalıştığını gözlemleyin.
\begin{verbatim}
 sudo pisi info pidgin
\end{verbatim}

\item ntp-client paketi kurulumu sonrası:

Eğer ntp server servisi açık ise kapatmalısınız. 

Aşağıda bulunan komutun sistem saatini değiştirdiğini gözlemleyin.  ("date" komutunu kullanarak kontrol edebilirsiniz.)
\begin{verbatim}
ntpdate -b "clock.nc.fukuoka-u.ac.jp" 
\end{verbatim}

\item systemtap paketi kurulumu sonrası:

Aşağıda bulunan komutun sorunsuz çalıştığını gözlemleyin.
\begin{verbatim}
 stap-report
\end{verbatim}

\item systemtap-grapher paketi kurulumu sonrası:
Aşağıda bulunan komutun sorunsuz çalıştığını gözlemleyin.
\begin{verbatim}
stapgraph
\end{verbatim}


\item latencytop paketi kurulumu sonrası:

Aşağıda bulunan komutu çalıştırın ve sistemde açık olan uygulamaların işlem zamanlarının sorunsuz bir şekilde listelendiğini gözlemleyin.
\begin{verbatim}
sudo latencytop 
\end{verbatim}

 \item phpmyadmin paketi kurulumu sonrası:
\begin{itemize}
 \item Apache ve mysql sunucularını servis yöneticisinden başlatın. Aşağıda bulunan komutu çalıştırın ve root parolasını "test" yapın:
\begin{verbatim}
 mysqladmin -u root password 'test'
\end{verbatim}

 \item http://localhost/phpmyadmin/ adresine firefox kullanarak girin. (Kullanıcı adı için root yazın, parola ise test olacak)
 \item Mysql bağlantı sayfasınının sorunsuz açıldığını gözlemleyin.
\end{itemize}

 \item iotop paketi kurulumu sonrası:

Aşağıda bulunan komutu çalıştırın ve sistemde çalışan tüm uygulamaların I/O bant genişliğini listelediğini gözlemleyin.
\begin{verbatim}
 # iotop
\end{verbatim}

\end{enumerate}
\section{Shell alt bileşeni}
\begin{enumerate}
\item Aşağıda bulunan paketler sadece kurulum testine tabidir.
\begin{verbatim}
 bashdb
\end{verbatim}
\item tcsh paketi kurulumu sonrası:
Aşağıda bulunan komutu çalıştırın ve yeni bir kabuğun açıldığını gözlemleyin.

\begin{verbatim}
 csh
\end{verbatim}

 \item zsh paketi kurulumu sonrası:

Aşağıda bulunan komutu çalıştırın ve yeni bir kabuğun açıldığını gözlemleyin.
 \begin{verbatim}
  zsh
  0
 \end{verbatim}

 \item bash-completion paketi kurulumu sonrası:

Aşağıda bulunan komutun hata alınmadan çalıştığını gözlemleyin.
\begin{verbatim}
 pisi --help
\end{verbatim}

\item command-not-found paketi kurulumu sonrası:

Amsn paketi sisteminizde kurulu değil ise, aşağıda bulunan komutu yazdığınızda:
\begin{verbatim}
 amsn
\end{verbatim}

Aşağıda bulunan çıktıyı verdiğini gözlemleyiniz:
\begin{verbatim}
'amsn' uygulaması sisteminizde kurulu değil. Bu paketi, paket yöneticisini kullanarak ya da aşağıdaki komutu çalıştırarak kurabilirsiniz:
sudo pisi it amsn
bash: amsn: command not found
\end{verbatim}

\end{enumerate}
\section{Misc alt bileşeni}
\begin{enumerate}

\item Aşağıda bulunan paketler sadece kurulum testine tabidir.
\begin{verbatim}
 augeas
 task
\end{verbatim}
\item vlock paketi kurulumu sonrası:

Konsoldan "vlock" komutunu çalıştırın ve daha sonra "ENTER" tuşuna basın ve sorunsuz çalıştığını gözlemleyin.


\item byobu paketi kurulumu sonrası:

Aşağıda bulunan komutun yeni bir konsol açtığını gözlemleyin:

\begin{verbatim}
 byobu
\end{verbatim}

\item tree paketi kurulumu sonrası:

İçinde veri bulunan bir dizin için aşağıdaki komutu çalıştırın ve komutun sorunsuz bir şekilde dizin içeriğini ağaç şeklinde listelediğini gözlemleyin.

\begin{verbatim}
 tree <dizin adı>
\end{verbatim}

\item multitail paketi kurulumu sonrası:

Aşağıda bulunan komutun sorunsuz çalıştığını gözlemleyin.
\begin{verbatim}
sudo multitail /var/log/comar3/trace.log 
\end{verbatim}


\item strace paketi kurulumu sonrası: 

Aşağıda bulunan komutununu a2p bin dosyasının sistem çağrılarını, bu çağrılara verilen parametreleri ve dönüş değerlerini gösterdiğini gözlemleyin.
\begin{verbatim}
strace /usr/bin/a2p 
\end{verbatim}



\item ncdu paketi kurulumu sonrası: 

Aşağıda bulunan komutunun programların disk kullanım alanlarını sorunsuz bir şekilde görüntülediğini gözlemleyin.
\begin{verbatim}
 ncdu 
\end{verbatim}


\item puding paketi kurulumu sonrası:

Kmenu'den programın sorunsuz çalıştığını gözlemleyin.

\item pax-utils paketi kurulumu sonrası:

Aşağıda bulunan komutları çalıştırın sorunsuz çalıştıklarını gözlemleyin.
\begin{verbatim}
 scanelf -lqtmyR /
 dumpelf /bin/hostname
\end{verbatim}

 \item fslint paketi kurulumu sonrası:

Kmenüden açılabildiğini gözlemleyin.

İki farklı dizine birer dosya kopyalayın ve uygulamaya bu iki farklı dizini ekleyin, kopya dosyalara tıklayıp bul tuşuna basın kopyalamış olduğunuz dosyanın listelenmiş olduğunu gözlemleyin.
 \item ltrace paketi kurulumu sonrası:

Aşağıda bulunan komutun hatasız çalıştığını gözlemleyin:
\begin{verbatim}
 # ltrace ls
\end{verbatim}
\item elfutils paketi kurulumu sonrası:

util-tr.pdf ltrace testini gerçekleştirin.
\end{enumerate}



\end{document}

