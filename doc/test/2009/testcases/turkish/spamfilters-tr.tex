\documentclass[a4paper,10pt]{article}
\usepackage[utf8x]{inputenc}
\usepackage[left=1cm,top=1cm,right=2cm,bottom=1cm]{geometry}
\title{Spam Filtre Test Aşamaları}
\author{Semen Cirit}

\renewcommand{\labelenumi}{\arabic{enumi}.}
\renewcommand{\labelenumii}{\arabic{enumi}.\arabic{enumii}.}
\renewcommand{\labelenumiii}{\arabic{enumi}.\arabic{enumii}.\arabic{enumiii}.}
\renewcommand{\labelenumiv}{\arabic{enumi}.\arabic{enumii}.\arabic{enumiii}.\arabic{enumiv}.}

\begin{document}

\maketitle

\begin{enumerate}
	\item İlgili paketi kurduktan sonra:
	\item Menüden uygulamalar $\rightarrow$ Kmail'i açın ve spam filtresini aktifleştirmek için:
		
	\begin{enumerate}
		\item Kmail menü çubuğundan Araçlar $\rightarrow$  Spam engelleme sihirbazı yolunu izleyin.
		\item İndirdiğiniz ilgili spam filtresini seçin.
		\item Bir postaya sağ tıklayın ve combobox'tan Filtreyi uygula $\rightarrow$ Filtreyi çöp posta olarak sınıflandır yolunu izleyin.
		Bu spam'in ilgili spam klasörüne gittiğini gözlemleyin.( Default spam klasörü eğer değitirmediyseniz çöp klasörü olacaktır.)

		\item Aşağıdaki linkten gtube.txt'yi indirin: 
		\begin{verbatim}
 		http://spamassassin.apache.org/gtube/
		\end{verbatim}
		\item  Konsoldan komutu çalıştırın:
		\begin{verbatim}
 		cat  gtube.txt | spamc 
		\end{verbatim}
		
		Bu komut size içerisinde şifrelenmiş bir satır içeren buna benzer bir çıktı gönderecek:
		
		\emph{If your spam filter supports it, the GTUBE provides a test by which you
	    	can verify that the filter is installed correctly and is detecting incoming
    		spam. You can send yourself a test mail containing the following string of
    		characters (in upper case and with no white spaces and line breaks):}
		\begin{verbatim}
 		XJS*C4JDBQADN1.NSBN3*2IDNEN*GTUBE-STANDARD-ANTI-UBE-TEST-EMAIL*C.34X
		\end{verbatim}
    		\emph{You should send this test mail from an account outside of your network.}

		\item Daha sonra bu ilgili şifrelenmiş kısmı kopyalayıp mail olarak kendinize gönderin.
		
		Bu mailin direk olarak ilgili spam klasörüne gittiğini gözlemleyin.
	\end{enumerate} 
 


\end{enumerate}

\end{document}




