\documentclass[a4paper,10pt]{article}
\usepackage[turkish]{babel}
\usepackage[utf8]{inputenc}
\usepackage[left=1cm,top=1cm,right=2cm,bottom=2cm]{geometry}

\title{Hardware Bileşeni Test Aşamaları}
\author{Semen Cirit}

\renewcommand{\labelenumi}{\arabic{enumi}.}
\renewcommand{\labelenumii}{\arabic{enumi}.\arabic{enumii}.}
\renewcommand{\labelenumiii}{\arabic{enumi}.\arabic{enumii}.\arabic{enumiii}.}
\renewcommand{\labelenumiv}{\arabic{enumi}.\arabic{enumii}.\arabic{enumiii}.\arabic{enumiv}.}

\begin{document}

\maketitle
\section{CPU alt Bileşeni}
\begin{enumerate}
 \item Aşağıda bulunan paketler sadece kurulum testine tabidir.
  \begin{verbatim}
   microcode-data
  \end{verbatim}

\end{enumerate}

\section{Irda alt Bileşeni}
\begin{enumerate}
 \item lirc paketi kurulumu sonrası:

(Eğer bilgisayarınız infrared destekliyor ise test edebilirsiniz.)

Aşağıda bulunan komutaları çalıştırın ve daha sonra uzaktan kumanda düğmelerinize basın ve kumandanın tanınmış olduğunu gözlemleyin.
\begin{verbatim}
modprobe lirc_gpio
lircd
irw  
\end{verbatim}

\end{enumerate}


\section{Graphics alt Bileşeni}
\begin{enumerate}
 \item ati-control-center paketi kurulumu sonrası:

(sadece ati ekran kartı sürücüsü olanlar test edebilecektir.)
 
Kmenu'den Ati Catalyst Denetim Merkezi'ni açın ve ati kartınızı bulabildiğini ve tanıdığını gözlemleyin.
\end{enumerate}

\section{Firmware alt Bileşeni}
\begin{enumerate}
 \item Aşağıda bulunan paketler kurulum testine tabidir.
\begin{verbatim}
ql2x00-firmware
aic94xx-firmware
ar9170-firmware
bluez-firmware
ql2x00-firmware
v4l-dvb-firmware
alsa- firmware
linux-firmware
\end{verbatim}
\end{enumerate}

\section{Printer alt Bileşeni}
\begin{enumerate}

\item Aşağıda bulunan paketler kurulum testine tabidir.
\begin{verbatim}
gutenprint 
foomatic-db
foomatic-db-engine
foomatic-db-hpijs
foomatic-filters
hplip-doc
cups-devel
cups-php
system-config-printer
system-config-printer-gtk
\end{verbatim}
\item epson-alc1100 paketi kurulumu sonrası:

Eğer Epson AcuLaser C1100 yazıcınız var ise bu yazıcının tanındığını gözlemleyin.

\item cups paketi kurulumu sonrası:

Eğer bir yazıcınız var ise deneme çıktı almaya çalışın ve sorunsuz bir şekilde alabildiğinizi gözlemleyin.

\item hplip-gui ve hplip kurulumu sonrası:

kmenu'den HP aygıt yöneticisini açın ve bir HP yazıcınız var ise aygıtı bulduğunu, yok ise aygıt bulunmamaktadır uyarısı verdiğini gözlemleyiniz.
    
\end{enumerate}

\section{Smart Card alt Bileşeni}
\begin{enumerate}
 \item Aşağıda bulunan paketler sadece kurulum testine tabidir.
\begin{verbatim}
ccid 
mozilla-opensc-signer
openct
opensc
pcsc-lite
pcsc-lite-devel
pcsc-tools
\end{verbatim}


\end{enumerate}

\section{Info alt Bileşeni}
\begin{enumerate}
\item Aşağıda bulunan paketler sadece kurulum testine tabidir.
\begin{verbatim}
hal-devel
hal-docs
hal-libs
media-player-info
\end{verbatim}

\item dmidecode paketi kurulumu sonrası:

Aşağıda bulunan komutun sorunsuz çalıştığını gözlemleyin.
\begin{verbatim}
 sudo dmidecode
\end{verbatim}

 \item smolt paketi kurulumu sonrası:

Aşağıdaki komutu çalıştırın ve datalarınızı gönderin.

Profilinizi paylaşmak için bölümünden, donanımınıza ait linki kopyalayın ve firefox ile sayfanızı görüntüleyin, donanımınızın sunucuya gitmiş olduğunu gözlemleyin.

\begin{verbatim}
 smoltSendProfile
\end{verbatim}

\begin{verbatim}
 kaptan
\end{verbatim}

 \item  hal-info ve hal paketleri kurulumu sonrası:
Makinenize bir usbdisk takın ve sistemin uyarı verdiğini gözlemleyin.

 \item  x86info paketi kurulumu sonrası:

Aşağıda bulunan komutun x86 mimarisi ile oluşturulmuş cpu'ları listelediğini gözlemleyin.
\begin{verbatim}
  x86info
\end{verbatim}


 \item gkrellm paketi kurulumu sonrası:

Servis yöneticisinden gkrellm servisini başlatın.

Aşağıda bulunan komut ile başlatıldığından emin olun.
\begin{verbatim}
 service gkrellm status
\end{verbatim}

\end{enumerate}

\section{Powermanagement alt Bileşeni}
\begin{enumerate}
\item Aşağıda bulunan paketler sadece kurulum testine tabidir.
\begin{verbatim}
 acpid
 knutclient
\end{verbatim}

\item powertop paketi kurulumu sonrası:

Aşağıda bulunan komutu çalıştırın ve sorunsuz bir şekilde çalıştığını gözlemleyin.

\begin{verbatim}
 powertop
\end{verbatim}

\item nut paketi kurulumu sonrası:

Servis yöneticisinden nut servisini başlatın.

Aşağıda bulunan komutu çalıştırın ve servisin başlatıldığını gözlemleyin. 
\begin{verbatim}
 service nut status
\end{verbatim}


 \item lm\_sensors paketi kurulumu sonrası:

hardware-tr.pdf gkrellm testini gerçekleştirin.

Aşağıda bulunan komutunun voltaj ve cpu fanı gibi bilgileri hatasız bir şekilde listelediğini gözlemleyin.
\begin{verbatim}
 # sensors 
\end{verbatim}

\end{enumerate}

\section{Scanner alt Bileşeni}
\begin{enumerate}
\item Aşağıda bulunan paketler sadece kurulum testine tabidir.
\begin{verbatim}
sane-backends-devel
sane-backends-docs
\end{verbatim}

\item sane-backends paketi kurulumu sonrası:

Eğer bir tarayıcınız var ise bu paketi test edebilirsiniz!!
\begin{verbatim}
sane-find-scanner
scanimage 
\end{verbatim}

İlk komutun sorunsuz olarak tarayıcınızı bulduğunu gözlemleyin.
İkinci komutun ise düzgün bir şekilde tarama yaptığını ve bir .pnm uzantılı bir resim dosyası ürettiğini gözlemleyin.

\end{enumerate}

\section{Optical alt Bileşeni}
\begin{enumerate}

\item Aşağıda bulunan paketler sadece kurulum testine tabidir:

\begin{verbatim}
 k3b-docs
\end{verbatim}

\item lsdvd paketi kurulumu sonrası:

Bir muzik veya video DVD'sini bilgisayarınızın sürücüsüne takın. Aşağıda bulunan konutu çalıştırın ve DVD'de bulunan verilerin listelendiğini gözlemleyin. 
\begin{verbatim}
 lsdvd
\end{verbatim}

\item cdrdao paketi kurulumu sonrası:

Aşağıdaki bağlantıda bulunan iso'yu DAO modunda DVD ve CD'ye yazdırın. 
\begin{verbatim}
 wget http://cekirdek.pardus.org.tr/~semen/dist/test/hardware/optical/boot.iso
\end{verbatim}

Bilgisayarınızı CD/DVD sürücünüzden başlatın ve iso'nun düzgün bir şekilde boot ettiğini gözlemleyin.


 \item cdrtools paketi kurulumu sonrası:

Aşağıda bulunan dosyaları k3b ile bir audio cd olarak yazdırın. Yazdırabildiğinizi gözlemleyin.
 \begin{verbatim}
  wget http://cekirdek.pardus.org.tr/~semen/dist/test/multimedia/sound/sound/sample.wav
  wget http://cekirdek.pardus.org.tr/~semen/dist/test/multimedia/sound/sound/sample.mp3
  wget http://cekirdek.pardus.org.tr/~semen/dist/test/multimedia/sound/sound/sample.ogg
 \end{verbatim}

Daha sonra bu cd'de bulunanların okunabildiğini gözlemleyin.
 \item cdparanoia paketi kurulumu sonrası:

Aşağıda bulunan dosyaları k3b ile bir audio cd olarak yazdırın. Yazdırabildiğinizi gözlemleyin.
 \begin{verbatim}
  wget http://cekirdek.pardus.org.tr/~semen/dist/test/multimedia/sound/sound/sample.wav
  wget http://cekirdek.pardus.org.tr/~semen/dist/test/multimedia/sound/sound/sample.mp3
  wget http://cekirdek.pardus.org.tr/~semen/dist/test/multimedia/video/cokluortam/linux.ogg
 \end{verbatim}

Audio CD'yi mount edin ve daha sonra aşağıda bulunan klomutları çalıştırın:
\begin{verbatim}
 cdparanoia -vsQ
 cdparanoia -B
\end{verbatim}
\item k3b-extra-themes paketi kurulumu sonrası:

k3b uygulamasını çalıştırın, Ayarlar $\rightarrow$ k3b uygulamasını yapılandır $\rightarrow$ Temalar yolunu takip edin ve farklı bir tema seçin. Bu temanın uygulanabildiğini gözlemleyin.

\item k3b paketi kurulumu sonrası:
\begin{itemize}
\item Aşağıdaki bağlantıda bulunan iso'yu DVD ve CD'ye yazdırın. 
\begin{verbatim}
 wget http://cekirdek.pardus.org.tr/~semen/dist/test/hardware/optical/boot.iso
\end{verbatim}

Bilgisayarınızı CD/DVD sürücünüzden başlatın ve iso'nun düzgün bir şekilde boot ettiğini gözlemleyin.

\item Aşağıdaki bağlantıda bulunan video ve müzikleri DVD ve CD'ye audio (ses cd'si) olarak yazdırın. (RAW mod ile)
\begin{verbatim}
 wget http://cekirdek.pardus.org.tr/~semen/dist/test/multimedia/sound/sound.tar
 wget http://cekirdek.pardus.org.tr/~semen/dist/test/multimedia/video/cokluortam.tar
\end{verbatim}
Yazdırdığınız CD veya DVD'den video veya müziklerinizi açın, ses ve görüntünün sorunsuz bir şekilde olduğunu gözlemleyin.

\end{itemize}

\end{enumerate}
\section{Emulator alt Bileşeni}
\begin{enumerate}
\item Aşağıda bulunan paketler sadece kurulum testine tabidir:

\begin{verbatim}
 wine-devel
\end{verbatim}
\item virtinst ve virt-viewer paketleri kurulumu sonrası:

Aşağıda buluna  komutu çalıştırın ve sırayla istenen bilgileri girin. Sanal sistemin açıldığını gözlemleyin.
\begin{verbatim}
wget http://cekirdek.pardus.org.tr/~semen/dist/test/hardware/optical/boot.iso
virt-install --prompt
\end{verbatim}

 \item zsnes paketi kurulumu sonrası:

Aşağıda bulunan nintendo oyununu uygulama ile açın ve çalıştırın, sorunsuz bir şekilde çalıştırğını gözlemleyin.
\begin{verbatim}
wget http://cekirdek.pardus.org.tr/~semen/dist/test/hardware/emulator/10%20Yard%20Fight%20(A&S%20NES%20Hack).smc 
\end{verbatim}

 \item wine paketi kurulumu sonrası:
\begin{verbatim}
 wget http://cekirdek.pardus.org.tr/~semen/dist/test/hardware/emulator/Firefox%20Setup%203.5.5.exe
 wine Firefox\ Setup\ 3.5.5.exe
\end{verbatim}
\begin{itemize}
 \item Düzgün bir şekilde firefox'un kurulduğunu ve açıldığını gözlemleyin.
 \end{itemize}

\end{enumerate}

\section{Virtualization alt Bileşeni}
\begin{enumerate}
 \item libvirt paketi kurulumu sonrası:

hardware-tr.pdf virtualbox testini gerçekleştiriniz.
 \item virtualbox, virtualbox-guestadditions, virtualbox-modules ve virt-wrapper paketleri kurulumu sonrası:
 
Virtualbox uygulamasını çalıştırın ve yeni bir sanal sistem oluşturun. Aşağida bulunan iso'yu bu sanal sisteme kurun ve çalıştığını gözlemleyin.
\begin{verbatim}
 wget http://cekirdek.pardus.org.tr/~gokmen/Pardus_Minimal_2009.iso
\end{verbatim}
 \item kvm paketi kurulumu sonrası:

/proc/cpuinfo dosyasından hangi işlemciyi kullandığınızı öğrenin.

Aşağıda bulunan komutların sorunsuz çalıştığını ve iso'nun boot edildiğini gözlemleyin. (Aynı anda diğer sannallaştırma uygulamalarının çalışmadığından emin olun.) (Eğer kvm-intel veya kvm-amd modulleriniz bulunmamakta ise bu testi gerçekleştiremeyeceksiniz.)

\begin{verbatim}
 su -
 modprobe kvm-intel (işlemci intel ise)
 modprobe kvm-amd  (işlemci amd ise)
 lsmod | grep kvm 
 qemu-img create -f raw disk_imajim.img 10G 
 wget http://cekirdek.pardus.org.tr/~semen/dist/test/hardware/optical/boot.iso
 qemu-kvm -m 512 -hda disk_imajim.img -cdrom boot.iso -boot d 
\end{verbatim}

\item qemu paketi kurulumu sonrası.
Aşağıda bulunan komutların sorunsuz çalıştığını ve iso'nun boot edildiğini gözlemleyin. (Aynı anda diğer sannallaştırma uygulamalarının çalışmadığından emin olun.)
\begin{verbatim}
 qemu-img create pardus.img 5G
 wget http://cekirdek.pardus.org.tr/~semen/dist/test/hardware/optical/boot.iso
 qemu -m 512 -soundhw all -localtime -hda pardus.img -cdrom boot.iso -boot d
\end{verbatim}


\end{enumerate}

\section{Misc alt Bileşeni}
\begin{enumerate}
 \item Aşağıda bulunan paketler sadecekurulum testine tabidir.
\begin{verbatim}
  acpi4asus
  libmtp
  libmtp-devel
\end{verbatim}

\item fnfx paketi kurulumu sonrası:

Eğer toshiba dizüstü bilgisayarınız var ise LCD parlaklığı, ses ve fan gibi ayarların tuşlar üzerinden yapılmasına olanak tanındığını gözlemleyin.
\end{enumerate}

\section{Bluetooth alt Bileşeni}
\begin{enumerate}
\item Aşağıda bulunan paketler sadece kurulum testine tabidir. 
\begin{verbatim}
  obexd
  bluez-compat
  cwiid-devel
  bluedevil
  bluedevil-devel
  bluez-libs
  bluez-libs-devel
  libbluedevil
  libbluedevil-devel
\end{verbatim}

 \item obex-data-server paketi kurulumu sonrası: 
 
(Eğer bluetooth aygıtınız var ise bu testi gerçekleştirebilirsiniz.)

Bluetooth aygıtınızı bilgisayarınıza takın ve dosya paylaşımı yapmaya çalışın. Sorunsuz bir şekilde yapılabildiğini gözlemleyin.
 \item bluez paketi kurulumu sonrası: 
 
(Eğer bluetooth aygıtınız var ise bu testi gerçekleştirebilirsiniz.)

Bluetooth aygıtınızı bilgisayarınıza takınız ve aşağıda bulunan komutun bu aygıtı listelediğini gözlemleyiniz.
\begin{verbatim}
 list-devices
\end{verbatim}

 \item cwiid paketi kurulumu sonrası: (Nintendo video oyun kumandanız var ise bu testi gerçekleştirebilirsiniz.)

wiimote uygulamasını açın ve Nintendo kumandanızı aktif hale getirmeye çalışın.

\item blueman paketi kurulumu sonrası:

\begin{itemize}
 \item Cep telefonunuz ve sisteminizde de bluetooth var ise uygulamayı açın ve taratın; uygulamanın cep telefonunuzu bulabildiğini gözlemleyin.

 \item Sistem çekmecesinde bulunan blueman ikonuna sağ tıklayarak dosya gönderi seçin ve cep telefonunuza bir dosya gönderin, sorunsuz bir şekilde gönderilebildiğini gözlemleyin.

\end{itemize}

\item kbluetooth paketi kurulumu sonrası:

\begin{itemize}
 \item Cep telefonunuz ve sisteminizde de bluetooth var ise uygulamayı açın ve taratın; uygulamanın cep telefonunuzu bulabildiğini gözlemleyin.

 \item Sistem çekmecesinde bulunan kbluetooth ikonuna sağ tıklayarak dosya gönderi seçin ve cep telefonunuza bir dosya gönderin, sorunsuz bir şekilde gönderilebildiğini gözlemleyin.

\end{itemize}

\end{enumerate}

\section{Mobile alt Bileşeni}
\begin{enumerate}
 \item Aşağıda bulunan uygulamalar sadece kurulum testine tabidir:
\begin{verbatim}
 gobi_loader
 usb-modeswitch
 pilot-link
\end{verbatim}

\end{enumerate}

\section{Disk alt Bileşeni}
\begin{enumerate}
 \item Aşağıda bulunan uygulamalar sadece kurulum testine tabidir:
\begin{verbatim}
  squashfs-tools
  device-mapper
  device-mapper-static
  lvm2
  lvm2-static
  udftools
  extundelete
  ntfsprogs-devel
  hdparm
  sg3_utils-devel
\end{verbatim}
\item ntfprogs paketi kurulumu sonrası:

gparted ile bir usb çubuğu ntfs olarak formatlayın, formatlama işlemi bittikten sonra usb çubuğu makinenizden çıkarıp tekrar takın ve mount edilebildiğini gözlemleyin. 

\item testdisk paketi kurulumu sonrası:

 "testdisk" komutunu çalıştırın ve Create $\rightarrow$ Select a Media $\rightarrow$ None $\rightarrow$ Analyse $\rightarrow$ QuickSearch yolunu izleyin ve sorunsuz bir şekilde seçilen disk içerisinde bulunan bölümlerin listelendiğini gözlemleyin.

\item ddrescue paketi kurulumu sonrası:

Aşağıda bulunan komutun test2 dosyasına test\_ddrescue içeriğini kopyaladığını gözlemleyin.
 \begin{verbatim}
  wget http://cekirdek.pardus.org.tr/~semen/dist/test/hardware/disk/test_ddrescue
  ddrescue test_ddrescue test2
 \end{verbatim}

\item dosfstools paketi kurulumu sonrası:

gparted ile bir usb çubuğu dos veya fat olarak formatlayın, formatlama işlemi bittikten sonra usb çubuğu makinenizden çıkarıp tekrar takın ve mount edilebildiğini gözlemleyin. 

\item ntfs\_3g paketi kurulumu sonrası:

gparted ile bir usb çubuğu ntfs olarak formatlayın, formatlama işlemi bittikten sonra usb çubuğu makinenizden çıkarıp tekrar takın ve mount edilebildiğini gözlemleyin. 

\item gparted paketi kurulumu sonrası:

Bir usbdisk takın ve gparted uygulamasını Kmenu'den çalıştırın. 

Daha sonra uygulama yardımı ile usbdiske format atın, sorunsuz bir şekilde işlemin gerçekleştiğini gözlemleyin.

\item fuseiso paketi kurulumu sonrası:

Aşağıdaki komutları çalıştırın ve test dizini altında iso dosyasının açıldığını gözlemleyin.
\begin{verbatim}
 wget http://cekirdek.pardus.org.tr/~semen/dist/test/hardware/optical/boot.iso
 fuseiso -p boot.iso test
 su -
 cd test
\end{verbatim}

\item fuse paketi kurulumu sonrası:
hardware-tr.pdf fuse-python testini gerçekleştirin.

\item fuse-python paketi kurulumu sonrası:

ipython paketini kurun ve aşağıda bulunan komutları çalıştırın:
\begin{verbatim}
 ipython
 import fuse
\end{verbatim}

\item sg3\_utils  paketi kurulumu sonrası:

Eğer bir SCSI disk kullanıyorsanız, aşağıda bulunan komutları ilgili SCSI disk bölümünüzü kullanarak çalıştırın.
\begin{verbatim}
 sg_modes <SCSIdisk>
 sg_logs <SCSIdisk>
\end{verbatim}

Örnek:
\begin{verbatim}
 sg_logs /dev/sda1
\end{verbatim}

\item sdparm paketi kurulumu sonrası:

Eğer bir SCSI disk kullanıyorsanız, aşağıda bulunan komutu ilgili SCSI disk bölümünüzü kullanarak çalıştırın.
\begin{verbatim}
 sdparm <SCSIdisk>
\end{verbatim}

Örnek:
\begin{verbatim}
 sdparm /dev/sda1
\end{verbatim}


\item filelight paketi kurulumu sonrası:

Uygulamayı kmenuden açın ve disk kullanımının görsel olarak yapıldığını gözlemleyin. Tara $\rightarrow$ Ev dizinini tara bölümünden dizininizi taratın ve surunsuz bir şekilde tarandığını gözlemleyin.
 \item partimage paketi kurulumu sonrası:

Aşağıda bulunan komutun sorunsuz bir şekilde çalıştığını gözlemleyin:
\begin{verbatim}
  su -
  partimage
\end{verbatim}

 \item mountmanager paketi kurulumu sonrası:

Uygulamayı kmenuden açın ve daha sonra bir usb stick takın ve mountmanager'a bu aygıtın eklendiğini gözlemleyin.
 \item partitionmanager paketi kurulumu sonrası:

Partitionmanager'ı kmenüden açın ve diskinizin küçük bir kısmını bölmeye çelışın sorunsuz bir şekilde bölünebildiğini gözlemleyin.

USB bellek takarak, bu bellek üzerinden de işlem yapabilirsiniz.

\item reiserfsprogs paketi kurulumu sonrası:

Bir usb disk takın ve /dev dizini altında bulunan uzantısını kullanarak aşağıda bulunan komutları çalıştırın. Düzgün bir şekilde çalıştıklarını gözlemleyin. 

\begin{verbatim}
 mkreiserfs /dev/<aygıt uzantısı> -f
 reiserfsck /dev/<aygıt uzantısı>
\end{verbatim}

\end{enumerate}
\section{Sound alt Bileşeni}
\begin{enumerate}
 \item alsa-driver,alsa-lib-32bit, alsa-headers, alsa-lib, alsa-lib-devel, alsa-plugins, alsa-plugins-pulseaudio, alsa-tools, alsa-utils paketleri kurulumu sonrası.
\begin{itemize}
 \item Bilgisayarınızı yeniden başlatın ve açılış sesinin sorunsuz bir şekilde çalıştığını gözlemleyin.
 \item Aşağıda bulunan dosyanın düzgün çalıştığını gözlemleyin.
\begin{verbatim}
 wget http://cekirdek.pardus.org.tr/~semen/dist/test/multimedia/sound/sound/sample.mp3
 mplayer sample.mp3
\end{verbatim}

\item alsa-tools-gui paketi kurulumu sonrası:

Eğer  EchoAudio, Envy24, Hammerfall HDSP, RMedigicontrol ses kartlarından birini kullanıyorsanız bu testi gerçekleştirebilirsiniz.

Echomixer uygulaması EchoAudio aygıtı için.

Envy24control uygulaması Envy24 aygıtı için.

HDSPconf ve HDSPmixer uygulaması Hammerfall HDSP aygıtı için.

Rmedigicontrol uygulaması RMedigicontrol aygıtı için.
\end{itemize}


\end{enumerate}
\section{Mobile alt Bileşeni}
\begin{enumerate}
\item Aşağıda bulunan paketler sadece kurulum testine tabidir.
\begin{verbatim}
 usb-modeswitch
libopensync-plugin-irmc
gammu-devel
\end{verbatim}

\item gammu paketi kurulumu sonrası

hardware-tr.pdf python-gammu testini gerçekleştiriniz.

\item python-gammu paketi kurulumu sonrası:

Aşağıda bulunan komutların sorunsuz çalıştığını gözlemleyin.

\begin{verbatim}
 ipython
 import gammu
\end{verbatim}


\item wammu paketi kurulumu sonrası:

Eğer bilgisayara usb, bluetooth veya IrDA aygıtlarından biri ile bağlanabilen bir telefonunuz var ise wammu uygulamasını açın ve telefonunuzda bulunan bilgilerin bilgisayara aktarılabildiğini gözlemleyin.

\item libopensync-plugin-kdepim paketi kurulumu sonrası:

Aşağıda bulunan komutun kdepim-sync eklentisini listelediğini gözlemleyin.
\begin{verbatim}
 msynctool --listplugins
\end{verbatim}

Aşağıda bulunan komutların sorunsuz çalıştığını gözlemleyin:

\begin{verbatim}
 msynctool --addgroup sync2kontact
 msynctool --addmember sync2kontact kdepim-sync
\end{verbatim}


 \item libopensync-plugin-google-calendar paketi kurulumu sonrası:

Eğer bir gmail üyeliğiniz var ise bu testi gerçekleştirebilirsiniz.
\begin{itemize}
 \item msynctool paketini kurun.
 \item http://www.google.com/calendar/feeds/ adresinden google üyeliğinize girin ve takviminizi aktive edin.
 \item Aşağıda bulunan komutları çalıştırın, üçüncü ve dördüncü komutlarda gmail ile üyeliğiniz ile ilgili istenen bilgileri girin.

\begin{verbatim}
 msynctool --addgroup GoogleCalendar
 msynctool --addmember GoogleCalendar google-calendar
 msynctool --configure GoogleCalendar 1
 msynctool --configure GoogleCalendar 2
 msynctool --sync GoogleCalendar	
\end{verbatim}

\end{itemize}

Sorunsuz bir şekilde takvimlerin senkronize olduklarını gözlemleyin.	
\end{enumerate}


\end{document}

