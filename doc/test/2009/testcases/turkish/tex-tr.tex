\documentclass[a4paper,10pt]{article}
\usepackage[turkish]{babel}
\usepackage[utf8]{inputenc}
\usepackage[left=1cm,top=1cm,right=2cm,bottom=2cm]{geometry}


\renewcommand{\labelenumi}{\arabic{enumi}.}
\renewcommand{\labelenumii}{\arabic{enumi}.\arabic{enumii}.}
\renewcommand{\labelenumiii}{\arabic{enumi}.\arabic{enumii}.\arabic{enumiii}.}
\renewcommand{\labelenumiv}{\arabic{enumi}.\arabic{enumii}.\arabic{enumiii}.\arabic{enumiv}.}

\title{Tex Bileşeni Test Aşamaları}
\author{Semen Cirit}

\begin{document}

\maketitle

\section{Base alt bileşeni}
\begin{enumerate}
 \item texlive-basic paketi kurulumu sonrası:
 texlive-latex paketini kurun ve aşağıda bulunan komutların düzgün çalıştığını gözlemleyin.

\begin{verbatim}
 wget http://cekirdek.pardus.org.tr/~semen/dist/test/tex/base/test_texlivelatex.tex
 pdflatex test_texlivelatex.tex
 okular test_texlivelatex.pdf
\end{verbatim}

\item texlive-core paketi kurulumu sonrası:
 texlive-latex paketini kurun ve aşağıda bulunan komutların düzgün çalıştığını gözlemleyin.

\begin{verbatim}
 wget http://cekirdek.pardus.org.tr/~semen/dist/test/tex/base/test_texlivelatex.tex
 pdflatex test_texlivelatex.tex
 okular test_texlivelatex.pdf
\end{verbatim}
\item texlive-documentation-base paketi kurulumu sonrası:

Aşağıda bulunan paketlerin sorunsuz bir şekilde kurulduğunu gözlemleyin.	
\begin{verbatim}
 texlive-documentation-greek
 texlive-documentation-italian
 texlive-documentation-japanese
\end{verbatim}
\item texlive-latex paketi kurulumu sonrası:

Kile paketini kurun ve yukarıda verilen dosyayı kile ile açın ve QuickBuild  seçeneğini tıklayın ve sorunsuz olarak pdf dosyasının oluşturğunu gözlemleyin.
\begin{verbatim}
# wget http://cekirdek.pardus.org.tr/~semen/dist/test/tex/base/test_texlivelatex.tex
\end{verbatim}

\item texlive-latexrecommended paketi kurulumu sonrası:

latex-leaflet paketini kurun
\begin{verbatim}
 # wget http://cekirdek.pardus.org.tr/~semen/dist/test/tex/addon/test_latexleaflet.tex
\end{verbatim}

Kile paketini kurun ve yukarıda verilen dosyayı kile ile açın ve QuickBuild  seçeneğini tıklayın ve sorunsuz olarak pdf dosyasının oluşturğunu gözlemleyin.

\item texlive-metapost paketi kurulumu sonrası:

latex-feynmf paketini kurun
\begin{verbatim}
 # wget http://cekirdek.pardus.org.tr/~semen/dist/test/tex/addon/test_latexfeynmf.tex
\end{verbatim}

Kile paketini kurun ve yukarıda verilen dosyayı kile ile açın ve QuickBuild  seçeneğini tıklayın ve sorunsuz olarak pdf dosyasının oluşturğunu gözlemleyin.

\end{enumerate}


\section{Addon alt bileşeni}
\begin{enumerate}
\item  latex-beamer kurulumu sonrası:
\begin{verbatim}
 # wget http://cekirdek.pardus.org.tr/~semen/dist/test/tex/addon/test_latexbeamer.tex
\end{verbatim}

Kile paketini kurun ve yukarıda verilen dosyayı kile ile açın ve QuickBuild  seçeneğini tıklayın ve sorunsuz olarak pdf dosyasının oluştuğunu gözlemleyin.
\item  latex-curvita kurulumu sonrası:
\begin{verbatim}
 # wget http://cekirdek.pardus.org.tr/~semen/dist/test/tex/addon/test_latexcurrvita.tex
\end{verbatim}

Kile paketini kurun ve yukarıda verilen dosyayı kile ile açın ve QuickBuild  seçeneğini tıklayın ve sorunsuz olarak pdf dosyasının oluşturğunu gözlemleyin.
\item  latex-envlab kurulumu sonrası:
\begin{verbatim}
 # wget http://cekirdek.pardus.org.tr/~semen/dist/test/tex/addon/test_latexenvlab.tex
\end{verbatim}

Kile paketini kurun ve yukarıda verilen dosyayı kile ile açın ve QuickBuild  seçeneğini tıklayın ve sorunsuz olarak pdf dosyasının oluşturğunu gözlemleyin.
\item  latex-europecv kurulumu sonrası:
\begin{verbatim}
 # wget http://cekirdek.pardus.org.tr/~semen/dist/test/tex/addon/test_latexeuropecv.tex
\end{verbatim}

Kile paketini kurun ve yukarıda verilen dosyayı kile ile açın ve QuickBuild  seçeneğini tıklayın ve sorunsuz olarak pdf dosyasının oluşturğunu gözlemleyin.
\item  latex-feynmf kurulumu sonrası:
\begin{verbatim}
 # wget http://cekirdek.pardus.org.tr/~semen/dist/test/tex/addon/test_latexfeynmf.tex
\end{verbatim}

Kile paketini kurun ve yukarıda verilen dosyayı kile ile açın ve QuickBuild  seçeneğini tıklayın ve sorunsuz olarak pdf dosyasının oluşturğunu gözlemleyin.
\item  latex-gbrief kurulumu sonrası:
\begin{verbatim}
 # wget http://cekirdek.pardus.org.tr/~semen/dist/test/tex/addon/test_latexgbrief.tex
\end{verbatim}

Kile paketini kurun ve yukarıda verilen dosyayı kile ile açın ve QuickBuild  seçeneğini tıklayın ve sorunsuz olarak pdf dosyasının oluşturğunu gözlemleyin.
\item  latex-glossaries kurulumu sonrası:
\begin{verbatim}
 # wget http://cekirdek.pardus.org.tr/~semen/dist/test/tex/addon/test_latexglossaries.tex
\end{verbatim}

Kile paketini kurun ve yukarıda verilen dosyayı kile ile açın ve QuickBuild  seçeneğini tıklayın ve sorunsuz olarak pdf dosyasının oluşturğunu gözlemleyin.
\item  latex-leaflet kurulumu sonrası:
\begin{verbatim}
 # wget http://cekirdek.pardus.org.tr/~semen/dist/test/tex/addon/test_latexleaflet.tex
\end{verbatim}

Kile paketini kurun ve yukarıda verilen dosyayı kile ile açın ve QuickBuild  seçeneğini tıklayın ve sorunsuz olarak pdf dosyasının oluşturğunu gözlemleyin.
\item  latex-maltese kurulumu sonrası:
\begin{verbatim}
 # wget http://cekirdek.pardus.org.tr/~semen/dist/test/tex/addon/test_latexmaltese.tex
\end{verbatim}

Kile paketini kurun ve yukarıda verilen dosyayı kile ile açın ve QuickBuild  seçeneğini tıklayın ve sorunsuz olarak pdf dosyasının oluşturğunu gözlemleyin.

\item  latex-mh kurulumu sonrası:
\begin{verbatim}
 # wget http://cekirdek.pardus.org.tr/~semen/dist/test/tex/addon/test_latexmh.tex
\end{verbatim}

Kile paketini kurun ve yukarıda verilen dosyayı kile ile açın ve QuickBuild  seçeneğini tıklayın ve sorunsuz olarak pdf dosyasının oluşturğunu gözlemleyin.
\item  latex-svninfo kurulumu sonrası:
\begin{verbatim}
 # wget http://cekirdek.pardus.org.tr/~semen/dist/test/tex/addon/test_latexsvninfo.tex
\end{verbatim}

Kile paketini kurun ve yukarıda verilen dosyayı kile ile açın ve QuickBuild  seçeneğini tıklayın ve sorunsuz olarak pdf dosyasının oluşturğunu gözlemleyin.

\item  latex-xcolor kurulumu sonrası:
\begin{verbatim}
 # wget http://cekirdek.pardus.org.tr/~semen/dist/test/tex/addon/test_latexxcolor.tex
\end{verbatim}

Kile paketini kurun ve yukarıda verilen dosyayı kile ile açın ve QuickBuild  seçeneğini tıklayın ve sorunsuz olarak pdf dosyasının oluşturğunu gözlemleyin.

\item texlive-bibtexextra kurulumu sonrası:
\begin{verbatim}
 # wget http://cekirdek.pardus.org.tr/~semen/dist/test/tex/addon/test_texlivebibtexextra.tex
\end{verbatim}

Kile paketini kurun ve yukarıda verilen dosyayı kile ile açın ve QuickBuild  seçeneğini tıklayın ve sorunsuz olarak pdf dosyasının oluşturğunu gözlemleyin.
\item texlive-fontsextra kurulumu sonrası:
\begin{verbatim}
 # wget http://cekirdek.pardus.org.tr/~semen/dist/test/tex/addon/test_texlivefontsextra.tex
\end{verbatim}

Kile paketini kurun ve yukarıda verilen dosyayı kile ile açın ve QuickBuild  seçeneğini tıklayın ve sorunsuz olarak pdf dosyasının oluşturğunu gözlemleyin.

\item texlive-fontsrecommended kurulumu sonrası:
\begin{verbatim}
 # wget http://cekirdek.pardus.org.tr/~semen/dist/test/tex/addon/test_texlivefontsrecommended.tex
\end{verbatim}

Kile paketini kurun ve yukarıda verilen dosyayı kile ile açın ve QuickBuild  seçeneğini tıklayın ve sorunsuz olarak pdf dosyasının oluşturğunu gözlemleyin.

\item texlive-formatextra kurulumu sonrası:
\begin{verbatim}
 # wget http://cekirdek.pardus.org.tr/~semen/dist/test/tex/addon/test_texliveformatsextra.tex
\end{verbatim}

Kile paketini kurun ve yukarıda verilen dosyayı kile ile açın ve QuickBuild  seçeneğini tıklayın ve sorunsuz olarak pdf dosyasının oluşturğunu gözlemleyin.

\item texlive-games kurulumu sonrası:
\begin{verbatim}
 # wget http://cekirdek.pardus.org.tr/~semen/dist/test/tex/addon/test_texlivegames.tex
\end{verbatim}

Kile paketini kurun ve yukarıda verilen dosyayı kile ile açın ve QuickBuild  seçeneğini tıklayın ve sorunsuz olarak pdf dosyasının oluşturğunu gözlemleyin.

\item texlive-humanities kurulumu sonrası:
\begin{verbatim}
 # wget http://cekirdek.pardus.org.tr/~semen/dist/test/tex/addon/test_texlivehumanities.tex
\end{verbatim}

Kile paketini kurun ve yukarıda verilen dosyayı kile ile açın ve QuickBuild  seçeneğini tıklayın ve sorunsuz olarak pdf dosyasının oluşturğunu gözlemleyin.

\item texlive-latex3 kurulumu sonrası:
\begin{verbatim}
 # wget http://cekirdek.pardus.org.tr/~semen/dist/test/tex/addon/test_texlivelatex3.tex
\end{verbatim}

Kile paketini kurun ve yukarıda verilen dosyayı kile ile açın ve QuickBuild  seçeneğini tıklayın ve sorunsuz olarak pdf dosyasının oluşturğunu gözlemleyin.
\item texlive-latexextra kurulumu sonrası:
\begin{verbatim}
 # wget http://cekirdek.pardus.org.tr/~semen/dist/test/tex/addon/test_texlivelatexextra.tex
\end{verbatim}

Kile paketini kurun ve yukarıda verilen dosyayı kile ile açın ve QuickBuild  seçeneğini tıklayın ve sorunsuz olarak pdf dosyasının oluşturğunu gözlemleyin.
\item texlive-mathxextra kurulumu sonrası:
\begin{verbatim}
 # wget http://cekirdek.pardus.org.tr/~semen/dist/test/tex/addon/test_texlivemathextra.tex
\end{verbatim}

Kile paketini kurun ve yukarıda verilen dosyayı kile ile açın ve QuickBuild  seçeneğini tıklayın ve sorunsuz olarak pdf dosyasının oluşturğunu gözlemleyin.

\item texlive-music kurulumu sonrası:
\begin{verbatim}
 # wget http://cekirdek.pardus.org.tr/~semen/dist/test/tex/addon/test_texlivemusic.tex
\end{verbatim}

Kile paketini kurun ve yukarıda verilen dosyayı kile ile açın ve QuickBuild  seçeneğini tıklayın ve sorunsuz olarak pdf dosyasının oluşturğunu gözlemleyin.
\item texlive-plainextra kurulumu sonrası:
\begin{verbatim}
 # wget http://cekirdek.pardus.org.tr/~semen/dist/test/tex/addon/test_texliveplainextra.tex
\end{verbatim}

Kile paketini kurun ve yukarıda verilen dosyayı kile ile açın ve QuickBuild  seçeneğini tıklayın ve sorunsuz olarak pdf dosyasının oluşturğunu gözlemleyin.
\item texlive-publishers kurulumu sonrası:
\begin{verbatim}
 # wget http://cekirdek.pardus.org.tr/~semen/dist/test/tex/addon/test_texlivepublishers.tex
\end{verbatim}

Kile paketini kurun ve yukarıda verilen dosyayı kile ile açın ve QuickBuild  seçeneğini tıklayın ve sorunsuz olarak pdf dosyasının oluşturğunu gözlemleyin.

\item texlive-pictures kurulumu sonrası:
\begin{verbatim}
 # wget http://cekirdek.pardus.org.tr/~semen/dist/test/tex/addon/test_texlivepictures.tex
\end{verbatim}

Kile paketini kurun ve yukarıda verilen dosyayı kile ile açın ve QuickBuild  seçeneğini tıklayın ve sorunsuz olarak pdf dosyasının oluşturğunu gözlemleyin.

\item texlive-xetex kurulumu sonrası:
\begin{verbatim}
 # wget http://cekirdek.pardus.org.tr/~semen/dist/test/tex/addon/test_texlivexetex.tex
\end{verbatim}

Kile paketini kurun ve yukarıda verilen dosyayı kile ile açın ve QuickBuild  seçeneğini tıklayın ve sorunsuz olarak pdf dosyasının oluşturğunu gözlemleyin.
\item texlive-science kurulumu sonrası:
\begin{verbatim}
 # wget http://cekirdek.pardus.org.tr/~semen/dist/test/tex/addon/test_texlivescience.tex
\end{verbatim}

Kile paketini kurun ve yukarıda verilen dosyayı kile ile açın ve QuickBuild  seçeneğini tıklayın ve sorunsuz olarak pdf dosyasının oluşturğunu gözlemleyin.

\item texlive-genericrecommended kurulumu sonrası:
\begin{verbatim}
 # wget http://cekirdek.pardus.org.tr/~semen/dist/test/tex/addon/test_texlivegenericrecommended.tex
\end{verbatim}

Kile paketini kurun ve yukarıda verilen dosyayı kile ile açın ve sırasıyla Alt+2 ve Alt+3 tuşlarına basın ve sorunsuz birşekilde DVI uzantılı bir dosyanın oluştuğunu gözlemleyin.

\item texlive-pstricks kurulumu sonrası:
\begin{verbatim}
 # wget http://cekirdek.pardus.org.tr/~semen/dist/test/tex/addon/test_texlivepstricks.tex
\end{verbatim}

Kile paketini kurun ve yukarıda verilen dosyayı kile ile açın ve sırasıyla Alt+2 ve Alt+3 tuşlarına basın ve sorunsuz birşekilde DVI uzantılı bir dosyanın oluştuğunu gözlemleyin.
\item  latex-passivetex ve latex-xmltex kurulumu sonrası:
\begin{verbatim}
 # wget http://cekirdek.pardus.org.tr/~semen/dist/test/tex/addon/test-latexpassivetex.fo
 # pdfxmltex test-latexpassivetex.fo
\end{verbatim}

Yukarıda bulunan komutun düzgün bir şekilde pdf çıktısı ürettiğini gözlemleyin.

\item texlive-texinfo kurulumu sonrası:

Asymptote paketini kurun ve aşağıda bulunan komutları çalıştırın.
\begin{verbatim}
 # wget http://cekirdek.pardus.org.tr/~semen/dist/test/multimedia/graphics/test_asymptote.asy
 # asy test_asymptote.asy
 # okular test_asymptote.eps
\end{verbatim}

Eps dosyasının sorunsuz bir şekilde üretildiğini gözlemleyin.

\item Aşağıdaki paketler kurulum testine tabidir.
texlive-genericextra
texlive-htmlxml
texlive-context
texlive-omega
texlive-psutils
\end{enumerate}
\section{Doc alt bileşeni} 
\begin{enumerate}
\item  Aşağıda bulunan dosyalar sadece kurulum testine tabidir.

\begin{verbatim}
 texlive-documentation-bulgarian
 texlive-documentation-chinese
 texlive-documentation-czechslovak
 texlive-documentation-dutch
 texlive-documentation-english
 texlive-documentation-finnish
 texlive-documentation-french
 texlive-documentation-german
 texlive-documentation-greek
 texlive-documentation-italian
 texlive-documentation-japanese
 texlive-documentation-korean
 texlive-documentation-mongolian
 texlive-documentation-polish
 texlive-documentation-portuguese
 texlive-documentation-russian
 texlive-documentation-slovenian
 texlive-documentation-spanish
 texlive-documentation-thai
 texlive-documentation-turkish
 texlive-documentation-ukrainian
 texlive-documentation-vietnamese
\end{verbatim}

\end{enumerate}

\section{Doc alt bileşeni} 
\begin{enumerate}
 \item Aşağıda bulunan paketler sadece kurulum testine tabidir.
\begin{verbatim}
 texlive-langafrican
 texlive-langarab
 texlive-langarmenian
 texlive-langcjk
 texlive-langcroatian
 texlive-langczechslovak
 texlive-langdanish
 texlive-langdutch
 texlive-langfinnish
 texlive-langfrench
 texlive-langhebrew
 texlive-langhungarian
 texlive-langindic
 texlive-langitalian
 texlive-langlatin
 texlive-langmongolian
 texlive-langnorwegian
 texlive-langpolish
 texlive-langportuguese
 texlive-langspanish
 texlive-langswedish
 texlive-langtibetan
 texlive-langukenglish
 texlive-langvietnamese
\end{verbatim}
\item  texlive-langgerman, texlive-langgreek, texlive-langother, texlive-langcyrillic paketleri kurulumu sonrası:

texlive-latex paketini kurun kile ile aşağıda verilen dosyayı çalıştırın ve sorunsuz bir şekilde pdf dosyasına dönüştüğünü gözlemleyin.

\begin{verbatim}
# wget http://cekirdek.pardus.org.tr/~semen/dist/test/tex/base/test_texlivelatex.tex
\end{verbatim}

\end{enumerate}

\section{Tool alt bileşeni}
\begin{enumerate}
 \item Aşağıda bulunan paketler sadece kurulum testine tabidir.
\begin{verbatim}
 chktex
 detex
 jadetex
 lcdf-typetools
 xindy
\end{verbatim}

 \item auctex paketi kurulumu sonrası:
  
\begin{verbatim}
 # wget http://cekirdek.pardus.org.tr/~semen/dist/test/tex/base/test_texlivelatex.tex
 # emacs test_texlivelatex.tex 
\end{verbatim}

Emacs editörü açıldıktan sonra tex $\rightarrow$ Tex File seçin ve sorunsuz bir şekilde .dvi uzantılı bi dosyanın oluştuğunu gözlemleyin.

 \item dvi2tty paketi kurulumu sonrası:
  
\begin{verbatim}
 # wget http://cekirdek.pardus.org.tr/~semen/dist/test/tex/tool/test_texlivelatex.dvi
 # dvi2tty -o test.tty test_texlivelatex.dvi
\end{verbatim}

test.tty dosyasının düzgün bir şekilde oluştuğunu gözlemleyin.
\item dvipdfm paketi kurulumu sonrası:
  
\begin{verbatim}
 # wget http://cekirdek.pardus.org.tr/~semen/dist/test/tex/tool/test_texlivelatex.dvi
 # dvipdfm test_texlivelatex.dvi
 # okular test_texlivelatex.pdf
\end{verbatim}

 Düzgün bir şekilde dosyasının açıldığını gözlemleyin.
\item dvipng paketi kurulumu sonrası:
 \begin{verbatim}
  wget http://cekirdek.pardus.org.tr/~semen/dist/test/tex/tool/test_texlivelatex.dvi
  dvipng -o test.png test_texlivelatex.dvi
 \end{verbatim}

 test.png dosyasının düzgün bir şekilde açıldığını gözlemleyin.
\item dvipost paketi kurulumu sonrası:
  \begin{verbatim}
 # wget http://cekirdek.pardus.org.tr/~semen/dist/test/tex/tool/test_texlivelatex.dvi
 # dvipost test_texlivelatex.dvi test.pdf
 \end{verbatim}
 test.pdf dosyasının düzgün bir şekilde açıldığını gözlemleyin.

\item hevea paketi kurulumu sonrası:
 \begin{verbatim}
 wget http://cekirdek.pardus.org.tr/~semen/dist/test/tex/base/test_texlivelatex.tex
 hevea test_texlivelatex.tex
 \end{verbatim}
Firefox Dosya $\rightarrow$ Dosya aç yolunu izleyerek test\_texlivelatex.html dosyasını açın. Ve düzgün bir şekilde açıldığını gözlemleyin.

\item kile paketi kurulumu sonrası:
 \begin{verbatim}
 wget http://cekirdek.pardus.org.tr/~semen/dist/test/tex/base/test_texlivelatex.tex
 \end{verbatim}

Kile ile yukarıda bulunan dosyayı açın ve QuickBuild butonuna basın. Düzgün bir şekilde pdf dosyasının oluştuğunu gözlemleyin.
\item lyx paketi kurulumu sonrası:
 \begin{verbatim}
 wget http://cekirdek.pardus.org.tr/~semen/dist/test/tex/tool/test_lyx.lyx
 \end{verbatim}

Lyx ile yukarıda bulunan dosyayı açın ve Görünüm $\rightarrow$  DVI yolunu izleyin. Düzgün bir şekilde dvi dosyasının oluştuğunu gözlemleyin.

\item mplib paketi kurulumu sonrası:
  
 texlive-latex paketini kurun ve aşağıda bulunan komutların düzgün çalıştığını gözlemleyin.

\begin{verbatim}
wget http://cekirdek.pardus.org.tr/~semen/dist/test/tex/base/test_texlivelatex.tex
pdflatex test_texlivelatex.tex
okular test_texlivelatex.pdf
\end{verbatim}
\item luatex paketi kurulumu sonrası:
  
 texlive-latex paketini kurun ve aşağıda bulunan komutların düzgün çalıştığını gözlemleyin.

\begin{verbatim}
# wget http://cekirdek.pardus.org.tr/~semen/dist/test/tex/base/test_texlivelatex.tex
# pdflatex test_texlivelatex.tex
# okular test_texlivelatex.pdf
\end{verbatim}

\item  pgf kurulumu sonrası:

latex-beamer ve kile paketlerini kurun:
\begin{verbatim}
 # wget http://cekirdek.pardus.org.tr/~semen/dist/test/tex/addon/test_latexbeamer.tex
\end{verbatim}

Yukarıda verilen dosyayı kile ile açın ve QuickBuild  seçeneğini tıklayın ve sorunsuz olarak pdf dosyasının oluştuğunu gözlemleyin.

\item ps2eps paketi kurulumu sonrası:

\begin{verbatim}
# wget http://cekirdek.pardus.org.tr/~semen/dist/test/tex/tool/test_texlivelatex.ps
# ps2eps test_texlivelatex.ps
# okular test_texlivelatex.eps
\end{verbatim}

Düzgün bir şekilde eps dosyasının üretilmiş olduğunu gözlemleyin.
\item xfig ve transfig paketleri kurulumu sonrası:

Xfig uygulamasını açın ve File $\rightarrow$ Open ile yeni bir dosya açın.

Dosyaya bir kare çizin ve File $\rightarrow$ Export bölümünden dosyayı export edin.

Öntanımlı eps dosyasının sorunsuz bir şekilde oluşmuş olduğunu gözlemleyin.

\item translator paketi kurulumu sonrası:
\begin{verbatim}
 # wget http://cekirdek.pardus.org.tr/~semen/dist/test/tex/tool/translator-manual-en-macros.tex
 # wget http://cekirdek.pardus.org.tr/~semen/dist/test/tex/tool/translator-manual-en.tex
\end{verbatim}

Yukarıda bulunan dosyaları bir dizine kopyalayın ve ikinci dosyayı kile ile açın. QuickBuild  seçeneğini tıklayın ve sorunsuz olarak pdf dosyasının oluştuğunu gözlemleyin.
\item xdvik paketi kurulumu sonrası:

Aşağıda bulunan dosyanın xdvik uygulaması ile sorunsuz bir şekilde açıldığını gözlemleyin.
\begin{verbatim}
 # wget http://cekirdek.pardus.org.tr/~semen/dist/test/tex/tool/test_texlivelatex.dvi
 \end{verbatim}

\end{enumerate}

\end{document}

