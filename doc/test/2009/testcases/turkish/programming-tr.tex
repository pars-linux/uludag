
\documentclass[a4paper,10pt]{article}
\usepackage[turkish]{babel}
\usepackage[utf8]{inputenc}
\usepackage[left=1cm,top=1cm,right=2cm,bottom=2cm]{geometry}

\title{Programming Bileşeni Test Aşamaları}
\author{Semen Cirit}

\renewcommand{\labelenumi}{\arabic{enumi}.}
\renewcommand{\labelenumii}{\arabic{enumi}.\arabic{enumii}.}
\renewcommand{\labelenumiii}{\arabic{enumi}.\arabic{enumii}.\arabic{enumiii}.}
\renewcommand{\labelenumiv}{\arabic{enumi}.\arabic{enumii}.\arabic{enumiii}.\arabic{enumiv}.}

\begin{document}

\maketitle
\section{Build alt bileşeni}
\begin{enumerate}
 \item Aşağıda bulunan paketler sadece kurulum testine tabidir.
\begin{verbatim}
 autogen
\end{verbatim}

\end{enumerate}


\section{Profiler alt bileşeni}
\begin{enumerate}
 \item valgrind paketi kurulumu sonrası:

Gimp uygulamasının işlem ve hafıza durumunun listelendiğini gözlemleyin.

  \begin{verbatim}
   valgrind gimp
  \end{verbatim}


 \item oprofile paketi kurulumu sonrası:

Aşağıda bulunan komutların hatasız çalıştığını gözlemleyin.
  \begin{verbatim}
   su -
   opcontrol --no-vmlinux
   opcontrol --start
   opcontrol --dump
   opreport
  \end{verbatim}

\end{enumerate}

\section{Debug alt bileşeni}
\begin{enumerate}
\item cgdb paketi kurulumu sonrası:

"cgdb" komutunu çalıştırım ve yeni bir konsol arayüzünün açıldığını gözlemleyin.
 \item gdb paketi kurulumu sonrası:

Aşağıda bulunan komutları çalıştırın ve gdb-test.c'in derlendiğini ve daha sonra gdb-test ile ilgili gdb'nin bir hata bulduğunu gözlemleyin.
\begin{verbatim}
wget http://cekirdek.pardus.org.tr/~semen/dist/test/programming/debug/gdb-test.c
gcc -g gdb-test.c -o gdb-test
./gdb-test
gdb gdb-example
run
backtrace
quit
\end{verbatim}


\end{enumerate}


\section{Microcontroller alt bileşeni}
\begin{enumerate}
 \item Aşağıda bulunan paketler kurulum testine tabidir.
\begin{verbatim}
 avr-libc
 avrdude
 binutils-avr
 gcc-avr
\end{verbatim}

\end{enumerate}

\section{Tool alt bileşeni}
\begin{enumerate}
 \item Aşağıda bulunan paketler sadece kurulum testine tabidir.
\begin{verbatim}
fcgi
mcpp
translate-toolkit-docs
ctags
\end{verbatim}

\item translate-toolkit paketi kurulumu sonrası:

ipython paketini kurun ve aşağıda bulunan komutları çalıştırın:
\begin{verbatim}
 ipython
 import translate
\end{verbatim}

\item cdecl paketi kurulumu sonrası:

Aşağıda bulunan kopmutun düzgün çalıştığını gözlemleyin.
\begin{verbatim}
  cdecl
 cdecl> #include <string.h>
\end{verbatim}

\end{enumerate}

\section{vcs alt bileşeni}
\begin{enumerate}
\item Aşağıda bulunan paketler sadece kurulum testine tabidir.
\begin{verbatim}
mod_dav_svn
abicheck
git-cvs
git-emacs
git-gui
git-svn
gitweb
\end{verbatim}

\item bzr paketi kurulumu sonrası:
\begin{verbatim}
 bzr init test
 bzr branch test test2
 cd test
 touch testfile
 bzr add testfile
 bzr diff
\end{verbatim}

\item SVK paketi kurulumu sonrası:

Aşağıda bulunan komutların sorunsuz çalıştığını gözlemleyin:
\begin{verbatim}
svk mirror http://svn.ofbiz.org/svn/ofbiz/trunk //uplink
svk sync //svk/uplink
\end{verbatim}


\item tig paketi kurulumu sonrası:

tig comutunun tüm geçmişi listelediğini gözlemleyin.
\begin{verbatim}
git clone git://github.com/git/hello-world.git 
cd hello-world
tig
\end{verbatim}


\item mercurial paketi kurulumu sonrası:

Aşağıda bulunan komutların sorunsuz çalıştığını gözlemleyiniz.
\begin{verbatim}
 hg clone http://selenic.com/hg mercurial-repo
 cd mercurial-repo
 hg parents
\end{verbatim}


\item gitk paketi kurulumu sonrası:

gitk komutunun tüm geçmişi listelediğini gözlemleyin.
\begin{verbatim}
 git clone git://github.com/git/hello-world.git 
 cd hello-world
 gitk 
\end{verbatim}

\item git-daemon paketi kurulumu sonrası:

Servis yöneticisinden git\_daemon başlatın, aşağıdaki komut ile başlatıldığına emin olun:
\begin{verbatim}
 service git_daemon status
\end{verbatim}

\item git paketi kurulumu sonrası:

Aşağıdaki komutları çalıştırın. Ve sorunsuz bir şekilde Git deposu oluşturduğunu ve klonlandığını gözlemleyin.
\begin{verbatim}
  cd ~
  mkdir test_git
  cd test_git
  git init
  cd ..
  git clone test_git test_clone
\end{verbatim}

\item subversion paketi kurulumu sonrası:

test dizininin sorunsuz bir şekilde eklenmiş olduğunu gözlemleyin:
\begin{verbatim}
   svn co http://svn.pardus.org.tr/uludag/trunk/doc/test/2009/testguide/turkish/
   cd turkish
   svn mkdir test
   svn st
 \end{verbatim}
testfile dosyası içerisine bir kaç kelime yazın ve kaydedin. Yapılan değişiklik farkının alınabildiğini gözlemleyin:
\begin{verbatim}
   vi testfile
   svn add testfile
   svn diff
 \end{verbatim}
\end{enumerate}

\section{Environment alt bileşeni}
\begin{enumerate}

\item Aşağıda bulunan paketler sadece kurulum testine tabidir.

 \begin{verbatim}
  kdevelop-devel
 \end{verbatim}

\item eclipse-binary paketi kurulumu sonrası:

Kmenu'den eclipse uygulamasını açın sorunsuz bir şekilde açılabildiğini gözlemleyin.

\item eclipse-cdt-binary paketi kurulumu sonrası:

Kmenu'den eclipse uygulamasını açın ve C/C++ geliştirme araçlarının eklendiğini gözlemleyin. (İlk açılan geniş pencerede C/C++ Development yazmalı)

\item eclipse-jdt-binary paketi kurulumu sonrası:

Kmenu'den eclipse uygulamasını açın ve Java geliştirme araçlarının eklendiğini gözlemleyin. (İlk açılan geniş pencerede Java Development yazmalı)

\item eclipse-pde-binary paketi kurulumu sonrası:

Kmenu'den eclipse uygulamasını açın ve eklenti geliştirme araçlarının eklendiğini gözlemleyin. (İlk açılan geniş pencerede Eclipse plug-in development yazmalı)

\item qdevelop paketi kurulumu sonrası:

Kmenuden uygulamayı açın, açılabildiğini gözlemleyin

 \item kdevelop paketi kurulumu sonrası: 

Kmenuden uygulamayı açın açılabildiğini gözlemleyin.

 \item lazarus paketi kurulumu sonrası:

Kmenüden programın sorunsuz açıldığını gözlemleyin.

 \item eric paketi kurulumu sonrası:
 
Aşağıda bulunan dosyayı eric uygulaması ile açın ve Start $\rightarrow$ Run Script yolunu izleyerek çalıştırın. 

Sorunsuz bir şekilde çalıştığını gözlemleyin.
\begin{verbatim}
 # wget http://cekirdek.pardus.org.tr/~semen/dist/test/programming/environment/test.py
\end{verbatim}
 \item Aşağıda bulunan paketlerin kurulumu sonrasında, yerel dilinizi değiştirip, konsoldan aynı dizinde bir open office uygulaması açın ve yardım dosyasının ilgili dilde olduğunu gözlemleyin.
\begin{verbatim}
eric-i18n-cs
eric-i18n-de
eric-i18n-es
eric-i18n-fr
eric-i18n-ru
eric-i18n-tr
 \end{verbatim}

Yerel dili değiştirmek için:
\begin{verbatim}
export LC_ALL= <lang_LANG>
\end{verbatim}

lang\_LANG şeklinde yazılmış olan, pt-BT için pt\_BT, diğer diller için örneğin de\_DE olacaktır.

Daha sonra bu çalıştırdığınız komut dizininde eric4 komutunu çalıştırın, paket eğer help ile ilgili ise help dosyasının, uygulama dili ise uygulamanın sorunsuz bir şekilde istenilen dilde açıldığını gözlemleyin.

\item ipython-gui paketi kurulumu sonrası:

Aşağıda bulunan komutları çalıştırdığınızda ipython guilerinin sorunsuz bir şekilde açıldığını gözlemleyin.
\begin{verbatim}
 ipython-wx
 ipythonx
\end{verbatim}


\item ipython paketi kurulumu sonrası: 

Aşağıda bulunan komutları çalıştırdığınızda, bulunduğunuz dizinde test adında bir dosya oluştuğunu ve içerisinde "test ipython" yazdığını gözlemleyin:
\begin{verbatim}
 ipython
 a = open("test", "a")
 a.write("test ipython")
\end{verbatim}

\item drscheme paketi kurulumu sonrası:

Kmenüden uygulamayı açın ve sorunsuz bir şekilde çalıştığını gözlemleyin.

\item qt-creator paketi kurulumu sonrası:

Kmenüden uygulamayı açın ve sorunsuz bir şekilde açıldığını gözlemleyin.
\end{enumerate}


\section{Language alt bileşeni}

\subsection{---}
\item Aşağıda bulunan paketler sadece kurulum testine tabidir.

\begin{verbatim}
llvm
llvm-docs
llvm-ocaml
\end{verbatim}

\subsection{Ruby alt bileşeni}

\begin{enumerate}
 \item ruby paketi kurulumu sonrası:

\begin{verbatim}
irb --simple-prompt 
puts "Test Pardus"
\end{verbatim}

Yukarıda bulunan komutun aşağıda bulunan çıktıyı döndürdüğünü gözlemleyin.
\begin{verbatim}
Test Pardus
=> nil
\end{verbatim}

\item ruby-mode paketi kurulumu sonrası:

Emacs uygulamasını açın ALT+x tuşlarına basın ve ruby-mode yazın ENTER tuşuna basın, emacs'in bu mod değişkliğini yaptığını gözlemleyin.
\end{enumerate}

\subsection{tcl alt bileşeni}

\begin{enumerate}
 \item tcl, tcltk paketleri kurulumu sonrası:

Aşağıda bulunan komutun bir gui ekranı çıkardığını gözlemleyin.
\begin{verbatim}
 wget http://cekirdek.pardus.org.tr/~semen/dist/test/programming/language/test_tcl.tcl
 wish test_tcl.tcl
\end{verbatim}


\end{enumerate}

\subsection{Php alt bileşeni}
\begin{enumerate}

\item php-gtk paketi kurulumu sonrası:

Aşağıda bulunan komutun küçük bir pencere açtığını gözlemleyin.
\begin{verbatim}
 wget http://cekirdek.pardus.org.tr/~semen/dist/test/programming/language/php/test_phpgtk.phpw
 php test_phpgtk.phpw
\end{verbatim}

\item php-cli, php-common, mod\_php paketleri kurulumu sonrası:

Servis yöneticisinden apache servisini başlatın.

Aşağıda bulunan komutları çalıştırdıktan sonra http://localhost/test.php adresine firefox ile girin ve php ile ilgili bilgilerin sayfalandığını gözlemleyin.

\begin{verbatim}
 cd /var/www/localhost/htdocs/
 sudo wget http://cekirdek.pardus.org.tr/~semen/dist/test/programming/language/php/test.php 
\end{verbatim}

\end{enumerate}


\subsection{Perl alt bileşeni}
\begin{enumerate}
\item Aşağıda bulunan paketler sadece kurulum testine tabidir:
\begin{verbatim}
 perl-HTML-Template-Pro
 perl-JSON
 perl-JSON-XS
 perl-common-sense
 perlmod
 perl-Attribute-Handlers
 perl-DBD-Pg
 perl-Image-ExifTool
 perl-Image-ExifTool-docs
 perl-XML-SAX
 perl-AnyEvent
 perl-Authen-SASL
 perl-Crypt-Blowfish
 perl-Data-Dumper
 perl-Event-ExecFlow
 perl-FreezeThaw
 perl-HTML-Parser
 perl-MIME-Types
 perl-RPC-XML
 perl-Time-HiRes
 perl-URI
 perl-Git
 perl-Barcode-Code128
\end{verbatim}

\item perl-SDL ve perl-YAML paketleri kurulumu sonrası:

game-tr.pdf frozen-bubble testini gerçekleştiriniz.

\item perl-MailTools paketi kurulumu sonrası:
\begin{verbatim}
 wget http://cekirdek.pardus.org.tr/~semen/dist/test/programming/language/perl/MailTools.tar.gz
 perl "-MExtUtils::Command::MM" "-e" "test_harness(0,'blib/lib', 'blib/arch')" t/*.t
\end{verbatim}

\item perl-libwww paketi kurulumu sonrası:

Aşağıda bulunan dosyayı indirin ve açın.
\begin{verbatim}
wget http://cekirdek.pardus.org.tr/~semen/dist/test/programming/language/perl/libwww-perl.tar.gz
\end{verbatim}

Konsoldan;
\begin{verbatim}
 cd libwww-perl
 perl t/TEST
\end{verbatim}

Komutlarını çalıştırın ve testlerden "ok" sonuçlarının döndüğünü gözlemleyin.

\item perl-Yaml paketi kurulumu sonrası:

Aşağıda bulunan komutu çalıştırın ve sorunsuz bir şekilde çalıştığını gözlemleyin.
\begin{verbatim}
 wget http://cekirdek.pardus.org.tr/~semen/dist/test/programming/language/perl/perl-yaml-test.pm
 perl perl-yaml-test.pm
\end{verbatim}

\item perl-TimeDate paketi kurulumu sonrası:

Aşağıda bulunan dosyayı indirin ve açın.
\begin{verbatim}
wget http://cekirdek.pardus.org.tr/~semen/dist/test/programming/language/perl/TimeDate.tar.gz
\end{verbatim}

Konsoldan;
\begin{verbatim}
# cd TimeDate
# /usr/bin/perl5.10.0 "-MExtUtils::Command::MM" "-e" "test_harness(0,'blib/lib', 'blib/arch')" t/*.t
\end{verbatim}

Komutlarını çalıştırın ve testlerden "ok" sonuçlarının döndüğünü gözlemleyin.

\item perl-Storable paketi kurulumu sonrası:

Aşağıda bulunan dosyayı indirin ve açın.
\begin{verbatim}
wget http://cekirdek.pardus.org.tr/~semen/dist/test/programming/language/perl/Storable.tar.gz
\end{verbatim}

Konsoldan;
\begin{verbatim}
# cd St
\end{verbatim}

Komutlarını çalıştırın ve testlerden "ok" sonuçlarının döndüğünü gözlemleyin.

% \item perl-RPC-XML paketi kurulumu sonrası:
% 
% Aşağıda bulunan dosyayı indirin ve açın.
% \begin{verbatim}
% wget http://cekirdek.pardus.org.tr/~semen/dist/test/programming/language/perl/RPC-XML.tar.gz
% \end{verbatim}
% 
% Konsoldan;
% \begin{verbatim}
%  cd RPC-XML
%  perl "-MExtUtils::Command::MM" "-e" "test_harness(0,'blib/lib', 'blib/arch')" t/*.t
% \end{verbatim}
% 
% Komutlarını çalıştırın ve testlerden "ok" sonuçlarının döndüğünü gözlemleyin.

\item perl-IP-Country paketi kurulumu sonrası:

Aşağıda bulunan dosyayı indirin ve açın.
\begin{verbatim}
wget http://cekirdek.pardus.org.tr/~semen/dist/test/programming/language/perl/IP-Country.tar.gz
\end{verbatim}

Konsoldan;
\begin{verbatim}
cd IP-Country
perl "-MExtUtils::Command::MM" "-e" "test_harness(0,'blib/lib', 'blib/arch')" t/*.t
\end{verbatim}

Komutlarını çalıştırın ve testlerden "ok" sonuçlarının döndüğünü gözlemleyin.

% \item perl-HTML-Parser paketi kurulumu sonrası:
% 
% Aşağıda bulunan dosyayı indirin ve açın.
% \begin{verbatim}
% wget http://cekirdek.pardus.org.tr/~semen/dist/test/programming/language/perl/HTML-Parser.tar.gz
% \end{verbatim}
% 
% Konsoldan;
% \begin{verbatim}
% cd HTML-Parser
% perl "-MExtUtils::Command::MM" "-e" "test_harness(0,'blib/lib', 'blib/arch')" t/*.t
% \end{verbatim}
% 
% Komutlarını çalıştırın ve testlerden "ok" sonuçlarının döndüğünü gözlemleyin.

\item perl-Date-Calc paketi kurulumu sonrası:

Aşağıda bulunan dosyayı indirin ve açın.
\begin{verbatim}
wget http://cekirdek.pardus.org.tr/~semen/dist/test/programming/language/perl/Date-Calc.tar.gz
\end{verbatim}

Konsoldan;
\begin{verbatim}
 cd Date-Calc
 /usr/bin/perl5.10.0 "-MExtUtils::Command::MM" "-e" "test_harness(0,'blib/lib', 'blib/arch')" t/*.t
\end{verbatim}

Komutlarını çalıştırın ve testlerden "ok" sonuçlarının döndüğünü gözlemleyin.

\item perl-Class-ISA paketi kurulumu sonrası:

Aşağıda bulunan dosyayı indirin ve açın.
\begin{verbatim}
wget http://cekirdek.pardus.org.tr/~semen/dist/test/programming/language/perl/Class-ISA.tar.gz
\end{verbatim}

Konsoldan;
\begin{verbatim}
 cd Class-ISA
 /usr/bin/perl5.10.0 "-MExtUtils::Command::MM" "-e" "test_harness(0,'blib/lib', 'blib/arch')" t/*.t
\end{verbatim}

Komutlarını çalıştırın ve testlerden "ok" sonuçlarının döndüğünü gözlemleyin.

\item perl-Class-Accessor paketi kurulumu sonrası:

Aşağıda bulunan dosyayı indirin ve açın.
\begin{verbatim}
wget http://cekirdek.pardus.org.tr/~semen/dist/test/programming/language/perl/Class-Accessor.tar.gz
\end{verbatim}

Konsoldan;
\begin{verbatim}
# cd Class-Accessor
# /usr/bin/perl5.10.0 "-MExtUtils::Command::MM" "-e" "test_harness(0,'blib/lib', 'blib/arch')" t/*.t
\end{verbatim}

Komutlarını çalıştırın ve testlerden "ok" sonuçlarının döndüğünü gözlemleyin.


\item perl-IO-Socket-SSL paketi kurulumu sonrası:

Aşağıda bulunan dosyayı indirin ve açın.
\begin{verbatim}
 wget http://cekirdek.pardus.org.tr/~semen/dist/test/programming/language/perl/IO-Socket-SSL-1.26.tar.gz
\end{verbatim}

Konsoldan;
\begin{verbatim}
 cd IO-Socket-SSL-1.26/
 /usr/bin/perl5.10.0 "-MExtUtils::Command::MM" "-e" "test_harness(0,'blib/lib', 'blib/arch')" t/*.t
\end{verbatim}

Komutlarını çalıştırın ve testlerden "ok" sonuçlarının döndüğünü gözlemleyin.
\item perl-Compress-Zlib paketi kurulumu sonrası:

programming-tr.pdf git testini gerçekleştirin.

\item perl-Email-MIME-Encodings paketi kurulumu sonrası:
\begin{verbatim}
 wget http://cekirdek.pardus.org.tr/~semen/dist/test/programming/language/perl/Email-MIME-Encodings.t
 perl Email-MIME-Encodings.t
\end{verbatim}

Tüm sonuçların "ok" döndürdüğünü gözlemleyin. 

\item perl-Email-MIME-Encodings paketi kurulumu sonrası:
\begin{verbatim}
 wget http://cekirdek.pardus.org.tr/~semen/dist/test/programming/language/perl/test_perl_Test_Simple.t
 perl test_perl_Test_Simple.t
\end{verbatim}

Tüm sonuçların "ok" döndürdüğünü gözlemleyin. 


\end{enumerate}
\subsection{Python alt bileşeni}
\begin{enumerate}

\item Aşağıda bulunan paketler sadece kurulum testine tabidir.
\begin{verbatim}
 python-docs
 gnome-python-docs
 PyQt-doc
 python-addons
 python-bytecodeassembler
 python-extremes
 epydoc-doc
 reportlab-docs
 pyinotify-doc 
 python-e_dbus
 python-Genshi-doc
 python-EasyGui-docs
 python-pytables-docs
 python-mako-docs
\end{verbatim}

\item pyalsaaudio paketi kurulumu sonrası:

ipython paketini kurun ve aşağıda bulunan komutları çalıştırın:
\begin{verbatim}
 ipython
 import alsaaudio
\end{verbatim}

\item tagpy paketi kurulumu sonrası:

ipython paketini kurun ve aşağıda bulunan komutları çalıştırın:
\begin{verbatim}
 ipython
 import tagpy
\end{verbatim}

\item python-EasyGui paketi kurulumu sonrası:

ipython paketini kurun ve aşağıda bulunan komutları çalıştırın:
\begin{verbatim}
 ipython
 import easygui
\end{verbatim}

\item python-SilverCity paketi kurulumu sonrası:

ipython paketini kurun ve aşağıda bulunan komutları çalıştırın:
\begin{verbatim}
 ipython
 import SilverCity
\end{verbatim}

\item python-biggles paketi kurulumu sonrası:

ipython paketini kurun ve aşağıda bulunan komutları çalıştırın:
\begin{verbatim}
 ipython
 import biggles
\end{verbatim}

\item python-pp paketi kurulumu sonrası:

ipython paketini kurun ve aşağıda bulunan komutları çalıştırın:
\begin{verbatim}
 ipython
 import pp
\end{verbatim}

\item python-pyMbus paketi kurulumu sonrası:

ipython paketini kurun ve aşağıda bulunan komutları çalıştırın:
\begin{verbatim}
 ipython
 import mbus
\end{verbatim}

\item python-cssutils paketi kurulumu sonrası:

ipython paketini kurun ve aşağıda bulunan komutları çalıştırın:
\begin{verbatim}
 ipython
 import cssutils
\end{verbatim}

\item python-iniparse paketi kurulumu sonrası:

ipython paketini kurun ve aşağıda bulunan komutları çalıştırın:
\begin{verbatim}
 ipython
 import iniparse
\end{verbatim}

\item python-mailer paketi kurulumu sonrası:

ipython paketini kurun ve aşağıda bulunan komutları çalıştırın:
\begin{verbatim}
 ipython
 import mailer
\end{verbatim}

\item python-numexpr paketi kurulumu sonrası:

ipython paketini kurun ve aşağıda bulunan komutları çalıştırın:
\begin{verbatim}
 ipython
 import numexpr
\end{verbatim}

\item python-pytables paketi kurulumu sonrası:

ipython paketini kurun ve aşağıda bulunan komutları çalıştırın:
\begin{verbatim}
 ipython
 import tables
\end{verbatim}

\item python-eventlet paketi kurulumu sonrası:

ipython paketini kurun ve aşağıda bulunan komutları çalıştırın:
\begin{verbatim}
 ipython
 import eventlet
\end{verbatim}

\item python-greenlet paketi kurulumu sonrası:

ipython paketini kurun ve aşağıda bulunan komutları çalıştırın:
\begin{verbatim}
 ipython
 import greenlet
\end{verbatim}

\item python-MarkupSafe paketi kurulumu sonrası:

ipython paketini kurun ve aşağıda bulunan komutları çalıştırın:
\begin{verbatim}
 ipython
 import markupsafe
\end{verbatim}

\item python-Unidecode paketi kurulumu sonrası:

ipython paketini kurun ve aşağıda bulunan komutları çalıştırın:
\begin{verbatim}
 ipython
 import unidecode
\end{verbatim}

\item python-decoratortools paketi kurulumu sonrası:

ipython paketini kurun ve aşağıda bulunan komutları çalıştırın:
\begin{verbatim}
 ipython
 import peak.util.decorators
\end{verbatim}

\item python-paste paketi kurulumu sonrası:

ipython paketini kurun ve aşağıda bulunan komutları çalıştırın:
\begin{verbatim}
 ipython
 import paste
\end{verbatim}

\item python-turbokid paketi kurulumu sonrası:

ipython paketini kurun ve aşağıda bulunan komutları çalıştırın:
\begin{verbatim}
 ipython
 import turbokid
\end{verbatim}

\item python-webtest paketi kurulumu sonrası:

ipython paketini kurun ve aşağıda bulunan komutları çalıştırın:
\begin{verbatim}
 ipython
 import webtest
\end{verbatim}


\item python-opentts paketi kurulumu sonrası:

ipython paketini kurun ve aşağıda bulunan komutları çalıştırın:
\begin{verbatim}
 ipython
 import opentts
\end{verbatim}

\item python-speech-dispatcher paketi kurulumu sonrası:

ipython paketini kurun ve aşağıda bulunan komutları çalıştırın:
\begin{verbatim}
 ipython
 import speechd
\end{verbatim}

\item python-pyatspi paketi kurulumu sonrası:

ipython paketini kurun ve aşağıda bulunan komutları çalıştırın:
\begin{verbatim}
 ipython
 import pyatspi
\end{verbatim}

\item python-v4l2capture paketi kurulumu sonrası:

ipython paketini kurun ve aşağıda bulunan komutları çalıştırın:
\begin{verbatim}
 ipython
 import v4l2capture
\end{verbatim}

\item PyKDE paketi kurulumu sonrası:

ipython paketini kurun ve aşağıda bulunan komutları çalıştırın:
\begin{verbatim}
 ipython
 import PyKDE4
\end{verbatim}

\item python-pyliblzma paketi kurulumu sonrası:

ipython paketini kurun ve aşağıda bulunan komutları çalıştırın:
\begin{verbatim}
 ipython
 import lzma
\end{verbatim}

\item python-foolscap paketi kurulumu sonrası:

ipython paketini kurun ve aşağıda bulunan komutları çalıştırın:
\begin{verbatim}
 ipython
 import foolscap
\end{verbatim}

\item python-Genshi paketi kurulumu sonrası:

ipython paketini kurun ve aşağıda bulunan komutları çalıştırın:
\begin{verbatim}
 ipython
 import genshi
\end{verbatim}

\item dnspython paketi kurulumu sonrası:

ipython paketini kurun ve aşağıda bulunan komutları çalıştırın:
\begin{verbatim}
 ipython
 import dns
\end{verbatim}
\item pyPdf paketi kurulumu sonrası:

ipython paketini kurun ve aşağıda bulunan komutları çalıştırın:
\begin{verbatim}
 ipython
 import pyPdf
\end{verbatim}

\item python-evas paketi kurulumu sonrası:

ipython paketini kurun ve aşağıda bulunan komutları çalıştırın:
\begin{verbatim}
 ipython
 import evas
\end{verbatim}

\item python-ethumb paketi kurulumu sonrası:

ipython paketini kurun ve aşağıda bulunan komutları çalıştırın:
\begin{verbatim}
 ipython
 import ethumb
\end{verbatim}

\item python-elementary paketi kurulumu sonrası:

ipython paketini kurun ve aşağıda bulunan komutları çalıştırın:
\begin{verbatim}
 ipython
 import elementary
\end{verbatim}

\item python-edje paketi kurulumu sonrası:

ipython paketini kurun ve aşağıda bulunan komutları çalıştırın:
\begin{verbatim}
 ipython
 import edje
\end{verbatim}



\item python-ecore paketi kurulumu sonrası:

ipython paketini kurun ve aşağıda bulunan komutları çalıştırın:
\begin{verbatim}
 ipython
 import ecore
\end{verbatim}


\item urwid paketi kurulumu sonrası:

ipython paketini kurun ve aşağıda bulunan komutları çalıştırın:
\begin{verbatim}
 ipython
 import urwid
\end{verbatim}


\item Pygments paketi kurulumu sonrası:
ipython paketini kurun ve aşağıda bulunan komutları çalıştırın:
\begin{verbatim}
 ipython
 import pygments
\end{verbatim}

\item chardet paketi kurulumu sonrası:
ipython paketini kurun ve aşağıda bulunan komutları çalıştırın:
\begin{verbatim}
 ipython
 import chardet
\end{verbatim}

\item matplotlib paketi kurulumu sonrası:
ipython paketini kurun ve aşağıda bulunan komutları çalıştırın:
\begin{verbatim}
 ipython
 import matplotlib
\end{verbatim}

\item paramiko paketi kurulumu sonrası:
ipython paketini kurun ve aşağıda bulunan komutları çalıştırın:
\begin{verbatim}
 ipython
 import paramiko
\end{verbatim}

\item pydns paketi kurulumu sonrası:
ipython paketini kurun ve aşağıda bulunan komutları çalıştırın:
\begin{verbatim}
 ipython
 import DNS
\end{verbatim}


\item python-reportlab paketi kurulumu sonrası:
ipython paketini kurun ve aşağıda bulunan komutları çalıştırın:
\begin{verbatim}
 ipython
 import reportlab
\end{verbatim}

\item pyinotify paketi kurulumu sonrası:
ipython paketini kurun ve aşağıda bulunan komutları çalıştırın:
\begin{verbatim}
 ipython
 import pyinotify
\end{verbatim}

\item pyserial  paketi kurulumu sonrası:
ipython paketini kurun ve aşağıda bulunan komutları çalıştırın:
\begin{verbatim}
 ipython
 import serial
\end{verbatim}

\item python-beautifulsoup  paketi kurulumu sonrası:
ipython paketini kurun ve aşağıda bulunan komutları çalıştırın:
\begin{verbatim}
 ipython
 import BeautifulSoup
\end{verbatim}

\item python-bsddb3  paketi kurulumu sonrası:
ipython paketini kurun ve aşağıda bulunan komutları çalıştırın:
\begin{verbatim}
 ipython
 import bsddb3
\end{verbatim}

\item python-pep8  paketi kurulumu sonrası:
ipython paketini kurun ve aşağıda bulunan komutları çalıştırın:
\begin{verbatim}
 ipython
 import pep8
\end{verbatim}

\item pylint  paketi kurulumu sonrası:
ipython paketini kurun ve aşağıda bulunan komutları çalıştırın:
\begin{verbatim}
 ipython
 import pylint
\end{verbatim}

\item pysvn  paketi kurulumu sonrası:
ipython paketini kurun ve aşağıda bulunan komutları çalıştırın:
\begin{verbatim}
 ipython
 import pysvn
\end{verbatim}

\item python-logilab-astng ve python-logilab-common  paketleri kurulumu sonrası:
ipython paketini kurun ve aşağıda bulunan komutları çalıştırın:
\begin{verbatim}
 ipython
 import ast
 import logilab
\end{verbatim}


\item pytextile  paketi kurulumu sonrası:
ipython paketini kurun ve aşağıda bulunan komutları çalıştırın:
\begin{verbatim}
 ipython
 import textile
\end{verbatim}

\item Turbogears  paketi kurulumu sonrası:
ipython paketini kurun ve aşağıda bulunan komutları çalıştırın:
\begin{verbatim}
 ipython
 import turbogears
\end{verbatim}

\item TurboGears-mochikit  paketi kurulumu sonrası:
ipython paketini kurun ve aşağıda bulunan komutları çalıştırın:
\begin{verbatim}
 ipython
 import tgmochikit
\end{verbatim}


\item python-wordaxe paketi kurulumu sonrası:
ipython paketini kurun ve aşağıda bulunan komutları çalıştırın:
\begin{verbatim}
 ipython
 import wordaxe
\end{verbatim}


\item sqlalchemy paketi kurulumu sonrası:
ipython paketini kurun ve aşağıda bulunan komutları çalıştırın:
\begin{verbatim}
 ipython
 import sqlalchemy
\end{verbatim}


\item python-bsddb3 paketi kurulumu sonrası:
ipython paketini kurun ve aşağıda bulunan komutları çalıştırın:
\begin{verbatim}
 ipython
 import bsddb3
\end{verbatim}

\item python-webob paketi kurulumu sonrası:
ipython paketini kurun ve aşağıda bulunan komutları çalıştırın:
\begin{verbatim}
 ipython
 import webob
\end{verbatim}

\item python-routes paketi kurulumu sonrası:
ipython paketini kurun ve aşağıda bulunan komutları çalıştırın:
\begin{verbatim}
 ipython
 import routes
\end{verbatim}


\item python-prioritized-methods paketi kurulumu sonrası:
ipython paketini kurun ve aşağıda bulunan komutları çalıştırın:
\begin{verbatim}
 ipython
 import prioritized-methods
\end{verbatim}


\item python-beaker paketi kurulumu sonrası:
ipython paketini kurun ve aşağıda bulunan komutları çalıştırın:
\begin{verbatim}
 ipython
 import beaker
\end{verbatim}

\item Twisted paketi kurulumu sonrası:
ipython paketini kurun ve aşağıda bulunan komutları çalıştırın:
\begin{verbatim}
 ipython
 import twisted
\end{verbatim}

\item python-mlt paketi kurulumu sonrası:
ipython paketini kurun ve aşağıda bulunan komutları çalıştırın:
\begin{verbatim}
 ipython
 import mlt
\end{verbatim}


\item python-Babel paketi kurulumu sonrası:
ipython paketini kurun ve aşağıda bulunan komutları çalıştırın:
\begin{verbatim}
 ipython
 import babel
\end{verbatim}

\item pyfits paketi kurulumu sonrası:
ipython paketini kurun ve aşağıda bulunan komutları çalıştırın:
\begin{verbatim}
 ipython
 import pyfits
\end{verbatim}


\item epydoc paketi kurulumu sonrası:
ipython paketini kurun ve aşağıda bulunan komutları çalıştırın:
\begin{verbatim}
 ipython
 import epydoc
\end{verbatim}

\item python-peak-rules paketi kurulumu sonrası:

ipython paketini kurun ve aşağıda bulunan komutları çalıştırın:
\begin{verbatim}
 ipython
 import peak
\end{verbatim}

\item python-symboltype paketi kurulumu sonrası:

ipython paketini kurun ve aşağıda bulunan komutları çalıştırın:
\begin{verbatim}
 ipython
 import symbol
\end{verbatim}

\item 4suite paketi kurulumu sonrası:
ipython paketini kurun ve aşağıda bulunan komutları çalıştırın:
\begin{verbatim}
 ipython
 import Ft
\end{verbatim}


\item setuptools paketi kurulumu sonrası:
\begin{verbatim}
 svn co http://svn.pardus.org.tr/uludag/trunk/kde/boot-manager/manager/
 cd manager
 sudo python setup.py build 
 sudo python setup.py install
\end{verbatim}

\item django-tagging paketi kurulumu sonrası:
\begin{verbatim}
 ipython
 import tagging
\end{verbatim}

\item django-south paketi kurulumu sonrası:
\begin{verbatim}
 ipython
 import south
\end{verbatim}


\item pycha paketi kurulumu sonrası:
ipython paketini kurun ve aşağıda bulunan komutları çalıştırın:
\begin{verbatim}
 ipython
 import pycha
 import chavier
\end{verbatim}

\item python-xlrd paketi kurulumu sonrası:
ipython paketini kurun ve aşağıda bulunan komutları çalıştırın:
\begin{verbatim}
 ipython
 import xlrd
\end{verbatim}

\item pytz paketi kurulumu sonrası:
ipython paketini kurun ve aşağıda bulunan komutları çalıştırın:
\begin{verbatim}
 ipython
 import pytz
\end{verbatim}

\item python-Jinja2 paketi kurulumu sonrası:
ipython paketini kurun ve aşağıda bulunan komutları çalıştırın:
\begin{verbatim}
 ipython
 import jinja2
\end{verbatim}

\item python-sphinx paketi kurulumu sonrası:
ipython paketini kurun ve aşağıda bulunan komutları çalıştırın:
\begin{verbatim}
 ipython
 import sphinx
\end{verbatim}


\item mysql-python paketi kurulumu sonrası:
ipython paketini kurun ve aşağıda bulunan komutları çalıştırın:
\begin{verbatim}
 ipython
 import MySQLdb
\end{verbatim}

\item python-markdown paketi kurulumu sonrası:
ipython paketini kurun ve aşağıda bulunan komutları çalıştırın:
\begin{verbatim}
 ipython
 import markdown
\end{verbatim}

\item python-preprocess paketi kurulumu sonrası:
ipython paketini kurun ve aşağıda bulunan komutları çalıştırın:
\begin{verbatim}
 ipython
 import preprocess
\end{verbatim}

\item python-avahi paketi kurulumu sonrası:
ipython paketini kurun ve aşağıda bulunan komutları çalıştırın:
\begin{verbatim}
 ipython
 import avahi
\end{verbatim}

\item python-tornado paketi kurulumu sonrası:
ipython paketini kurun ve aşağıda bulunan komutları çalıştırın:
\begin{verbatim}
 ipython
 import tornado
\end{verbatim}

\item pyorbit paketi kurulumu sonrası:
ipython paketini kurun ve aşağıda bulunan komutları çalıştırın:
\begin{verbatim}
 ipython
 import ORBit
\end{verbatim}

\item gnome-python paketi kurulumu sonrası:
ipython paketini kurun ve aşağıda bulunan komutları çalıştırın:
\begin{verbatim}
 ipython
 import bonobo
\end{verbatim}


\item PyXML paketi kurulumu sonrası:

ipython paketini kurun ve aşağıda bulunan komutları çalıştırın:
\begin{verbatim}
 ipython
 import xml
\end{verbatim}

\item sip paketi kurulumu sonrası:

ipython paketini kurun ve aşağıda bulunan komutları çalıştırın:
\begin{verbatim}
 ipython
 import sip
\end{verbatim}
\item pysqlite paketi kurulumu sonrası:

ipython paketini kurun ve aşağıda bulunan komutları çalıştırın:
\begin{verbatim}
 ipython
 import sqlite3
\end{verbatim}


\item PyQt paketi kurulumu sonrası:

Aşağıdaki komutları çalıştırın ve bir pencerenin açıldığını gözlemleyin.
\begin{verbatim}
 wget http://cekirdek.pardus.org.tr/~semen/dist/test/programming/language/python/pyqt-test.py
 python pyqt-test.py
\end{verbatim}


\item sip paketi kurulumu sonrası:

ipython paketini kurun ve aşağıda bulunan komutları çalıştırın:
\begin{verbatim}
 ipython
 import sip
\end{verbatim}

\item python-configobj paketi kurulumu sonrası:

ipython paketini kurun ve aşağıda bulunan komutları çalıştırın:
\begin{verbatim}
 ipython
 import configobj
\end{verbatim}
\item python-decorator paketi kurulumu sonrası:

ipython paketini kurun ve aşağıda bulunan komutları çalıştırın:
\begin{verbatim}
 ipython
 import decorator
\end{verbatim}
\item python-mako paketi kurulumu sonrası:

ipython paketini kurun ve aşağıda bulunan komutları çalıştırın:
\begin{verbatim}
 ipython
 import mako
\end{verbatim}

\item python-FormEncode paketi kurulumu sonrası:

ipython paketini kurun ve aşağıda bulunan komutları çalıştırın:
\begin{verbatim}
 ipython
 import formencode
\end{verbatim}

\item cheetah  paketi kurulumu sonrası:

ipython paketini kurun ve aşağıda bulunan komutları çalıştırın:
\begin{verbatim}
 ipython
 import Cheetah
\end{verbatim}


\item SQLObject  paketi kurulumu sonrası:

ipython paketini kurun ve aşağıda bulunan komutları çalıştırın:
\begin{verbatim}
 ipython
 import sqlobject
\end{verbatim}

\item PyOpenGL  paketi kurulumu sonrası:

ipython paketini kurun ve aşağıda bulunan komutları çalıştırın:
\begin{verbatim}
 ipython
 import OpenGL
\end{verbatim}

\item imdbpy  paketi kurulumu sonrası:

ipython paketini kurun ve aşağıda bulunan komutları çalıştırın:
\begin{verbatim}
 ipython
 import imdb
\end{verbatim}

\item mutagen  paketi kurulumu sonrası:

Aşağıda bulunan komutun sorunsuz bir şekilde müzik dosyası etiklerini listelediğini gözlemleyin.
\begin{verbatim}
wget http://cekirdek.pardus.org.tr/~semen/dist/test/multimedia/sound/sound/sample.mp3 
mutagen-inspect sample.mp3
\end{verbatim}


\item python-libcap-ng  paketi kurulumu sonrası:

ipython paketini kurun ve aşağıda bulunan komutları çalıştırın:
\begin{verbatim}
 ipython
 import capng
\end{verbatim}

\item python-twitter paketi kurulumu sonrası:

ipython paketini kurun ve aşağıda bulunan komutları çalıştırın:
\begin{verbatim}
 ipython
 import twitter
\end{verbatim}


\item python-cx\_freeze paketi kurulumu sonrası:

ipython paketini kurun ve aşağıda bulunan komutları çalıştırın:
\begin{verbatim}
 ipython
 import cx_Freeze
\end{verbatim}

Aşağıda bulunan komutları çalıştırın ve küçük bir pencerenin açıldığını gözlemleyin.
\begin{verbatim}
 cd /usr/lib/python2.6/site-packages/cx_Freeze/samples/wx
 sudo python setup.py build
 sudo python setup.py install
 python wxapp.py
\end{verbatim}


\item pybluez paketi kurulumu sonrası:

ipython paketini kurun ve aşağıda bulunan komutları çalıştırın:
\begin{verbatim}
 ipython
 import bluetooth
\end{verbatim}

Eğer bluetooth aygıtınız var ise aşağıda bulunan komutları çalıştırın ve hatasız bir şekilde çalıştığını gözlemleyiniz.
\begin{verbatim}
  cd /usr/share/doc/pybluez/examples/advanced
  python inquiry-with-rssi.py
\end{verbatim}



\item pylirc paketi kurulumu sonrası:

ipython paketini kurun ve aşağıda bulunan komutları çalıştırın:
\begin{verbatim}
 ipython
 import pylirc
\end{verbatim}

Eğer infrared bir kumandanız var ise bu kumandayı aktive ettikten sonra aşağıda bulunan testi çalıştırınız ve hatasız bir şekilde çalıştığını gözlemleyiniz.
\begin{verbatim}
  wget http://cekirdek.pardus.org.tr/~semen/dist/test/programming/language/python/pylirc_test.py
  python pylirc_test.py
\end{verbatim}


\item pygobject ve pygobject-docs paketi kurulumu sonrası:

ipython paketini kurun ve aşağıda bulunan komutları çalıştırın:
\begin{verbatim}
 ipython
 import codegen
 import gio
 import glib
\end{verbatim}
\item pygame ve pygame-doc paketi kurulumu sonrası:

ipython paketini kurun ve aşağıda bulunan komutları çalıştırın:
\begin{verbatim}
 ipython
 import pygame
\end{verbatim}

\item gst-python paketi kurulumu sonrası:
ipython paketini kurun ve aşağıda bulunan komutları çalıştırın:
\begin{verbatim}
 ipython
 import pygst
\end{verbatim}

\item pyFltk paketi kurulumu sonrası:

ipython paketini kurun ve aşağıda bulunan komutları çalıştırın:
\begin{verbatim}
 ipython
 import fltk
\end{verbatim}

\item python-turboflot paketi kurulumu sonrası:

ipython paketini kurun ve aşağıda bulunan komutları çalıştırın:
\begin{verbatim}
 ipython
 import turboflot
\end{verbatim}

\item python-ldap paketi kurulumu sonrası:

ipython paketini kurun ve aşağıda bulunan komutları çalıştırın:
\begin{verbatim}
 ipython
 import ldap
\end{verbatim}

\item python-iptables paketi kurulumu sonrası:

ipython paketini kurun ve aşağıda bulunan komutları çalıştırın:
\begin{verbatim}
 ipython
 import iptables
\end{verbatim}

\item numpy paketi kurulumu sonrası:

Aşağıda bulunan komutu çalıştırdığınızda testlerden ok sonucu döndüğünü gözlemleyin.
\begin{verbatim}
 python -c "import numpy;print numpy.test()"
\end{verbatim}

\item sympy paketi kurulumu sonrası:

Aşağıda bulunan komutu çalıştırdığınızda testlerden ok sonucu döndüğünü gözlemleyin.
\begin{verbatim}
 # python -c "import sympy;print sympy.test()"
\end{verbatim}

\item scipy paketi kurulumu sonrası:

Aşğıda bulunan komutu çalıştırdığınızda <nose.result.TextTestResult run=XXXX errors=0 failures=0> gibi bir çıktı ürettiğini gözlemleyin.
\begin{verbatim}
 python -c "import scipy;print scipy.test()"
\end{verbatim}

\item PyXML paketi kurulumu sonrası:

ipython paketini kurun ve aşağıda bulunan komutları çalıştırın:
\begin{verbatim}
 ipython
 import xml
\end{verbatim}


\item PyX paketi kurulumu sonrası:

ipython paketini kurun ve aşağıda bulunan komutları çalıştırın:
\begin{verbatim}
 ipython
 import pyx
\end{verbatim}

\item notify-python paketi kurulumu sonrası:

ipython paketini kurun ve aşağıda bulunan komutları çalıştırın:
\begin{verbatim}
 ipython
 import pynotify
\end{verbatim}


\item httplib2 paketi kurulumu sonrası:
ipython paketini kurun ve aşağıda bulunan komutları çalıştırın:
\begin{verbatim}
 ipython
 import httplib2
\end{verbatim}

 \item Django paketi kurulumu sonrası:
\begin{itemize}
 \item Aşağıda bulunan komutu çalıştırın:
 \begin{verbatim}
  django-admin.py startproject test
  cd test
 \end{verbatim}
 test adında bir dizinin oluştuğunu ve bu dizin altında aşağıda bulunan dosyaların oluştuğunu gözlemleyin.
  \begin{verbatim}
  __init__.py
  manage.py
  settings.py
  urls.py 
  \end{verbatim}
 \item Aşağıdaki komutu çalıştırın ve daha sonra firefox'tan http://localhost:8080/ adresini girin ve sunucuya bağlanabildiğinizi gözlemleyin
  \begin{verbatim}
   python manage.py runserver 8080
  \end{verbatim}
 \item settings.py içerisinde DATABASE\_ENGINE DATABASE\_NAME değişkenlerini aşağıdaki database değişkenlerini atayın:
  \begin{verbatim}
   DATABASE_ENGINE = 'sqlite3'
   DATABASE_NAME = 'sqlite3_'   
  \end{verbatim}
\item Aşağıdaki komutu çalıştırın ve istemiş olduğu işlemleri sırasıyla gerçekleştirin ve sorunsuz bir şekilde Django onay sisteminin kurulduğunu gözlemleyin:
\begin{verbatim}
python manage.py syncdb 
\end{verbatim}
\item Aşağıdaki komutu çalıştırın ve polls adında bir dizinin oluştuğunu gözlemleyin:
\begin{verbatim}
 python manage.py startapp polls 
\end{verbatim}
\item Ve dizin içerişinde aşağıdaki dosyaların oluştuğunu gözlemleyin:
\begin{verbatim}
 __init__.py
 models.py
 views.py
\end{verbatim}

\end{itemize}

\item python-memcached paketi kurulumu sonrası:

memcached sunucusunu servis yöneticisinden başlatın.

Aşağıdaki komutları çalıştırın ve sonucun "True" döndürdüğünü gözlemleyin.
\begin{verbatim}
 wget http://cekirdek.pardus.org.tr/~semen/dist/test/programming/language/
python/test_python-memcache.py
 python test_python-memcache.py
\end{verbatim}

\item pygtk, pygtk-docs ve pygtk-demo paketi kurulumu sonrası: 

Aşağıda bulunan komutu çalıştırın ve çıkan listeden bir uygulama seçip çift tıklayın ve sorunsuz bir şekilde çalıştığını gözlemleyin.
\begin{verbatim}
 pygtk-demo
\end{verbatim}


\item mpmath paketi kurulumu sonrası:  

ipython paketini kurun ve aşağıda bulunan komutları çalıştırın:
\begin{verbatim}
 # ipython
 import mpmath
\end{verbatim}

\item mpmath paketi kurulumu sonrası:  

ipython paketini kurun ve aşağıda bulunan komutları çalıştırın:
\begin{verbatim}
 # ipython
 import mpmath
\end{verbatim}

\item python-M2Crypto paketi kurulumu sonrası:

ipython paketini kurun ve aşağıda bulunan komutları çalıştırın:
\begin{verbatim}
 # ipython
 import M2Crypto
\end{verbatim}

\item winpdb paketi kurulumu sonrası:

ipython paketini kurun ve aşağıda bulunan komutları çalıştırın:
\begin{verbatim}
 # ipython
 import winpdb
\end{verbatim}
(DeprecationWarning önemli değildir.)

\item cython paketi kurulumu sonrası:

ipython paketini kurun ve aşağıda bulunan komutları çalıştırın:
\begin{verbatim}
 ipython
 import cython
\end{verbatim}

\item lxml paketi kurulumu sonrası:  

ipython paketini kurun ve aşağıda bulunan komutları çalıştırın:
\begin{verbatim}
 ipython
 import lxm
\end{verbatim}
\item python-RuleDispatch paketi kurulumu sonrası:  

ipython paketini kurun ve aşağıda bulunan komutları çalıştırın:
\begin{verbatim}
 ipython
 import dispatch
\end{verbatim}

\item python-nose paketi kurulumu sonrası:  

ipython paketini kurun ve aşağıda bulunan komutları çalıştırın:
\begin{verbatim}
 ipython
 import nose
\end{verbatim}

\item PyICU paketi kurulumu sonrası:  

ipython paketini kurun ve aşağıda bulunan komutları çalıştırın:
\begin{verbatim}
 ipython
 import PyICU
\end{verbatim}

\item python-simplejson paketi kurulumu sonrası:  

ipython paketini kurun ve aşağıda bulunan komutları çalıştırın:
\begin{verbatim}
 ipython
 import simplejson
\end{verbatim}
\item python-openid paketi kurulumu sonrası:  

ipython paketini kurun ve aşağıda bulunan komutları çalıştırın:
\begin{verbatim}
 ipython
 import openid
\end{verbatim}


\end{enumerate}

\subsection{Java alt bileşeni}
\begin{enumerate}
 \item Aşağıdaki paketlerin kurulumu sonrası:
\begin{verbatim}
 sun-jre
 sun-jdk
 sun-jdk-demo
 sun-jdk-samples
 sun-jdk-doc
\end{verbatim}

Oturumu kapatıp tekrar açın.

Aşağıda bulunan komutların düzgün bir şekilde çalıştığını gözlemleyin.
\begin{verbatim}
 java -version
 wget http://cekirdek.pardus.org.tr/~semen/dist/test/programming/language/java/test.java
 javac test.java
 java test
\end{verbatim}

\item libmatthew, slf4j paketleri kurulumu sonrası:

office-tr.pdf zemberek-openoffice testini gerçekleştirin.
\end{enumerate}



\subsection{Lisp alt bileşeni}
\begin{enumerate}
 \item clisp paketi kurulumu sonrası: (Warningleri önemsemeyiniz.)

Aşağıdaki komutların çalıştırın ve hata olmadığını gözlemleyin.
\begin{verbatim}
# wget http://cekirdek.pardus.org.tr/~semen/dist/test/programming/language/lisp/test_clisp.lisp 
# clisp -c test_clisp.lisp
\end{verbatim}

\end{enumerate}
\subsection{Dotnet alt bileşeni}
\begin{enumerate}

\item Aşağıda bulunan paketler sadece kurulum testine tabidir
\begin{verbatim}
taglib-sharp
ndesk-dbus
ndesk-dbus-glib
mono-addins
mono-bytefx-data-mysql
mono-data
mono-data-firebird
mono-data-oracle
mono-data-postgresql
mono-data-sqlite
mono-devel
mono-extras
mono-ibm-data-db2
mono-jay
mono-jscript
mono-locale-extras
mono-nunit
mono-runtime
mono-sybase
mono-wcf
mono-web
mono-winforms
mono-winfxcore
mono-zeroconf
mono-zeroconf-docs
monodoc-core
webkit-sharp
webkit-sharp-docs
gecko-sharp
gecko-sharp-docs
\end{verbatim}

\item boo paketi kurulumu sonrası:

Aşağıda bulunan komutların sorunsuz çalıştığını gözlemleyin.
\begin{verbatim}
 wget http://cekirdek.pardus.org.tr/~semen/dist/test/programming/language/dotnet/boo-test.boo
 booi boo-test.boo
\end{verbatim}
 \item gmime-sharp, gtk-sharp paketi kurulumu sonrası:

pkgconfig paketini kurun

Aşağıda bulunan komutun küçük bir pencere açtığını gözlemleyin. 
\begin{verbatim}
 wget http://cekirdek.pardus.org.tr/~semen/dist/test/programming/language/dotnet/test_gmimesharp.cs
 mcs -pkg:gtk-sharp-2.0 test_gmimesharp.cs
 mono test_gmimesharp.exe
\end{verbatim}


 \item gmime, gmime-docs paketleri kurulumu sonrası:

Aşağıda bulunan komutun jpeg dosyasını encode ettiğini gözlemleyin.
\begin{verbatim}
 wget http://cekirdek.pardus.org.tr/~semen/dist/test/multimedia/graphics/test_dcraw.jpg
 gmime-uuencode -m test_dcraw.jpg jpeg
\end{verbatim}


 \item mono paketi kurulumu sonrası:
Aşağıdaki komutların çalıştırın ve hata olmadığını gözlemleyin.
\begin{verbatim}
 wget http://cekirdek.pardus.org.tr/~semen/dist/test/programming/language/dotnet/test_mono.cs
 mcs test_mono.cs
 mono test_mono.exe
\end{verbatim}

\item libgdiplus paketi kurulumu sonrası:

programming-tr.pdf mono testini gerçekleştiriniz.

\end{enumerate}


\begin{itemize}
\item ocaml paketi kurulumu sonrası:

Aşağıda bulunan komutun sorunsuz bir şekilde çalıştırğını gözlemleyin.
\begin{verbatim}
wget http://cekirdek.pardus.org.tr/~semen/dist/test/programming/language/ocaml-test.ml 
ocaml ocaml-test.ml
\end{verbatim}

 \item R paketi kurulumu sonrası:

Aşağıda bulunan komutları çalıştırın ve bir grafiğin oluştuğunu gözlemleyin.
\begin{verbatim}
 wget http://cekirdek.pardus.org.tr/~semen/dist/test/programming/language/test_R.R
 R --vanilla --slave < test_R.R
\end{verbatim}
\item R-mathlib paketi kurulumu sonrası:

Aşağıda bulunan komutları çalıştırın ve sorunsuz bir şekilde çalıştıklarını gözlemleyin. (gcc paketini öncelikle kurmanız gerekmektedir.)	
\begin{verbatim}
 wget http://cekirdek.pardus.org.tr/~semen/dist/test/programming/language/test-r-mathlib.c
 gcc -o test-r-matlib test-r-matlib.c -lm -lRmath
\end{verbatim}
\end{itemize}


\end{document}

