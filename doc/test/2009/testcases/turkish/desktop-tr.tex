\documentclass[a4paper,10pt]{article}
\usepackage[turkish]{babel}
\usepackage[utf8]{inputenc}
\usepackage[left=1cm,top=1cm,right=2cm,bottom=2cm]{geometry}

\title{Masaüstü Bileşeni Test Aşamaları}
\author{Semen Cirit}

\renewcommand{\labelenumi}{\arabic{enumi}.}
\renewcommand{\labelenumii}{\arabic{enumi}.\arabic{enumii}.}
\renewcommand{\labelenumiii}{\arabic{enumi}.\arabic{enumii}.\arabic{enumiii}.}
\renewcommand{\labelenumiv}{\arabic{enumi}.\arabic{enumii}.\arabic{enumiii}.\arabic{enumiv}.}

\begin{document}

\maketitle
\section{Gnome alt Bileşeni}
\begin{itemize}
 \item Aşağıda bulunan paketler sadece kurulum testine tabidir.
\begin{verbatim}
 libgnomecanvasmm
 gnome-keyring
 gnome-keyring-docs
 gnome-doc-utils
\end{verbatim}

\end{itemize}

\section{Look and Feel alt Bileşeni}
\begin{itemize}
\item Aşağıda bulunan paketler sadece kurulum testine tabidir.
\begin{verbatim}
 icon-theme-oxygen-svg-4.2.4-39-6.pisi
\end{verbatim}
\item icon-theme-oxygen-index ve icon-theme-oxygen-png paketleri kurulumu sonrası:

Bilgisayarınızı yeniden başlatın ve ikonların düzgün bir şekilde görüntülendiğni gözlemleyin.
 \item Aşağıdaki paketler aynı şekilde test edilecektir.
\begin{verbatim}
cursor-theme-oxygen-black
cursor-theme-oxygen-black-big
cursor-theme-oxygen-blue
cursor-theme-oxygen-blue-big
cursor-theme-oxygen-white
cursor-theme-oxygen-white-big
cursor-theme-oxygen-yellow
cursor-theme-oxygen-yellow-big
cursor-theme-oxygen-zion
cursor-theme-oxygen-zion-big
\end{verbatim}

Sistem ayarları $\rightarrow$  Klavye ve Fare $\rightarrow$ İmleç teması altında listelendiğini gözlemleyin. İmleç temasını değiştirin ve sorunsuz bir şekilde değiştiğini gözlemleyin.

\end{itemize}


\section{Font alt Bileşeni}
\begin{itemize}

\item  urw-fonts paketi kurulumu sonrası:

Open ofis yazıcı açın ve URW fontunun ekli olduğunu gözlemleyin.


\item  dejavu-fonts paketi kurulumu sonrası:

Open ofis yazıcı açın ve Dejavu fontlarının ekli olduğunu gözlemleyin.
 \item fontforge paketi kurulumu sonrası:

aquafont paketini kurun, kmenü'den fontforge uygulamasını açın ve /usr/share/fonts/aquafont/aquafont.ttf uzantısını seçin ve listelenen aquafont.ttf  dosyası üzerine çift tıklayın ve sorunsuz bir şekilde bu fontun karakterleri ile ilgili bir pencere açıldıpğını gözlemleyin.
\item liberation-fonts paketi kurulumu sonrası:

Open ofis yazıcı açın ve LiberationMono, LiberationSans, LiberationSerif fontlarının ekli olduğunu gözlemleyin.

\item ecofont paketi kurulumu sonrası:

Open ofis yazıcı açın ve Sproanq eco sans fontunun ekli olduğunu gözlemleyin.
\item gbfed paketi kurulumu sonrası:

Kmenüden uygulamayı açın ve sorunsuz bir şekilde açılabildiğini gözlemleyin.
\end{itemize}

\section{Misc alt Bileşeni}
\begin{itemize}
\item Aşağıda bulunan paketler sadece kurulum testine tabidir:
\begin{verbatim}
 iTest
 shared-mime-info
\end{verbatim}

\item krename paketi kurulumu sonrası:

Kmenüden uygulamayı açın, aşağıda bulunan resim dosyalarını Ekle butonuna basarak ekleyin. Daha sonra Dosya adı tab'ına gidip Örnek düşen kutusundan sayıyı seçin ve bitir butonuna basın ve sorunsuz bir şekilde yeni ismlendirmelerin yapıldığını gözlemleyin.
\begin{verbatim}
 wget http://cekirdek.pardus.org.tr/~semen/dist/test/desktop/kde/base/circus-bw_hats.jpg
 wget http://cekirdek.pardus.org.tr/~semen/dist/test/desktop/kde/base/tepecik_01.png
\end{verbatim}

\item google-gadgets, google-gadgets-qt ve google-gadgets-gtk paketleri kurulumu sonrası:

Uygulamalar $\rightarrow$ İnternet yolunu izleyerek uygulamanızı çalıştırın ve programcığın efektler bozulmadan eklendiğini gözlemleyin.

\item basket paketi kurulumu sonrası:

Uygulamanın kemenüden gdüzgün bir şekilde çalıştığını gözlemleyin.

\end{itemize}

\section{Toolkit Bileşeni}
\subsection*{Qt ve Qt4}

(Bu kısımda verilen paket adlarının qt ile başlayan bölümleri, 2008'de qt için qt, qt4 için qt4, 2009'da qt için qt3, qt4 için qt olacaktır.)
\begin{enumerate}
 \item Aşağıda bulunan paketler sadece kurulum testine tabidir.
\begin{verbatim}
 qt-doc
 qt-sql-ibase
 qt-sql-odbc
 qt-sql-postgresql
\end{verbatim}
 \item qt paketi kurulumu sonrası

\begin{verbatim}
 # sudo pisi it -c system.devel
 # mkdir test
 # cd test
 # wget http://cekirdek.pardus.org.tr/~semen/dist/test/desktop/toolkit/test.cpp
 # qmake-qt4 -project
 # qmake-qt4
 # make
 # ./test
\end{verbatim}

"Hello world!" yazan bir pencerenin açıldığını gözlemleyin.
\item qt-designer paketi kurulumu sonrası

Menu $\rightarrow$ Programlar $\rightarrow$ Geliştirme yolunu izleyerek sorunsuz bir şekilde açıldığını gözlemleyin.

\item qt-linguist paketi kurulumu sonrası

Menu $\rightarrow$ Programlar $\rightarrow$ Geliştirme yolunu izleyerek sorunsuz bir şekilde açıldığını gözlemleyin.

\item qt-sql-mysql, qt-sql-sqlite paketi kurulumu sonrası

qt-sql-mysql için Mysql'i servis yöneticisinden başaltınız.
\begin{verbatim}
 # mkdir test
 # cd test
 # wget http://cekirdek.pardus.org.tr/~semen/dist/test/desktop/ 
toolkit/test-qt-sql-<ilgili_veritabanı>.cpp
 # qmake-qt4 -project
 # qmake-qt4	
\end{verbatim}
qmake komutundan sonra oluşan .pro dosyanıza QT += sql satırını eklemelisiniz. Daha sonra aşağıdaki komutları çalıştırın.
\begin{verbatim}
 # make
 # ./test
\end{verbatim}

Bağlamtının sorunsuz bir şekilde gerçekleştğini gözlemleyin.

\end{enumerate}

\subsection*{Gtk}
\begin{enumerate}
\item Aşağıda bulunan paketler sadece kurulum testine tabidir:
\begin{verbatim}
gtk+extra
gtk-doc
\end{verbatim}
\item libglademm paketi kurulumu sonrası:

multimedia-tr.pdf pavucontrol testini gerçekleştiriniz.
 
 \item gtk2, gtk2-docs ve gtk2-demo paketleri kurulumu sonrası: 

Aşağıda bulunan komutun demo kodlar için bir gui açtığını gözlemleyin:
\begin{verbatim}
 # gtk-demo
\end{verbatim}

Bu gui'de bulunan listeden:

Drawing Area, Clipboard, Color Selector bölümlerine çift tıklayarak çalıştırın ve sorunsuz bir şekilde çalıştıklarını gözlemleyin.

\item gtk2 paketi kurulumu sonrası: 
\begin{itemize}
 \item multimedia-tr.pdf avidemux testini gerçekleştirin
\item desktop-tr.pdf gtk2-demo testini gerçekleştirin.
\end{itemize}
\item pango ve pango-docs paketi kurulumu sonrası: 
\begin{itemize}
 \item progrmming-tr.pdf pygtk testini gerçekleştirin.
 \item multimedia-tr.pdf inkscape testini gerçekleştirin.
\end{itemize}

\end{enumerate}
\subsection*{Others}
\begin{enumerate}
\item wxGTK ve wxGTK-devel paketleri kurulumu sonrası:

science-tr.pdf wxMaxima testini gerçekleştirin.

 \item newt paketi kurulumu sonrası:
\begin{itemize}
 \item ipython paketini kurun ve aşağıda bulunan komutların sorunsuz çalıştığını gözlemleyin:
\begin{verbatim}
 # ipython
 # import snack
\end{verbatim}
  \item hardware-tr.pdf partimage testini gerçekleştirin.

\end{itemize}
 


\end{enumerate}

\section{Kde3 alt Bileşeni}
\subsection*{Base alt bileşeni}
\begin{enumerate}
 \item kdebase paketi kurulumu sonrası:
Kmenu'den Programlar $\rightarrow$ Yardımcı Programlar $\rightarrow$ Düzenleyiciler $\rightarrow$ Kwrite çalıştırın ve sorunsuz çalıştığını gözlemleyin.

Kmenu'den Programlar $\rightarrow$ İnternet $\rightarrow$ Konqueror çalıştırın ve düzgün çalıştığını gözlemleyin.

Kmenu'den Programlar $\rightarrow$ Sistem $\rightarrow$ Konsole çalıştırın ve düzgün çalıştığını gözlemleyin.

Aşağıda bulunan komutu çalıştırın ve klipper'ı başlatın sorunsuz bir şekilde başladığını gözlemleyin.

\begin{verbatim}
 klipper
\end{verbatim}

\end{enumerate}

\section{Kde alt Bileşeni}
\subsection*{Admin alt bileşeni}

\begin{enumerate} 
 \item history-manager paketi kurulumu sonrası:

 http://svn.pardus.org.tr/uludag/trunk/doc/test/2009/testguide/turkish/alfabeta/history-manager-tr.pdf  testini gerçekleştiriniz.
\end{enumerate}

\subsection*{Base alt bileşeni}

2008'de tüm kde4 paketleri 4 versiyon numarasını içermekte iken, 2009'da bu versiyon numarasını içermemektedir.
\begin{enumerate} 
\item  Aşağıda bulunan paketler sadece kurulum testine tabidir.
\begin{verbatim}
kdelibs-devel
kdelibs-apidox 
\end{verbatim}
\item kbd paketi kurulumu sonrası:

Makinenizi yeniden başlatın, klavyenizin dilinin ve fonksiyon tuşlarının düzgün olduğunu gözlemleyiniz. 
\item kdebase-emoticons paketi kurulumu sonrası:

Sistem ayarları $\rightarrow$ Görünüm $\rightarrow$ Emoticon yolunu izleyin ve kde4 için emoticon eklendiğini gözlemleyin.
\item kdebase-sound paketi kurulumu sonrası:

Sistemi yeniden başlatın bitiş ve başlangıç seslerinin çıktığını gözlemleyin.

K3b ile bir cd yazdırın ve sesin cd yazdırma bitiş muziğinin çıktığını gözlemleyin.
\item kdebase-runtime ve kdebase-runtime-doc paketleri kurulumu sonrası:

network-tr.pdf choqok testini gerçekleştirin

Aşağıda bulunan komutu çalıştırın ve sorunsuz bir şekilde çalıştırğını gözlemleyin:
\begin{verbatim}
 # nepomukserver
\end{verbatim}

 \item kdelibs paketi kurulumu sonrası:
\begin{itemize}
 \item network-tr.pdf choqok testini gerçekleştirin


 \item kdegraphics paketini kurun: (2008 için kdegraphics4)
\begin{verbatim}
 # wget http://cekirdek.pardus.org.tr/~semen/dist/test/desktop/kde/base/circus-bw_hats.jpg
 # wget http://cekirdek.pardus.org.tr/~semen/dist/test/desktop/kde/base/tepecik_01.png
\end{verbatim}
Yukarıda bulunan dosyaların okular ve gwenview ile açıldığını gözlemleyin.
\item amarok paketini kurun: (2008 için amarok-kde4) 

\begin{verbatim}
/usr/kde/4/share/sounds/k3b_error1.wav
/usr/kde/4/share/sounds/KDE-Im-Irc-Event.ogg
\end{verbatim}

Dosyalarının düzgün bir şekilde amarok ile açıldığını gözlemleyin.

\item yakuake paketini kurun: (2008 içi yakuake4)

F12 tuşuna basıldığında sorunsuz bir şekilde yakuake'nin açıldığını gözlemleyin.
\end{itemize}
\item kdebase-workspace ve kdebase-workspace-doc paketi kurulumu sonrası:
\begin{itemize}

 \item Bilgisayarınızı kapatın ve yeniden başlatın ve düzgün bir şekilde kde'yi kapattığını ve başlattığını gözlemleyin.

 \item Aşağıda bulunan komutların düzgün bir şekilde çalıştığını gözlemleyin:
\begin{verbatim}
# plasmoidviewer nm-applet 
# klipper
# krunner
# kfontview
\end{verbatim}

\end{itemize}

\item kdebase-wallpapers paketi kurulumu sonrası:

masaüstüne sağ tıklayıp, görünüm ayarlarından duvar kağıdını değiştiri seçiniz. Red Leaf ve Vector Sunset duvar kağıtlarının eklenmiş olduğunu gözlemleyiniz.

\item kdm paketi kurulumu sonrası:

Bilgisayarınızı yeniden başlatın. Açılışta, çıkışta ve kullanıcı değiştirirken çıkan grafiksel giriş ekranınının düzgün bir şekilde açıldığını gözlemleyin.

\item kdeplasma-addons paketi kurulumu sonrası:

Masaüstüne sağ tıklayarak programcık kilidini açın.

Panel Üzerine sağ tıklayarak, programcık ekleyi seçin ve lacelot'u programcık olarak eklemeye çalışın. Düzgün bir şekilde eklenediğini ve çalıştığını gözlemleyin.

Aynı şekilde LCD Weather Station, Twitter Microblogging, RSSNOW, Blue Marble programcıklarını da deneyiniz.

\item kdegraphics paketi kurulumu sonrası:

Aşağıda bulunan dosyaların gwenview  ile çalıştıklarını gözlemleyin.  
\begin{verbatim} 
 # wget http://cekirdek.pardus.org.tr/~semen/dist/test/office/openoffice/test_oodraw.jpg
 # wget http://cekirdek.pardus.org.tr/~semen/dist/test/office/openoffice/test_oodraw.mng
 # wget http://cekirdek.pardus.org.tr/~semen/dist/test/office/openoffice/test_oodraw.png
 # wget http://cekirdek.pardus.org.tr/~semen/dist/test/office/openoffice/test_oodraw.ps
 # wget http://cekirdek.pardus.org.tr/~semen/dist/test/office/openoffice/test_oodraw.tif
 # wget http://cekirdek.pardus.org.tr/~semen/dist/test/office/openoffice/test_oodraw.xcf
 # wget http://cekirdek.pardus.org.tr/~semen/dist/test/office/openoffice/test_openoffice-extension-pdfimport.pdf
 # gwenview
\end{verbatim}
Aşapıda bulunan dosyaların okular ile düzgün çalıştığını gözlemleyin:
\begin{verbatim} 
 # wget http://cekirdek.pardus.org.tr/~semen/dist/test/office/postscript/test_ghostscript.dvi
 # wget http://cekirdek.pardus.org.tr/~semen/dist/test/office/openoffice/test_openoffice-extension-pdfimport.pdf
 # wget http://cekirdek.pardus.org.tr/~semen/dist/test/office/openoffice/test_oodraw.ps
 # okular
 \end{verbatim}

Aşağıda bulunan ugulamaların düzgün çalıştığını gözlemleyin.
\begin{verbatim}
 # kcolorchooser
 # kruller
 # ksnapshot
\end{verbatim}


\end{enumerate}

\subsection*{Addon alt bileşeni}
\begin{enumerate}
 \item  QtCurve-KDE4 paketi kurulumu sonrası:

Sistem ayarları $\rightarrow$ Görünüm $\rightarrow$ Stil yolunu izleyerek QtCurve stilini seçiniz ve stilin sorunsuz bir şekilde değiştiğini gözlemleyiniz.
\item plasmoid-daisy paketi kurulumu sonrası:

Masaüstüne sağ tıklayıp programcık ekleyi seçin. Daisy'un eklenmiş olduğunu gözlemleyin.

Panele eklenen boşluk üzerine sağ tıklayın ve panel ayarlarını seçin. Daha sonra plasmoid üzerinde çıkan ok üzerine tıklayın ve daisy ayarlarının açıldığını gözlemleyin.

 \item plasmoid-adjustable-clock paketi kurulumu sonrası:

Masaüstüne sağ tıklayıp programcık ekleyi seçin. Adjustable Clock'un eklenmiş olduğunu gözlemleyin.


 \item plasmoid-translatoid paketi kurulumu sonrası:

Masaüstüne sağ tıklayıp programcık ekleyi seçin. Translatoid'in eklenmiş olduğunu gözlemleyin.

Programcığı açın ve türkçeden ingilizceye bir çeviri yapmayı deneyin.
 \item kshutdown paketi kurulumu sonrası:

Aşağıda bulunan komutu çalıştırın ve kapatma penceresinin açıldığını gözlemleyin.
\begin{verbatim}
 # kshutdown 
\end{verbatim}

 \item kaptan paketi kurulumu sonrası:

  http://svn.pardus.org.tr/uludag/trunk/doc/test/2009/testguide/turkish/alfabeta/kaptan-tr.pdf testini gerçekleştirin.

\end{enumerate}

\end{document}

