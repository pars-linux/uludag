\documentclass[a4paper,10pt]{article}
\usepackage[turkish]{babel}
\usepackage[utf8]{inputenc}
\usepackage[left=1cm,top=1cm,right=2cm,bottom=2cm]{geometry}

\title{Game bileşeni test aşamaları}
\author{Semen Cirit}

\renewcommand{\labelenumi}{\arabic{enumi}.}
\renewcommand{\labelenumii}{\arabic{enumi}.\arabic{enumii}.}
\renewcommand{\labelenumiii}{\arabic{enumi}.\arabic{enumii}.\arabic{enumiii}.}
\renewcommand{\labelenumiv}{\arabic{enumi}.\arabic{enumii}.\arabic{enumiii}.\arabic{enumiv}.}

\begin{document}

\maketitle

\begin{enumerate}
\item Aşağıda bulunan paketler aynı yöntem ile test edilecektir.
\begin{verbatim}
abuse
blobby
extreme-tuxtracer
rocksndiamonds 
blobAndConquer
blobwars
freecol
xmoto
wormux ve wormux-bonusmaps
eboard
ppracer
AssaultCube
glest ve glest-data
connectagram
btanks
egoboo ve egoboo-data
neverball
nexuiz ve nexuiz-data
scorched3d
supertuxkart
supertux
trigger
warsow ve warsow-data
warzone2100
pokerth
alienarena
alienarena-data
daimonin
daimonin-music
freedroidrpg
funguloids
ioPaintball
ioPaintball-data
memonix
memonix-data
scourge
scourge-data
stormbaancoureur
tecnoballz
westernquake3
westernquake3-data
lbreakout2
ltris
widelands
frozen-bubble
fish-fillets ve fish-fillets-data
gmult
dreamchess
armagetronad
bzflag
teeworlds
smc
wesnoth
flightgear
flightgear-data-base
chromium-bsu
liquidwar
\end{verbatim}

Oyunları çalıştırın.

Oyunların görsel ve işitsel olarak düzgün çalıştıklarını gözlemleyin.

\item xmoto-edit paketi kurulumu sonrası:

inkscape uygulmasını açın, eklentiler $\rightarrow$ X-moto seçeneğinin eklenmiş olduğunu gözlemleyin.
\end{enumerate}

\end{document}

