
\documentclass[a4paper,10pt]{article}
\usepackage[turkish]{babel}
\usepackage[utf8]{inputenc}
 \usepackage[left=1cm,top=1cm,right=2cm,bottom=2 cm]{geometry}
\renewcommand{\labelenumi}{\arabic{enumi}.}
\renewcommand{\labelenumii}{\arabic{enumi}.\arabic{enumii}.}
\renewcommand{\labelenumiii}{\arabic{enumi}.\arabic{enumii}.\arabic{enumiii}.}
\renewcommand{\labelenumiv}{\arabic{enumi}.\arabic{enumii}.\arabic{enumiii}.\arabic{enumiv}.}
\title{Server Bileşeni Test Aşamaları}
\author{Semen Cirit}



\begin{document}

\maketitle
\section{Auth alt bileşeni}
\begin{enumerate}
\item freeradius paketi kurulumu sonrası:

Servis yöneticisinden freeradius servisini başlatın.

Serisin başlatılmış olduğunu gözlemleyin:
\begin{verbatim}
 service freeradius status
\end{verbatim}


% \item mit-kerberos paketi kurulumu sonrası:
% \begin{verbatim}
% sudo pisi it -c system.devel
% wget ftp://ftp.funet.fi/pub/unix/shells/tcsh/tcsh-6.16.00.tar.gz
% tar xf tcsh-6.16.00.tar.gz
% cd tcsh-6.16.00 
% ./configure --prefix=/usr
% sudo make
% sudo make install
% cd ..
% 
% \end{verbatim}
\end{enumerate}

\section{Proxy alt bileşeni}
\begin{enumerate}
\item squid paketi kurulumu sonrası:

Servis yöneticisinden squid başlatın, aşağıdaki komut ile başlatıldığına emin olun:
\begin{verbatim}
  service squid status
\end{verbatim}
 \item dansguardian paketi kurulumu sonrası:

Servis yöneticisinden dansguardian başlatın, aşağıdaki komut ile başlatıldığına emin olun:
\begin{verbatim}
  service dansguardian status
\end{verbatim}
\item ntlmaps paketi kurulumu sonrası:

Servis yöneticisinden ntlmaps başlatın, aşağıdaki komut ile başlatıldığına emin olun:
\begin{verbatim}
  service ntlmaps status
\end{verbatim}
\end{enumerate}


\section{Auth alt bileşeni}
\begin{enumerate}
 \item ypserv paketi kurulumu sonrası:

Servis yöneticisinden ypserv başlatın, aşağıdaki komut ile başlatıldığına emin olun:
\begin{verbatim}
  service ypserv status
\end{verbatim}

\item yp-tools paketi kurulumu sonrası:

Servis yöneticisinden ypserv başlatıldıktan sonra:
\begin{verbatim}
  su -
  domainname localdomain
  domainname
  nisdomainname
\end{verbatim}

\item ypbind paketi kurulumu sonrası:

Aşağıda bulunan dosyayı /etc altına kopylayın,
\begin{verbatim}
 wget http://cekirdek.pardus.org.tr/~semen/dist/test/server/auth/yp.conf 
 sudo cp yp.conf /etc/
 sudo domainname localdomain
\end{verbatim}

Servis yöneticisinden ypbind başlatın, aşağıdaki komut ile başlatıldığına emin olun:
\begin{verbatim}
  service ypbind status
\end{verbatim}

\end{enumerate}

\section{Mta alt bileşeni}
\begin{enumerate}
 \item dovecot paketi kurulumu sonrası:

Servis yöneticisinde dovecot başlatılır.
Aşağıdaki komut ile başlatıldığını gözlemleyin:
\begin{verbatim}
  service dovecot status
\end{verbatim}
Aşağıdaki komut ile dovecot kullanıcısı tarafından servisin başlatıldığını gözlemleyin.
\begin{verbatim}
 ps aux|grep dovecot 
\end{verbatim}

Aşağıda bulunan komutlar ile imap ve pop3 paketlerini kontrol edin:
\begin{verbatim}
 netstat -ln|grep 110
 netstat -ln|grep 143
\end{verbatim}

\end{enumerate}

\section{Web alt bileşeni}
\begin{enumerate}
\item Aşağıda bulunan paketler sadece kurulum testine tabidir:
\begin{verbatim}
mod_wsgi
mod_dav_svn
gitweb
openldap-client
openldap-server
openldap-slurpd
\end{verbatim}

\item webalizer paketi kurulumu sonrası:

Service yöneticisinden apache servisini başlatın.

Aşağıda bulunan dosyayı /var/log/apache2 altına kopyalayın.
\begin{verbatim}
wget http://cekirdek.pardus.org.tr/~semen/dist/test/server/web/access_log
\end{verbatim}

Aşağıda bulunan komutu çalıştırın ve daha sonra http://localhost/webalizer/ bağlantısına firefox ile bağlanın. Ve apache kullanım grafiklerini gözlemleyin.
\begin{verbatim}
sudo webalizer
\end{verbatim}

\item apache paketi kurulumu sonrası:
\begin{itemize}
\item Servis yöneticisinden apache sunucusunu başlatın. Aşağıda bulunan komutu kullanarak sunucunun başlatılmıl olduğunu gözlemleyin.
\begin{verbatim}
service list
\end{verbatim}
\item Firefox üzerinden http://localhost adresine bağlanın ve sorunsuz bir şekilde bağlanabildiğinizi gözlemleyin. 
\item Aşağıdaki komutu çalıştırın ve komut çıktısında `Syntax Ok` aldığınzı gözlemleyin.
\begin{verbatim}
apachectl -M 
\end{verbatim}

\end{itemize}

\item mod\_php paketi kurulumu sonrası:

util-tr.pdf phpmyadmin testini gerçekleştiriniz.

\end{enumerate}

\section{Database alt bileşeni}
\begin{enumerate}
\item Aşağıda bulunan paketler sadece kurulum testine tabidir.

\begin{verbatim}
virtuoso
virtuoso-apps
virtuoso-conductor
virtuoso-doc
virtuoso-java
virtuoso-utils
\end{verbatim}

\item memcached paketi kurulumu sonrası:

programming-tr.pdf python-memcached testini gerçekleştirin.

\item postgresql-python paketi kurulumu sonrası:

ipython paketini kurun ve aşağıda bulunan komutların düzgün çalıştığını gözlemleyin.
 \begin{verbatim}
  ipython
  import pgdb \end{verbatim}


 \item Aşağıda bulunan paketlerin kurulumu sonrası:
\begin{verbatim}
 postgresql-doc
 postgresql-lib
 postgresql-pl
 postgresql-server
 postgresql-odbc
\end{verbatim}

Servis yöneticisinden postgreql sunucusunu başlatın. Aşağıdaki komut ile sunucunun başlatılmış olduğunu gözlemleyin.
\begin{verbatim}
  service postgresql-server status
\end{verbatim}

Süreçlerin postgres kullanıcısı ile çalıştığını gözlemleyin:
\begin{verbatim}
 ps aux|grep postgres 
\end{verbatim}

Aşağıdaki komutu kullanarak sql komut satırına geçin ve ikinci satırdaki sql komutunu çalıştırın:
\begin{verbatim}
 psql -h localhost -d postgres -U postgres
 select * from information_schema.tables ;
\end{verbatim}

İşlemlerin sorunsuz olarak gerçekleştiğini gözlemleyin.

 \item firebird-superserver ve firebird-client paketleri kurulumu sonrası:

Bilgisayarınızı yeniden başlatın,

Servis yöneticisinden firebird-superserver'ı başlatın.

Aşağıda bulunan komutları sırası ile çalıştırın ve sorunsuz bir şekilde çalıştığını gözlemleyin:
\begin{verbatim}
# cd /opt/firebird/examples/empbuild
# isql (2008 için)
# fb_isql (2009 için)

SQL> CONNECT employee.fdb user sysdba password masterkey;
SQL> show tables;
SQL> select *from COUNTRY
\end{verbatim}

 \item mysql-client, mysql-server, mysql-lib paketleri kurulumu sonrası:
\begin{itemize}
 \item Servis yöneticisinden Mysql'i başlatın ve  aşağıda bulunan komutu kullanarak başlatılmış olduğundan emin olun:

\begin{verbatim}
 service list
\end{verbatim}
 \item Aşağıda bulunan komutları çalıştırın ve sorunsuz bir şekilde çalıştıklarını gözlemleyin.
\begin{verbatim}
 sudo mysql 
 show databases;
 use mysql;
 show tables;
\end{verbatim}


\end{itemize}

\item mysql-man-pages paketi kurulumu sonrası:

Aşağıda bulunan komutun man sayfasını düzgün açtığından emin olun.
\begin{verbatim}
man myisampack 
\end{verbatim}

\end{enumerate}

\section{Diğerleri}

\begin{itemize}
\item Aşağıda bulunan paketler sadece kurulum testine tabidir.

\begin{verbatim}
 ntp-docs
\end{verbatim}


\item ntp-server paketi kurulumu sonrası:

Servis yöneticisinden ntp deamon'u başlatın, aşağıda bulunan komut ile servisin başlatılmış olduğunu gözlemleyin.
\begin{verbatim}
 service ntpd status
\end{verbatim}


 \item dhcp paketi kurulumu sonrası:

Ağ yöneticisinden dhcp kullanarak bir ağa bağlanmayı deneyin. Daha sonra konsoldan aşağıda bulunan komutu çalıştırın ve ağa bağlı olduğununuzu gözlemleyin.
\begin{verbatim}
 ping 4.2.2.1
\end{verbatim}

\item bind paketleri kurulumu sonrası:
\begin{verbatim}
 dig www.google.com
\end{verbatim}
Yukarıda bulunan komutun düzgün bir şekilde dns sunucuları listelediğini gözlemleyin.

\item samba paketi kurulumu sonrası:

Servis yöneticisinden samba servisini başlatın.

Aşağıda bulunan komut ile servisin başlatıldığını gözlemleyin.
\begin{verbatim}
  service samba status 
\end{verbatim}

 Aşağıda bulunan komutun sorunsuz çalıştığını gözlemleyin:
\begin{verbatim}
 sudo testparm /etc/samba/smb.conf
\end{verbatim}

\end{itemize}


\end{document}

