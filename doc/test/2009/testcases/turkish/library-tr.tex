\documentclass[a4paper,10pt]{article}
\usepackage[turkish]{babel}
\usepackage[utf8]{inputenc}
\usepackage[left=1cm,top=1cm,right=2cm,bottom=2cm]{geometry}

\title{Library Test Aşamaları}
\author{Semen Cirit}

\renewcommand{\labelenumi}{\arabic{enumi}.}
\renewcommand{\labelenumii}{\arabic{enumi}.\arabic{enumii}.}
\renewcommand{\labelenumiii}{\arabic{enumi}.\arabic{enumii}.\arabic{enumiii}.}
\renewcommand{\labelenumiv}{\arabic{enumi}.\arabic{enumii}.\arabic{enumiii}.\arabic{enumiv}.}

\begin{document}

\maketitle
\begin{enumerate}
\item Aşağıda bulunan paketler sadece kurulum testine tabidir.
\begin{verbatim}
yazpp
xmlsec
libftdi
libp11
libuuid
cddlib
atlas
blas
lapack
libcap-ng
libcap-ng-utils
libtasn1
netcdf
neon
avce00
e00compr
webkit-gtk
webkit-gtk-devel
facile
qca2
qca2-apidocs
qca2-ossl
omniORB
libunistring
FusionSound
DevIL
obexftp
FreeImage
redland-mysql
redland-postgresql
log4c
matio
lensfun
zziplib
gmm
\end{verbatim}
\item minixml paketi kurulumu sonrası:
game-tr.pdf dreamchess testini gerçekleştiriniz.

\item libmcrypt paketi kurulumu sonrası:

util-tr.pdf mcrypt testini gerçekleştiriniz.

\item libnfnetlink ve libnetfilter\_conntrack paketleri kurulumu sonrası:

network-tr.pdf conntrack-tools testini gerçekleştiriniz.

\item simgear paketi kurulumu sonrası:
game-tr.pdf flightgear testini gerçekleştiriniz.

\item libwww paketi kurulumu sonrası:

util-tr.pdf ntp-client testini gerçekleştiriniz.

\item physfs paketi kurulumu sonrası:

game-tr.pdf supertux paketi kurulumu sonrası.

\item libqjson paketi kurulumu sonrası:

desktop-tr.pdf plasmois-translatoid testini gerçekleştirin.

\item gupnp, gupnp-igd, gssdp paketleri kurulumu sonrası:

network-tr.pdf amsn testini gerçekleştiriniz.

\item farsight2 paketi kurulumu sonrası:

network-tr.pdf amsn testini gerçekleştiriniz.

network-tr.pdf pidgin testini gerçekleştiriniz.

\item gnutls paketi kurulumu sonrası:

network-tr.pdf pidgin testini gerçekleştiriniz.

network-tr.pdf wireshark testini gerçekleştiriniz.

\item avahi-glib, avahi-qt paketleri kurulumu sonrası:

network-tr.pdf pidgin testini gerçekleştiriniz.

\item phonon, phonon-gstreamer, phonon-xine paketleri kurulumu sonrası:

office-tr.pdf koffice-kpresenter testini gerçekleştiriniz.

Sistem Ayarları $\rightarrow$ Çokluortam $\rightarrow$ Arka uç bölümünden ilk önce xine daha sonra gstreamer seçerek aşağıda bulunan testi gerçekleştiriniz.

desktop-tr.pdf kdegames testini gerçekleştirin.
\item ode paketi kurulumu sonrası.

game-tr.pdf stormbaancoureur testini gerçekleştiriniz.

\item libiphone,ifuse, libplist, libusb1, usbmuxd paketleri kurulumu sonrası:

Bir iphone cihazınız var ise bunu sisteminize takın ve sistemin algılayıp, cihazınızı mount ettiğini gözlemleyin.

\item libmp3splt paketi kurulumu sonrası:

multimedia-tr.pdf mp3splt testini gerçekleştirin.

\item gd paketi kurulumu sonrası:

Aşağıda bulunan komutun sorunsuz çalıştığını gözlemleyiniz.

\begin{verbatim}
 wget http://cekirdek.pardus.org.tr/~semen/dist/test/library/gdtest.png
 pngtogd  gdtest.png test.gd2
\end{verbatim}

\item taglib ve taglib-extras paketleri kurulumu sonrası:

multimedia-tr.pdf kid3 testini gerçekleştirin.

\item libgphoto2 ve libgphoto2-doc paketleri kurulumu sonrası:

multimedia-tr.pdf digikam testini gerçekleştirin.

\item libtheora paketi kurulumu sonrası:

multimedia-tr.pdf k3b testini gerçekleştirin.

\item libquicktime paketi kurulumu sonrası:

multimedia-tr.pdf transcode testini gerçekleştirin.

\item libdvdread paketi kurulumu sonrası:

hardware-tr.pdf k3b DVD testini gerçekleştirin.

\item libdvdnav paketi kurulumu sonrası:

multimedia-tr.pdf vlc testini gerçekleştirin.

\item qimageblitz paketi kurulumu sonrası:

blitztest uygulaması ile aşağıda bulunan resmi açın ve effect testinin sorunsuz bir şekilde gerçekleştiğini gözlemleyin.
\begin{verbatim}
 # wget http://cekirdek.pardus.org.tr/~semen/dist/test/office/openoffice/test_oodraw.png
 # blitztest
\end{verbatim}


\item tre paketi kurulumu sonrası:

multimedia-tr.pdf streamripper testini gerçekleştiriniz.

\item gstreamer, gst-plugin-ugly, gst-plugin-bad, gst-plugin-base ve gst-ffmeg paketleri kurulumu sonrası:

Sistem ayarları $\rightarrow$ multimedia $\rightarrow$ Arka uç $\rightarrow$ gstreamer seçilir. 

dragon player ve kaffein ile aşağıda bulunan dosyaları çalıştırın ve sesin düzgün bir şekilde çıktığını gözlemleyin.
\begin{verbatim}
 # wget http://cekirdek.pardus.org.tr/~semen/dist/test/multimedia/video/cokluortam/niceday.asf
 # wget http://cekirdek.pardus.org.tr/~semen/dist/test/multimedia/video/cokluortam/MPEG-1_with_VCD_extensions.mpeg
 # wget http://cekirdek.pardus.org.tr/~semen/dist/test/multimedia/video/cokluortam/Lake_dance_XviD.AVI
\end{verbatim}

\item exempi ve yaz paketi kurulumu sonrası:

office-tr.pdf tellico paketi testini gerçekleştiriniz.

\item eet ve eina paketi kurulumu sonrası
\begin{itemize}
 \item [2008 için] edb paketini kurun. Aşağıda buluna komutların sorunsuz bir şekilde çalıştığını ve sonuç olarak "default" diye bir çıktı döndürdüğünü gözlemleyin.
\begin{verbatim}
 # wget http://cekirdek.pardus.org.tr/~semen/dist/test/library/test_edb
 # chmod 755 test_edb
 # ./test_edb
 # edb_ed test.db get /foo/theme str
\end{verbatim}
\item [2009 için] library-tr.pdf qedje paketi testini gerçekleştiriniz.
\end{itemize}

\item qedje kurulumu sonrası:

Aşağıdaki komutu çalıştırdığınızda hata vermeden çalıştığını gözlemleyiniz.
\begin{verbatim}
# qedje_viewer
\end{verbatim}

\item geoip paketi kurulumu sonrası:
\begin{verbatim}
# geoiplookup www.google.com 
\end{verbatim}
"GeoIP Country Edition: US, United States" gibi bir sonuç döndürdüğünü gözlemleyin.

\item nss ve nspr paketleri kurulumu sonrası:

network-tr.pdf firefox ve office-tr.pdf openoffice testlerini gerçekleştiriniz.

\item openexr paketi kurulumu sonrası:
\begin{itemize}
 \item multimedia-tr.pdf gimp, digikam ve imagemagick testlerini geçekleştiriniz.
 \item Aşağıdaki bağlantıda bulunan resim dosyalarının gwenview ile sorunsuz bir şekilde açıldığını gözlemleyin.
  \begin{verbatim}
   http://cekirdek.pardus.org.tr/~semen/dist/test/multimedia/graphics/graphics.tar
  \end{verbatim}
\end{itemize}
\item iksemel paketi kurulumu sonrası:
\begin{verbatim}
 # wget http://cekirdek.pardus.org.tr/~semen/dist/test/library/component.xml
 # ikslint -s component.xml
 # iksperf -a component.xml 
\end{verbatim}


Yukarıda bulunan komutların düzgün çalıştığını gözlemleyin.

\item xulrunner ve xulrunner-devel paketleri kurulumu sonrası:
\begin{itemize}
\item office-tr.pdf openoffice testlerini gerçekleştirin.
\item network-tr.pdf firefox testlerini gerçekleştirin.
\item network-tr.pdf gecko-mediaplayer testlerini gerçekleştirin.
\item multimedia-tr.pdf vlc-firefox testlerini gerçekleştirin.
\end{itemize}

\item libvorbis paketi kurulumu sonrası:

multimedia-tr.pdf vorbis-tools testini gerçekleştirin.

\item xerces-c paketi kurulumu sonrası:

Aşağıda bulunan dosyaları aynı dizin içerisine indirin.
\begin{verbatim}
# wget http://cekirdek.pardus.org.tr/~semen/dist/test/library/shiporder.xml
# wget http://cekirdek.pardus.org.tr/~semen/dist/test/library/shiporder.xsd
# wget http://cekirdek.pardus.org.tr/~semen/dist/test/library/test-xerces-c.sh
\end{verbatim}

./test-xerces-c.sh dosyasını çalıştırın ve çıktısında hata çıktısı olup olmadığını gözlemleyin:
\begin{verbatim}
# chmod 755 test-xerces-c.sh
# ./test-xerces-c.sh | less
\end{verbatim}

\item apr ve apr-util paketi kurulumu sonrası:

programming-tr.pdf subversion testini gerçekleştirin.

\item portaudio paketi kurulumu sonrası:

multimedia-tr.pdf qpitch testini gerçekleştirin.

\item libv4l paketi kurulumu sonrası: (Kamerası olanlar test edebilecektir.)

Tüm kamera ile ilgili diğer ugulamaların kapalı olduğundan emin olduktan sonra aşağıda bulunan komutun düzgün çalıştığını gözlemleyin:
\begin{verbatim}
 test-webcam
\end{verbatim}

\item qca2 ve qca2-apidocs paketi kurulumu sonrası:
\begin{itemize}
 \item Aşağıda bulunan komutu çalıştırın:
\begin{verbatim}
# qcatool2 plugins 
\end{verbatim}
\item network-tr.pdf konversation testini gerçekleştirin.
\end{itemize}


\item icu4c paketi kurulumu sonrası:

programming-tr.pdf PyICU testini gerçekleştirin.

\item raptor ve redland paketleri kurulumu sonrası:

office-tr.pdf openoffice yazıcı testini gerçekleştirin.

desktop-tr.pdf kdegraphics testini gerçekleştirin.

\item libnice paketi kurulumu sonrası:

network-tr.pdf pidgin testtini gerçekleştiriniz.

\item libtdb paketi kurulumu sonrası:

samba paketini kurunuz. Servis yöneticisinden samba servisini başlatınız.

Aşağıda bulunan komutların sorunsuz çalıştığını gözlemleyiniz.
\begin{verbatim}
 # tdbtool
 # create test
 # open test 
 # insert testkey testdata
 # show testkey
\end{verbatim}
\item qzion paketi kurulumu sonrası:

library-tr.pdf qedje testini gerçekleştiriniz.

\end{enumerate}

\end{document}

