\documentclass[a4paper,10pt]{article}
\usepackage[turkish]{babel}
\usepackage[utf8]{inputenc}
\usepackage[left=1cm,top=1cm,right=2cm,bottom=2cm]{geometry}

\title{Multimedia Bileşeni Test Aşamaları}
\author{Semen Cirit}

\renewcommand{\labelenumi}{\arabic{enumi}.}
\renewcommand{\labelenumii}{\arabic{enumi}.\arabic{enumii}.}
\renewcommand{\labelenumiii}{\arabic{enumi}.\arabic{enumii}.\arabic{enumiii}.}
\renewcommand{\labelenumiv}{\arabic{enumi}.\arabic{enumii}.\arabic{enumiii}.\arabic{enumiv}.}

\begin{document}

\maketitle

\section{Share}
\begin{enumerate}
 \item smb4k paketi kurulumu sonrası:

Uygulamayı açın "Ağı tara" butonuna basın, daha sonra alt kısımda bulunan "Network neighborhood" tab'ına basın ve etrafta bulunan komşu paylaşımların listelendiğini gözlemleyin.

\end{enumerate}

\section{TV alt Bileşeni}
\begin{enumerate}
 \item mlt paketi kurulumu sonrası:

programming-tr.pdf python-mlt testini gerçekleştiriniz.
\end{enumerate}

\section{Stream alt Bileşeni}
\begin{enumerate}

\item Aşağıda bulunan paketler sadece kurulum testine tabidir.
\begin{verbatim}
 mpd
\end{verbatim}

\item ushare paketi kurulumu sonrası:

/etc/ushare.conf dosyası içerisinde USHARE\_DIR değerine gerçerli bir dizin verin.

Servis yöneticisinden ushare sunucusunu başlatın ve sorunsuz bir şekilde başlatıldığını gözlemleyin.

\begin{verbatim}
 
\end{verbatim}

\item mpeg4ip paketi kurulumu sonrası:

Aşağıda bulunan komutların düzgün çalıştığını gözlemleyin.

Aşağıda bulunan komutun avi wav dönüşümünü gerçekleştirdiğini gözlemleyin.
\begin{verbatim}
 wget http://cekirdek.pardus.org.tr/~semen/dist/test/multimedia/video/cokluortam/Lake_dance_XviD.AVI
 avi2wav Lake_dance_XviD.AVI
\end{verbatim}

\item streamripper paketi kurulumu sonrası:

Aşağıda bulunan komutu çalıştırın ve bir takım şarkıları stream.mp3 adı altında birleştirdiğini gözlemleyin:
\begin{verbatim}
streamripper http://yp.shoutcast.com/sbin/tunein-station.pls?id=509645 -a stream 
\end{verbatim}

Daha sonra stream.mp3 dosyasını amarok, mplayer gibi uygulamalardan biri ile sorunsuz bir şekilde dinleyebildiğinizi gözlemleyin.

\end{enumerate}


\section{Plugin alt Bileşeni}
\begin{enumerate}
 \item Aşağıda bulunan paketler kurulum testine tabidir.
\begin{verbatim}
 hexter-dssi
 xsynth-dssi
\end{verbatim}


 \item kipi-plugins paketi kurulumu sonrası:

Menü $\rightarrow$ digikam $\rightarrow$ İçeriye Aktar $\rightarrow$ Facebook'tan içeriye aktar yolunu izleyerek facebook hesabınızdan fotoğraf indirmeye çalışın. Sorunsuz bir şekilde indirebildiğinizi gözlemleyin. (Digikam uygulamasını ilk defa açıyorsanız bir albüm oluşturunuz.)
\end{enumerate}

\section{Sound alt Bileşeni}
\begin{enumerate}

\item Aşağıda bulunan paketler sadece kurulum testine tabidir.
\begin{verbatim}
jack-audio-connection-kit 
jack-audio-connection-kit-docs
mpc
ncmpc
ncmpcpp
espeak-devel
flite-devel
opentts-devel
speech-dispatcher-devel
\end{verbatim}
\item clementine paketi kurulumu sonrası:

Aşağıda bulunan dosyayı indirin ve açın. Daha sonra bu dosyada bulunan ses dosyalarını uyglama ile birlikte çalıştırın.
\begin{verbatim}
 wget http://cekirdek.pardus.org.tr/~semen/dist/test/multimedia/sound/sound.tar
\end{verbatim}

Sorunsuz çalıştıklarını gözlemleyin.

\item opentts paketi kurulumu sonrası:

programming-tr.pdf python-opentts testini gerçekleştirin.


\item flite ve speech-dispatcher paketi kurulumu sonrası:

programming-tr.pdf python-speech-dispatcher testini gerçekleştirin.

\item espeak paketi kurulumu sonrası:

Aşağıda bulunan komutun sorunsuz çalıştığını gözlemleyin.

\begin{verbatim}
 espeak "This is a test"
\end{verbatim}

\item solfege paketi kurulumu sonrası:

Kmenu'den solfege uygulamasını açın ve sorusuz açılabildiğini gözlemleyin.

\item ZynAddSubFX paketi kurulumu sonrası:

Kmenü'den uygulamayı açın ve sorunsuz bir şekilde çalıştığını gözlemleyin.

\item ardour paketi kurulumu sonrası:

Kmenü'den uygulamayı açın ve sorunsuz bir şekilde çalıştığını gözlemleyin.

\item qmmp paketi kurulumu sonrası:
 
Aşağıda bulunan tar dosyasını indirdikten sonra uygulamayı çalıştırın ve seslerin düzgün çıktığını gözlemleyin.
 \begin{verbatim}
  wget wget http://cekirdek.pardus.org.tr/~semen/dist/test/multimedia/sound/sound.tar
 \end{verbatim}	
\item rosegarden paketi kurulumu sonrası:

Kmenu'den uygulamayı açın ve sorunsuz bir şekilde açıldığını gözlemleyin.

\item faad2 paketi kurulumu sonrası:

Aşağıda bulunan komutun dosya ile ilgili bilgileri listelediğinii gözlemleyin. 
\begin{verbatim}
 wget http://cekirdek.pardus.org.tr/~semen/dist/test/multimedia/sound/sound/sample_mpeg4.mp4
 faad sample_mpeg4.mp4
\end{verbatim}

\item faac paketi kurulumu sonrası:

soundconverter paketini kurun. 

\begin{verbatim}
wget http://cekirdek.pardus.org.tr/~semen/dist/test/multimedia/sound/sound/sample.mp3
\end{verbatim}

Kmenü'den soundconverter uygulamasını açın ve yukarıda bulunan dosyayı açın ve daha sonra "dönüştür" butonuna basın. Dosyanın kopyalandığı dizinde sample.ogg adında bir dosya oluştuğunu gözlemleyin.

\item padevchooser paketi kurulumu sonrası:

\begin{itemize}
 \item Uygulamanın sistem çekmecesine yerleştiğini gözlemleyin. 
 \item Uygulamaya sağ tıklayın ve ses kontrolünü seçin.
 \item Amarok ile bir mp3 dosyası açın.
 \item padevchooser uygulamasında sesi kapat butonuna basın ve sesin kapandığını gözlemleyin.
\end{itemize}

\item mpck paketi kurulumu sonrası:

Aşağıda bulunan komutun mp3 dosyası ile ilgili istatiskleri listelediğini gözlemleyin.
\begin{verbatim}
wget http://cekirdek.pardus.org.tr/~semen/dist/test/multimedia/sound/sound/sample.mp3 
mpck sample.mp3
\end{verbatim}


\item fluidsynth paketi kurulumu sonrası:

multimedia-tr.pdf vlc testini gerçekleştiriniz.

\item mixxx paketi kurulumu sonrası:

Kmenüden uygulamanın açılabildiğini gözlemleyin.

\item pavucontrol paketi kurulumu sonrası:

Uygulamayı Kmenüden açın ve daha sonra amarok ile aşağıda bulunan müzik dosyasını çalıştırın, çalma bölümünde titreşimlerin olduğunu gözlemleyin.
Aşağıda bulunan
\begin{verbatim}
wget http://cekirdek.pardus.org.tr/~semen/dist/test/multimedia/sound/sound/sample.mp3
\end{verbatim}

\item jamin paketi kurulumu sonrası:

Kmenüden uygulamayı açın ve /usr/share/jamin/examples dosyası altında bulunan .jam uzantılı dosyalardan birini açın ve sorunsuz olarak açıldığını gözlemleyin.
\item frescobaldi paketi kurulumu sonrası:
 \begin{verbatim}
 wget http://cekirdek.pardus.org.tr/~semen/dist/test/multimedia/sound/test_frescobaldi.ly
 \end{verbatim}
Yukarıda bulunan dosyayı frescobaldi ile açıp, uygulamanın sol tarafında bulunan Lilypond butonuna basın ve sorunsuz bir şekilde pdf dökümanının oluştuğunu gözlemleyin.
\item pulseaudio, pulseaudio-docs, pulseaudio-gconf, pulseaudio-jack paketi kurulumu sonrası:
\begin{itemize}
 \item multimedia-tr.pdf pavucontrol testini gerçekleştirin.
 \item Sisteminizi tekrar başlatın ve açılış sesinin düzgün bir şekilde çıktığını gözlemleyin. 
 \item multimedia-tr.pdf amarok paketi testini gerçekleştiriniz.
\end{itemize}
\item sox paketi kurulumu sonrası:

hardware-tr.pdf k3b testini gerçekleştiriniz.
\item amarok paketi kurulumu sonrası:
 
Aşağıda bulunan zip dosyasını indirdikten sonra seslerin düzgün çıktığını gözlemleyin.
 \begin{verbatim}
  wget wget http://cekirdek.pardus.org.tr/~semen/dist/test/multimedia/sound/sound.tar
 \end{verbatim}	

\item listen paketi kurulumu sonrası: 

Uygulamayı açın dosya sistemi bölümünden /usr/kde/3.5/share/sounds/ dizinine girin ve sorunsuz bir şekilde müziklerin listelendiğini ve müzikleri çalabildiğini gözlemleyin.

\item vorbis-tools paketi kurulumu sonrası: 
 \begin{verbatim}
   wget http://cekirdek.pardus.org.tr/~semen/dist/test/multimedia/sound/sound/sample.ogg
   oggdec game.ogg
   mplayer game.wav
   oggenc game.wav
   mplqayer game.ogg
 \end{verbatim}

\item qpitch paketi kurulumu sonrası:

Kmenüden uygulamayı açın düzgün bir şekilde çalıştığını gözlemleyin.

\item qjackctl paketi kurulumu sonrası: 

Kmenüden qjackct uygulamasını açın. (Bu sırada hiçbir ses aygıtının açık olmadığından emin olun)

Başla butonuna basın, status ve mesajlardan Jack ses sunucusunun başlatılmış olduğunu gözlemleyin.

\item lame ve lame-docs paketleri kurulumu sonrası: 

Aşağıda bulunan komutların sorunsuz çalıştığını gözlemleyin:
\begin{verbatim}
 wget http://cekirdek.pardus.org.tr/~semen/dist/test/multimedia/sound/sound/sample.mp3
 lame sample.mp3 music.mpeg
 mplayer music.mpeg
\end{verbatim}

\item mpg123 paketi kurulumu sonrası:

Aşağıda bulunan komutların sorunsuz bir şekilde çalıştığını gözlemleyin:
 \begin{verbatim}
   wget http://cekirdek.pardus.org.tr/~semen/dist/test/multimedia/sound/sound/sample.mp3
   mpg123 sample.mp3
 \end{verbatim}

\end{enumerate}

\section{Video alt Bileşeni}
\begin{enumerate}
\item Aşağıda bulunan paketler sadece kurulum testine tabidir. 
\begin{verbatim}
 libvpx
 libvpx-devel
\end{verbatim}

\item minitube paketi kurulumu sonrası:

minitube uygulamasını açın ve YouTube'te bulunan bir arama yapın ve sorunsuz bir şekilde bulunduğunu ve izlenebildiğini gözlemleyin.
\item recordmydesktop paketi kurulumu sonrası:

Aşağıda bulunan komutu çalıştırın ve bilgisayar üzerinde dosya açma kapama, uygulama açma kapama gibi işlemler yapın. 
\begin{verbatim}
 recordmydesktop
\end{verbatim}

Ctrl+C tuşlarına basın ve videonun encode edilmesini bekleyin. Daha sonra aşağıda bulunan komut ile video'nun sorunsuz bir şekilde oluştuğunu gözlemleyin.

\begin{verbatim}
 mplayer out.ogv 
\end{verbatim}

\item mpgtx paketi kurulumu sonrası:

\begin{verbatim}
 wget http://cekirdek.pardus.org.tr/~semen/dist/test/multimedia/video/cokluortam/vcd.mpg
 mpgsplit -10M vcd.mpg
 mplayer chunk-01.mpg
 mpgjoin chunk-01.mpg chunk-02.mpg 
 mplayer chunk.mpg
\end{verbatim}

\item gpac paketi kurulumu sonrası:

multimedia-tr.pdf x264 testini gerçekleştirin.

\item dvdrip paketi kurulumu sonrası:

Bir DVD albümünüz var ise bu testi gerçekleştirebilirsiniz.

dvdrip uygulmasını kmenüden açın ve dvd'de bulunan bir parçayı rip edin, sorunsuz bir şekilde .rip uzantılı dosyanın oluştuğunu gözlemleyin.
 \item mplayer, vlc, kaffeine, smplayer, gnome-mplayer paketleri kurulumu sonrası:
 \begin{verbatim}
  wget http://cekirdek.pardus.org.tr/~semen/dist/test/multimedia/video/cokluortam.tar
 \end{verbatim}
Yukarıda bulunan dosyayı indirdikten sonra, uygulamayı her türlü dosya formatı ile çalıştırın ve sorunsuz bir şekilde çalıştığını gözlemleyin.
 \item vlc-firefox kurulumu sonrası:
 \begin{itemize}
  \item Firefox $\rightarrow$ Düzen $\rightarrow$ Seçenekler $\rightarrow$ Eklentileri Yönet $\rightarrow$ Yan Uygulamalar bölümünde VLC eklentisinin eklenmiş olduğunu gözlemleyin.
  \item Aşağıda bulunan uzantıdaki videoları firefox üzerinden açınız. Ve düzgün bir şekilde çalıştıklarını gözlemleyiniz.
  \begin{verbatim}
  http://cekirdek.pardus.org.tr/~semen/dist/test/multimedia/video/cokluortam/  
  \end{verbatim}
 \end{itemize}
\item ffmpeg paketi kurulumu sonrası:
 
Aşağıda bulunan ikinci komutun düzgün çıktılar ürettiğini ve üçüncü komutun da bu çıktıları çalıştırabildiğini gözlemleyin.
\begin{verbatim}
   wget http://cekirdek.pardus.org.tr/~semen/dist/test/multimedia/video/cokluortam/Lake_dance_XviD.AVI
   ffmpeg -i Lake_dance_XviD.AVI -r 24 test.mpg
   ffplay test.mpg
  \end{verbatim}
\item x264 paketi kurulumu sonrası:

Aşağıdaki komutların sorunsuz bir şekilde çalıştığını gözlemleyin:
\begin{verbatim}
 wget  http://cekirdek.pardus.org.tr/~semen/dist/test/multimedia/video/example.y4m
 x264 -o test.mp4 example.y4m 300x300 
 mplayer test.mp4 (Kısa bir video olmalı)
\end{verbatim}

\end{enumerate}

\section{Converter alt Bileşeni}
\begin{enumerate}
 \item Aşağıdaki paketler sadece kurulum testine tabidir:
\begin{verbatim}
 nrg2iso
 vnc2swf
 subtitleripper
uif2iso
\end{verbatim}
\item soundconverter paketi kurulumu sonrası:
\begin{verbatim}
 wget http://cekirdek.pardus.org.tr/~semen/dist/test/multimedia/sound/sound/sample.mp3
\end{verbatim}

Kmenü'den uygulamayı açın ve yukarıda bulunan dosyayı açın ve daha sonra "dönüştür" butonuna basın. Dosyanın kopyalandığı dizinde sample.ogg adında bir dosya oluştuğunu gözlemleyin.

Daha sonra bu dosyayı mplayer ile çalıştırın ve sorunsuz olarak çalıştığını gözlemleyin.
\begin{verbatim}
 mplayer sample.ogg
\end{verbatim}

 \item amrwb ve amrnb paketleri kurulumu sonrası:

 multimedia-tr.pdf sox ve mplayer testlerini gerçekleştiriniz.

\item ccd2iso paketi kurulumu sonrası:
\begin{verbatim}
  wget http://cekirdek.pardus.org.tr/~semen/dist/test/multimedia/converter/default.img
  ccd2iso default.img test.iso
\end{verbatim}

test.iso dosyasının sorunsuz bir şekilde oluştuğunu gözlemleyin.

Sorunsuz bir şekilde iso dosyasının oluştuğunu gözlemleyin.
\item dvdbackup paketi kurulumu sonrası:
\begin{itemize}
 \item Aşağıdaki bağlantıda bulunan iso'yu DVD'ye yazdırın. 
\begin{verbatim}
  wget http://cekirdek.pardus.org.tr/~semen/dist/test/hardware/optical/boot.iso
\end{verbatim}
 \item Eğer DVD yeniden yazdırabilir bir DVD ise /dev/dvdrw değil ise /dev/dvd girdilerini kullanarak aşağıdaki bulunan komutu çalıştırın ve sorunsuz bir şekilde DVD'nin yedeklendiğini gözlemleyin.
\begin{verbatim}
  dvdbackup -i <girdiyolu> -o <çıktıyolu> -M
 Örnek 
  dvdbackup -i /dev/dvdrw -o /home/pardus/dvd -M
\end{verbatim}
 
\end{itemize}
\item emovix paketi kurulumu sonrası:

Aşağıda bulunan dosyayı indirin:
\begin{verbatim}
  wget http://cekirdek.pardus.org.tr/~semen/dist/test/multimedia/converter/default.img
\end{verbatim}
 
Komutların sorunsuz bir şeklilde çalıştıklarını gözlemleyin.
\begin{verbatim}
  movix-version
  movix-files
  movix-conf
  mkmovixiso default.img --output-file=default.iso
\end{verbatim}

\item ffmpeg2theora paketi kurulumu sonrası:
\begin{verbatim}
  wget http://cekirdek.pardus.org.tr/~semen/dist/test/multimedia/video/cokluortam/DVD.mpg
  ffmpeg2theora DVD.mpg
\end{verbatim}
\begin{itemize}
 \item Yukarıda bulunan komutun sorunsuz bir şekilde DVD.ogv dosyasını oluşturduğunu gözlemleyin.
 \item mplayer ile bu dosyayı çalıştırın ve sorunsuz bir şekilde çalıştığını gözlemleyin.
\end{itemize}
\item ffmpeg2theora paketi kurulumu sonrası:
\begin{verbatim}
 wget http://cekirdek.pardus.org.tr/~semen/dist/test/multimedia/converter/lazarus.png
 icns2png lazarus.icns
\end{verbatim}
\begin{itemize}
 \item Yukarıda bulunan komutun sorunsuz bir şekilde lazarus.png dosyasını oluşturduğunu gözlemleyin.
 \item gwenview ile bu dosyayı açın ve sorunsuz bir şekilde açıldığını gözlemleyin.
\end{itemize}
\item kaudiocreator paketi kurulumu sonrası:
\begin{itemize}
 \item Aşağıdaki bağlantıda bulunan ses dosyalarını k3b ile audio CD olarak yazdırın.
\begin{verbatim}
 wget http://cekirdek.pardus.org.tr/~semen/dist/test/multimedia/video/cokluortam/
\end{verbatim}
 \item kaudiocreator uygulamasını çalıştırın:

Sorunsuz bir şekilde çalışıp CD'de bulunan dosyalarını listeleyebildiğini gözlemleyin.

Dosyaları seçin ve Rip butonuna basın ve bu işlemi düzgün bir şekilde yaptığını gözlemleyin.

Ev dizininiz altında rip ettiğiniz format adı (mp3, wav, ogg, flac olabilir) ile bir dizin yaratıldığını gözlemleyin, ve mplayer ile bu dosyaların sorunsuz bir şekilde çalıştığını gözlemleyin.
\end{itemize}

\item libnut paketi kurulumu sonrası:

multimedia-tr.pdf mplayer ve ffmpeg testlerini gerçekleştirin.

\item mkvtoolnix paketi kurulumu sonrası:

Uygulamalar $\rightarrow$ Çokluortam $\rightarrow$ mkvmerge GUI uygulamasını açın:

Aşağıdaki bağlantıda bulunan dosyaları, uygulamayı kullanarak .mkv dosyasına dönüştürün ve mplayer ile bu dosyaların düzgün bir şekilde oluştuklarını gözlemleyin: (Ekle tuşuna basarak ilgili dosyayı ekleyin ve muxing'e başla tuşuna basın.)
\begin{verbatim}
   wget http://cekirdek.pardus.org.tr/~semen/dist/test/multimedia/video/cokluortam.tar
 \end{verbatim}

\item mpeg2vidcodec paketi kurulumu sonrası:
\begin{verbatim}
   mkdir flower
   cd flower
   wget http://cekirdek.pardus.org.tr/~semen/dist/test/multimedia/converter/flowgard.mpg
   mpeg2decode -b flowgard.mpg -f -r -o0 sflowg.%d
   cd ..
   wget http://cekirdek.pardus.org.tr/~semen/dist/test/multimedia/converter/flower2.par
   mpeg2encode flower2.par flowgard.m2v
   mplayer flowgard.m2v
\end{verbatim}

Yukarıdaki komutların sorunsuz bir şekilde çalıştığını gözlemleyin.
\item ogmtools paketi kurulumu sonrası:

Aşağıdaki komutların düzgün bir şekilde çalıştıklarını gözlemleyin.
\begin{verbatim}
 wget http://cekirdek.pardus.org.tr/~semen/dist/test/multimedia/sound/sound/music.mp3 
 ogmmerge music.mp3 -o test.ogg
 mplayer test.ogg
\end{verbatim}

\item potrace paketi kurulumu sonrası:
\begin{verbatim}
 wget http://cekirdek.pardus.org.tr/~semen/dist/test/multimedia/converter/tepecik_01.pbm 
 potrace tepecik_01.pbm -o test.png
 gwenview test.png
\end{verbatim}

Sorunsuz bir şekilde png dosyasının oluştuğunu ve görüntülenebildiğini gözlemleyin.

\item shntool paketi kurulumu sonrası:
\begin{verbatim}
 wget http://cekirdek.pardus.org.tr/~semen/dist/test/multimedia/sound/sound/sample.wav
 wget http://cekirdek.pardus.org.tr/~semen/dist/test/multimedia/sound/sound/11k16bitpcm2.wav
 shncat 11k16bitpcm.wav
 shncmp 11k16bitpcm.wav 11k16bitpcm2.wav
\end{verbatim}
\item shorten paketi kurulumu sonrası:
\begin{verbatim}
  wget http://cekirdek.pardus.org.tr/~semen/dist/test/multimedia/sound/sound/sample.wav
  shorten 11k16bitpcm.wav
  mplayer 11k16bitpcm.shn
\end{verbatim}

Sorunsuz bir şekilde .shn uzantılı dosyanın oluştuğunu ve çalıştırılabildiğini  gözlemleyin.
\item transcode paketi kurulumu sonrası:

\begin{itemize}
 \item hardware-tr.pdf k3b paketini test ediniz.
 \item Aşağıdaki komutları çalıştırın:
\begin{verbatim}
 wget http://cekirdek.pardus.org.tr/~semen/dist/test/multimedia/video/cokluortam/
Lake_dance_XviD.AVI
 transcode -i Lake_dance_XviD.AVI -y xvid -o test.avi -k -z
 mplayer test.avi
\end{verbatim}
 Oluşan .avi uzantılı dosyanın baş aşağı çalıştığını gözlemleyin.
\end{itemize}

\item vcdimager paketi kurulumu sonrası:

\begin{itemize}
 \item hardware-tr.pdf k3b paketini test ediniz.
 \item Aşağıdaki komutları çalıştırın ve sorunsuz bir şekilde çalıştıklarını gözlemleyin:
\begin{verbatim}
 wget http://cekirdek.pardus.org.tr/~semen/dist/test/multimedia/video/cokluortam/DVD.mpg
 vcdimager DVD.mpg
 vcd-info -b videocd.bin
 vcdxgen DVD.mpg
 vcdxminfo -i DVD.mpg
\end{verbatim}

\end{itemize}
\item vobcopy paketi kurulımu sonrası:

\end{enumerate}
\section{Graphics alt Bileşeni}
\begin{enumerate}
 \item Aşağıda bulunan paketler kurulum testine tabidir
\begin{verbatim}
 gimp-data-extras
 tachyon
 graphviz-docs
 GraphicsMagick-devel
 ufraw
 hugin-docs
 gimp-devel
 colorize-gimp
 gimp-layer-effects
 gimp-save-for-web
 gimpfx-foundry
 gimp-ufraw-plugin
 imagemagick-doc
 tuxpaint-doc 
\end{verbatim}
\item mypaint paketi kurulumu sonrası:

Uygulamayı açın ve fırçaların birkaçını kullanarak çizim yapmayı deneyin, yapabildiğinizi gözlemleyin.

\item autopano-sift-C paketi kuruluu sonrası:

multimedia-tr.pdf hugin testini gerçekleştirin.

\item enblend paketi kuruluu sonrası:

multimedia-tr.pdf hugin testini gerçekleştirin.

\item hugin paketi kurulumu sonrası:

Aşağıda bulunan dosyayı indirin ve açın, daha sonra hugin uygulamasını çalıştırın. "Load Images" butonuna basın ve "test\_hugin" dosyası içerisinde bulunan resimleri yükleyin. "Align" butonuna basın ve sorunsuz bir şekilde align edildiğini gözlemleyin. 

\begin{verbatim}
 wget http://cekirdek.pardus.org.tr/~semen/dist/test/multimedia/graphics/test-hugin.zip
\end{verbatim}


\item tesseract paketi kurulumu sonrası:

Aşağıda bulunan komutları çalıştırın ve sorunsuz bir şekilde text dosyasının oluştuğunu gözlemleyin.
\begin{verbatim}
 wget http://cekirdek.pardus.org.tr/~semen/dist/test/multimedia/graphics/test_tesseract.tif
 tesseract test_tesseract.tif test_tesseract
 vi test_tesseract.txt
\end{verbatim}

\item uniconverter paketi kurulumu sonrası:

Aşağıda bulunan komutun sorunsuz çalıştığını gözlemleyin.
\begin{verbatim}
wget http://cekirdek.pardus.org.tr/~semen/dist/test/multimedia/graphics/drawing.svg
uniconv  drawing.svg test.ps
\end{verbatim}


\item kphotoalbum paketi kurulumu sonrası:

Kmenuden uygulamayı açın, tanıtımı yükleyi seçin butonuna basın. Daha sonra slideshow'u çalıştır butonuna basın ve sorunsuz bir şekilde çalıştığını gözlemleyin.

\item yafaray ve yafaray-blender paketleri kurulumu sonrası:

multimedia-tr.pdf blender testini gerçekleştiriniz.
\item jasper paketi kurulumu sonrası:

\begin{verbatim}
 wget http://cekirdek.pardus.org.tr/~semen/dist/test/multimedia/graphics/test_jasper.jpg
 jiv test_jasper.jpg
 jasper --input test_jasper.jpg --output test.jp2 --output-format jp2
 jiv test.jp2
\end{verbatim}


 \item gocr paketi kurulumu sonrası:

Aşağıda bulunan komutları çalıştırın ve uygulamanın karakterleri taryabildiğini ve test adında bir dosyaya yazabildiğini gözlemleyin.
\begin{verbatim}
 wget http://cekirdek.pardus.org.tr/~semen/dist/test/multimedia/graphics/font1.pbm.gz
 gocr -i font1.pbm.gz -o test
 vi test
\end{verbatim}

\item graphviz paketi kurulumu sonrası:

Aşağıda bulunan komutların sorunsuz olarak çalıştığını gözlemleyin.
\begin{verbatim}
 wget http://cekirdek.pardus.org.tr/~semen/dist/test/multimedia/graphics/test_graphviz.mm
 mm2gv  test_graphviz.mm -o test.gv
 dotty test.gv
 gv2gxl test.gv -o test.gxl
 gxl2dot test.gxl test.dot
 acyclic test.dot test_asyclic.dot
 lneato test.dot
 vimdot test.dot
\end{verbatim}

 \item Aşağıda bulunan paketlerin kurulumu sonrasında, yerel dilinizi değiştirip, konsoldan aynı dizinde gimp uygulamasını açın ve uygulamanın ilgili dilde olduğunu gözlemleyin.

Yerel dili değiştirmek için:

İlgili gimp dil paketi ile glibc-locales-<lang> paketini kurun. Örneğin: gimp-i18n-es için için glibc-locales-es.

\begin{verbatim}
export LC_ALL=<lang_LANG>
\end{verbatim}

lang\_LANG şeklinde yazılmış olan, pt-BT için pt\_BT, diğer diller için örneğin de\_DE olacaktır.

Daha sonra bu çalıştırdığınız komut dizininde gimp komutunu çalıştırın, uygulamanın sorunsuz bir şekilde istenilen dilde açıldığını gözlemleyin.
\begin{verbatim}
gimp-i18n-es
gimp-i18n-sk
gimp-i18n-sl
gimp-i18n-sr
gimp-i18n-sr_Latn
gimp-i18n-sv
gimp-i18n-ta
gimp-i18n-th
gimp-i18n-tt
gimp-i18n-uk
gimp-i18n-vi
gimp-i18n-et
gimp-i18n-eu
gimp-i18n-fa
gimp-i18n-fi
gimp-i18n-fr
gimp-i18n-ga
gimp-i18n-gl
gimp-i18n-gu
gimp-i18n-he
gimp-i18n-hi
gimp-i18n-xh
gimp-i18n-yi
gimp-i18n-zh_CN
gimp-i18n-zh_HK
gimp-i18n-zh_TW
gimp-i18n-hr
gimp-i18n-hu
gimp-i18n-id
gimp-i18n-is
gimp-i18n-it
gimp-i18n-ja
gimp-i18n-ka
gimp-i18n-km
gimp-i18n-kn
gimp-i18n-ko
gimp-i18n-lt
gimp-i18n-lv
gimp-i18n-mk
gimp-i18n-ml
gimp-i18n-mr
gimp-i18n-ms
gimp-i18n-nb
gimp-i18n-ne
gimp-i18n-nl
gimp-i18n-nn
gimp-i18n-oc
gimp-i18n-or
gimp-i18n-pa
gimp-i18n-pl
gimp-i18n-pt
gimp-i18n-pt_BR
gimp-i18n-ro
gimp-i18n-ru
gimp-i18n-rw
gimp-i18n-si 
\end{verbatim}

 \item GraphicsMagick  paketi kurulumu sonras:

office-tr.pdf koffice-krita testinin gerçekleştiriniz.

 \item İmageJ paketi kurulumu sonrası:

 Kmenüden uygulamayı çalıştırın ve Dosya $\rightarrow$ Aç yolunu izleyerek aşağıda bulunan dosyanın açılabildiğini gözlemleyin.
  \begin{verbatim}
   wget http://cekirdek.pardus.org.tr/~semen/dist/test/office/openoffice/test_oodraw.jpg
  \end{verbatim}

\item autotrace paketi kurulumu sonrası:

Aşağıda bulunan komutları çalıştırın ve sorunsuz bir şekilde çalıştığını gözlemleyin.
  \begin{verbatim}
   wget http://cekirdek.pardus.org.tr/~semen/dist/test/multimedia/graphics/bmp_24.bmp
   autotrace bmp_24.bmp -output-file test.eps  -output-format eps
   gwenview test.eps 
  \end{verbatim}

 \item gimp paketi kurulumu sonrası:
  \begin{verbatim}
   http://cekirdek.pardus.org.tr/~semen/dist/test/multimedia/graphics/graphics.tar
  \end{verbatim}

 Yukarıda bulunan farklı formattaki dosyaları gimp ile açınız, sorunsuz bir şekilde açıldıklarını gözlemleyiniz.

\item smile paketi kurulumu sonrası:

Kmenü'den uygulamanın açıldığını gözlemleyin.

\item digikam paketi kurulumu sonrası:

  \begin{verbatim}
   wget http://cekirdek.pardus.org.tr/~semen/dist/test/multimedia/graphics/graphics.tar
  \end{verbatim}

 Digikam uygulaması için seçtiğiniz dosya dizinine yukarıda klasör içinde bulunan dosyaları kopyalayınız ve sorunsuz bir şekilde çalıştığını gözlemleyiniz.
\item imagemagick  paketi kurulumu sonrası:
  \begin{verbatim}
   wget http://cekirdek.pardus.org.tr/~semen/dist/test/multimedia/graphics/graphics.tar
  \end{verbatim}

Yukarıdaki bağlantıda bulunan dosyaların aşağıda bulunan komutlar ile düzgün çalıştığını gözlemleyiniz.
  \begin{verbatim}
   cd graphics
   animate test_animate.gif
   display test.* (diğer tüm dosyalar için)
  \end{verbatim}
\item tuxpaint, tuxpaint-stamps paketleri kurulumu sonrası:
  \begin{itemize}
   \item Uygulamayı açın çeşitli çizimler yapın ve kaydedin, sorunsuz bir şekilde çalıştığını ve ilgili yerlerde ses dosyaları çalıştırdığını gözlemleyin.
   \item Stamps butonuna basın ve sağ tarafta bulunan pullardan birini eklemeye çalışın sorunsuz bir şekilde eklendiğini gözlemleyin.
   \item Aşağıda bulunan komutu çalıştırın ve daha sonra uygulamanın sol tarafında bulunan open butonuna basın ve import ettiğiniz resmin uygulmada görüntülenmiş olduğunu gözlemleyin.
\begin{verbatim}
   # tuxpaint-import /usr/share/tuxpaint/stamps/vehicles/ship/walnutBoat.png
\end{verbatim} 
\end{itemize}

\item inkscape paketi kurulumu sonrası:

Aşağida bulunan dosyayı inkscape uygulaması ile açın ve üzerinde birkaç değişiklik yapın. Sorunsuz bir şekilde çalışabildiklerini gözlemleyin.
\begin{verbatim}
# wget http://cekirdek.pardus.org.tr/~semen/dist/test/multimedia/graphics/drawing.svg 
\end{verbatim}

\item asymptote paketi kurulumu sonrası:

Aşağıda bulunan komutların sorunsuz bir şekilde çalıştığını gözlemleyin:
\begin{verbatim}
# wget http://cekirdek.pardus.org.tr/~semen/dist/test/multimedia/graphics/test_asymptote.tex
# latex test_asymptote
# asy test_asymptote
# latex test_asymptote
# okular test_asymptote.dvi
\end{verbatim}
\item  dcmtk paketi kurulumu sonrası:

Aşağıda bulunan komutlaruın düzgün bir şekilde çalıştırğını gözlemleyin:
\begin{verbatim}
# wget http://cekirdek.pardus.org.tr/~semen/dist/test/multimedia/graphics/test_dcmtk.dcm 
# dcmj2pnm test_dcmtk.dcm  test.png
# gwenview test.png
\end{verbatim}

\item dcraw paketi kurulumu sonrası:

Aşağıda bulunan komutların düzgün bir şekilde çalıştıklarını gözlemleyin.  
\begin{verbatim}
# wget http://cekirdek.pardus.org.tr/~semen/dist/test/multimedia/graphics/test_dcraw.jpg 
# dcparse test_dcraw.jpg 
\end{verbatim}

\end{enumerate}

\section{Editor alt Bileşeni}

\begin{enumerate}
\item openshot paketi kurulumu sonrası:

Uygulamayı kmenu'den açın ve sorunsuz çalıştığını gözlemleyin.

\item pitivi paketi kurulumu sonrası:

Uygulamayı kmenü'den açın. 

Aşağıda bulunan dosyayı indirin. 
\begin{verbatim}
 wget http://cekirdek.pardus.org.tr/~semen/dist/test/multimedia/video/cokluortam/Lake_dance_XviD.AVI
\end{verbatim}

Uygulamanın boş olan beyaz alanına sağ tıklayın. "Import clips" seçeneğini seçin ve indirmiş olduğunuz dosyayı buraya ekleyin. Bu dosya beyaz alana eklenecektir, bu eklenmiş görüntüyü sağ bölümde bulunan video penceresine götürün ve dosyanın ses ve görüntü olarak sorunsuz bir şekilde çalıştığını gözlemleyin.

\item python-eyeD3 paketi kurulumu sonrası:

Aşğıda bulunan komutları çalıştırın ve sorunsuz bir şekilde id3 tag'inin listelediğini gözlemleyin.
\begin{verbatim}
 # wget http://cekirdek.pardus.org.tr/~semen/dist/test/multimedia/sound/sound/sample.mp3
 # eyeD3 linux.mp3 
\end{verbatim}

\item lilypond paketi kurulumu sonrası:

  multimedia-tr.pdf frescobaldi testini gerçaklaştiriniz.

\item lilycomp paketi kurulumu sonrası:
   
 Kmenüden uygulamayı çalıştırın ve uygulamanın yukarısında bulunan notalara basın ve alt bölümde bu notaların kodlarının basıldığını gözlemleyin.

\item kino paketi kurulumu sonrası:

  Kmenüden uygulamayı çalıştırın ve aşağıda bulunan dosyayı açın ve trim butonuna basarak çeşitli noktalardan kesmeye çalışın.
\begin{verbatim}
 # wget http://cekirdek.pardus.org.tr/~semen/dist/test/multimedia/editor/sample.dv
\end{verbatim}


\item kid3 paketi kurulumu sonrası:

Kmenüden uygulamayı çalıştırın ve aşağıda bulunan dosyayı Dosya $\rightarrow$ Aç yolunu izleyerek açın ve tag bölümünü editlemeye çalışın, editleyebildiğinizi gözlemleyin.

\begin{verbatim}
# wget http://cekirdek.pardus.org.tr/~semen/dist/test/multimedia/video/cokluortam/linux.mp3 
\end{verbatim}

 \item blender paketi kurulumu sonrası:

Aşağıda bulunan dosyayı blender ile açın ve Render butonuna basın. Resmi render edebildiğinizi gözlemleyin.
\begin{verbatim}
wget http://cekirdek.pardus.org.tr/~semen/dist/test/multimedia/editor/MATtests1.blend 
\end{verbatim}

 \item dvd-slideshow paketi kurulumu sonrası:

Aşağıdaki komutları çalıştırın ve bir slideshow oluştuğunu gözlemleyin.
\begin{verbatim}
wget http://cekirdek.pardus.org.tr/~semen/dist/test/multimedia/editor/image.tar.gz
dir2slideshow -n test -s "slide test" image
dvd-slideshow image.txt
mplayer image.vob 
\end{verbatim}
\item kdenlive paketi kurulumu sonrası:

Kdenlive uygulamasını açın ve Projeler $\rightarrow$ Klip ekle yolunu izleyerek aşağıda bulunan dosyayı ekleyin ve çalıştırın sesinin ve görüntüsünün sorunsuz olduğunu gözlemleyin.
\begin{verbatim}
wget http://cekirdek.pardus.org.tr/~semen/dist/test/multimedia/video/cokluortam/DVD.mpg
\end{verbatim}

\item kiconedit paketi kurulumu sonrası:

Menüden uygulamayı açın, Dosya $\rightarrow$ Aç yolunu izleyerek /usr/kde/4/share/apps/amarok/icons/hicolor/16x16/actions/ dizini altından bir ikonu açın ve düzgün bir şekilde açılabildiğini gözlemleyin.

\item dvdauthor paketi kurulumu sonrası:

multimedia-tr.pdf dvd-slideshow ve kdenlive testini gerçekleştiriniz.

 \item avidemux-common paketi kurulumu sonrası:

 multimedia-tr.pdf avidemux testini gerçekleştiriniz.

 \item avidemux ve avidemux-qt paketleri kurulumu sonrası:

\begin{verbatim}
  wget http://cekirdek.pardus.org.tr/~semen/dist/test/multimedia/video/cokluortam/Lake_dance_XviD.AVI
  wget http://cekirdek.pardus.org.tr/~semen/dist/test/multimedia/video/cokluortam/MPEG-1_with_
VCD_extensions.mpeg
\end{verbatim}
Yukarıda bulunan dosyaları ilgili uygulama ile açın. Go $\rightarrow$ Play/Stop düymesine basın ve açtığınız videonun ses ve görüntü bakımından sorunsuz bir şekilde çalıştığını gözlemleyin.
\item avidemux-cli paketi kurulumu sonrası:

video.mpeg dosyasının sorunsuz bir şekilde oluşmuş olduğunu gözlemleyin.
\begin{verbatim}
  wget http://cekirdek.pardus.org.tr/~semen/dist/test/multimedia/video/cokluortam/Lake_dance_XviD.AVI
  avidemux2_cli --force-alt-h264 --load "Lake_dance_XviD.AVI" --save "video.mpeg" 
--output-format MPEG --quit 
  mplayer video.mpeg
\end{verbatim}

\end{enumerate}

\end{document}

