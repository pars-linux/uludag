\documentclass[a4paper,10pt]{article}

\usepackage[turkish]{babel}
\usepackage[utf8]{inputenc}

%opening
\title{Efficient Bug Reporting}

\begin{document}

\maketitle

Eficient Bug Report Draft

\begin{itemize}
 
  	\item \subsection*{A smooth bug report is so valuable for its developer or maintainer. Therefore, the first thing, you should read the informations on this document very carefully:}
	\begin{verbatim} 
 	http://bugs.pardus.org.tr/page.cgi?id=bug-writing.html#why
	\end{verbatim}

  	\item For the description part of bugzilla, a template is prepared. All Pardus testers will use this template to report bugs.
  	In Description textbox

   	Reproducible: (Always or arbitrary)	
   
    	Give a short description of the bug here.
   
   	Steps to reproduce:
   	\begin{enumerate}
    	\item First step.
    	\item Second step
    	\item ...
   	\end{enumerate}

	Actual results:
	Describe the actual results here. You can add the bug outputs.
	
	Expected results:
	Describe the expected according your computer hardware and system settings.
	
	\item An efficient bug reporting example:
	\begin{verbatim}
	http://bugs.pardus.org.tr/show_bug.cgi?id=10043
	\end{verbatim}
	\item And also don't forget to propose your solution/patch If you think you solved the problem.

  	\item If you can't figure out to which product you have to assign the bug but know the name of the application, you can find it with the following command:
	\begin{verbatim}
	pisi sf `which <filename>`
	\end{verbatim}
	The filename is the name of the file(a configuration file, a library, an executable application, etc.) which seems problematic to you.

	Let's assume that the systemsettings application crashes when trying to switch the color scheme but you don't know which package actually ships systemsettings. Just type the command below to find it out:
	\begin{verbatim}
	pisi sf `which systemsettings`
	Searching for /usr/kde/4/bin/systemsettings
	Package kdebase-workspace has file usr/kde/4/bin/systemsettings
	\end{verbatim}
	So this means that, the bug should be assigned to the kdebase-workspace package.

  \item \subsection*{The attachments are also important for developers in order to completely understand the problem:}

	Attention: You have to switch to the root user in order to correctly execute some of the commands below. Type the following command and enter your root password:
	\begin{verbatim}
	 su -
	\end{verbatim}

	\begin{enumerate}
	\item For X server related bugs:
	\begin{itemize}
		\item The outputs of the following commands shoud be attached:
		\begin{verbatim}
		lspci -nn > lspci.txt
		dmesg > dmesg.txt
		lsmod > lsmod.txt
		\end{verbatim}
		\item If the computer or the keyboard is still alive, it's really useful to also grab the X server logs:
		\begin{verbatim}
		cat /var/log/Xorg.0.log > xserver.txt
		\end{verbatim}
		\item If not, restart your computer and open it in vesa mode and take the log from:
		\begin{verbatim}
		cat /var/log/Xorg.0.log.old
		\end{verbatim}

		For all outputs, if X crashed, you can take the outputs of these command with the below procedure.
		\begin{itemize}
			\item Plug an usb stick to the computer.
			\item Mount it manually.
				\begin{verbatim}
    				mount /dev/<your_usb_stick_partition> /mnt/flash
				\end{verbatim}
			\item Copy the outputs to /mnt/flash
				\begin{verbatim}
     				cp <output> /mnt/flash
				\end{verbatim}
			\item Unmount it manually.
				\begin{verbatim}
  				umount /dev/<your_usb_stick_partition>
				\end{verbatim}
		\end{itemize}
	\end{itemize}
	\item For Pardus specific applications' bugs:
	
	COMAR's log file is really helping in a lot of situations:
	\begin{verbatim}
	cat /var/log/comar3/trace.log > comar.txt
	\end{verbatim}

	\begin{itemize}
		\item For network-manager:
			\begin{verbatim}
			lspci -nn > lspci.txt
			\end{verbatim}
			\begin{itemize}
  			\item Ethernet specific problems:
				\begin{verbatim}
    				ifconfig -a > ifconfig.txt
				\end{verbatim}
  			\item Wireless specific problems:
				\begin{verbatim}
    				iwconfig > iwconfig.txt
				\end{verbatim}
			\end{itemize}
		\item For disk-manager:
			\begin{verbatim}
		    	fdisk -l > fdisk.txt
    			cat /etc/fstab > fstab.txt
			\end{verbatim}
		\item For service-manager:
			\begin{verbatim}
			service -N > service.txt
			\end{verbatim}
		\item For boot-manager:
			\begin{verbatim}
			cat /boot/grub/grub.conf > grub.txt
			\end{verbatim}
		\item For firewall-manager:
			\begin{verbatim}
			service -N > service.txt
			iptables > iptables.txt
			\end{verbatim}
	\end{itemize}
	\item For camera/video device related problems:
		
		The outputs of these commands shoud be attached after closing all applications which may use the camera device:
		\begin{verbatim}
		dmesg > dmesg.txt
		cat /var/log/syslog > syslog.txt
		lsusb > lsusb.txt
		test-webcam > webcam.txt
		\end{verbatim}
	\item For sound card related problems:

		Run the following command as root user and write down the WWW link that it gives you at the end:
		\begin{verbatim}
        alsa-info
		\end{verbatim}
	\end{enumerate}
\end{itemize}
\end{document}
