\documentclass[a4paper,10pt]{article}
%opening
\title{Efficient Bug Reporting}
\author{Semen Cirit}

\begin{document}

\maketitle

\section{Efficient Bug Report Draft}

\begin{itemize}
 
  	\item \subsection*{A smooth bug report is so valuable for its develpor or maintainer. Therefore, the first thing, you should read the informations of this webpage very carefully:}
	\begin{verbatim} 
 	http://bugs.pardus.org.tr/page.cgi?id=bug-writing.html#why
	\end{verbatim}

  	\item For the description part of bugzilla, a template is prepared. All Pardus testers will use this template to report bugs.
  	In Description textbox

   	Reproducible: (Always or arbitrary)	
   
    	Give a short description of the bug here.
   
   	Steps to reproduce:
   	\begin{enumerate}
    	\item First step.
    	\item Second step
    	\item ...
   	\end{enumerate}

	Actual results:
	Describe the actual results here. You can add the bug outputs.
	
	Expected results:
	Describe the expected according your computer hardware and system settings.
	
	\item An efficient bug reporting example:
	\begin{verbatim}
	http://bugs.pardus.org.tr/show_bug.cgi?id=10043
	\end{verbatim}
	\item And also if you solved the problem you can add the related patch to your bug.

  	\item And also one additional thing, if you don't know the erroneous product exactly, you can find it with this command
	\begin{verbatim}
	pisi sf `which <variable>`
	\end{verbatim}
	The variable can be any related file about the product. For example a bin file, a library file.

	For our example the effects can be configured from system settings. But there is no product about system settings so we can use the its binary file. So the command:
	\begin{verbatim}
	pisi sf `which systemsettings`
	\end{verbatim}

	And the output:
	\begin{verbatim}
	/usr/kde/4/bin/systemsettings file searching
	kdebase-workspace paketi içinde usr/kde/4/bin/systemsettings file is find.
	\end{verbatim}
	So this means, the bug is for the kdebase-workspace package.

  \item \subsection*{The attachments are also important for developers in order to understand the bug:}

	!!!For some commands you have to be root
	Execute this command try to be root enter your root password:
	\begin{verbatim}
	 su
	\end{verbatim}

	\begin{enumerate}
	\item For X bugs:
	\begin{itemize}
		\item The outputs of these commands shoud be attached.
		\begin{verbatim}
		lspci -nn > lspci.txt
		dmesg > dmesg.txt
		lsmod > lsmod.txt
		\end{verbatim}
		\item If the computer or the keyboard was not lock, you can take the log from 
		\begin{verbatim}
		cat /var/log/Xorg.0.log
		\end{verbatim}
		\item If not, restart your computer and open it in vesa mode and take the log from
		\begin{verbatim}
		cat /var/log/Xorg.conf.log.old
		\end{verbatim}

		For all outputs, if X crashed, you can take the outputs of these command with the below procedure.
		\begin{itemize}
			\item Plug an usb stick to the computer.
			\item Mount it manually.	
				\begin{verbatim}
    				mount /dev/<your_usb_stick_partition> /mnt/flash
				\end{verbatim}
			\item Copy the outputs to /mnt/flash	
				\begin{verbatim}
     				cp <output> /mnt/flash	
				\end{verbatim}
			\item Unmount it manually.
				\begin{verbatim}
  				umount /dev/<your_usb_stick_partition>
				\end{verbatim}
		\end{itemize}
	\end{itemize}
	\item For Pardus specific application bugs:
	
	All Pardus specific applications need:
	\begin{verbatim}
	cat /var/log/comar3/trace.log 
	\end{verbatim}

	\begin{itemize}
		\item For network-manager:
			\begin{verbatim}
			lspci > lspci.txt
			\end{verbatim}
			\begin{itemize}
  			\item If the problem is about ethernet.
				\begin{verbatim}
    				ifconfig -a > ifconfig.txt
				\end{verbatim}
  			\item If the problem is about ethernet.
				\begin{verbatim}
    				iwconfig > iwconfig.txt
				\end{verbatim}
			\end{itemize}
		\item For disk-manager:
			\begin{verbatim}
		    	fdisk -l >fdisk.txt
    			cat /etc/fstab
			\end{verbatim}
		\item For service-manager:
			\begin{verbatim}
			service -N > service.txt
			\end{verbatim}
		\item For boot-manager:
			\begin{verbatim}
			cat /boot/grub/grub.conf
			\end{verbatim}
		\item For boot-manager:
			\begin{verbatim}
			service -N > service.txt
			iptables > iptables.txt
			\end{verbatim}
	\end{itemize}
	\item  For camera devices:
		
		The outputs of these commands shoud be attached:
		\begin{verbatim}
		dmesg > dmesg.txt
		cat /var/log/syslog > syslog.txt
		lsusb > lsusb.txt
		test-webcam > webcame.txt (Please close all applications use camera device before execute test-webcam.)
		\end{verbatim}
	\item  For audio devices:

		The outputs of these commands shoud be attached:
		\begin{verbatim}
		ps axu|grep pulse > pulse.txt
		dmesg > dmesg.txt
		cat /var/log/syslog > syslog.txt
		cat /proc/asound/cards
		cat /proc/asound/*/codec*
		\end{verbatim}
	\end{enumerate}
\end{itemize}
\end{document}
