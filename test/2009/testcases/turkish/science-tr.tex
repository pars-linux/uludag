\documentclass[a4paper,10pt]{article}
\usepackage[turkish]{babel}
\usepackage[utf8]{inputenc}
\usepackage[left=1cm,top=1cm,right=2cm,bottom=2cm]{geometry}

\title{Science Bileşeni Test Aşamaları}
\author{Semen Cirit}

\renewcommand{\labelenumi}{\arabic{enumi}.}
\renewcommand{\labelenumii}{\arabic{enumi}.\arabic{enumii}.}
\renewcommand{\labelenumiii}{\arabic{enumi}.\arabic{enumii}.\arabic{enumiii}.}
\renewcommand{\labelenumiv}{\arabic{enumi}.\arabic{enumii}.\arabic{enumiii}.\arabic{enumiv}.}

\begin{document}

\maketitle
\section{Electronics alt Bileşeni}
\begin{enumerate}
 \item Aşağıda bulunan paketler sadece kurulum testine tabidir.
\begin{verbatim}
 gpsim
\end{verbatim}

\end{enumerate}


\section{Gis alt Bileşeni}
\begin{enumerate}
 \item gpsd ve libgps paketleri kurulumu sonrası:

Eğer gps'iniz var ise bu testi gerçekleştirebilirsiniz.

Gps'inizi takın. Servis yöneticisinden gpsd servisini başlatın.

Aşağıda bulunan komut ile servisin başlatıldığını gözlemleyin.
\begin{verbatim}
 service gpsd status
\end{verbatim}


\end{enumerate}


\section{Astronomy alt Bileşeni}
\begin{enumerate}
 \item stellarium paketi kurulumu sonrası:

Kmenüden uygulamayı açın ve düzgün bir şekilde açılabildiğini gözlemleyin. 
\end{enumerate}

\section{Chemistry alt Bileşeni}
\begin{enumerate}
 \item openbabel paketi kurulumu sonrası:

Aşağıda bulunan komutları çalıştırın ve test.smi dosyasında "c1cccc(c1C(=O)O)OC(=O)C·C9H8O4" formülünün yazılmış olduğunu gözlemleyin.
\begin{verbatim}
 # babel -H sdf
 # wget http://cekirdek.pardus.org.tr/~semen/dist/test/science/aspirin.sdf
 # babel -isdf 'aspirin.sdf' -osmi 'test.smi'
 # vi test.smi
\end{verbatim}

\end{enumerate}

\section{Mathematics alt Bileşeni}
\begin{enumerate}
\item octave paketi kurulumu sonrası:

\item rkward paketi kurulumu sonrası:

Uygulamayı açın ve Uygulama paneli üzerinden Plots bölümünü tıklayın, Barplot'u seçin ve burada listelenen verilerden birini seçin, ekleyin ve onaylayın. 

Bu durumun sonunda ilgili grafiğin sorunsuz bir şekilde oluştuğunu gözlemleyin.

\item wxMaxima paketi kurulumu sonrası:

Kmenüden uygulamanın sorunsuz olarak açılabildiğini gözlemleyin.

Birkaç matematiksel işlem yapın ve sorunsuz bir şekilde yapılabildiğini gözlemleyin.
\item maxima paketi kurulumu sonrası:

Aşağıdaki komutların sorunsuz bir şekilde çalıştığını gözlemleyin:
\begin{verbatim}
 # maxima
 144*17 - 9;
 144^25;
\end{verbatim}

\item Aşağıda bulunan paketler kurulum testine tabidir.
\begin{verbatim}
gfan 
\end{verbatim}

\end{enumerate}

\section{Robotics alt Bileşeni}
\begin{enumerate}
 \item opencv paketi kurulumu sonrası: (kamerası olanlar test edebilecektir.)

Resim çek butonuna basın sorunsuz bir şekilde ekranınyenilendiğini gözlemleyin
\begin{verbatim}
# wget http://svn.pardus.org.tr/projeler/facelock/pardus.py
# wget http://svn.pardus.org.tr/projeler/facelock/pardus.png
# python pardus.py
\end{verbatim}


\end{enumerate}

\end{document}

