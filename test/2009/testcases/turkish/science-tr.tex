\documentclass[a4paper,10pt]{article}
\usepackage[turkish]{babel}
\usepackage[utf8]{inputenc}
\usepackage[left=1cm,top=2cm,right=2cm,bottom=1cm]{geometry}

\title{Science Bileşeni Test Aşamaları}
\author{Semen Cirit}

\renewcommand{\labelenumi}{\arabic{enumi}.}
\renewcommand{\labelenumii}{\arabic{enumi}.\arabic{enumii}.}
\renewcommand{\labelenumiii}{\arabic{enumi}.\arabic{enumii}.\arabic{enumiii}.}
\renewcommand{\labelenumiv}{\arabic{enumi}.\arabic{enumii}.\arabic{enumiii}.\arabic{enumiv}.}

\begin{document}

\maketitle

\section{Mathematics alt Bileşeni}
\begin{enumerate}
\item rkward paketi kurulumu sonrası:

Uygulamayı açın ve Uygulama paneli üzerinden Plots bölümünü tıklayın, Barplot'u seçin ve burada listelenen verilerden birini seçin, ekleyin ve onaylayın. 

Bu durumun sonunda ilgili grafiğin sorunsuz bir şekilde oluştuğunu gözlemleyin.
\end{enumerate}

\end{document}

