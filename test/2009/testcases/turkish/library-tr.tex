\documentclass[a4paper,10pt]{article}
\usepackage[turkish]{babel}
\usepackage[utf8]{inputenc}
\usepackage[left=1cm,top=2cm,right=2cm,bottom=1cm]{geometry}

\title{Library Test Aşamaları}
\author{Semen Cirit}

\renewcommand{\labelenumi}{\arabic{enumi}.}
\renewcommand{\labelenumii}{\arabic{enumi}.\arabic{enumii}.}
\renewcommand{\labelenumiii}{\arabic{enumi}.\arabic{enumii}.\arabic{enumiii}.}
\renewcommand{\labelenumiv}{\arabic{enumi}.\arabic{enumii}.\arabic{enumiii}.\arabic{enumiv}.}

\begin{document}

\maketitle
\begin{enumerate}
\item yazpp sadece kurulum testine tabidir.
\item exempi ve yaz paketi kurulumu sonrası:

office-tr.pdf tellico paketi testini gerçekleştiriniz.

\item eet ve eina paketi kurulumu sonrası
\begin{itemize}
 \item [2008 için] edb paketini kurun. Aşağıda buluna komutların sorunsuz bir şekilde çalıştığını ve sonuç olarak "default" diye bir çıktı döndürdüğünü gözlemleyin.
\begin{verbatim}
 # wget http://cekirdek.pardus.org.tr/~semen/dist/test/library/test_edb
 # chmod 755 test_edb
 # ./test_edb
 # edb_ed test.db get /foo/theme str
\end{verbatim}
\item [2009 için] library-tr.pdf qedje paketi testini gerçekleştiriniz.
\end{itemize}

\item qedje kurulumu sonrası:

Aşağıdaki komutu çalıştırdığınızda hata vermeden çalıştığını gözlemleyiniz.
\begin{verbatim}
# qedje_viewer
\end{verbatim}

\item geoip paketi kurulumu sonrası:
\begin{verbatim}
# geoiplookup www.google.com 
\end{verbatim}
"GeoIP Country Edition: US, United States" gibi bir sonuç döndürdüğünü gözlemleyin.

\item nss ve nspr paketleri kurulumu sonrası:

network-tr.pdf firefox ve office-tr.pdf openoffice testlerini gerçekleştiriniz.

\item openexr paketi kurulumu sonrası:
\begin{itemize}
 \item multimedia-tr.pdf gimp, digikam ve imagemagick testlerini geçekleştiriniz.
 \item Aşağıdaki bağlantıda bulunan resim dosyalarının gwenview ile sorunsuz bir şekilde açıldığını gözlemleyin.
  \begin{verbatim}
   http://cekirdek.pardus.org.tr/~semen/dist/test/multimedia/graphics/graphics.tar
  \end{verbatim}
\end{itemize}
\item iksemel paketi kurulumu sonrası:
\begin{verbatim}
 # wget http://cekirdek.pardus.org.tr/~semen/dist/test/library/component.xml
 # ikslint -s component.xml
 # iksperf -a component.xml
 
\end{verbatim}

\begin{itemize}
\item  Yukarıda bulunan komutların düzgün çalıştığını gözlemleyin.
\item office-tr.pdf imposter testini gerçekleştirin.
\end{itemize}

\item xulrunner paketi kurulumu sonrası:
\begin{itemize}
\item office-tr.pdf openoffice testlerini gerçekleştirin.
\item network-tr.pdf firefox testlerini gerçekleştirin.
\item network-tr.pdf gecko-mediaplayer testlerini gerçekleştirin.
\item multimedia-tr.pdf vlc-firefox testlerini gerçekleştirin.
\item desktop-tr.pdf google-gadgets-qt ve google-gadgets-gtk testlerini gerçekleştirin.
\end{itemize}

\end{enumerate}

\end{document}

