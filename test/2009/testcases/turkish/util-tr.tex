\documentclass[a4paper,10pt]{article}
\usepackage[turkish]{babel}
\usepackage[utf8]{inputenc}
\usepackage[left=1cm,top=1cm,right=2cm,bottom=2cm]{geometry}

\title{Util Bileşeni Test Aşamaları}
\author{Semen Cirit}

\renewcommand{\labelenumi}{\arabic{enumi}.}
\renewcommand{\labelenumii}{\arabic{enumi}.\arabic{enumii}.}
\renewcommand{\labelenumiii}{\arabic{enumi}.\arabic{enumii}.\arabic{enumiii}.}
\renewcommand{\labelenumiv}{\arabic{enumi}.\arabic{enumii}.\arabic{enumiii}.\arabic{enumiv}.}

\begin{document}

\maketitle
\section{Archive alt bileşeni}
\begin{enumerate}
\item rar paketi kurulumu sonrası:

Aşağıda bulunan komutun sorunsuz bir şekilde test.tar dosyasını oluşturduğunu gözlemleyin.
\begin{verbatim}
 mkdir test
 tar a test.tar test
\end{verbatim}


 \item lrzip paketi kurulumu sonrası:

Aşağıda bulunan komutların dosyayı önce sıkıştırıp sonra açtığını gözlemleyin.
\begin{verbatim}
 # wget http://cekirdek.pardus.org.tr/~semen/dist/test/util/test_lrzip
 # lrzip test_lrzip
 # mkdir test
 # lrzip -d test_lrzip.lrz -O test/
\end{verbatim}


\end{enumerate}


\section{Crypt alt bileşeni}
\begin{enumerate}
 \item mcrypt paketi kurulumu sonrası:

Aşağıda bulunan komutların sorunsuz çalıştığını gözlemleyin.
\begin{verbatim}
 # mcrypt --list
 # wget http://cekirdek.pardus.org.tr/~semen/dist/test/util/test_mcrypt
 # mcrypt -a blowfish test_mcrypt
 # mcrypt -d test_mcrypt.nc
 # vi test_mcrypt
\end{verbatim}


\end{enumerate}


\section{Antivirus alt bileşeni}
\begin{enumerate}
\item clamtk paketi kurulumu sonrası:

Kmenüden uygulamayı açın, imzanızı kaydedin ve çık butonuna basın, daha sonra Ev butaonuna basıp Ev dizini altında antivirus için tarama yapın.
Sorunsuz bir şekilde yapılabildiğini gözlemleyin.

\item Klamav paketi kurulumu sonrası:

Aşağıda bulunan dosyayı belirli bir dizin içerisine kopyalayın. 
\begin{verbatim}
# wget https://secure.eicar.org/eicar.com.txt
\end{verbatim}

Klamav uygulamasını kullanarak, bu dizini virüs taramasından geçirin, ve klamavın bu virüsü bulup karantinaya almak itediğini gözlemleyin.

\item Clamav paketi kurulumu sonrası:

Aşağıda bulunan dosyayı belirli bir dizin içerisine kopyalayın. 
\begin{verbatim}
# wget https://secure.eicar.org/eicar.com.txt
# clamscan eicar.com.txt
\end{verbatim}

Yukarıda verilen ikinci komutun, virüsü tespit edebildiğini gözlemleyin.

\end{enumerate}
\section{Admin alt bileşeni}
\begin{enumerate}
\item latencytop paketi kurulumu sonrası:

Aşağıda bulunan komutu çalıştırın ve sistemde açık olan uygulamaların işlem zamanlarının sorunsuz bir şekilde listelendiğini gözlemleyin.
\begin{verbatim}
sudo latencytop 
\end{verbatim}

 \item phpmyadmin paketi kurulumu sonrası:
\begin{itemize}
 \item Apache ve mysql sunucularını servis yöneticisinden başlatın. Aşağıda bulunan komutu çalıştırın ve root parolasını "test" yapın:
\begin{verbatim}
 mysqladmin -u root password 'test'
\end{verbatim}

 \item http://localhost/phpmyadmin/ adresine firefox kullanarak girin. (Kullanıcı adı için root yazın, parola ise test olacak)
 \item Mysql bağlantı sayfasınının sorunsuz açıldığını gözlemleyin.
\end{itemize}



 \item iotop paketi kurulumu sonrası:

Aşağıda bulunan komutu çalıştırın ve sistemde çalışan tüm uygulamaların I/O bant genişliğini listelediğini gözlemleyin.
\begin{verbatim}
 # iotop
\end{verbatim}

\end{enumerate}
\section{Shell alt bileşeni}
\begin{enumerate}
 \item bash-completion paketi kurulumu sonrası:

Aşağıda bulunan komutun hata alınmadan çalıştığını gözlemleyin.
\begin{verbatim}
 # pisi --help
\end{verbatim}

\item command-not-found paketi kurulumu sonrası:

Amsn paketi sisteminizde kurulu değil ise, aşağıda bulunan komutu yazdığınızda:
\begin{verbatim}
 # amsn
\end{verbatim}

Aşağıda bulunan çıktıyı verdiğini gözlemleyiniz:
\begin{verbatim}
'amsn' uygulaması sisteminizde kurulu değil. Bu paketi, paket yöneticisini kullanarak ya da aşağıdaki komutu çalıştırarak kurabilirsiniz:
sudo pisi it amsn
bash: amsn: command not found
\end{verbatim}

\end{enumerate}
\section{Misc alt bileşeni}
\begin{enumerate}
 \item fslint paketi kurulumu sonrası:

Kmenüden açılabildiğini gözlemleyin.

İki farklı dizine birer dosya kopyalayın ve uygulamaya bu iki farlı dizini ekleyin, kopya dosyalara tıklayıp bul tuşuna basın kopyalamış olduğunuz dosyanın listelenmiş olduğunu gözlemleyin.
 \item ltrace paketi kurulumu sonrası:

Aşağıda bulunan komutun hatasız çalıştığını gözlemleyin:
\begin{verbatim}
 # ltrace ls
\end{verbatim}
\item elfutils paketi kurulumu sonrası:

util-tr.pdf ltrace testini gerçekleştirin.
\end{enumerate}



\end{document}

