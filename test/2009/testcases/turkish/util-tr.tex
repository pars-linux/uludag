\documentclass[a4paper,10pt]{article}
\usepackage[turkish]{babel}
\usepackage[utf8]{inputenc}
\usepackage[left=1cm,top=1cm,right=2cm,bottom=2cm]{geometry}

\title{Util Bileşeni Test Aşamaları}
\author{Semen Cirit}

\renewcommand{\labelenumi}{\arabic{enumi}.}
\renewcommand{\labelenumii}{\arabic{enumi}.\arabic{enumii}.}
\renewcommand{\labelenumiii}{\arabic{enumi}.\arabic{enumii}.\arabic{enumiii}.}
\renewcommand{\labelenumiv}{\arabic{enumi}.\arabic{enumii}.\arabic{enumiii}.\arabic{enumiv}.}

\begin{document}

\maketitle
\section{Archive alt bileşeni}
\begin{enumerate}
 
\end{enumerate}

\section{Antivirus alt bileşeni}
\begin{enumerate}
\item Klamav paketi kurulumu sonrası:

Aşağıda bulunan dosyayı belirli bir dizin içerisine kopyalayın. 
\begin{verbatim}
# wget https://secure.eicar.org/eicar.com.txt
\end{verbatim}

Klamav uygulamasını kullanarak, bu dizini virüs taramasından geçirin, ve klamavın bu virüsü bulup karantinaya almak itediğini gözlemleyin.

\item Clamav paketi kurulumu sonrası:

Aşağıda bulunan dosyayı belirli bir dizin içerisine kopyalayın. 
\begin{verbatim}
# wget https://secure.eicar.org/eicar.com.txt
# clamscan eicar.com.txt
\end{verbatim}

Yukarıda verilen ikinci komutun, virüsü tespit edebildiğini gözlemleyin.

\end{enumerate}
\section{Admin alt bileşeni}
\begin{enumerate}
 \item iotop paketi kurulumu sonrası

Aşağıda bulunan komutu çalıştırın ve sistemde çalışan tüm uygulamaların I/O bant genişliğini listelediğini gözlemleyin.
\begin{verbatim}
 # iotop
\end{verbatim}

\end{enumerate}
\section{Shell alt bileşeni}
\begin{enumerate}
 \item bash-completion paketi kurulumu sonrası:

Aşağıda bulunan komutun hata alınmadan çalıştığını gözlemleyin.
\begin{verbatim}
 # pisi --help
\end{verbatim}

\item command-not-found paketi kurulumu sonrası:

Amsn paketi sisteminizde kurulu değil ise, aşağıda bulunan komutu yazdığınızda:
\begin{verbatim}
 # amsn
\end{verbatim}

Aşağıda bulunan çıktıyı verdiğini gözlemleyiniz:
\begin{verbatim}
'amsn' uygulaması sisteminizde kurulu değil. Bu paketi, paket yöneticisini kullanarak ya da aşağıdaki komutu çalıştırarak kurabilirsiniz:
sudo pisi it amsn
bash: amsn: command not found
\end{verbatim}

\end{enumerate}
\section{Misc alt bileşeni}
\begin{enumerate}
 \item ltrace paketi kurulumu sonrası:

Aşağıda bulunan komutun hatasız çalıştığını gözlemleyin:
\begin{verbatim}
 # ltrace ls
\end{verbatim}
\item elfutils paketi kurulumu sonrası:

util-tr.pdf ltrace testini gerçekleştirin.
\end{enumerate}



\end{document}

