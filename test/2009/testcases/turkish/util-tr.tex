\documentclass[a4paper,10pt]{article}
\usepackage[turkish]{babel}
\usepackage[utf8]{inputenc}
\usepackage[left=1cm,top=2cm,right=2cm,bottom=1cm]{geometry}

\title{Util Bileşeni Test Aşamaları}
\author{Semen Cirit}

\renewcommand{\labelenumi}{\arabic{enumi}.}
\renewcommand{\labelenumii}{\arabic{enumi}.\arabic{enumii}.}
\renewcommand{\labelenumiii}{\arabic{enumi}.\arabic{enumii}.\arabic{enumiii}.}
\renewcommand{\labelenumiv}{\arabic{enumi}.\arabic{enumii}.\arabic{enumiii}.\arabic{enumiv}.}

\begin{document}

\maketitle
\section{Antivirus alt bileşeni}
\begin{enumerate}
\item Klamav paketi kurulumu sonrası:

Aşağıda bulunan dosyayı belirli bir dizin içerisine kopyalayın. 
\begin{verbatim}
# wget https://secure.eicar.org/eicar.com.txt
\end{verbatim}

Klamav uygulamasını kullanarak, bu dizini virüs taramasından geçirin, ve klamavın bu virüsü bulup karantinaya almak itediğini gözlemleyin.

\item Clamav paketi kurulumu sonrası:

Aşağıda bulunan dosyayı belirli bir dizin içerisine kopyalayın. 
\begin{verbatim}
# wget https://secure.eicar.org/eicar.com.txt
# clamscan eicar.com.txt
\end{verbatim}

Yukarıda verilen ikinci komutun, virüsü tespit edebildiğini gözlemleyin.

\end{enumerate}
\section{Admin alt bileşeni}
\begin{enumerate}
 \item iotop paketi kurulumu sonrası

Aşağıda bulunan komutu çalıştırın ve sistemde çalışan tüm uygulamaların I/O bant genişliğini listelediğini gözlemleyin.
\begin{verbatim}
 # iotop
\end{verbatim}


\end{enumerate}


\end{document}

