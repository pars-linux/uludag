\documentclass[a4paper,10pt]{article}
\usepackage[turkish]{babel}
\usepackage[utf8]{inputenc}
\usepackage[left=1cm,top=2cm,right=2cm,bottom=1cm]{geometry}

\title{Masaüstü Bileşeni Test Aşamaları}
\author{Semen Cirit}

\renewcommand{\labelenumi}{\arabic{enumi}.}
\renewcommand{\labelenumii}{\arabic{enumi}.\arabic{enumii}.}
\renewcommand{\labelenumiii}{\arabic{enumi}.\arabic{enumii}.\arabic{enumiii}.}
\renewcommand{\labelenumiv}{\arabic{enumi}.\arabic{enumii}.\arabic{enumiii}.\arabic{enumiv}.}

\begin{document}

\maketitle

\section{Toolkit Bileşeni}
\subsection*{Qt ve Qt4}

(Bu kısımda verilen paket adlarının qt ile başlayan bölümleri, 2008'de qt için qt, qt4 için qt4, 2009'da qt için qt3, qt4 için qt olacaktır.)
\begin{enumerate}
 \item Aşağıda bulunan paketler sadece kurulum testine tabidir.
\begin{verbatim}
 qt-doc
 qt-sql-ibase
 qt-sql-odbc
 qt-sql-postgresql
\end{verbatim}
 \item qt paketi kurulumu sonrası
\begin{verbatim}
 # mkdir test
 # cd test
 # wget http://cekirdek.pardus.org.tr/~semen/dist/test/desktop/toolkit/test.cpp
 # qmake-qt -project
 # qmake-qt
 # make
 # ./test
\end{verbatim}

"Hello qt4!" yazan bir pencerenin açıldığını gözlemleyin.
\item qt-designer paketi kurulumu sonrası

Menu $\rightarrow$ Programlar $\rightarrow$ Geliştirme yolunu izleyerek sorunsuz bir şekilde açıldığını gözlemleyin.

\item qt-linguist paketi kurulumu sonrası

Menu $\rightarrow$ Programlar $\rightarrow$ Geliştirme yolunu izleyerek sorunsuz bir şekilde açıldığını gözlemleyin.

\item qt-sql-mysql, qt-sql-sqlite paketi kurulumu sonrası
\begin{verbatim}
 # mkdir test
 # cd test
 # wget http://cekirdek.pardus.org.tr/~semen/dist/test/desktop/ 
toolkit/test-qt-sql-<ilgili_veritabanı>.cpp
 # qmake-qt -project
 # qmake-qt
\end{verbatim}
qmake komutundan sonra oluşan .pro dosyanıza QT += sql satırını eklemelisiniz. Daha sonra aşağıdaki komutları çalıştırın.
\begin{verbatim}
 # make
 # ./test
\end{verbatim}

Bağlamtının sorunsuz bir şekilde gerçekleştğini gözlemleyin.

\end{enumerate}
\section{Kde4 alt Bileşeni}
\subsection*{Base alt bileşeni}

2008'de tüm kde4 paketleri 4 versiyon numarasını içermekte iken, 2009'da bu versiyon numarasını içermemektedir.
\begin{enumerate} 
 \item kdelibs ve kdelibs-devel paketleri kurulumu sonrası:
\begin{itemize}
 \item choqok paketini kurun: 

Uygulamayı açın ve bir twitter üyeliğiniz var ise, bu üyelik bilgilerinizi kaydedin ve sorunsuz bir şekilde twitter'a bağlanabildiğinizi gözlemleyin.
 \item kdegraphics paketini kurun: (2008 içi kdegraphics4)
\begin{verbatim}
 # wget http://cekirdek.pardus.org.tr/~semen/dist/test/desktop/kde/base/circus-bw_hats.jpg
 # wget http://cekirdek.pardus.org.tr/~semen/dist/test/desktop/kde/base/tepecik_01.png
\end{verbatim}
Yukarıda bulunan dosyaların okular ve gwenview ile açıldığını gözlemleyin.
\item amarok paketini kurun: (2008 için amarok-kde4) 

\begin{verbatim}
/sda1/usr/kde/4/share/sounds/k3b_error1.wav
/sda1/usr/kde/4/share/sounds/KDE-Im-Irc-Event.ogg
# wget http://cekirdek.pardus.org.tr/~semen/dist/test/multimedia/sound/music.mp3
\end{verbatim}

Dosyalarının düzgün bir şekilde amarok ile açıldığını gözlemleyin.

\item yakuake paketini kurun: (2008 içi yakuake4)

F12 tuşuna basıldığında sorunsuz bir şekilde yakuake'nin açıldığını gözlemleyin.
\end{itemize}

\end{enumerate}

\end{document}

