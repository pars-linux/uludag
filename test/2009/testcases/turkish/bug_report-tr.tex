\documentclass[a4paper,10pt]{article}
\usepackage[left=1cm,top=2cm,right=2cm,bottom=1cm]{geometry}
\usepackage[turkish]{babel}
\usepackage[utf8]{inputenc}

%opening
\title{Etkili Bug Raporlama}
\author{Semen Cirit}
\begin{document}

\maketitle

\subsection*{Etkili Bug Raporu Taslağı}

\begin{itemize}
 
  	\item \subsection*{Düzgün bir  bug raporu, geliştiricisi veya sağlayıcısı için çok önemlidir.Bu sebeple ilk yapmanız gereken şey, aşağıdaki linkte bulunan bilgileri çok dikkatli bir şekilde okumaktır!!!}
	\begin{verbatim} 
 	http://bugs.pardus.org.tr/page.cgi?id=bug-writing.html#why
	\end{verbatim}


  	\item Bugzilla'da hata girişinin açıklama bölümü için bir şablon hazırlandı. Tüm Pardus test ekibi bu şablonu karşılaştıkları hataları raporlamak için kullanmalılar.

   	Tekrarlanabilirlik: (Daima, Bazen, veya Bazı sistemlerde)	
   
    	Bug'ın kısa bir açıklamasını buraya yazın.
   
   	Tekrarlanması için adımlar:
   	\begin{enumerate}
    	\item İlk adım.
    	\item İkinci adım
    	\item ...
   	\end{enumerate}

	Mevcut sonuçlar:

	Burada mevcut sonuçları açıklayınız. Bug çıktılarını ekleyebilirsiniz.
	
	Beklenen sonuçlar:

	Bilgisayarınızın donanım ve sistem ayarlarına dayanarak beklenenleri açıklayınız.
	
	\item Etkili bir bug raporlama örneğine aşağıda bulunan linkten bakabilirsiniz:
	\begin{verbatim}
	http://bugs.pardus.org.tr/show_bug.cgi?id=10135
	\end{verbatim}
	\item Ve ayrıca problemi çözdüğünüzü düşünüyorsanız, çözümünüzü ve/veya yamanızı eklemeyi unutmayın.

  	\item Eğer bug'ı hangi ürüne atayacağınızı belirleyemiyorsanız ama uygulamanın ismini veya hangi ilgili dosyada hata ile karşılaşıldığını biliyorsanız, onu aşağıdaki komutu kullanarak bulabilirsiniz:
	\begin{verbatim}
	pisi sf `which <dosyaadı>`
	\end{verbatim}
	dosyaadı, problemli gözüken dosyanın ismidir. (yapılandırma dosyası, kütüphane, çalıştırılabilir uygulama, vb.)

	Örneğin yukarıdaki örnek bug raporlama ile ilgili verilmiş linkte karşılaşığımız sorunu hangi ürüne atayacağımızı bulabilmek için:

	Bildiğimiz gibi efekteler ile ilgili değişiklikler sistem ayarlarından yapılmaktadır. Bu durumda bu örnekte karşılaşmış olduğumuz sorun sistem ayarları ile ilgilidir. Fakat bugzillada hata sayfası girişindeki ürünlere baktığımızda sistem ayarlarını burada  görememekteyiz. Ve diyelim ki hangi paketin sistem ayarları uygulamasından sorumlu olduğunu bilmiyorsunuz. Bu durumda bu uygulamayı çalıştıran bin dosyasını bulup (bu örnek için system settings) aşağıdaki komutu girdiğinizde ilgili paketi bulabileceksiniz:
	\begin{verbatim}
	pisi sf `which systemsettings`
	Searching for /usr/kde/4/bin/systemsettings
	Package kdebase-workspace has file usr/kde/4/bin/systemsettings
	\end{verbatim}
	Bu şu anlama geliyor: bug, kdebase-workspace paketine atanmalıdır.

  \item \subsection*{Problemi doğru bir şekilde anlayabilmek için ek dosyalar geliştiriciler için çok önemlidir.}

	Dikkat: Aşağıdaki bazı komutları çalıştırabilmeniz için root kullanıcısı olmalısınız. 

	Bu durumlarda aşağıdaki komutu girerek root olmalısınız:
	\begin{verbatim}
	 su -
	\end{verbatim}

	\begin{enumerate}
	\item X server bugları için:
	\begin{itemize}
		\item Aşağıdaki komutların çıktıları eklenmelidir:
		\begin{verbatim}
		lspci -nn > lspci.txt
		dmesg > dmesg.txt
		lsmod > lsmod.txt
		\end{verbatim}
		\item Eğer bilgisayar veya klavye hala çalışabiliyorsa, X server logları da çok yararlı olacaktır.
		\begin{verbatim}
		cat /var/log/Xorg.0.log > xserver.txt
		\end{verbatim}
		\item Eğer çalışmıyorlarsa, izlemeniz gereken yol:

		Bilgisayarınızı yeniden başlatın, vesa modunda açın ve gerekli olan logu şu şekilde alın:
		\begin{verbatim}
		cat /var/log/Xorg.0.log.old
		\end{verbatim}

		Bütün çıktılar için, eğer X çökmüş ise, bu çıktıları aşağıdaki prosedürle alabilirsiniz:
		\begin{itemize}
			\item $CTRL+ALT+F1$ tuşlarına aynı anda basınız.(Bu komut ile sitem konsoluna geçmiş olmalısınız.)
			\item Bilgisayarınıza usb bellek takınız.
			\item Manuel bir şekilde bağlayınız.
				\begin{verbatim}
    				mount /dev/<your_usb_stick_partition> /mnt/flash
				\end{verbatim}
			\item X için gerekli olan çıktıları /mnt/flash dizinine kopyalayınız
				\begin{verbatim}
     				cp <output> /mnt/flash
				\end{verbatim}
			\item Bağı manuel bir şekilde kaldırınız.
				\begin{verbatim}
  				umount /dev/<your_usb_stick_partition>
				\end{verbatim}
		\end{itemize}
	\end{itemize}
	\item Pardus'a özel uygulamaların bugları için:
	
	Birçok durum için COMAR'ın log dosyası yardımcı olabilir
	\begin{verbatim}
	cat /var/log/comar3/trace.log > comar.txt
	\end{verbatim}

	\begin{itemize}
		\item  network-manager için:
			\begin{verbatim}
			lspci -nn > lspci.txt
			\end{verbatim}
			\begin{itemize}
  			\item Ethernet'e özel problemler:
				\begin{verbatim}
    				ifconfig -a > ifconfig.txt
				\end{verbatim}
  			\item Wireless'a özel problemler:
				\begin{verbatim}
    				iwconfig > iwconfig.txt
				\end{verbatim}
			\end{itemize}
		\item disk-manager için:
			\begin{verbatim}
		    	fdisk -l > fdisk.txt
    			cat /etc/fstab > fstab.txt
			\end{verbatim}
		\item service-manager için:
			\begin{verbatim}
			service -N > service.txt
			\end{verbatim}
		\item boot-manager için:
			\begin{verbatim}
			cat /boot/grub/grub.conf > grub.txt
			\end{verbatim}
		\item firewall-manager için:
			\begin{verbatim}
			service -N > service.txt
			iptables > iptables.txt
			\end{verbatim}
	\end{itemize}
	\item Kamera/video aygıtları ile ilgili problemler için:
		
		Bu komutların çıktıları kamera aygıtını kullanabilecek olan tüm uygulamaları kapattıktan sonra alınmalıdır.
		\begin{verbatim}
		dmesg > dmesg.txt
		cat /var/log/syslog > syslog.txt
		lsusb > lsusb.txt
		test-webcam > webcam.txt
		\end{verbatim}
	\item Ses kartı ile ilgili problemler için:

		Aşağıdaki komutu root kullanıcısı olarak çalıştırın ve en son olarak çıktıda bulunacak olan WWW linkini not alın:
		\begin{verbatim}
        	alsa-info
		\end{verbatim}
	\end{enumerate}
\end{itemize}
\end{document}
