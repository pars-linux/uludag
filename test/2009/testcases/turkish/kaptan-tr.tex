\documentclass[a4paper,10pt]{article}
\usepackage[turkish]{babel}
\usepackage[utf8]{inputenc}

\title{Kaptan Test Kriterleri}
\author{Semen Cirit}

\renewcommand{\labelenumi}{\arabic{enumi}.}
\renewcommand{\labelenumii}{\arabic{enumi}.\arabic{enumii}.}
\renewcommand{\labelenumiii}{\arabic{enumi}.\arabic{enumii}.\arabic{enumiii}.}
\renewcommand{\labelenumiv}{\arabic{enumi}.\arabic{enumii}.\arabic{enumiii}.\arabic{enumiv}.}

\begin{document}

\maketitle

\begin{enumerate}

\item 1440*900 den 800*600 a kadar tüm çözünürlükleri deneyin.

\item Yeni bir kullanıcı yaratın ve bu hesabı açın.
\begin{enumerate}
    \item Kaptan hemen açılıyor mu?
    \item Kaptan açıldıkdan sonra

          şu komutu çalıştırın:
\begin{verbatim}
    # vi ~/.kde4/share/config/kaptanrc
\end{verbatim} 
        RunOnStart' ın False olduğunu gözlemleyin.

    \item Fare davranışı
    \begin{enumerate}
        \item Çift tıklamayı seçin.

            Durumu gözlemleyin.
        \item Tek tıklamayı seçin.

            Durumu gözlemleyin.

        \item Sol eli seçin.

            Durumu gözlemleyin.
        \item Sağ eli seçin.

            Durumu gözlemleyin.

        \item Ters scroll'ü aktif hale getirin,

            Durumu gözlemleyin.
    \end{enumerate}
\end{enumerate}

\item Kaptan sonlandığında alttaki bütün değişiklikler gözlenebilmelidir. Her durum tek tek gözlenmelidir.
(Her bir durum için kaptan, console dan ``kaptan'' komutuyla başlatılabilir.)
    \begin{enumerate}
    \item Tema ve masaüstü tipi
        \begin{enumerate}
        \item Her bir durumu seçin ve gözlemleyin.
        \item Masaüstü tipi için Masaüstü görünümünü seçin ve değiştiğini onaylayın.
        \item Masaüstü tipi için Dosya görünümünü seçin ve değiştiğini onaylayın.
        \item Masaüstü sayısını değiştirin ve gerçekleştiğini onaylayın.
        \end{enumerate}

    \item Menü Stili.
        \begin{enumerate}
        \item Her durumu gözlemleyin.
        \end{enumerate}

    \item Duvar Kağıdı
        \begin{enumerate}
        \item Bir duvar kağıdı seçmeyi
            \begin{enumerate}
            \item Kaptandan deneyin, duvar kağıdının değiştiğini onaylayın.
            \item Sistem ayarlarından değiştirin ve değiştiğini onaylayın.
            \end{enumerate}
        \item Duvar kağıdı seçmeden devam edin.
        \end{enumerate}

    \item Nepomuk Semantik Masaüstü
        \begin{enumerate}
        \item Etkinleştirin ve durumu gözlemleyin.
        \item Devre dışı bırakın ve durumu gözlemleyin.
        \end{enumerate}

    \item Ağ ayarları
        \begin{enumerate}
        \item Ağ Yöneticisi test kriterlerini uygulayın.
        \end{enumerate}

    \item Paket Yöneticisi ayarları
        \begin{enumerate}
        \item Güncellemeler
            \begin{enumerate}
            \item Paket Yöneticisinin sistem çubuğuna yerleşmesini etkinleştirin.
                Sistem çubuğunda olduğunu onaylayın.
            \item Paket Yöneticisinin sistem çubuğuna yerleşmesini devre dışı bırakın.
                Sistem çubuğunda olmadığını onaylayın.
            \item Periyodik olarak güncellemeleri kontrol etmesini etkinleştirin ve gözlemlenebilir bir periyot zamanı verin.
                Sistem çubuğunda güncelleme için uyarı çıktığını onaylayın.
            \item Peryiodik güncellemeyi devre dışı bırakın.
                Sistem çubuğunda güncelleme için herhangi bir uyarı gelmediğini onaylayın.
            \end{enumerate}
        \end{enumerate}

        \begin{enumerate}
        \item Depo
            \begin{enumerate}
            \item Katkı deposunu etkinleştirin
                ardından şu komutu girin
\begin{verbatim}
    # pisi lr 
\end{verbatim}
                 Komutun çıktısında deponun eklendiğini onaylayın.
            \item Katkı deposunu devre dışı bırakın
                ardından şu komutu girin
\begin{verbatim}
    # pisi lr 
\end{verbatim}
                 Komutun çıktısında deponun olmadığını onaylayın.
            \end{enumerate}
        \end{enumerate}

    \item Pardus ayarları
        \begin{enumerate}
        \item Sistem ayarlarını açın
        \item Kullanıcı gruplarını ve yardım sayfalarını açmaya çalışın.
        \end{enumerate}

    \item Smolt donanım profilcisi
        \begin{enumerate}
        \item Smolt 'u etkinleştirin.
            Bilgisayar donanım profilinin smolt.pardus.org.tr 'ya yollandığını onaylayın.
        \end{enumerate}
    \end{enumerate}
\end{enumerate}

\end{document}
