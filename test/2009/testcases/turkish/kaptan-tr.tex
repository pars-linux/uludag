\documentclass[a4paper,10pt]{article}
\usepackage[turkish]{babel}
\usepackage[utf8]{inputenc}
\usepackage[left=1cm,top=2cm,right=2cm,bottom=1cm]{geometry}

\title{Kaptan Test Aşamaları}
\author{Semen Cirit}

\renewcommand{\labelenumi}{\arabic{enumi}.}
\renewcommand{\labelenumii}{\arabic{enumi}.\arabic{enumii}.}
\renewcommand{\labelenumiii}{\arabic{enumi}.\arabic{enumii}.\arabic{enumiii}.}
\renewcommand{\labelenumiv}{\arabic{enumi}.\arabic{enumii}.\arabic{enumiii}.\arabic{enumiv}.}

\begin{document}

\maketitle

\begin{enumerate}

\item 1440*900 den 800*600 a kadar tüm çözünürlüklerde Kaptanın denenmesi.

	Kaptanın otomatik olarak açıldığını gözlemleyin.
\item 4:3 ve 16:9 ekranda Kaptanın denenmesi.

	Kaptanın otomatik olarak açıldığını gözlemleyin.

\item Yeni bir kullanıcı yaratın ve bu hesabı açın.

\begin{enumerate}
    \item Kaptanın otomatik olarak açıldığını gözlemleyin.
    \item Kaptan açıldıkdan sonra

          şu komutu çalıştırın:
\begin{verbatim}
    # vi ~/.kde4/share/config/kaptanrc
\end{verbatim} 
        RunOnStart değişkeninin False olduğunu gözlemleyin.

    \item Fare davranışı
    \begin{enumerate}
        \item Çift tıklamayı seçin.

           Bir dosyanın çift tıklamayla açılabildiğini gözlemleyin.
        \item Tek tıklamayı seçin.

            Bir dosyanın çift tıklamayla açılabildiğini gözlemleyin.
        \item Sol eli seçin.

            Farenin sol el ile kullanılabildiğini gözlemleyin.
        \item Sağ eli seçin.

            Farenin sağ el ile kullanılabildiğini gözlemleyin.

        \item Ters kaydırmayı aktif hale getirin,

            Firefox'ta açılan bir sayfanın ters kaydırma kullanılarak kaydırılabildiğini gözlemleyin.
    \end{enumerate}
\end{enumerate}

\item Kaptan sonlandığında alttaki bütün değişiklikler gözlenebilmelidir. 

   Her durumun test edilebilmesi için, Kaptan her seferinde tekrar başlatılmalıdır.

   Kullanılacak komut:
	\begin{verbatim}
	# kaptan
	\end{verbatim}

    \begin{enumerate}
    \item Tema ve masaüstü tipi
        \begin{enumerate}
        \item Masaüstü için Masaüstü teması seçin.

	Kaptan sonlandığında, bu temanın şeçilmiş olduğunu gözlemleyin.
	
	Bu temanın, sistem ayarları $\rightarrow$ Görünüm $\rightarrow$ Biçim yolu izlendiğinde, Parçacık biçimi için ilgili temanın seçilmiş olduğunu gözlemleyin.

	Bu temanın, sistem ayarları $\rightarrow$ Görünüm $\rightarrow$ Simgeler yolu izlendiğinde, Simge için ilgili temanın seçilmiş olduğunu gözlemleyin.

	Bu temanın, sistem ayarları $\rightarrow$ Görünüm $\rightarrow$ Pencereler yolu izlendiğinde, Pencere Dekorasyonu için ilgili temanın seçilmiş olduğunu gözlemleyin.

        \item Masaüstü için Masaüstü tipi seçin. (Masaüstü veya Dizin görünümü)

	Kaptan sonlandığında, bu tipin seçilmiş olduğunu gözlemleyin.

	Bu tipin, Masaüstü sağ tık $\rightarrow$ Görünüm Ayarları $\rightarrow$ Masaüstü Eylemi yolu izlendiğinde, tip için ilgili tipin seçilmiş olduğunu gözlemleyin.

        \item Masaüstü sayısını değiştirin.

	Kaptan sonlandığında, masaüstü sayısınının verdiğiniz sayıya eşit olduğunu gözlemleyin.
        \end{enumerate}

    \item Bir menü stili seçin

        Kaptan sonlandığında, bu stilin seçilmiş olduğunu gözlemleyin.

    \item Duvar Kağıdı (Her bir aşama Kaptan tekrar başlatılarak denenmelidir.)
	\begin{enumerate}
	\item Kaptan üzerinden duvar kağıdını şeçmeyi deneyin.

	Kaptan sonlandığında, bu duvar kağıdının seçilmiş olduğunu gözlemleyin.
	\item Sistem ayarlarından duvar kağıdını değiştirin.

	Kaptan sonlandığında, bu duvar kağıdının seçilmiş olduğunu gözlemleyin.
	\item Duvar kağıdını değiştirme seçeneğini işaretleyin.

	Kaptan sonlandığında, bu duvar kağıdının seçilmiş olduğunu gözlemleyin.
	\end{enumerate}

    \item Nepomuk Semantik Masaüstü
        \begin{enumerate}
        \item Etkinleştirin.
	
	Kaptan sonlandığında, Nepomuk Semantik Masaüstü'nün etkinleştirilmiş olduğunu gözlemleyin. (Sistem çekmesine ilgili ikonun çıkması gerekmekte.)
        \item Devre dışı bırakın.

	Kaptan sonlandığında, Nepomuk Semantik Masaüstü'nün devre dışı bırakılmış olduğunu gözlemleyin.
        \end{enumerate}

    \item Ağ ayarları
        \begin{enumerate}
        \item İlgili dökümandan Ağ Yöneticisi test aşamalarını uygulayın.
        \end{enumerate}

    \item Paket Yöneticisi ayarları
        \begin{enumerate}
        \item Güncellemeler
            \begin{enumerate}
            \item Paket Yöneticisinin sistem çekmecesine yerleşmesini etkinleştirin.

                Kaptan sonlandığında, paket yöneticisi ikonunun sistem çekmecesine eklendiğini gözlemleyin.
            \item Paket Yöneticisinin sistem çekmecesine yerleşmesini devre dışı bırakın.

                Kaptan sonlandığında, paket yöneticisi ikonunun sistem çekmecesinde bulunmadığını gözlemleyin.
            \item Periyodik olarak güncellemeleri kontrol etmesini etkinleştirin ve gözlemlenebilir bir periyot zamanı verin.
                
		Kaptan sonlandığında, güncelleme kontrolünün belirlediğiniz aralıklar ile gerçekleştiğini gözlemleyin
            \end{enumerate}
        \end{enumerate}

        \begin{enumerate}
        \item Depo
            \begin{enumerate}
            \item Katkı deposunu etkinleştirin.

                Ardından şu komutu girin:
		\begin{verbatim}
		# pisi lr 
		\end{verbatim}
                 Kaptan sonlandığında, komutun çıktısında deponun eklendiğini gözlemleyin.
            \item Katkı deposunu devre dışı bırakın.
                Ardından şu komutu girin:
		\begin{verbatim}
		# pisi lr 
		\end{verbatim}
                 Kaptan sonlandığında, komutun çıktısında deponun olmadığını gözlemleyin.
            \end{enumerate}
        \end{enumerate}

    \item Pardus ayarları
        \begin{enumerate}
        \item Sistem ayarlarını başlatın.

	Sorunsuz bir şekilde sistem ayarlarının başlatılabildiğini gözlemleyin.

        \item Kullanıcı gruplarını ve yardım sayfalarını açmaya çalışın.

	Sorunsuz bir şekilde kullanıcı gruplarını ve yardım sayfalarının açıldığını gözlemleyin.
        \end{enumerate}

    \item Smolt'u etkinleştirin. 
 
      Bilgisayar donanım profilinin http://smolt.pardus.org.tr:8090/'ye yollandığını gözlemleyin. (Size ait UUID numarasını girerek gözlemleyebilirsiniz.)
    \end{enumerate}
\end{enumerate}

\end{document}
