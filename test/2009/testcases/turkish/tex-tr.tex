\documentclass[a4paper,10pt]{article}
\usepackage[turkish]{babel}
\usepackage[utf8]{inputenc}
\usepackage[left=1cm,top=2cm,right=2cm,bottom=1cm]{geometry}

\title{Tex Bileşeni Test Aşamaları}
\author{Semen Cirit}

\renewcommand{\labelenumi}{\arabic{enumi}.}
\renewcommand{\labelenumii}{\arabic{enumi}.\arabic{enumii}.}
\renewcommand{\labelenumiii}{\arabic{enumi}.\arabic{enumii}.\arabic{enumiii}.}
\renewcommand{\labelenumiv}{\arabic{enumi}.\arabic{enumii}.\arabic{enumiii}.\arabic{enumiv}.}

\begin{document}

\maketitle
\section{Addon alt bileşeni}
\begin{enumerate}
\item  latex-beamer kurulumu sonrası:
\begin{verbatim}
 # wget http://cekirdek.pardus.org.tr/~semen/dist/test/tex/addon/test_latexbeamer.tex
\end{verbatim}

Kile paketini kurun ve yukarıda verilen dosyayı kile ile açın ve QuickBuild  seçeneğini tıklayın ve sorunsuz olarak pdf dosyasının oluşturğunu gözlemleyin.
\item  latex-curvita kurulumu sonrası:
\begin{verbatim}
 # wget http://cekirdek.pardus.org.tr/~semen/dist/test/tex/addon/test_latexcurrvita.tex
\end{verbatim}

Kile paketini kurun ve yukarıda verilen dosyayı kile ile açın ve QuickBuild  seçeneğini tıklayın ve sorunsuz olarak pdf dosyasının oluşturğunu gözlemleyin.
\item  latex-envlab kurulumu sonrası:
\begin{verbatim}
 # wget http://cekirdek.pardus.org.tr/~semen/dist/test/tex/addon/test_latexenvlab.tex
\end{verbatim}

Kile paketini kurun ve yukarıda verilen dosyayı kile ile açın ve QuickBuild  seçeneğini tıklayın ve sorunsuz olarak pdf dosyasının oluşturğunu gözlemleyin.
\item  latex-europecv kurulumu sonrası:
\begin{verbatim}
 # wget http://cekirdek.pardus.org.tr/~semen/dist/test/tex/addon/test_latexeuropecv.tex
\end{verbatim}

Kile paketini kurun ve yukarıda verilen dosyayı kile ile açın ve QuickBuild  seçeneğini tıklayın ve sorunsuz olarak pdf dosyasının oluşturğunu gözlemleyin.
\item  latex-feynmf kurulumu sonrası:
\begin{verbatim}
 # wget http://cekirdek.pardus.org.tr/~semen/dist/test/tex/addon/test_latexfeynmf.tex
\end{verbatim}

Kile paketini kurun ve yukarıda verilen dosyayı kile ile açın ve QuickBuild  seçeneğini tıklayın ve sorunsuz olarak pdf dosyasının oluşturğunu gözlemleyin.
\item  latex-gbrief kurulumu sonrası:
\begin{verbatim}
 # wget http://cekirdek.pardus.org.tr/~semen/dist/test/tex/addon/test_latexgbrief.tex
\end{verbatim}

Kile paketini kurun ve yukarıda verilen dosyayı kile ile açın ve QuickBuild  seçeneğini tıklayın ve sorunsuz olarak pdf dosyasının oluşturğunu gözlemleyin.
\item  latex-glossaries kurulumu sonrası:
\begin{verbatim}
 # wget http://cekirdek.pardus.org.tr/~semen/dist/test/tex/addon/test_latexglossaries.tex
\end{verbatim}

Kile paketini kurun ve yukarıda verilen dosyayı kile ile açın ve QuickBuild  seçeneğini tıklayın ve sorunsuz olarak pdf dosyasının oluşturğunu gözlemleyin.
\item  latex-leaflet kurulumu sonrası:
\begin{verbatim}
 # wget http://cekirdek.pardus.org.tr/~semen/dist/test/tex/addon/test_latexleaflet.tex
\end{verbatim}

Kile paketini kurun ve yukarıda verilen dosyayı kile ile açın ve QuickBuild  seçeneğini tıklayın ve sorunsuz olarak pdf dosyasının oluşturğunu gözlemleyin.
\item  latex-maltese kurulumu sonrası:
\begin{verbatim}
 # wget http://cekirdek.pardus.org.tr/~semen/dist/test/tex/addon/test_latexmaltese.tex
\end{verbatim}

Kile paketini kurun ve yukarıda verilen dosyayı kile ile açın ve QuickBuild  seçeneğini tıklayın ve sorunsuz olarak pdf dosyasının oluşturğunu gözlemleyin.

\item  latex-mh kurulumu sonrası:
\begin{verbatim}
 # wget http://cekirdek.pardus.org.tr/~semen/dist/test/tex/addon/test_latexmh.tex
\end{verbatim}

Kile paketini kurun ve yukarıda verilen dosyayı kile ile açın ve QuickBuild  seçeneğini tıklayın ve sorunsuz olarak pdf dosyasının oluşturğunu gözlemleyin.
\item  latex-svninfo kurulumu sonrası:
\begin{verbatim}
 # wget http://cekirdek.pardus.org.tr/~semen/dist/test/tex/addon/test_latexsvninfo.tex
\end{verbatim}

Kile paketini kurun ve yukarıda verilen dosyayı kile ile açın ve QuickBuild  seçeneğini tıklayın ve sorunsuz olarak pdf dosyasının oluşturğunu gözlemleyin.

\item  latex-xcolor kurulumu sonrası:
\begin{verbatim}
 # wget http://cekirdek.pardus.org.tr/~semen/dist/test/tex/addon/test_latexxcolor.tex
\end{verbatim}

Kile paketini kurun ve yukarıda verilen dosyayı kile ile açın ve QuickBuild  seçeneğini tıklayın ve sorunsuz olarak pdf dosyasının oluşturğunu gözlemleyin.

\item texlive-bibtexextra kurulumu sonrası:
\begin{verbatim}
 # wget http://cekirdek.pardus.org.tr/~semen/dist/test/tex/addon/test_texlivebibtexextra.tex
\end{verbatim}

Kile paketini kurun ve yukarıda verilen dosyayı kile ile açın ve QuickBuild  seçeneğini tıklayın ve sorunsuz olarak pdf dosyasının oluşturğunu gözlemleyin.
\item  latex-passivetex ve latex-xmltex kurulumu sonrası:
\begin{verbatim}
 # wget http://cekirdek.pardus.org.tr/~semen/dist/test/tex/addon/test_latexpassivetex_xmltex.fo
 # pdfxmltex test_latexpassivetex_xmltex.fo
\end{verbatim}

Yukarıda bulunan komutun düzgün bir şekilde pdf çıktısı ürettiğini gözlemleyin.

\end{enumerate}


\end{document}

