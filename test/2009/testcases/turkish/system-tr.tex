\documentclass[a4paper,10pt]{article}
\usepackage[turkish]{babel}
\usepackage[utf8]{inputenc}
\usepackage[left=1cm,top=1cm,right=2cm,bottom=2cm]{geometry}

\title{System Bileşeni Test Aşamaları}
\author{Semen Cirit}

\renewcommand{\labelenumi}{\arabic{enumi}.}
\renewcommand{\labelenumii}{\arabic{enumi}.\arabic{enumii}.}
\renewcommand{\labelenumiii}{\arabic{enumi}.\arabic{enumii}.\arabic{enumiii}.}
\renewcommand{\labelenumiv}{\arabic{enumi}.\arabic{enumii}.\arabic{enumiii}.\arabic{enumiv}.}

\begin{document}

\maketitle


\section{Devel alt Bileşeni}
\begin{enumerate}
 \item xtrans paketi kurulum testine tabidir.
\end{enumerate}


\section{Base alt Bileşeni}

\begin{enumerate}
\item dnsmasq paketi kurulumu sonrası:

\begin{itemize}
 \item Service yöneticinizden dnsmasq servisini başlatın.
 \item Ağ yöneticisinden bağlantınızı durdurun ve yeniden başlatın.
 \item Aşağıdaki komutu çalıştırın ve sorunsuz bir şekilde sorfu zamanını döndürdüğünü gözlemleyin.
\begin{verbatim}
 # dig http://archlinux.org | grep "Query time"
\end{verbatim}

\end{itemize}


\item file paketi kurulumu sonrası:

\begin{verbatim}
 # wget http://cekirdek.pardus.org.tr/~semen/dist/test/office/openoffice/test_oodraw.mng
 # wget http://cekirdek.pardus.org.tr/~semen/dist/test/office/openoffice/test_oodraw.odg
 # wget http://cekirdek.pardus.org.tr/~semen/dist/test/office/openoffice/test_oodraw.jpg
 # wget http://cekirdek.pardus.org.tr/~semen/dist/test/office/openoffice/test_oodraw.gif
 # wget http://cekirdek.pardus.org.tr/~semen/dist/test/office/openoffice/test_oodraw.png
 # wget http://cekirdek.pardus.org.tr/~semen/dist/test/office/openoffice/test_oodraw.tif
 # wget http://cekirdek.pardus.org.tr/~semen/dist/test/office/openoffice/test_oowriter.txt
 # wget http://cekirdek.pardus.org.tr/~semen/dist/test/office/openoffice/test_oodraw.ps
 # wget http://cekirdek.pardus.org.tr/~semen/dist/test/office/openoffice/
   test_openoffice-extension-pdfimport.pdf
\end{verbatim}

Yukarıda bulunan dosyaları aşağıda bulunan komut ile çalıştırın, dosya formatlarını düzgün bir şekilde bulduğunu gözlemleyin.
\begin{verbatim}
 # file <dosya adı>
\end{verbatim}

\item mudur paketi kurulumu sonrası:

\begin{itemize}
  \item Makinenizi yeniden başlatın ve sistemin düzgün bir şekilde açıldığını  gözlemleyin.
 \item Ctrl+Alt+F1 tuşuna basıp sistem konsoluna geçin ve sistem dilinizin daha önceki dil ve kalvye düzeni ile aynı olduğunu gözlemleyin.
  \item /etc/mudur/ altında bulunan locale, language, keymap dosyalarının sistem dil ve klavye dilinize eskisi gibi uygun olduğunu gözlemleyin.
 \item Pisi komutlarının düzgün bir şekilde çalıştığını gözlemleyin. 

\end{itemize}
\item tiff paketi kurulumu sonrası:
\begin{verbatim}
 # wget http://cekirdek.pardus.org.tr/~semen/dist/test/desktop/kde/base/doga.tiff
 # wget http://cekirdek.pardus.org.tr/~semen/dist/test/desktop/kde/base/istanbul.tiff
\end{verbatim}

Resimlerin üzerine sağ tıklayarak gwenview, kolourPaint, gimp, showfoto ile açılabildiklerini gözlemleyin.
\item pisi paketi kurulumu sonrası:
\begin{itemize}
 \item package-manager-tr.pdf test aşamalarını gerçekleştiriniz.
 \item history-manager-tr.pdf test aşamalarını gerçekleştiriniz.
\end{itemize}
\item baselayout paketi kurulumu sonrası:
\begin{itemize}
 \item Bilgisayarınızı kapatın düzgün bir şekilde kapanabildiğini gözlemleyin.
 \item Bilgisayarınızı yeniden başlatın ve düzgün bir şekilde açılabildiğini gözlemleyin.
 \item 2009 için:
  \begin{verbatim}
   # pisi blame baselayout
  \end{verbatim}
      2008 için: http://svn.pardus.org.tr/pardus/2008/stable/system/base/baselayout/pspec.xml linkinde bulunan en son eklenmiş history tag'i yorumuna (comment) bakınız.

   Bu her iki durumda da yorumlarda belirtilmiş olan kullanıcının /etc/passwd dosyasına eklenmiş olduğunu gözlemleyin.

\end{itemize}
\item libxml2 paketi kurulumu sonrası:

\begin{itemize}
\item Aşağıda bulunan komutun bir katalog oluşturduğunu gözlemleyin.
  \begin{verbatim}
   #  xmlcatalog --create
  \end{verbatim}
\item multimedia-tr.pdf avidemux-qt, avidemux ve inkscape testlerini gerçekleştirin.
\end{itemize}
\item curl paketi kurulumu sonrası:

\begin{itemize}
\item http://pardus.org.tr adresinin içeriğinin hata alınmadan çıktı olarak alındığını gözlemleyin.
\begin{verbatim}
# wget http://cekirdek.pardus.org.tr/~semen/dist/test/system/base/test_curl.php
# php test_curl.php
\end{verbatim}

\item network-tr.pdf sylpheed testini geçekleştirin.
\end{itemize}

\item glib2 paketi kurulumu sonrası:
\begin{itemize}
 \item system-tr.pdf openssh testini gerçekleştirin.
\item system-tr.pdf getext testini gerçekleştirin.
\end{itemize}

\item gettext paketi kurulumu sonrası:

Aşağıda bulunan komutun "network-manager.pot" dosyasını oluşturduğunu gözlemleyin.
\begin{verbatim}
 # svn co http://svn.pardus.org.tr/uludag/trunk/kde4/network-manager/
 # cd network-manager/manager
 # rm -rf po
 # mkdir po
 # xgettext setup.py 
 # ./setup.py update_messages
\end{verbatim}


\item openssh paketi kurulumu sonrası:

Servis yöneticisinden openssh'ı çalıştırın ve aşağıda bulunan komutun çıktısında openssh'ın çalıştığını gözlemleyin:
\begin{verbatim}
 # service list 
\end{verbatim}

\item mkinitramfs paketi kurulumu sonrası:

Bilgisayarınızı yeniden başlatın. Düzgün bir şekilde açıldığını gözlemleyin.

/boot/ dizini altında kullanmakta olduğunuz kernelin initramfs dosyasının var olduğunu gözlemleyin.

\item pardus-python paketi kurulum testine tabidir.

\end{enumerate}
\section{Service alt Bileşeni}
\begin{enumerate}
 \item memcached paketi kurulumu sonrası:
  programming-tr.pdf python-memcached testini gerçekleştirin. 
\end{enumerate}



\end{document}

