\documentclass[a4paper,10pt]{article}
\usepackage[turkish]{babel}
\usepackage[utf8]{inputenc}
\usepackage[left=1cm,top=2cm,right=2cm,bottom=1cm]{geometry}


\renewcommand{\labelenumi}{\arabic{enumi}.}
\renewcommand{\labelenumii}{\arabic{enumi}.\arabic{enumii}.}
\renewcommand{\labelenumiii}{\arabic{enumi}.\arabic{enumii}.\arabic{enumiii}.}
\renewcommand{\labelenumiv}{\arabic{enumi}.\arabic{enumii}.\arabic{enumiii}.\arabic{enumiv}.}

\title{Kernel Test Aşamaları}
\author{Semen Cirit}

\begin{document}

\maketitle

\section{Default ve Pae alt bileşeni}
\begin{enumerate}
 \item  Aşağıda bulunan paketlerin kurulumundan sonrası:
 \begin{verbatim}
 kernel
 kernel-firmware
 kernel-module-headers
 kernel-headers
 kernel-doc
 kernel-source 
 kernel-pae
 kernel-module-headers-pae
\end{verbatim}

İlk olarak bilgisyarınızı yeniden başlatınız.
\begin{itemize}
\item Açılış ekranın görüntüsünün titremediğini gözlemleyiniz
\item Bilgisayarınızın yeni kernel ile düzgün bir şekilde açılabildiğini gözlemleyiniz.
\item Eğer dizüstü bilgisayar kullanıyorsanız kablonuzu çıkarıp taktığınızda uyarı verdiğini ve pil seviyesinin düzgün bir şekilde görüntülendiğini gözlemleyiniz.
\item USB bellek takınız ve algılandığını gözlemleyiniz.
\end{itemize}

\item module-alsa-driver ve module-alsa-driver-userspace paketleri kurulumu sonrası.

Bilgisayarınız yeniden başlatınız.
\item Açılış sesinin düzgün bir şekilde takılmadan çalıştığını gözlemleyiniz.

\item Amarok ile aşağıdaki bağlantıda bulunan ses dosyalrından birkaçını deneyiniz. Sorunsuz bir şekilde çalıştıklarını gözlemleyiniz.
\begin{verbatim}
http://cekirdek.pardus.org.tr/~semen/dist/test/multimedia/sound/sound.tar 
\end{verbatim}

\end{enumerate}

\end{document}

