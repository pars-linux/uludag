\documentclass[a4paper,10pt]{article}
\usepackage[turkish]{babel}
\usepackage[utf8]{inputenc}
\usepackage[left=1cm,top=2cm,right=2cm,bottom=1cm]{geometry}

\title{Network Bileşeni Test Aşamaları}
\author{Semen Cirit}

\renewcommand{\labelenumi}{\arabic{enumi}.}
\renewcommand{\labelenumii}{\arabic{enumi}.\arabic{enumii}.}
\renewcommand{\labelenumiii}{\arabic{enumi}.\arabic{enumii}.\arabic{enumiii}.}
\renewcommand{\labelenumiv}{\arabic{enumi}.\arabic{enumii}.\arabic{enumiii}.\arabic{enumiv}.}

\begin{document}

\maketitle

\section{Chat alt Bileşeni}
\begin{enumerate}
\item choqok paketi kurulumu sonrası:

Uygulamayı açın ve bir twitter üyeliğiniz var ise, bu üyelik bilgilerinizi kaydedin ve sorunsuz bir şekilde twitter'a bağlanabildiğinizi gözlemleyin.	
\end{enumerate}
\section{Web alt Bileşeni}
\begin{enumerate}
\item firefox paketi kurulumu sonrası:
\begin{itemize}
 \item Aşağıda bulunan sayfanın ilgili butonlarına basıldığında sorunsuz bir şekilde çalıştığını gözlemleyin.
	\begin{verbatim}
	 http://www.croczilla.com/~alex/old-site/dom2.xml
	\end{verbatim}
	http://www.croczilla.com/~alex/old-site/dom2.xml
 \item Home dizininiz altında bulunan .firefox dizininin baştan oluşturulmadığını gözlemleyin.
	
	Bookmarklarınızın kaybolmadığını gözlemleyin.
	
 	Daha önce açmış olduğunuz sayfaların yeniden yüklendiğinde açıldığını gözlemleyin.

\item Aşağıda bulunan siteyi açın ve videoyu tam ekran olarak oynatmaya çalışın, sesinin ve görüntüsünün sorunsuz bir şekilde olduğunu gözlemleyin.
	\begin{verbatim}
	http://www.dailymotion.com/video/x3akre_loreena-mckennitt-all-souls-night-l  
	\end{verbatim}
\item http://svn.pardus.org.tr/uludag/trunk/test/2009/testguide/turkish/ dizini altında bulunan dökümanlardan birini indirmeye çalışın kaydetme penceresinin açıldığını gözlemleyin.

Bu dosyayı indirin ve indirme penceresinin düzgün bir şekilde açıldığını gözlemleyin.
\end{itemize}


\end{enumerate}

\section{Monitor alt Bileşeni}
\begin{enumerate}
 \item wireshark paketi kurulumu sonrası:
	Wireshark uygulamasını açın, interface listesinden eth0 seçin ve bu interface ile ilgili paketleri listelendiğini gözlemleyin.
\end{enumerate}

\end{document}

