\documentclass[a4paper,10pt]{article}
\usepackage[turkish]{babel}
\usepackage[utf8]{inputenc}
\usepackage[left=1cm,top=1cm,right=2cm,bottom=2cm]{geometry}

\title{Network Bileşeni Test Aşamaları}
\author{Semen Cirit}

\renewcommand{\labelenumi}{\arabic{enumi}.}
\renewcommand{\labelenumii}{\arabic{enumi}.\arabic{enumii}.}
\renewcommand{\labelenumiii}{\arabic{enumi}.\arabic{enumii}.\arabic{enumiii}.}
\renewcommand{\labelenumiv}{\arabic{enumi}.\arabic{enumii}.\arabic{enumiii}.\arabic{enumiv}.}

\begin{document}

\maketitle

\section{Download alt Bileşeni}
\begin{enumerate}
 \item youtube-dl paketi kurulumu sonrası:

Aşağıdaki komutu çalıştırdığınızda .flv uzantılı bir dosyanın indirildiğini gözlemleyin.
\begin{verbatim}
# youtube-dl http://www.youtube.com/watch?v=5u2q3P60XEk
\end{verbatim}

\end{enumerate}

\section{Chat alt Bileşeni}
\begin{enumerate}
\item choqok paketi kurulumu sonrası:

Uygulamayı açın ve bir twitter üyeliğiniz var ise, bu üyelik bilgilerinizi kaydedin ve sorunsuz bir şekilde twitter'a bağlanabildiğinizi gözlemleyin.	
\item konversation ve konversation-docs paketleri kurulumu sonrası:

Uygulamayı çalıştırın ve öntanımlı kanallara bağlandığınızı gözlemleyin.

\end{enumerate}
\section{Web alt Bileşeni}
\begin{enumerate}
\item firefox ve arora paketi kurulumu sonrası:
\begin{itemize}
 \item Aşağıda bulunan sayfanın ilgili butonlarına basıldığında sorunsuz bir şekilde çalıştığını gözlemleyin.
	\begin{verbatim}
	 http://www.croczilla.com/~alex/old-site/dom2.xml
	\end{verbatim}
 \item (Firefox için)Home dizininiz altında bulunan .firefox dizininin baştan oluşturulmadığını gözlemleyin.
	
	Bookmarklarınızın kaybolmadığını gözlemleyin.
	
 	Daha önce açmış olduğunuz sayfaların yeniden yüklendiğinde açıldığını gözlemleyin.

\item Aşağıda bulunan siteyi açın ve videoyu tam ekran olarak oynatmaya çalışın, sesinin ve görüntüsünün sorunsuz bir şekilde olduğunu gözlemleyin.
	\begin{verbatim}
	http://www.dailymotion.com/video/x3akre_loreena-mckennitt-all-souls-night-l 
	\end{verbatim}
\item http://cekirdek.pardus.org.tr/~semen/dist/test/office/openoffice/ dizini altında bulunan dökümanlardan birini indirmeye çalışın kaydetme penceresinin açıldığını gözlemleyin.

Bu dosyayı indirin ve indirme penceresinin düzgün bir şekilde açıldığını gözlemleyin.
\item http://cekirdek.pardus.org.tr/~semen/dist/test/multimedia/video/cokluortam/ dizini altında bulunan videolardan birkaçını çalıştırın ve firefox üzerinden çalışabildiğini gözlemleyin.

\end{itemize}


\end{enumerate}

\section{Monitor alt Bileşeni}
\begin{enumerate}
 \item wireshark paketi kurulumu sonrası:
	Wireshark uygulamasını açın, interface listesinden eth0 seçin ve bu interface ile ilgili paketleri listelendiğini gözlemleyin.
\end{enumerate}

\section{Mail alt Bileşeni}
\begin{enumerate}
 \item thunderbird ve sylpheed paketleri kurulumu sonrası:
\begin{itemize}
\item Bir e-posta hesabı oluşturuyoruz.
\item Mail alabildiğimizi gözlemliyoruz.
\item Mail gönderebildiğimizi gözlemliyoruz.
\item Filtre oluşturabildiğimizi gözlemliyoruz.
\item Eğer daha önce thunderbird kullanıyor isek, ev dizini altında .thunderbird dizininin silinmediğini gözlemliyoruz.
\end{itemize}
 \item spamassassin ve spamd paketleri kurulumu sonrası:
\begin{enumerate}
	\item İlgili paketi kurduktan sonra:
	\item Menüden uygulamalar $\rightarrow$ Kmail'i açın ve spam filtresini aktifleştirmek için:
		
	\begin{enumerate}
		\item Kmail menü çubuğundan Araçlar $\rightarrow$  Spam engelleme sihirbazı yolunu izleyin.
		\item İndirdiğiniz ilgili spam filtresini seçin.
		\item Bir postaya sağ tıklayın ve combobox'tan Filtreyi uygula $\rightarrow$ Filtreyi çöp posta olarak sınıflandır yolunu izleyin.
		Bu spam'in ilgili spam klasörüne gittiğini gözlemleyin.( Default spam klasörü eğer değitirmediyseniz çöp klasörü olacaktır.)

		\item Aşağıdaki linkten gtube.txt'yi indirin: 
		\begin{verbatim}
 		http://spamassassin.apache.org/gtube/
		\end{verbatim}
		\item  Konsoldan komutu çalıştırın:
		\begin{verbatim}
 		cat  gtube.txt | spamc 
		\end{verbatim}
		
		Bu komut size içerisinde şifrelenmiş bir satır içeren buna benzer bir çıktı gönderecek:
		
		\emph{If your spam filter supports it, the GTUBE provides a test by which you
	    	can verify that the filter is installed correctly and is detecting incoming
    		spam. You can send yourself a test mail containing the following string of
    		characters (in upper case and with no white spaces and line breaks):}
		\begin{verbatim}
 		XJS*C4JDBQADN1.NSBN3*2IDNEN*GTUBE-STANDARD-ANTI-UBE-TEST-EMAIL*C.34X
		\end{verbatim}
    		\emph{You should send this test mail from an account outside of your network.}

		\item Daha sonra bu ilgili şifrelenmiş kısmı kopyalayıp mail olarak kendinize gönderin.
		
		Bu mailin direk olarak ilgili spam klasörüne gittiğini gözlemleyin.
	\end{enumerate} 
\end{enumerate} 
\end{enumerate}
\section{Plugin alt Bileşeni}
\begin{enumerate}
\item flashplugin paketi kurulumu sonrası:

Opera paketini kurun ve aşağıdaki bağlantıda bulunan videonun düzgün bir şekilde çalıştığını gözlemeyin.

Tam ekran olabiliyor mu? Ses sorunu var mı?
\begin{verbatim}
http://www.dailymotion.com/relevance/search/lorena+mckennit/video/xd9s3_princesse
-mononoke-studioslorenna 
\end{verbatim}
\item gecko-mediaplayer paketi kurulumu sonrası:
\begin{itemize}
  \item Firefox $\rightarrow$ Düzen $\rightarrow$ Seçenekler $\rightarrow$ Eklentileri Yönet $\rightarrow$ Yan Uygulamalar bölümünde gecko-mediaplayer eklentisinin eklenmiş olduğunu gözlemleyin.
  \item Aşağıda bulunan uzantıdaki videoları firefox üzerinden açınız. Ve düzgün bir şekilde çalıştıklarını gözlemleyiniz.
  \begin{verbatim}
  http://cekirdek.pardus.org.tr/~semen/dist/test/multimedia/video/cokluortam/  
  \end{verbatim}
\end{itemize}
\end{enumerate}

\section{Ftp alt Bileşeni}
\begin{enumerate}
 \item lftp paketi kurulumu sonrası:
\begin{verbatim}
 # lftp http://ftp.pardus.org.tr/pub/
 # ls 
 # cd pardus
\end{verbatim}
Yukarıda bulunan komutların sorunsuz bir şekilde çalıştığını gözlemleyin.

\end{enumerate}

\section{Connection alt Bileşeni}
\begin{enumerate}
 \item Aşağıda bulunan paketler kurulum testine tabidir.
\begin{verbatim}
iw
wireless-regdb
\end{verbatim}


\end{enumerate}


\end{document}

