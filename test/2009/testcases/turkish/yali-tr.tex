\documentclass[a4paper,10pt]{article}
\usepackage[left=1cm,top=2cm,right=2cm,bottom=1cm]{geometry}
\usepackage[turkish]{babel}
\usepackage[utf8]{inputenc}

\renewcommand{\labelenumi}{\arabic{enumi}.}
\renewcommand{\labelenumii}{\arabic{enumi}.\arabic{enumii}.}
\renewcommand{\labelenumiii}{\arabic{enumi}.\arabic{enumii}.\arabic{enumiii}.}
\renewcommand{\labelenumiv}{\arabic{enumi}.\arabic{enumii}.\arabic{enumiii}.\arabic{enumiv}.}

\title{YALI Test Aşamaları}
\author{Semen Cirit}

\begin{document}
\maketitle
\begin{enumerate}
\item Çevirilerde bir sorun olmadığını gözlemleyin.
\item Sürüm notları ve lisans dosyasının mevcut olduğunu gözlemleyin.
\item Farklı klavye düzenlerini seçin.
	\begin{enumerate}	
	\item Pencerenin altında bulunan kutudan ilgili klavye düzeninin doğru çıktıyı verdiğini gözlemleyin.

	\item $CTRL+ALT+F1$ tuşlarına aynı anda basarak konsola geçin ve seçilen klavye düzeninin doğru çıktıyı verdiğini gözlemleyin.
	\end{enumerate}
\item Bir zaman dilimi seçin ve saat ve zamanı ayarlayın.


    \begin{enumerate}
	\item Kurulumdan sonra saatin ve tarihin doğru olduğunu gözlemleyin.
        \item /usr/share/zoneinfo/ altında bulunan sizin seçmiş olduğunuz zaman diliminin ve /etc/localtime dosyası içerisine kaydedilmiş olduğunu gözlemleyin
    \end{enumerate}
\item İki kullanıcı ekleyin.
\begin{enumerate}
	\item Birini silin, diğerinin bilgilerini değiştirin.

	Sorunsuz bir şekilde bu işlemleri gerçekleştiğini gözlemleyin.
	\item Kalan kullanıcıyı silin ve aynı isimden başka bir kullanıcı eklemeye çalışın.

	Sorunsuz bir şekilde bu işlemleri gerçekleştiğini gözlemleyin.
	\item Kullanıcı adı için kabul edilmeyen karakter kullanmaya çalışın. (ı,ş,ğ gibi)
	
	YALI'nın buna uyarı verdiğini gözlemleyin.
	\item Parola tekrarla kısmına yanlış bir parola girmeye çalışın.

	YALI'nın buna uyarı verdiğini gözlemleyin.

\item Daha önce eklemiş olduğunuz kullanıcıya root hakkı ve bu kullanıcı için hiç parola sorma seçeneğini işaretleyin.

\item Root hakkına sahip olmayan ve bu kullanıcı için hiç parola sorma seçeneği işaretlenmemiş bir kullanıcı ekleyin.

\item Daha sonra ilk kullanıcı için girişte parola sormayı işaretleyin.
\item Makina adı ve root şifresi girilmesi
	
	İlgili ekrana geldiğinizde öntanımlı olarak bir makine adının bulunduğunu gözlemleyin.
	\begin{enumerate}
		\item Root şifresi olarak kabul edilmeyen karakter kullanmaya çalışın.

		YALI'nın buna uyarı verdiğini gözlemleyin.
		\item Root şifresi için tekar parolayı gir kısmına farklı bir parola girmeye çalışın.
		YALI'nın buna uyarı verdiğini gözlemleyin.
	\end{enumerate}
\item Kurulum bittikten sonra

	İlk kullanıcı için:
	\begin{enumerate}
	\item Otomatik olarak login olduğunu gözlemleyin.
	\item Kaptan'da network ayarlarında kullanıcı parolası sormadığını gözlemleyin.
	\item Root olabildiğini gözlemleyin.
	Kullanılacak komut:
	\begin{verbatim}
	 # su
	\end{verbatim}
	\end{enumerate}
	İkinci kullanıcı için:
	\begin{enumerate}
	\item Otomatik olarak login olamadığını gözlemleyin.
	\item Kaptan'da network ayarlarında kullanıcı parolası sorulduğunu gözlemleyin.
	\item Root olamadığını gözlemleyin.
	Kullanılacak komut:
	\begin{verbatim}
	 # su
	\end{verbatim}
	\end{enumerate}


\end{enumerate}
\item Bölümlendirme kısmı için tüm seçenekleri deneyin;
    \begin{enumerate}
        \item Yeniden boyutlandırma için otomatik bölümlendirmeyi seçin. 
        \begin{enumerate}
            \item Yeniden boyutlandıracak olduğunuz bölümün 7 Gb'tan az boş yeri bulunuyor ise bu bölümün listelenmediğini gözlemleyiniz.
            \item Kurulum bittikten sonra yeni oluşan ve yeniden boyutlandırılan bölümlerin dosya sistemlerinin bozulmamış olduklarını ve sorunsuz bir şekilde açıldığını gözlemleyin.

% 		Gözlem için gerekli olan komut: (root olarak çalıştırın)
% 		\begin{verbatim}
% 		 # fdisk -l
% 		\end{verbatim}
% gokmene sor		

        \end{enumerate}
        \item Otomatik bölümlendirme bölümünden, tüm bölümleri sil seçeneğini seçin ve kurulumu gerçekleştirin.
	
	Kurulum bittikten sonra harddiskinizde tek bir bölüm olduğunu ve sorunsuz bir şekilde açıldığını gözlemleyin.

        \item Manuel bölümlendirmeyi seçin ve tüm dosya sistemi tipleri için:
        \begin{enumerate}
            \item Tüm dosya sistem tipleri için birer Pardus root, swap ve arşiv bölümü yaratın.

		Aynı anda iki farklı arşiv ekleyemediğinizi gözlemleyin.
	    \item Rastgele olacak şekilde bazı bölümleri silin.
		
		Seçtiğiniz bölümlerin silindiğini gözlemleyin.
            \item Hepsini silin
	
		Hepsinin silindiğini gözlemleyin.
            \item Tüm yapılan değişiklikleri geri alın.
	
		Herşeyin eski haline geldiğini gözlemleyin.
            \item Uygun bir bölümü yeniden boyutlandırmayı deneyin.
	
		Verilerinizin kaybolabileceğine dair bir uyarı verdiğini gözlemleyin. 
            \item Yeni bir usb disk takın ve diski yenile deyin.

		Taktığınız diskin bölümlendirme tablosuna eklendiğini gözlemleyin.

            \item Üzerinde değişiklik yapılmış bir bölüme farklı bir dosya sistemi vermeyi deneyin.

		Sorunsuz bir şekilde durumun gerçekleştiğini gözlemleyin.
        \end{enumerate}
        \item İlk önce otomatik bölümlendirmeyi seçin ve geri dön tuşuna basın, daha sonra manuel bölümlendirmeyi seçin ve gerekli işlemleri yapıp ileri tuşuna basın.

	Sorunsuz bir şekilde kuruluma geçtiğini gözlemleyin.
	\item Manuel bölümlendirmeyi seçin ve birkaç farklı bölüm yaratın ve daha sonra ileri tuşuna basın. Ve daha sonra geri tuşuna basın.

	Tüm yaptığınız değişikliklerin geri alınacağına dair bir uyarı çıktığını gözlemleyin. 

	Bu uyarıya ok deyin. 
		
	Yaptığınız değişikliklerin geri alındığını gözlemleyin.
	 
        \item Bir birincil bölüm (Padus sistemi) ve bir mantıksal (swap) veya boş bir bölüm oluşturun. Otomatik olarak yeniden boyutlandırmayı seçin.

	Bu diski yeniden boyutlandırabildiğinizi gözlemleyin.
    \end{enumerate}
\item Önyükleyici seçimi

Kurulum bittikten sonra Önyükleyici seçimine göre oluşmuş Grub durumunu aşağıdaki dosyadan gözlemleyebilirsiniz. 
   \begin{verbatim}
    cat /boot/grub/grub.conf 
   \end{verbatim}


   \begin{enumerate}
     \item Açılış diskinin başına kur

	Yukarıdaki komutu kullanarak, sistemin açılış diskinin başına eklendiğini gözlemleyin.
    \item Pardusun kurulu olduğu diskin başına kur.

	Yukarıdaki komutu kullanarak, sistemin pardusu kurmuş olduğunuz diskin başında olduğunu gözlemleyin
    \item Sistem yükleyicisini kurma

	Yukarıdaki komutu kullanarak, sistemin dosyaya eklenmemiş olduğunu gözlemleyin.

    \item Eğer makinanızda birden fazla hardisk veya harici bellek var ise önyükleyici seçiminde tüm disklerin listelendiğini gözlemleyin.
      \end{enumerate}
 \item Sisteminizin bir bölümünde windows var ise kurulum bittikten sonra önyükleyiciye bu disk bölümünün eklenmiş olduğunu gözlemleyin.
% gokmen
% \item Bilgisayarınızı yeniden başlatmadan önce Check whole log file before restart, yali doesn't stores grub logs in yaliInstall.log because of some (mount,grub-install) problems
\item Bozuk bir CD hazırlayın ve CD'yi kontrol et butonuna tıklayın.

CD'nin bozuk olduğuna dair bir meşaj çıktığını gözlemleyin.

Bu mesaja ok deyin.

Sorunsuz bir şekilde sistemin tekrar başlatıldığını gözlemleyin.

\end{enumerate}

\end{document}
