\documentclass[a4paper,10pt]{article}

\title{History Manager Test Cases}

\renewcommand{\labelenumi}{\arabic{enumi}.}
\renewcommand{\labelenumii}{\arabic{enumi}.\arabic{enumii}.}
\renewcommand{\labelenumiii}{\arabic{enumi}.\arabic{enumii}.\arabic{enumiii}.}
\renewcommand{\labelenumiv}{\arabic{enumi}.\arabic{enumii}.\arabic{enumiii}.\arabic{enumiv}.}

\begin{document}

\maketitle

\begin{enumerate}
\item sistem ayarlarından geçmiş yöneticisini açmayı deneyin.
\item Kmenüsünden geçmiş yöneticisini açmayı deneyin.
\item Sistem görüntüsü almayı deneyin.
\begin{enumerate}
    \item Paket yöneticisinden bir paket kurun veya kaldırın.
    \item Geçmiş yöneticisindeki görüntü alma butonuna basın.
    \item Görüntü aldıktan sonra, geçmiş yöneticisinin ekranı yenilediğini ve yeni değişiklikleri gösterdiğini onaylayın.
\end{enumerate}

\item Bir tarih için ayrıntıları gösterin.

    Aşağıdaki komutu çalıştırın:
\begin{verbatim}
    $ pisi hs
\end{verbatim} 

    Son değişiklikleri geçmiş yöneticisi ile ilgili tarihe göre karşılaştırın.

\item Bir tarih için işlem detaylarını gösterin.
    Aşağıdaki komutu çalıştırın:
\begin{verbatim}
     $ pisi hs
\end{verbatim} 
    Son değişiklikleri geçmiş yöneticisi ile ilgili tarih için karşılaştırın.

    Paketlerin doğru olarak güncellendiğini, yüklendiğini veya kaldırıldığını onaylayın.

\item Bir tarih seçip sistem geçmiş bir tarihe getirmeyi deneyin.

        Değişiklikler geri alındığında 
\begin{enumerate}
        \item İşlemden sonra geçmiş yöneticisi ekranın yenilendiğini gözlemleyin.
        \item İşlemi geri aldığınız tarih için işlemleri göster tuşuna basın.

			  Bu işlemlerin işlem planından kaldırıldığını onaylayın.
     	\item Konsolda şu komutu yazın·
\begin{verbatim}
    $ pisi hs
\end{verbatim} 

            Son değişiklikleri geçmiş yöneticisi ile karşılaştırın.

			İşlemin ardından geri alma işleminin doğru olarak listelendiğini onaylayın.
\end{enumerate}
\end{enumerate}

\end{document}
