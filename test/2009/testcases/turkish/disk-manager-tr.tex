\documentclass[a4paper,10pt]{article}
\usepackage[turkish]{babel}
\usepackage[utf8]{inputenc}
\usepackage[left=1cm,top=2cm,right=2cm,bottom=1cm]{geometry}


\renewcommand{\labelenumi}{\arabic{enumi}.}
\renewcommand{\labelenumii}{\arabic{enumi}.\arabic{enumii}.}
\renewcommand{\labelenumiii}{\arabic{enumi}.\arabic{enumii}.\arabic{enumiii}.}
\renewcommand{\labelenumiv}{\arabic{enumi}.\arabic{enumii}.\arabic{enumiii}.\arabic{enumiv}.}


\title{Disk Yöneticisi Test Aşamaları}
\author{Semen Cirit}

\begin{document}

\maketitle
\begin{enumerate}

\item Menü $\rightarrow$ Sistem Ayarları Disk Yöneticisi bölümünden açmayı deneyiniz.

Sorunzuz bir şekilde açıldığını gözlemlyiniz.
\item Menü $\rightarrow$ Sistem $\rightarrow$ Disk Yöneticisi'ni yolunu izleyerek açmayı deneyiniz.

\item Harici bir usb belleği veya benzeri bir aygıtı bağlama: (Başlangıçta disk bağlı değil durumda iken) 
    \begin{enumerate}
        \item Şu komutu çalıştırınız:
            \begin{verbatim}
                # mount
            \end{verbatim}
            Diskin hiçbir yere bağlı olmadığını gözlemleyiniz. 

	\item Ayrıca disk yöneticisinde bulunan diğer bağlı disklerin uygun bir şekilde bağlı olduklarını gözlemleyiniz.

    \item Bu diski ev dizininize bağlamayı deneyiniz.
          Disk yöneticisinin geçerli olmayan bu bağlama noktası hakkında bir uyarı verdiğini gözlemleyiniz.
    \item Bu diski bağlamayı deneyiniz.
        \begin{enumerate}
            \item Uygun bir bağlama noktası deneyiniz. Örneğin "/mnt/sample"
            \item Uygun bir dosya sistemi veriniz.
            \item Bağlı aygıt için uygun seçenekleri veriniz.
            \item Bunları uygulayınız.
            \item Şu komutu çalıştırınız:
            \begin{verbatim}
                # mount
            \end{verbatim}
                Diskin verdiğiniz yönlendirmelere uygun olarak bağlandığını gözlemleyiniz.
        \end{enumerate}

    \item Bu diskin bağlanma noktasını değiştirmeye çalışınız.
        \begin{enumerate}
            \item Şu komutu çalıştırınız:
            \begin{verbatim}
                # mount
            \end{verbatim}
            Disk yöneticisinin diskin zaten bağlı olduğuyla ilgili bir uyarı mesajı verip vermediğini gözlemleyiniz.
        \end{enumerate}

    \item Dosya sistemi türünü değiştirme:
        \begin{enumerate}
            \item Şu komutu çalıştırınız:
            \begin{verbatim}
                # blkid
            \end{verbatim}
                  Seçilmiş bir aygıt için, disk yöneticisindeki öntanımlı dosya sistemi türünün "blkid" çıktısı ile aynı olduğunu gözlemleyiniz.
            \item Dosya sistemi türünü değiştiriniz.
                  Eğer dosya sistemi türü bu aygıtla uyumlu değilse, disk yöneticisinin uyarı mesajı verdiğini gözlemleyiniz.
        \end{enumerate}

    \item Bağlama için bir seçenek eklemeyi deneyiniz:
        \begin{enumerate}
          \item Seçenek kısmı boş iken Tamam butonuna basmayı deneyiniz.
                Bir uyarı mesajı verdiğini gözlemleyiniz.
          \item Eğer seçenekler kısmına birkaç seçenek eklerseniz,
                "mount" komutunun çıktısının ilgili disk için bu seçenekleri içerdiğini gözlemleyiniz.
        \end{enumerate}
    \end{enumerate}

\item Root bölümünün bağlama noktasını değiştirmeyi deneyiniz

      Root böülmünün bağlanma noktasının değiştirilemeyeceğine dair bir uyarı mesajı aldığınızı gözlemleyiniz.

        \textbf{Not :} Root'un bağlama noktası "/" olmalı.

\item Genişletilmiş (extended) bir bölümün disk yöneticisinde listelenip listelenmediğini gözlemleyiniz. (Normal şartlarda listelenmemeli).

\item Herhangi bir işlem yapılmakta iken çıkan kimlik doğrulama penceresine iptal değiniz.

İptal işlemi sonucunda disk yöneticisinin sorunsuz bir şekilde yapılmakta olan işlemden önceki duruma geçtiğini gözlemleyiniz.

\item Disk yöneticisi ile bir işlem yapmakta iken, kimlik doğrulamayı anımsa bölümünü işaretleyiniz.

Disklerinizden biri için herhangi bir işlem uygulayınız.

Parola sorulmadan işlemin dorunsuz bir şekilde gerçekleştiğini gözlemleyiniz.
\end{enumerate}

\end{document}
