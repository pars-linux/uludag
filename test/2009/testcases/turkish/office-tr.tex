\documentclass[a4paper,10pt]{article}
\usepackage[turkish]{babel}
\usepackage[utf8]{inputenc}
\usepackage[left=1cm,top=2cm,right=2cm,bottom=1cm]{geometry}

\title{Ofis Bileşeni Test Aşamaları}
\author{Semen Cirit}

\renewcommand{\labelenumi}{\arabic{enumi}.}
\renewcommand{\labelenumii}{\arabic{enumi}.\arabic{enumii}.}
\renewcommand{\labelenumiii}{\arabic{enumi}.\arabic{enumii}.\arabic{enumiii}.}
\renewcommand{\labelenumiv}{\arabic{enumi}.\arabic{enumii}.\arabic{enumiii}.\arabic{enumiv}.}

\begin{document}

\maketitle
\section{Dictionary alt bileşeni}
\begin{enumerate}
 \item QstarDict dışındaki tüm sözlükler için ilgili text dosyasını aşağıda bulunan linkten indirebilirsiniz.
\begin{verbatim}
http://cekirdek.pardus.org.tr/~semen/dist/test/office/dictionary/
\end{verbatim}
İlgili sözlük için text dosyasını indirdikten sonra, bu dosyanın içinde, belirtilen dil ile ilgili, bir yanlış bir adette doğru olarak yazılmış kelime göreceksiniz.

Aşağıda verilen komut çıktısında .dic uzantılı dosyalar bulunmakta:
\begin{verbatim}
# pisi info -F <ilgili sözlüğün paket adı> 
\end{verbatim}

Bu dosya adlarını aşağıda bulunan çıktı için kullanacağız
\begin{verbatim}
#  enchant -d <dic uzantılı dosya adının uzantısız hali> <indirilen dosya> -a
\end{verbatim}

Bu çıktının verilen yanlış kelime ile ilgili alternatif doğru kelimeler verdiğini, doğru kelime ile ilgili de bir bilgi vermediğini gözlemleyin.

\begin{itemize}
 \item Örnek olarak: 
\begin{verbatim}
#  enchant -d en_US hunspell-dict-en.txt -a
\end{verbatim}

\end{itemize}
 \item Qstardict paketi için, contrib deposundan stardict-essential-turkish paketini kurun.

	Qstardict uygulmasının düzgün bir şekilde çalıştığını gözlemleyin.
\end{enumerate}

\section{Docbook alt bileşeni}
\begin{enumerate}
 \item Aşağıda bulunan paketler sadece kurulum testine tabidir.
\begin{verbatim}
build-docbook-catalog
SGMLSpm
docbook-dssl
docbook-sgml3_1
docbook-sgml4_1
docbook-sgml4_2
docbook-sgml4_3
docbook-sgml4_4
docbook-sgml4_5
docbook-xml4_1_2
docbook-xml4_2
docbook-xml4_3
docbook-xml4_4
docbook-xml4_5
docbook-xsl
opensp
\end{verbatim}
 \item asciidoc paketi kurulumu sonrası:

Aşağıda bulunan komutları çalıştırın:
\begin{verbatim}
# wget http://cekirdek.pardus.org.tr/~semen/dist/test/office/docbook/testasciidoc.txt
# asciidoc testasciidoc.txt
\end{verbatim}

Sorunsuz bir şekilde testasciidoc.html dosyasının oluştuğunu gözlemleyin.

\item docbook-utils paketi kurulumu sonrası:
\begin{verbatim}
# wget http://cekirdek.pardus.org.tr/~semen/dist/test/office/docbook/test.sgml
# pisi info -F docbook-utils
\end{verbatim}

Pisi çıktısı docbook-utils paketinin dosyalarının sistemde nereye yerleştiği ile ilgili bilgi içermektedir. Bu çıktıda /usr/bin altında bulunan tüm dosyalar ile test.sgml dosyasını çalıştırın.

örnek olarak:
\begin{verbatim}
#docbook2dvi test.sgml
\end{verbatim}

Tüm çalıştırılabilir dosyaların sorunsuz bir şekilde çalıştığını gözlemleyin.

\item openjade paketi kurulumu sonrası:
\begin{verbatim}
# wget http://cekirdek.pardus.org.tr/~semen/dist/test/office/docbook/test.sgml
# openjade -t sgml /usr/share/sgml/docbook/dsssl-stylesheets-1.79/html/docbook.dsl test.sgml
\end{verbatim}

Yukarıdaki komutları çalıştırın, openjade'ın sorunsuz bir şekilde çalıştığını gözlemleyin.

\item sgml-common paketi kurulumu sonrası:
\begin{verbatim}
 # sudo install-catalog --add /etc/sgml/sgml-ent.cat /usr/share/sgml/sgml-iso-entities-8879.1986/catalog
\end{verbatim}

Sorunsuz bir şekilde çalıştığını gözlemleyin.
\item xmlto paketi kurulumu sonrası:

\begin{verbatim}
# wget http://cekirdek.pardus.org.tr/~semen/dist/test/office/docbook/test.xml
# xmlto -o html_dir html test.xml
\end{verbatim}

Yukarıdaki komutları çalıştırın, xmlto paketinin sorunsuz bir şekilde çalıştığını gözlemleyin.
\end{enumerate}
\section{Koffice alt bileşeni}
\subsection*{base alt bileşeni}
\begin{enumerate}
\item Aşağıda bulunan paketler sadece kurulum testine tabidir.

\begin{verbatim}
 koffice-devel 
 koffice-filters
 koffice-kchart
\end{verbatim}
\item koffice-core kurulumu sonrası.

\begin{itemize}
\item KThesaurus uygulamasını menüden açın ve ilişkisi bulunabilecek bir ingilizce kelime girin. (Çünkü kthearus sadece ingilizce dilini desteklemektedir)

Menü $\rightarrow$ uygulamalar $\rightarrow$ ofis $\rightarrow$ daha fazla uygulama yolunu izleyerek sorunsuz bir şekilde açıldığını ve ilgili kelimenin ilişkili olduüğu diğer kelime gruplarını bulabildiğini gözlemleyin.

\item koconverter için:
\begin{verbatim}
# wget http://cekirdek.pardus.org.tr/~semen/dist/test/office/koffice/koconverter_test.html
# koconverter koconverter_test.html koconverter_test.pdf
# wget http://cekirdek.pardus.org.tr/~semen/dist/test/office/koffice/koconverter_test.xls
# koconverter koconverter_test.xls koconverter_test.txt
\end{verbatim}

Çalıştırılan koconverter komutlarının sorunsuz bir şekilde dönüşümü gerçekleştirdiğini gözlemleyin.
\end{itemize}
\item koffice-karbon paketi kurulumu sonrası:

Menü $\rightarrow$ uygulamalar $\rightarrow$ ofis yolunu izleyerek Karbon14 uygulamasının sorunsuz bir şekilde açıldığını gözlemleyin.

Bu uygulama ile basit bir çizim yapıp kaydetmeyi deneyin.

Hatasız bir şekilde işlemi gerçekleştirdiğini gözlemleyin.

\item koffice-kplato paketi kurulumu sonrası:

Menü $\rightarrow$ uygulamalar $\rightarrow$ ofis yolunu izleyerek kplato uygulamasının sorunsuz bir şekilde açıldığını gözlemleyin.

\item koffice-kpresenter paketi kurulumu sonrası:

Menü $\rightarrow$ uygulamalar $\rightarrow$ ofis yolunu izleyerek kpresenter uygulamasının sorunsuz bir şekilde açıldığını gözlemleyin.

Bu uygulamayı kullanarak slide içerisine bir resim bir text ekleyin, start presentation butonuna basın.

Sorunsuz bir şekilde bu işlemlerin gerçekleştiğini gözlemleyin.

\item koffice-krita paketi kurulumu sonrası:

Menü $\rightarrow$ uygulamalar $\rightarrow$ ofis yolunu izleyerek krita uygulamasının sorunsuz bir şekilde açıldığını gözlemleyin.

Bu uygulama ile basit bir çizim yapıp kaydetmeyi deneyin.

Hatasız bir şekilde işlemi gerçekleştirdiğini gözlemleyin.

\item koffice-kspread paketi kurulumu sonrası:

Menü $\rightarrow$ uygulamalar $\rightarrow$ ofis yolunu izleyerek kspread uygulamasının sorunsuz bir şekilde açıldığını gözlemleyin.

Bu uygulamada basit bir taslak açın, (örneğin öğrenci kartı taslağını), bu taslağa resim eklemeye ve bilgileri değiştirmeye çalışın.

Hatasız bir şekilde işlemi gerçekleştirdiğini gözlemleyin.

\item koffice-kword paketi kurulumu sonrası:

Menü $\rightarrow$ uygulamalar $\rightarrow$ ofis yolunu izleyerek kword uygulamasının sorunsuz bir şekilde açıldığını gözlemleyin.

Bu uygulamada çıkan sayfaya bir resim ekleyin, yazı yazın ve kaydedin.

Hatasız bir şekilde işlemi gerçekleştirdiğini gözlemleyin.
\end{enumerate}
\subsection*{l10n alt bileşeni}
Aşağıda bulunan paketlerin kurulum sonrası testleri için. 

Kword uygulamasını açın, yardım bölümünden uygulama dilini seçin, ve ilgili dili işaretleyin. 

Daha sonra uygulamayı kapatıp yeniden açın. Uygulama dilinin sorunsuz olarak seçtiğiniz dile dönüştüğünü gözlemleyin.

\begin{verbatim}
koffice-l10n-ca
koffice-l10n-cs
koffice-l10n-da
koffice-l10n-de
koffice-l10n-el
koffice-l10n-en_GB
koffice-l10n-es
koffice-l10n-et
koffice-l10n-fr
koffice-l10n-gl
koffice-l10n-it
koffice-l10n-ja
koffice-l10n-nl
koffice-l10n-pl
koffice-l10n-pt
koffice-l10n-pt_BR
koffice-l10n-sv
koffice-l10n-tr
koffice-l10n-uk
koffice-l10n-zh_CN
koffice-l10n-zh_TW
\end{verbatim}
\section{Misc alt bileşeni}
\begin{enumerate}
\item Sadece kurulum testine tabidir:
\begin{verbatim}
iso-codes
openclpart
recode

\end{verbatim}
\item aiksaurus paketi kurulumu sonrası:
\begin{verbatim}
 # aiksaurus <ingilizce bir kelime>
\end{verbatim}

Verdiğiniz kelimeye benzer kelimeler bulduğunu gözlemleyin.

\item antiword paketi kurulumu sonrası:
\begin{verbatim}
 # wget http://cekirdek.pardus.org.tr/~semen/dist/test/office/misc/antiword_test.doc
 # antiword antiword_test.doc
\end{verbatim}

Komut çıktısının sorunsuz çalıştığını gözlemleyin.

\item barcode paketi kurulumu sonrası:
\begin{verbatim}
 # barcode
\end{verbatim}
 Bu komudu çalıştırdıktan sonra bir kelime girin ve bu kelimenin barcode olarak sorunsuz bir şekilde kodlnadığını gözlemleyin.

\item djview4, djvu paketi kurulumu sonrası:
\begin{verbatim}
 # wget http://cekirdek.pardus.org.tr/~semen/dist/test/office/misc/djvu_test.djvu
\end{verbatim}

İndirdiğiniz dosya üzerine sağ tıklayarak, birlikte aç bölümünden djview4 seçin. 

Sorunsuz bir şekilde açıldığını gözlemleyin.

\item dos2unix paketi kurulumu sonrası:
\begin{verbatim}
 # wget http://cekirdek.pardus.org.tr/~semen/dist/test/office/misc/test_dos2unix.txt
 # dos2unix test_dos2unix.txt deneme1.txt
 # vi deneme1.txt
\end{verbatim}
Text dosyasının sorunsuz bir şekilde dos formatından unix formatına geçtiğini gözlemleyin. (İlgili satırlar uygun satırlarda olmalıdır.)

\begin{verbatim}
 # wget http://cekirdek.pardus.org.tr/~semen/dist/test/office/misc/test_mac2unix.txt
 # mac2unix test_mac2unix.txt deneme2.txt
 # vi deneme2.txt
\end{verbatim}
Text dosyasının sorunsuz bir şekilde mac formatından unix formatına geçtiğini gözlemleyin. (İlgili satırlar uygun satırlarda olmalıdır.)

\item doxygen paketi kurulumu sonrası:
\begin{verbatim}
 # wget http://cekirdek.pardus.org.tr/~semen/dist/test/office/misc/test_doxgen.c
 # doxygen -g test_doxygen.cfg test_doxygen.c
 # doxygen test_doxygen.cfg
\end{verbatim}

Bulunduğunuz dizinde bir html ve bir de latex dizinin sorunsuz bir şekilde oluştuğunu gözlemleyin.

\item htmldoc paketi kurulumu sonrası:
\begin{verbatim}
 # wget http://cekirdek.pardus.org.tr/~semen/dist/test/office/misc/test_htmldoc.html
\end{verbatim}

Menü $\rightarrow$ Uygulamalar $\rightarrow$ Ofis yolunu izleyerek htmldoc uygulmasını açın.

Sorunsuz bir şekilde açıldığını gözlemleyin.

Htmldoc uygulmasını kullanarak test\_htmldoc.html dosyasını pdf dosyasına dönüştürün.
 
Düzgün bir şekilde dönüşümün gerçekleştiğini gözlemleyin.
\item mftrace paketi kurulumu sonrası:
\begin{verbatim}
 # mftrace cmr10
\end{verbatim}
Sorunsuz bir şekilde cmrt10.pfa dosyasının oluştuğunu gözlemleyin.
\item t1utils paketi kurulumu sonrası:
\begin{verbatim}
 # wget http://cekirdek.pardus.org.tr/~semen/dist/test/office/misc/test_t1utils.pfa
 # t1ascii test_t1utils.pfa
 # t1asm test_t1utils.pfa
 # t1binary test_t1utils.pfa
 # t1disasm test_t1utils.pfa
 # t1mac test_t1utils.pfa
 # t1unmac test_t1utils.pfa
\end{verbatim}

Sorunsuz bir şekilde bu komutların çalıştığını gözlemleyin. 
\item tellico paketi kurulumu sonrası:

Kurulumdan sonra sorunsuz bir şekilde bu uygulamanın açıldığını gözlemleyin.

\item texi2html paketi kurulumu sonrası:
\begin{verbatim}
 # wget http://svn.pardus.org.tr/uludag/trunk/test/2009/testcases/turkish/bug_report-tr.tex
 # texi2html bug_report-tr.tex
\end{verbatim}

Sorunsuz bir şekilde bug\_report-tr.html dosyasının oluştuğunu gözlemleyin.

\item wv paketi kurulumu sonrası:
\begin{verbatim}
 # wget http://svn.pardus.org.tr/uludag/trunk/test/2009/testcases/turkish/test_wv.doc
# wvAbw test_wv.doc deneme.abw                                
# wvCleanLatex test_wv.doc deneme.tex                            
# wvConvert  test_wv.doc                              
# wvDVI   test_wv.doc deneme.dvi                                
# wvDocBook test_wv.doc deneme.sgml                               
# wvHtml test_wv.doc deneme.html                                 
# wvLatex test_wv.doc deneme.tex                                
# wvMime  test_wv.doc deneme.mime                                
# wvPDF test_wv.doc deneme.pdf                                  
# wvPS  test_wv.doc deneme.ps                                  
# wvRTF test_wv.doc deneme.rtf                                  
# wvSummary test_wv.doc            
# wvText   test_wv.doc deneme.txt                               
# wvVersion  test_wv.doc                             
# wvWare  test_wv.doc -X deneme.xml                                
# wvWml test_wv.doc deneme.wml       

\end{verbatim}

Tüm komutların sorunsuz bir şekilde çalıştığını gözlemleyin.


\end{enumerate}
\section{Openoffice alt bileşeni}
\begin{enumerate}
 \item mdbtools paketi kurulumu sonrası:
\begin{verbatim}
# wget http://cekirdek.pardus.org.tr/~semen/dist/test/office/openoffice/test_mdbtools.mdb
# mdb-ver test_mdbtools.mdb
# mdb-tables test_mdbtools.mdb
# mdb-ver test_mdbtools.mdb Client
\end{verbatim}

Yukarıdaki komutların sorunsuz bir şekilde çalıştığını gözlemleyin.

\item openoffice paketi kurulumu sonrası:
\begin{itemize}
\item Openoffice Kelime işlemcisi:
\begin{verbatim}
# wget http://cekirdek.pardus.org.tr/~semen/dist/test/office/openoffice/test_oowriter.doc
# wget http://cekirdek.pardus.org.tr/~semen/dist/test/office/openoffice/test_oowriter.odt
# wget http://cekirdek.pardus.org.tr/~semen/dist/test/office/openoffice/test_oowriter.sxw
# wget http://cekirdek.pardus.org.tr/~semen/dist/test/office/openoffice/test_oowriter.txt
# wget http://cekirdek.pardus.org.tr/~semen/dist/test/office/openoffice/test_oowriter.ott
# wget http://cekirdek.pardus.org.tr/~semen/dist/test/office/openoffice/test_oowriter.html
\end{verbatim}

Yukarıda linkleri verilen dosyaların düzgün bir şekilde açıldığını gözlemleyin.

\item Openoffice Sunum:
\begin{verbatim}
# wget http://cekirdek.pardus.org.tr/~semen/dist/test/office/openoffice/test_ooimpress.odp
# wget http://cekirdek.pardus.org.tr/~semen/dist/test/office/openoffice/test_ooimpress.ppt
# wget http://cekirdek.pardus.org.tr/~semen/dist/test/office/openoffice/test_ooimpress.pot
\end{verbatim}

Yukarıda linkleri verilen dosyaların düzgün bir şekilde açıldığını gözlemleyin.

\item Openoffice Hesap Tablosu:
\begin{verbatim}
# wget http://cekirdek.pardus.org.tr/~semen/dist/test/office/openoffice/test_oocalc.xls
# wget http://cekirdek.pardus.org.tr/~semen/dist/test/office/openoffice/test_oocalc.xlt
# wget http://cekirdek.pardus.org.tr/~semen/dist/test/office/openoffice/test_oocalc.ods
# wget http://cekirdek.pardus.org.tr/~semen/dist/test/office/openoffice/test_oocalc.ots
# wget http://cekirdek.pardus.org.tr/~semen/dist/test/office/openoffice/test_oocalc.csv	
\end{verbatim}

Yukarıda linkleri verilen dosyaların düzgün bir şekilde açıldığını gözlemleyin.

\item Openoffice Çizim:
\begin{verbatim}
# wget http://cekirdek.pardus.org.tr/~semen/dist/test/office/openoffice/test_oodraw.gif
# wget http://cekirdek.pardus.org.tr/~semen/dist/test/office/openoffice/test_oocalc.jpg
# wget http://cekirdek.pardus.org.tr/~semen/dist/test/office/openoffice/test_oocalc.png
# wget http://cekirdek.pardus.org.tr/~semen/dist/test/office/openoffice/test_oocalc.tif
# wget http://cekirdek.pardus.org.tr/~semen/dist/test/office/openoffice/test_oocalc.odg
\end{verbatim}

Yukarıda linkleri verilen dosyaların düzgün bir şekilde açıldığını gözlemleyin.
\item Veritabanı Düzenleyici:
\begin{verbatim}
# wget http://cekirdek.pardus.org.tr/~semen/dist/test/office/openoffice/test_openoffice-base.odb
\end{verbatim}

Yukarıda linki verilen dosyanın düzgün bir şekilde açıldığını gözlemleyin.

\item Web Sayfası Düzenleyici:
\begin{verbatim}
# wget http://cekirdek.pardus.org.tr/~semen/dist/test/office/openoffice/test_openoffice-base.odb
\end{verbatim}

Yukarıda linki verilen dosyanın düzgün bir şekilde açıldığını gözlemleyin.

\end{itemize}

 \item sadece kurulum testine tabidir.
\begin{verbatim}
openoffice-python
\end{verbatim}

 \item openoffice-extension-pdfimport paketi kurulumu sonrası.
\begin{verbatim}
# wget http://cekirdek.pardus.org.tr/~semen/dist/test/office/openoffice/test_openoffice-extension-pdfimport.pdf
\end{verbatim}

\begin{itemize}
\item Menü $\rightarrow$ uygulamalar $\rightarrow$ ofis yolunu izleyerek openoffice Çizim uygulamasını açın.

Yukarıda bulunan pdf dosyasının bu uygulama ile sorunsuz bir şekilde açıldığını gözlemleyin.
\item Üzerinde değişiklikler yapın kaydedin tekrar açın.

Değişiklikler ile birlikte sorunsuz bir şekilde açıldığını gözlemleyin.
\item openoffice Çizim Araçlar $\rightarrow$ Eklenti Yöneticisinden ilgili eklentiyi iptal etmeyi deneyin.

Sorunsuz bir şekilde iptal edebildiğinizi gözlemleyin.
\end{itemize}
 
\item openoffice-extension-presentation-minimizer paketi kurulumu sonrası.
\begin{verbatim}
# wget http://cekirdek.pardus.org.tr/~semen/dist/test/office/openoffice/test_ooimpress.odp
\end{verbatim}

\begin{itemize}
\item Yukarıda bulunan dosyayı openoffice Sunum uygulaması ile açın. Araçlar -> Sunumu Küçült seçeneğine tıklayın ve gerekli aşamaları uygulayın.

Sorunsuz bir şekilde .mini.p uzantılı bir dosyanın oluştuğunu gözlemleyin.

\item Oluşan dosyaya sağ tıklayarak openoffice Çizim uygulamasını seçin.

Düzgün bi şekilde açıldığını gözlemleyin.

\item openoffice Sunum Araçlar $\rightarrow$ Eklenti Yöneticisinden ilgili eklentiyi iptal etmeyi deneyin.

Sorunsuz bir şekilde iptal edebildiğinizi gözlemleyin.
\end{itemize}
 
\item openoffice-extension-presenter-screen paketi kurulumu sonrası.
\begin{verbatim}
# wget http://cekirdek.pardus.org.tr/~semen/dist/test/office/openoffice/test_ooimpress.odp
\end{verbatim}

\begin{itemize}
\item Yukarıda bulunan dosyayı openoffice Sunum uygulaması ile açın. Notlar bölümünden bir not ekleyin. 

F5 tuşuna basın ve bu eklediğiniz notun görüntülenmediğini gözlemleyin.

\item openoffice Sunum Araçlar $\rightarrow$ Eklenti Yöneticisinden ilgili eklentiyi iptal etmeyi deneyin.

Sorunsuz bir şekilde iptal edebildiğinizi gözlemleyin.
\end{itemize}
\item openoffice-extension-report-builder paketi kurulumu sonrası.
\begin{verbatim}
# wget http://cekirdek.pardus.org.tr/~semen/dist/test/office/openoffice/test_openoffice-base.odb
\end{verbatim}

\begin{itemize}
\item Yukarıda bulunan dosyayı openoffice VeriTabanı uygulaması ile açın. Araçlar -> Rapor seçeneğine tıklayın.

Yeni bir rapor yazmaya çalışın, sorunsuz bir şekilde yazılabildiğinizi gözlemleyin.

\item openoffice Sunum Araçlar $\rightarrow$ Eklenti Yöneticisinden ilgili eklentiyi iptal etmeyi deneyin.

Sorunsuz bir şekilde iptal edebildiğinizi gözlemleyin.
\end{itemize}

\item openoffice-extension-wiki-publisher paketi kurulumu sonrası.

Openoffice Kelime İşlemci'yi açın. Dosya $\rightarrow$ Gönder $\rightarrow$ MediaWiki'ye seçeneğini tıklayın. Wiki sunucusu olarak http://tr.pardus-wiki.org/ adresini ekleyip bağlanmaya çalışın.

Sorunsuz bir şekilde bağlandığınız gözlemleyin.
\item Aşağıda bulunan paketlerin kurulumu sonrasında, bir open office uygulması açın ve yardım dosyasının ilgili dilde olduğunu gözlemleyin.
\begin{verbatim}
openoffice-help-en 
openoffice-help-de
openoffice-help-es
openoffice-help-fr
openoffice-help-it
openoffice-help-nl
openoffice-help-pt-BR
openoffice-help-sv
openoffice-help-tr
\end{verbatim}
\item Aşağıda bulunan paketlerin kurulumu sonrasında, bir open office uygulması açın ve uygulamanın ilgili dilde açıldığını gözlemleyin.
\begin{verbatim}
openoffice-langpack-de
openoffice-langpack-es
openoffice-langpack-fr
openoffice-langpack-fr
openoffice-langpack-nl
openoffice-langpack-pt-BR
openoffice-langpack-sv
openoffice-langpack-tr
\end{verbatim}

\item openoffice-kde paketi kurulumu sonrası:

Dolphin uygulmasını açın solda bulunan Konumlar bölümünde sağ tıklayıp, yeni ekle deyin. Bir simge seçin ikonunun üzerine tıklayın ve diğer simgeleri seçin. Ara böümüne oo yazın.

Aşağıda çıkan ikon penceresinde openoffice uygulamaları ile ilgili ikonların bulunduğunu gözlemleyin.

Bir open office ikonu seçin ve tamam deyin.

İkonun sol bölüme sorunsuz bir şekilde eklendiğini gözlemleyin.

\item zemberek-openoffice

\begin{verbatim}
# wget http://cekirdek.pardus.org.tr/~semen/dist/test/office/openoffice/test-zemberek-openoffice.odt
\end{verbatim}

Yukarıda bulunan dosyayı açın ve Araçlar $\rightarrow$ İmla denetimi ve Dil bilgisi bölümünden imla denetiminin gerçekleştiğini gözlemleyin.
\end{enumerate}

\end{document}

