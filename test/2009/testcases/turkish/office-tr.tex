\documentclass[a4paper,10pt]{article}
\usepackage[turkish]{babel}
\usepackage[utf8]{inputenc}
\usepackage[left=1cm,top=2cm,right=2cm,bottom=1cm]{geometry}

\title{Ofis Bileşeni Test Aşamaları}
\author{Semen Cirit}

\renewcommand{\labelenumi}{\arabic{enumi}.}
\renewcommand{\labelenumii}{\arabic{enumi}.\arabic{enumii}.}
\renewcommand{\labelenumiii}{\arabic{enumi}.\arabic{enumii}.\arabic{enumiii}.}
\renewcommand{\labelenumiv}{\arabic{enumi}.\arabic{enumii}.\arabic{enumiii}.\arabic{enumiv}.}

\begin{document}

\maketitle
\subsection*{Dictionary}
\begin{enumerate}
 \item QstarDict dışındaki tüm sözlükler için ilgili text dosyasını aşağıda bulunan linkten indirebilirsiniz.
\begin{verbatim}
http://cekirdek.pardus.org.tr/~semen/dist/test/office/dictionary/
\end{verbatim}
İlgili sözlük için text dosyasını indirdikten sonra, bu dosyanın içinde, belirtilen dil ile ilgili, bir yanlış bir adette doğru olarak yazılmış kelime göreceksiniz.

Aşağıda verilen komut çıktısında .dic uzantılı dosyalar bulunmakta:
\begin{verbatim}
# pisi info -F <ilgili sözlüğün paket adı> 
\end{verbatim}

Bu dosya adlarını aşağıda bulunan çıktı için kullanacağız
\begin{verbatim}
#  enchant -d <dic uzantılı dosya adının uzantısız hali> <indirilen dosya> -a
\end{verbatim}

Bu çıktının verilen yanlış kelime ile ilgili alternatif doğru kelimeler verdiğini, doğru kelime ile ilgili de bir bilgi vermediğini gözlemleyin.

\begin{itemize}
 \item Örnek olarak: 
\begin{verbatim}
#  enchant -d en_US hunspell-dict-en.txt -a
\end{verbatim}

\end{itemize}
 \item Qstardict paketi için, contrib deposundan stardict-essential-turkish paketini kurun.

	Qstardict uygulmasının düzgün bir şekilde çalıştığını gözlemleyin.
\end{enumerate}

\subsection*{Docbook}
\begin{enumerate}
 \item Aşağıda bulunan paketler sadece kurulum testine tabidir.
\begin{verbatim}
build-docbook-catalog
SGMLSpm
docbook-dssl
docbook-sgml3_1
docbook-sgml4_1
docbook-sgml4_2
docbook-sgml4_3
docbook-sgml4_4
docbook-sgml4_5
docbook-xml4_1_2
docbook-xml4_2
docbook-xml4_3
docbook-xml4_4
docbook-xml4_5
docbook-xsl
opensp
\end{verbatim}
 \item asciidoc paketi kurulumu sonrası:

Aşağıda bulunan komutları çalıştırın:
\begin{verbatim}
# wget http://cekirdek.pardus.org.tr/~semen/dist/test/office/docbook/testasciidoc.txt
# asciidoc testasciidoc.txt
\end{verbatim}

Sorunsuz bir şekilde testasciidoc.html dosyasının oluştuğunu gözlemleyin.

\item docbook-utils paketi kurulumu sonrası:
\begin{verbatim}
# wget http://cekirdek.pardus.org.tr/~semen/dist/test/office/docbook/test.sgml
# pisi info -F docbook-utils
\end{verbatim}

Pisi çıktısı docbook-utils paketinin dosyalarının sistemde nereye yerleştiği ile ilgili bilgi içermektedir. Bu çıktıda /usr/bin altında bulunan tüm dosyalar ile test.sgml dosyasını çalıştırın.

örnek olarak:
\begin{verbatim}
#docbook2dvi test.sgml
\end{verbatim}

Tüm çalıştırılabilir dosyaların sorunsuz bir şekilde çalıştığını gözlemleyin.

\item openjade paketi kurulumu sonrası:
\begin{verbatim}
# wget http://cekirdek.pardus.org.tr/~semen/dist/test/office/docbook/test.sgml
# openjade -t sgml /usr/share/sgml/docbook/dsssl-stylesheets-1.79/html/docbook.dsl test.sgml
\end{verbatim}

Yukarıdaki komutları çalıştırın, openjade'ın sorunsuz bir şekilde çalıştığını gözlemleyin.

\item sgml-common paketi kurulumu sonrası:
\begin{verbatim}
 # sudo install-catalog --add /etc/sgml/sgml-ent.cat /usr/share/sgml/sgml-iso-entities-8879.1986/catalog
\end{verbatim}

Sorunsuz bir şekilde çalıştığını gözlemleyin.
\item xmlto paketi kurulumu sonrası:

\begin{verbatim}
# wget http://cekirdek.pardus.org.tr/~semen/dist/test/office/docbook/test.xml
# xmlto -o html_dir html test.xml
\end{verbatim}

Yukarıdaki komutları çalıştırın, xmlto paketinin sorunsuz bir şekilde çalıştığını gözlemleyin.
\end{enumerate}
\subsection*{Koffice}
\begin{enumerate}
\item Aşağıda bulunan paketler sadece kurulum testine tabidir.

\begin{verbatim}
 koffice-devel 
 koffice-filters
 koffice-kchart
\end{verbatim}
\item koffice-core kurulumu sonrası.

\begin{itemize}
\item KThesaurus uygulamasını menüden açın ve ilişkisi bulunabilecek bir ingilizce kelime girin. (Çünkü kthearus sadece ingilizce dilini desteklemektedir)

Menü $\rightarrow$ uygulamalar $\rightarrow$ ofis $\rightarrow$ daha fazla uygulama yolunu izleyerek sorunsuz bir şekilde açıldığını ve ilgili kelimenin ilişkili olduüğu diğer kelime gruplarını bulabildiğini gözlemleyin.

\item koconverter için:
\begin{verbatim}
# wget http://cekirdek.pardus.org.tr/~semen/dist/test/office/koffice/koconverter_test.html
# koconverter koconverter_test.html koconverter_test.pdf
# wget http://cekirdek.pardus.org.tr/~semen/dist/test/office/koffice/koconverter_test.xls
# koconverter koconverter_test.xls koconverter_test.txt
\end{verbatim}

Çalıştırılan koconverter komutlarının sorunsuz bir şekilde dönüşümü gerçekleştirdiğini gözlemleyin.
\end{itemize}
\item koffice-karbon paketi kurulumu sonrası:

Menü $\rightarrow$ uygulamalar $\rightarrow$ ofis yolunu izleyerek Karbon14 uygulamasının sorunsuz bir şekilde açıldığını gözlemleyin.

\item koffice-kplato paketi kurulumu sonrası:

Menü $\rightarrow$ uygulamalar $\rightarrow$ ofis yolunu izleyerek kplato uygulamasının sorunsuz bir şekilde açıldığını gözlemleyin.


\end{enumerate}

\end{document}

