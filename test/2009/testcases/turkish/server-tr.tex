\documentclass[a4paper,10pt]{article}
\usepackage[turkish]{babel}
\usepackage[utf8]{inputenc}
 \usepackage[left=1cm,top=1cm,right=2cm,bottom=2 cm]{geometry}

\title{Server Bileşeni Test Aşamaları}
\author{Semen Cirit}

\renewcommand{\labelenumi}{\arabic{enumi}.}
\renewcommand{\labelenumii}{\arabic{enumi}.\arabic{enumii}.}
\renewcommand{\labelenumiii}{\arabic{enumi}.\arabic{enumii}.\arabic{enumiii}.}
\renewcommand{\labelenumiv}{\arabic{enumi}.\arabic{enumii}.\arabic{enumiii}.\arabic{enumiv}.}

\begin{document}

\maketitle
\section{Web alt bileşeni}
\begin{enumerate}
\item apache paketi kurulumu sonrası:
\begin{itemize}
\item Servis yöneticisinden apache sunucusunu başlatın. Aşağıda bulunan komutu kullanarak sunucunun başlatılmıl olduğunu gözlemleyin.
\begin{verbatim}
# service list
\end{verbatim}
\item Firefox üzerinden http://localhost adresine bağlanın ve sorunsuz bir şekilde bağlanabildiğinizi gözlemleyin. 
\item Aşağıdaki komutu çalıştırın ve komut çıktısında `Syntax Ok` aldığınzı gözlemleyin.
\begin{verbatim}
# apachectl -M 
\end{verbatim}

\end{itemize}

\item mod\_php paketi kurulumu sonrası:

\begin{itemize}
 \item Contrib deposunda bulunan phpmyadmin paketini kurun.
 \item Apache ve mysql sunucularını servis yöneticisinden başlatın.
 \item http://localhost/phpmyadmin/ adresine firefox kullanarak girin. (Kullanıcı adı için root yazın, parola ise boş kalacak)
 \item Mysql bağlantı sayfasınının sorunsuz açıldığını gözlemleyin.
\end{itemize}


\end{enumerate}

\section{Database alt bileşeni}
\begin{enumerate}

 \item firebird-superserver ve firebird-client paketleri kurulumu sonrası:

Bilgisayarınızı yeniden başlatın,

Servis yöneticisinden firebird-superserver'ı başlatın.

Aşağıda bulunan komutları sırası ile çalıştırın ve sorunsuz bir şekilde çalıştığını gözlemleyin:
\begin{verbatim}
# cd /opt/firebird/examples/empbuild
# isql (2008 için)
# fb_isql (2009 için)

SQL> CONNECT employee.fdb user sysdba password masterkey;
SQL> show tables;
SQL> select *from COUNTRY
\end{verbatim}

 \item mysql-client, mysql-server, mysql-lib paketleri kurulumu sonrası:
\begin{itemize}
 \item Servise yöneticisinden Mysql'i başlatın ve  aşağıda bulunan komutu kullanarak başlatılmış olduğundan emin olun:

\begin{verbatim}
 # service list
\end{verbatim}
 \item desktop-tr.pdf qt-sql-mysql testini gerçekleştirin.

\end{itemize}

\item mysql-man-pages paketi kurulumu sonrası:

Aşağıda bulunan komutun man sayfasını düzgün açtığından emin olun.
\begin{verbatim}
# man myisampack 
\end{verbatim}

\end{enumerate}

\section{Diğerleri}

\begin{itemize}
 \item dhcp paketi kurulumu sonrası:

Ağ yöneticisinden dhcp kullanarak bir ağa bağlanmayı deneyin. Daha sonra konsoldan aşağıda bulunan komutu çalıştırın ve ağa bağlı olduğununuzu gözlemleyin.
\begin{verbatim}
# ping 4.2.2.1
\end{verbatim}

\item bind ve bind-tools paketleri kurulumu sonrası:
\begin{verbatim}
# dig www.google.com
\end{verbatim}
Yukarıda bulunan komutun düzgün bir şekilde dns sunucuları listelediğini gözlemleyin.

\end{itemize}


\end{document}

