\documentclass[a4paper,10pt]{article}
\usepackage[turkish]{babel}
\usepackage[utf8]{inputenc}
\usepackage[left=1cm,top=2cm,right=2cm,bottom=1cm]{geometry}

\title{Server Bileşeni Test Aşamaları}
\author{Semen Cirit}

\renewcommand{\labelenumi}{\arabic{enumi}.}
\renewcommand{\labelenumii}{\arabic{enumi}.\arabic{enumii}.}
\renewcommand{\labelenumiii}{\arabic{enumi}.\arabic{enumii}.\arabic{enumiii}.}
\renewcommand{\labelenumiv}{\arabic{enumi}.\arabic{enumii}.\arabic{enumiii}.\arabic{enumiv}.}

\begin{document}

\maketitle
\section{Web alt bileşeni}
\begin{enumerate}
\item Apache paketi kurulumu sonrası:
\begin{itemize}
\item Servis yöneticisinden apache sunucusunu başlatın. Aşağıda bulunan komutu kullanarak sunucunun başlatılmıl olduğunu gözlemleyin.
\begin{verbatim}
# service list
\end{verbatim}
\item Firefox üzerinden http://localhost adresine bağlanın ve sorunsuz bir şekilde bağlanabildiğinizi gözlemleyin. 
\item Aşağıdaki komutu çalıştırın ve komut çıktısında `Syntax Ok` aldığınzı gözlemleyin.
\begin{verbatim}
# apachectl -M 
\end{verbatim}

\end{itemize}

\end{enumerate}
\begin{itemize}
 \item dhcp paketi kurulumu sonrası:

Ağ yöneticisinden dhcp kullanarak bir ağa bağlanmayı deneyin. Daha sonra konsoldan aşağıda bulunan komutu çalıştırın ve ağa bağlı olduğununuzu gözlemleyin.
\begin{verbatim}
# ping 4.2.2.1
\end{verbatim}

\item bind ve bind-tools paketleri kurulumu sonrası:
\begin{verbatim}
# dig www.google.com
\end{verbatim}
Yukarıda bulunan komutun düzgün bir şekilde dns sunucuları listelediğini gözlemleyin.

\end{itemize}


\end{document}

