\documentclass[a4paper,10pt]{article}
\usepackage[utf8x]{inputenc}
\usepackage[left=1cm,top=2cm,right=2cm,bottom=1cm]{geometry}

%opening
\title{X11 Test Aşamaları}
\author{Semen Cirit}

\begin{document}

\maketitle

 Note:	
  	Tüm X hataları için  aşağıdaki dosyaya göz atabilirsiniz:
	/var/log/Xorg.0.log  

\section{Server alt Bileşeni}

Zorg ve xorg-server paketleri kurulumu sonrası aşağıda bulunan aşamalar uygulanmalıdır.

\begin{enumerate}
  \item Direct rendering testi
    \begin{enumerate}
    \item Eğer xterm paketi sisteminizde kurulu değil ise bu test için ilk olarak bu paketi kurunuz.
    \item   $Ctrl+Alt+F1$ tuşlarına aynı anda olacak şekilde basın.

	    X'in çökmediğini gözlemlemleyin.
    \item  Daha sonra bu komutu çalıştırın.
	  \begin{verbatim}
	  xinit -- :2
	  \end{verbatim}
	  Karşınıza koyu gri bir masaüstü, x işareti şeklinde bir fare imleci ve bir terminal penceresi çıktığını gözlemleyiniz..
    \item Grafik arayüzüne geçebilmek için \texttt{Ctrl+Alt+F7} tuşlarına aynı anda olacak şekilde basın.    
	   X'in çökmediğini gözlemlemleyin.
    \item "glxgears" ve "glxgears -fullscreen" komutlarını konsoldan çalıştırın.

	  Hiçbir problem olmadan çalışabildiklerini gözlemleyin.
    \end{enumerate}

  \item DPI (Dots Per Inch)testi
    
        Menüden bir openoffice yazıcı çalıştırın ve 100\% büyütülmüş bir A4 kağıt seçin. Aynı zamanda gerçek bir A4 kağıdı elinize alın.

        Gerçek A4 kağıdının boyutları ile openoffice yazıcıda açtığınız A4 kağıdının boyutlarının aynı olduğunu gözlemleyin.
  \item Çözünürlük testi
     \begin{enumerate}
      \item Menüden Sistem ayarları $\rightarrow$ Ekran yolunu izleyin

	Gerçek çözünürlük ile seçilmiş çözünürlüğün aynı olduğunu gözlemleyin.

      \item Gerçek ve seçilmiş çözünürlüğünüzü "xrandr" komutunu kullanarak bulabilirsiniz.
	  
	  Örnek komut çıktısı:
	  
	  1440x900  50.0*+
	  
	  Bu bilgi:
	  \begin{table}[h]
	  \centering
	  \begin{tabular}{|c|c|c|}
		  \hline
		  $1440x900$ & $\rightarrow$ & çözünürlük \\
		  50.0       & $\rightarrow$ & tazelenme sıklığı \\
		  $*$        & $\rightarrow$ & örnek çözünürlüğün seçilmiş çözünürlük olduğu \\
		  $+$        & $\rightarrow$ & örnek çözünürlüğün gerçek çözünürlük olduğu\\
		  \hline
	  \end{tabular} 
	  \label{tab:tbl}
	  \end{table}

	  Bu $+$ ve $*$ işaretlerinin aynı çözünürlüğe ait olduğunu gözlemleyin. 

    \item Bilgisayarınızı yeniden başlatıp, açılış ekranından F4 fonksiyon tuşu ile güvenli kipi seçerek bilgisayarınızı açın.  

	  Daha sonra ise tekar restart ederek F4 fonksiyon tuşu ile tekrar tanıyı seçerek açın.

	  Grafiksel olarak bir sorun yaşamadığınızı gözlemleyin.
    \end{enumerate}
 
 \item  Bir medya oynatıcıyı tam ekran olarak açın. 

      X'in çökmediğini gözlemleyin.
 \item(2009 için) Menüden Sistem ayarları $\rightarrow$ masaüstü yolunu izleyerek efektlerinizi açın
 
  X'in çökmediğini gözlemleyin.
  \item 2D (iki boyut) testi 
      
      Bir program penceresi açın (örneğin mplayer, dolphin, firefox) ve sağa sola hızlıca sallayın.
        
      Hiçbir problem olmadan sallandığını gözlemleyin.

 \item (2009 için)3D (üç boyut)testi
   %(For Pardus 2008 first execute glxinfo and find the Direct rendering is yes.)

    Menüden uygulamalar $\rightarrow$ eğitim $\rightarrow$ bilim $\rightarrow$ marble yolunu izleyin.
	
    Uygulamanın grafiksel olarak düzgün çalıştığını gözlemleyin.
    \end{enumerate}
\end{document}
