\documentclass[a4paper,10pt]{article}
\usepackage[utf8x]{inputenc}
\usepackage[left=1cm,top=1cm,right=2cm,bottom=2cm]{geometry}

%opening
\title{X11 Test Aşamaları}
\author{Semen Cirit}

\begin{document}

\maketitle

 Note:	
  	Tüm X hataları için  aşağıdaki dosyaya göz atabilirsiniz:
	/var/log/Xorg.0.log  

\section{Server alt Bileşeni}

\begin{itemize}
 \item Aşağıda bulunan paketler kurulum testine tabidir.
\begin{verbatim}
 xorg-server-common
 xorg-server-devel
 xorg-server-xdmx
 xorg-server-xvfb
\end{verbatim}

 \item xorg-server-xephyr paketi kurulumu sonrası:

Aşağıda bulunan komutun sorunsuz bir şekilde konsol açtığını gözlemleyin.
\begin{verbatim}
 xinit -- /usr/bin/Xephyr :1
\end{verbatim}

\item  Aşağıda bulunan paketlerin kurulumu sonrasında aşağıdaki testler uygulanmalıdır. 
\begin{verbatim}
zorg    
xorg-server
libdrm
mesa
\end{verbatim}


\begin{enumerate}
  \item Direct rendering testi
    \begin{enumerate}
    \item Eğer xterm paketi sisteminizde kurulu değil ise bu test için ilk olarak bu paketi kurunuz.
    \item   $Ctrl+Alt+F1$ tuşlarına aynı anda olacak şekilde basın.

	    X'in çökmediğini gözlemlemleyin.
    \item  Daha sonra bu komutu çalıştırın.
	  \begin{verbatim}
	  xinit -- :2
	  \end{verbatim}
	  Karşınıza koyu gri bir masaüstü, x işareti şeklinde bir fare imleci ve bir terminal penceresi çıktığını gözlemleyiniz..
    \item Grafik arayüzüne geçebilmek için \texttt{Ctrl+Alt+F7} tuşlarına aynı anda olacak şekilde basın.    
	   X'in çökmediğini gözlemlemleyin.
    \item "glxgears" ve "glxgears -fullscreen" komutlarını konsoldan çalıştırın.

	  Hiçbir problem olmadan çalışabildiklerini gözlemleyin.
    \end{enumerate}

  \item DPI (Dots Per Inch)testi
    
        Menüden bir openoffice yazıcı çalıştırın ve 100\% büyütülmüş bir A4 kağıt seçin. Aynı zamanda gerçek bir A4 kağıdı elinize alın.

        Gerçek A4 kağıdının boyutları ile openoffice yazıcıda açtığınız A4 kağıdının boyutlarının aynı olduğunu gözlemleyin.
  \item Çözünürlük testi
     \begin{enumerate}
      \item Menüden Sistem ayarları $\rightarrow$ Ekran yolunu izleyin

	Gerçek çözünürlük ile seçilmiş çözünürlüğün aynı olduğunu gözlemleyin.

      \item Gerçek ve seçilmiş çözünürlüğünüzü "xrandr" komutunu kullanarak bulabilirsiniz.
	  
	  Örnek komut çıktısı:
	  
	  1440x900  50.0*+
	  
	  Bu bilgi:
	  \begin{table}[h]
	  \centering
	  \begin{tabular}{|c|c|c|}
		  \hline
		  $1440x900$ & $\rightarrow$ & çözünürlük \\
		  50.0       & $\rightarrow$ & tazelenme sıklığı \\
		  $*$        & $\rightarrow$ & örnek çözünürlüğün seçilmiş çözünürlük olduğu \\
		  $+$        & $\rightarrow$ & örnek çözünürlüğün gerçek çözünürlük olduğu\\
		  \hline
	  \end{tabular} 
	  \label{tab:tbl}
	  \end{table}

	  Bu $+$ ve $*$ işaretlerinin aynı çözünürlüğe ait olduğunu gözlemleyin. 

    \item Bilgisayarınızı yeniden başlatıp, açılış ekranından F4 fonksiyon tuşu ile güvenli kipi seçerek bilgisayarınızı açın.  

	  Daha sonra ise tekar restart ederek F4 fonksiyon tuşu ile tekrar tanıyı seçerek açın.

	  Grafiksel olarak bir sorun yaşamadığınızı gözlemleyin.
    \end{enumerate}
 
 \item  Bir medya oynatıcıyı tam ekran olarak açın. 

      X'in çökmediğini gözlemleyin.
 \item(2009 için) Menüden Sistem ayarları $\rightarrow$ masaüstü yolunu izleyerek efektlerinizi açın
 
  X'in çökmediğini gözlemleyin.
  \item 2D (iki boyut) testi 
      
      Bir program penceresi açın (örneğin mplayer, dolphin, firefox) ve sağa sola hızlıca sallayın.
        
      Hiçbir problem olmadan sallandığını gözlemleyin.

 \item (2009 için)3D (üç boyut)testi
   %(For Pardus 2008 first execute glxinfo and find the Direct rendering is yes.)

    Menüden uygulamalar $\rightarrow$ eğitim $\rightarrow$ bilim $\rightarrow$ marble yolunu izleyin.
	
    Uygulamanın grafiksel olarak düzgün çalıştığını gözlemleyin.
    \end{enumerate}

\end{itemize}



\section{Driver alt Bileşeni}
\begin{enumerate}
 \item virtualbox-guest-utils paketi kurulumu sonrası:

hardware-tr.pdf virtualbox testini gerçekleştirin.

 \item xorg-input-synaptics  paketi kurulumu sonrası:

Eğer aşağıda bulunan komut bir çıktı üretiyor ise bu testi gerçekleştirebilirsiniz.
  \begin{verbatim}
   # grep -i synap /proc/bus/input/devices
  \end{verbatim}

Bilgisayırınızı yeniden başlatın ve mousepad'inizin düzgün çalıştığını gözlemleyin. 

\item xorg-input-vmmouse paketi kurulumu sonrası:

Eğer vmware sanal makine kullanıyor iseniz bu testi gerçekleştirebilirsiniz. 

VMware sanal makinelesinde fare uyumunu test ediniz.

\item xorg-input-wacom paketi kurulumu sonrası:

Wacom tabletiniz var ise bu testi gerçekleştirebilirsiniz.

Wacom tabletinizi takın ve çalışıp çalışmadığını gözlemleyin.

\item Aşağıda bulunan ekran kartı sürücüsü paketleri aynı şekilde test edilecektir elinizde bulunan ekran kartınına göre testlerimizi gerçekleştireceğiz. 

Örneğin: xorg-video-apm paketini Alliance Promotion ekran kartınız var ise test edebilirsiniz.

Paket hakkında ayrıntılı bilgiye aşağıda bulunan komut ile bakıp hangi ekran kartı ile ilgili olduğuna bakabilirsiniz.
\begin{verbatim}
 # pisi info <paketadı>
\end{verbatim}

Ekran kartınızın ne olduğuna dair bilgiye aşağıda bulunan komut ile ulaşabilirsiniz.
\begin{verbatim}
 # lspci
\end{verbatim}

\begin{verbatim}
 xorg-video-apm 
 xorg-video-ast
 xorg-video-cirru
 xorg-video-fbdev (bu sürücü ekran kartından bağımsızdır ve direk teste tabidir.)
 xorg-video-glint
 xorg-video-i128
 xorg-video-i740
 xorg-video-intel
 xorg-video-mach64
 xorg-video-mga
 xorg-video-neomagic
 xorg-video-r128
 xorg-video-radeon
 xorg-video-radeonhd
 xorg-video-s3
 xorg-video-s3virge
 xorg-video-savage
 xorg-video-siliconmotion
 xorg-video-sis
 xorg-video-sisusb
 xorg-video-tdfx
 xorg-video-trident,
 xorg-video-vesa (bu sürücü tüm testçiler tarafından yapılabilir, bilgisayarınızı güvenli kipte açarak, daha sonra diğer paketler ile aynı şekilde test edebilirsiniz.)
 xorg-video-vmware (vmware kullanıyor iseniz test edebilirsiniz.)
 xorg-video-voodoo
 xorg-video-chips
\end{verbatim}

\begin{itemize}
 \item Bilgisayarınızı yeniden başlatın, ve açtığınızda çözünürlüğün düzgün olduğunu gözlemleyin.

 Bilgisayarınızda bir yavaşlık olmadığını gözlemleyin
 \item Aşağıda bulunan viedoyu mplayer, kaffeine gibi bir oynatıcı ile oynatın ve sorunsuz bir şekilde çalıştığını gözlemleyin.
 \begin{verbatim}
  # wget http://cekirdek.pardus.org.tr/~semen/dist/test/multimedia/video/cokluortam/DVD.mpg
 \end{verbatim}


\end{itemize}

\end{enumerate}

\section{Terminal alt Bileşeni}
\begin{enumerate}
 \item rxvt-unicode paketi kurulumu sonrası:

Bir x konsol penceresi açıldığını gözlemleyin.
  \begin{verbatim}
   # urxvt
  \end{verbatim}

Aşağıda bulunan ilk komut sunucuyu başlatmaktadır, bir konsolda sunucuyu başlatın diğer konsola geçip ikinci satırda bulunan komutu çalıştırın ve bir x konsol penceresi açıldığını gözlemleyin.
  \begin{verbatim}
   # urxvtd (sunucuyu çalıştırın)
   # urxvtc
  \end{verbatim}

\item xterm paketi kurulumu sonrası:

Aşağıda bulunan komutun bir konsol çalıştırdığını gözlemleyin.
 \begin{verbatim}
  # xterm
 \end{verbatim}

\end{enumerate}


\end{document}
