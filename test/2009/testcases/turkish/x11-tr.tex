\documentclass[a4paper,10pt]{article}
\usepackage[utf8x]{inputenc}
\usepackage[left=1cm,top=1cm,right=2cm,bottom=2cm]{geometry}

%opening
\title{X11 Test Aşamaları}
\author{Semen Cirit}

\begin{document}

\maketitle

Not: Bu testlerde karşılaştığınız hataları bildirirken /var/log dizini altında yer alan Xorg.0.log
dosyasını iletmeyi unutmayınız.

\section{Server Alt Bileşeni}

\begin{itemize}
  \item Aşağıdaki paketler için kurulum testi uygulanacaktır.
  \begin{verbatim}
  xorg-server-common
  xorg-server-devel
  xorg-server-xdmx
  xorg-server-xvfb
  \end{verbatim}

  \item xorg-server-xephyr paketi kurulduktan sonra aşağıda bulunan komut ile sorunsuz bir şekilde konsol
  açıldığını gözlemleyin.
  \begin{verbatim}
  xinit -- /usr/bin/Xephyr :1
  \end{verbatim}

  \item Kurulumu yapılacak diğer paketler ve uygulanacak testler aşağıda verilmiştir.
  \begin{verbatim}
  zorg    
  xorg-server
  libdrm
  mesa
  \end{verbatim}


  \begin{enumerate}
  \item Temel işlevler
    \begin{enumerate}
    \item Sisteminiz yeniden başlatın ve grafiksel oturum açma ekranının sorunsuz bir şekilde açıldığını gözlemleyin.
    \item Yukarıdaki testi açılış ekranında "Grafik Ekran (F4)" menüsünden "Yeniden Tanı" seçeneğini seçerek tekrarlayın.
    \item xterm paketinin sisteminizde kurulu olduğundan emin olun ve şu komutu verin:
      \begin{verbatim}
xinit -- :2
      \end{verbatim}
      Yeni bir X ekranı içinde bir terminal penceresi açıldığını gözlemleyin. Fare imlecini terminal penceresi üzerine getirip "exit" komutunu vererek çıkabilirsiniz.
    \item Herhangi bir uygulama penceresini taşıyın ve boyutlandırın. Bu işlemlerin hızlı bir şekilde gerçekleştiğini gözlemleyin.
    \end{enumerate}

  \item 3B desteği
    \begin{enumerate}
    \item "glxgears" ve "glxgears -fullscreen" komutlarını konsoldan çalıştırın. Karşınıza dönen üç dişlinin sorunsuz bir şekilde geldiğini gözlemleyin.
    \end{enumerate}

  \item VT değiştirme
    \begin{enumerate}
    \item Grafiksel arayüzde iken \texttt{Ctrl+Alt+F1} tuşlarına aynı anda basın. TTY konsol ekranına sorunsuz bir şekilde geçtiğini gözlemleyin.
    \item Grafik arayüzüne geri dönmek için \texttt{Ctrl+Alt+F7} tuşlarına aynı anda basın. Bu aşamada da sorun çıkmadığını gözlemleyin.
    \end{enumerate}

  \item Çözünürlük testi
    \begin{enumerate}
    \item Menüden, Ayarlar $\rightarrow$ Ekran Ayarları (2008 için TASMA'da Görüntü Yöneticisi) yolunu izleyin.
    \item Ekranlar bölümünde kullandığınız çıkış aygıtı için bir çözünürlük seçin ve uygulayın.
    \item Yeniden oturum açın ve "xrandr" komutu ile o anki çözünürlüğün seçtiğiniz çözünürlük
          ile aynı olup olmadığını kontrol edin. Etkin çözünürlüğün yanında * işareti bulunacaktır.
    \end{enumerate}
 
  \item Video testi
    \begin{enumerate}
    \item Herhangi bir medya oynatıcıyı kullanarak tam ekranda bir video dosyası açın ve sorunsuz bir şekilde oynattığını gözlemleyin.
    \end{enumerate}
 
  \item (2009 için) Efekt testi
    \begin{enumerate}
    \item Menüden Sistem Ayarları $\rightarrow$ Masaüstü yolunu izleyerek efektlerinizi açın/kapatın. Bu sırada herhangi bir sorun yaşanmadığını gözlemleyin.
    \end{enumerate}
  \end{enumerate}
\end{itemize}

\section{Driver Alt Bileşeni}
\begin{enumerate}
  \item virtualbox-guest-utils paketi kurulumu sonrası:

hardware-tr.pdf virtualbox testini gerçekleştirin.

  \item xorg-input-synaptics  paketi kurulumu sonrası:

Eğer aşağıda bulunan komut bir çıktı üretiyor ise bu testi gerçekleştirebilirsiniz.
  \begin{verbatim}
   # grep -i synap /proc/bus/input/devices
  \end{verbatim}

Bilgisayarınızı yeniden başlatın ve dokunmatik giriş aygıtının (touchpad) düzgün çalıştığını gözlemleyin.

\item xorg-input-vmmouse paketi kurulumu sonrası:

Eğer vmware sanal makine kullanıyor iseniz bu testi gerçekleştirebilirsiniz. 

VMware sanal makinelesinde fare uyumunu test ediniz.

\item xorg-input-wacom paketi kurulumu sonrası:

Wacom tabletiniz var ise bu testi gerçekleştirebilirsiniz.

Wacom tabletinizi takın ve çalışıp çalışmadığını gözlemleyin.

\item Aşağıda bulunan ekran kartı sürücüsü paketleri aynı şekilde test edilecektir elinizde bulunan ekran kartınına göre testlerimizi gerçekleştireceğiz. 

Örneğin: xorg-video-apm paketini Alliance Promotion ekran kartınız var ise test edebilirsiniz.

Paket hakkında ayrıntılı bilgiye aşağıda bulunan komut ile bakıp hangi ekran kartı ile ilgili olduğuna bakabilirsiniz.
\begin{verbatim}
 # pisi info <paketadı>
\end{verbatim}

Ekran kartınızın ne olduğuna dair bilgiye aşağıda bulunan komut ile ulaşabilirsiniz.
\begin{verbatim}
 # lspci
\end{verbatim}

\begin{verbatim}
  xorg-video-apm 
  xorg-video-ast
  xorg-video-cirru
  xorg-video-fbdev (Bu sürücü ekran kartından bağımsızdır ve direk teste tabidir.)
  xorg-video-glint
  xorg-video-i128
  xorg-video-i740
  xorg-video-intel
  xorg-video-mach64
  xorg-video-mga
  xorg-video-neomagic
  xorg-video-r128
  xorg-video-radeon
  xorg-video-radeonhd
  xorg-video-s3
  xorg-video-s3virge
  xorg-video-savage
  xorg-video-siliconmotion
  xorg-video-sis
  xorg-video-sisusb
  xorg-video-tdfx
  xorg-video-trident,
  xorg-video-vesa (Bu sürücü tüm testçiler tarafından test edilebilir. Bilgisayarınızı güvenli kipte
                   açarak, diğer paketler ile aynı şekilde test edebilirsiniz.)
  xorg-video-vmware (vmware kullanıyor iseniz test edebilirsiniz.)
  xorg-video-voodoo
  xorg-video-chips
\end{verbatim}

\begin{itemize}
  \item Bilgisayarınızı yeniden başlatın, ve açtığınızda çözünürlüğün düzgün olduğunu gözlemleyin.

  Bilgisayarınızda bir yavaşlık olmadığını gözlemleyin
  \item Aşağıda bulunan videoyu mplayer, kaffeine gibi bir oynatıcı ile oynatın ve sorunsuz bir şekilde çalıştığını gözlemleyin.
  \begin{verbatim}
  # wget http://cekirdek.pardus.org.tr/~semen/dist/test/multimedia/video/cokluortam/DVD.mpg
  \end{verbatim}
\end{itemize}

\end{enumerate}

\section{Terminal alt Bileşeni}
\begin{enumerate}
  \item rxvt-unicode paketi kurulumu sonrası:

  Bir x konsol penceresi açıldığını gözlemleyin.
  \begin{verbatim}
  # urxvt
  \end{verbatim}

  Aşağıda bulunan ilk komut sunucuyu başlatmaktadır, bir konsolda sunucuyu başlatın diğer konsola geçip ikinci satırda bulunan komutu çalıştırın ve bir x konsol penceresi açıldığını gözlemleyin.
  \begin{verbatim}
  # urxvtd (sunucuyu çalıştırın)
  # urxvtc
  \end{verbatim}

  \item xterm paketi kurulumu sonrası:

  Aşağıda bulunan komutun bir konsol çalıştırdığını gözlemleyin.
  \begin{verbatim}
  # xterm
  \end{verbatim}
\end{enumerate}

\end{document}
