\documentclass[a4paper,10pt]{article}
\usepackage[turkish]{babel}
\usepackage[utf8]{inputenc}
\usepackage[left=1cm,top=1cm,right=2cm,bottom=2cm]{geometry}

\title{Multimedia Bileşeni Test Aşamaları}
\author{Semen Cirit}

\renewcommand{\labelenumi}{\arabic{enumi}.}
\renewcommand{\labelenumii}{\arabic{enumi}.\arabic{enumii}.}
\renewcommand{\labelenumiii}{\arabic{enumi}.\arabic{enumii}.\arabic{enumiii}.}
\renewcommand{\labelenumiv}{\arabic{enumi}.\arabic{enumii}.\arabic{enumiii}.\arabic{enumiv}.}

\begin{document}

\maketitle

\section{Sound alt Bileşeni}
\begin{enumerate}
\item frescobaldi paketi kurulumu sonrası:
 \begin{verbatim}
  # wget http://cekirdek.pardus.org.tr/~semen/dist/test/multimedia/sound/test_frescobaldi.ly
 \end{verbatim}
Yukarıda bulunan dosyayı frescobaldi ile açıp, uygulamanın sol tarafında bulunan Lilypond butonuna basın ve sorunsuz bir şekilde pdf dökümanının oluştuğunu gözlemleyin.
\item pulseaudio paketi kurulumu sonrası:
\begin{itemize}
 \item Sisteminizi tekrar başlatın ve açılış sesinin düzgün bir şekilde çıktığını gözlemleyin. 
 \item multimedia-tr.pdf amarok paketi testini gerçekleştiriniz.
\end{itemize}
\item sox paketi kurulumu sonrası:

hardware-tr.pdf k3b testini gerçekleştiriniz.
\item amarok paketi kurulumu sonrası:
 
Aşağıda bulunan zip dosyasını indirdikten sonra seslerin düzgün çıktığını gözlemleyin.
 \begin{verbatim}
  # wget wget http://cekirdek.pardus.org.tr/~semen/dist/test/multimedia/sound/sound.tar
 \end{verbatim}

\item listen paketi kurulumu sonrası: 

Uygulamayı açın dosya sistemi bölümünden /usr/kde/3.5/share/sounds/ dizinine girin ve sorunsuz bir şekilde müziklerin listelendiğini ve müzikleri çalabildiğini gözlemleyin.

\item vorbis-tools paketi kurulumu sonrası: 
 \begin{verbatim}
  # wget http://cekirdek.pardus.org.tr/~semen/dist/test/multimedia/sound/sound/game.ogg
  # oggdec game.ogg
  # mplayer game.wav
  # oggenc game.wav
  # mpleyer game.ogg
 \end{verbatim}

\item qpitch paketi kurulumu sonrası:

Kmenüden uygulamayı açın düzgün bir şekilde çalıştığını gözlemleyin.

\item qjackctl paketi kurulumu sonrası: 

Kmenüden qjackct uygulamasını açın. (Bu sırada hiçbir ses aygıtının açıkolmadığından emin olun)

Başla butonuna basın, status ve mesajlardan Jack ses sunucusunun başlatılmış olduğunu gözlemleyin.

\item lame ve lame-docs paketleri kurulumu sonrası: 

Aşağıda bulunan komutların sorunsuz çalıştığını gözlemleyin:
\begin{verbatim}
# wget http://cekirdek.pardus.org.tr/~semen/dist/test/multimedia/sound/sound/music.mp3
# lame music.mp3 music.mpeg
# mplayer music.mpeg
\end{verbatim}

\item mpg123 paketi kurulumu sonrası:

Aşağıda bulunan komutların sorunsuz bir şekilde çalıştığını gözlemleyin:
 \begin{verbatim}
  # wget http://cekirdek.pardus.org.tr/~semen/dist/test/multimedia/sound/sound/music.mp3
  # mpg123 music.mp3
 \end{verbatim}


\end{enumerate}

\section{Video alt Bileşeni}
\begin{enumerate}
 \item mplayer, vlc, kaffeine paketleri kurulumu sonrası:
 \begin{verbatim}
  # wget http://cekirdek.pardus.org.tr/~semen/dist/test/multimedia/video/cokluortam.tar
 \end{verbatim}
Yukarıda bulunan dosyayı indirdikten sonra, uygulamayı her türlü dosya formatı ile çalıştırın ve sorunsuz bir şekilde çalıştığını gözlemleyin.
 \item vlc-firefox kurulumu sonrası:
 \begin{itemize}
  \item Firefox $\rightarrow$ Düzen $\rightarrow$ Seçenekler $\rightarrow$ Eklentileri Yönet $\rightarrow$ Yan Uygulamalar bölümünde VLC eklentisinin eklenmiş olduğunu gözlemleyin.
  \item Aşağıda bulunan uzantıdaki videoları firefox üzerinden açınız. Ve düzgün bir şekilde çalıştıklarını gözlemleyiniz.
  \begin{verbatim}
  http://cekirdek.pardus.org.tr/~semen/dist/test/multimedia/video/cokluortam/  
  \end{verbatim}
 \end{itemize}
\item ffmpeg paketi kurulumu sonrası:
 
Aşağıda bulunan ikinci komutun düzgün çıktılar ürettiğini ve üçüncü komutun da bu çıktıları çalıştırabildiğini gözlemleyin.
\begin{verbatim}
  # wget http://cekirdek.pardus.org.tr/~semen/dist/test/multimedia/video/cokluortam.tar 
  # ffmpeg -i <multimedia dosyası> -r 24 <test dosyası>
  # ffplay <multimedia dosyası>
  \end{verbatim}
\item x264 paketi kurulumu sonrası:

Aşağıdaki komutların sorunsuz bir şekilde çalıştığını gözlemleyin:
\begin{verbatim}
# wget  http://cekirdek.pardus.org.tr/~semen/dist/test/multimedia/video/example.y4m.bz2
# x264 -o test.mp4 example.y4m.bz2 300x300 
# mplayer test.mp4
\end{verbatim}




\end{enumerate}


\section{Converter alt Bileşeni}
\begin{enumerate}
 \item Aşağıdaki paketler sadece kurulum testine tabidir:
\begin{verbatim}
 nrg2iso
 vnc2swf
\end{verbatim}

 \item amrwb ve amrnb paketleri kurulumu sonrası:

 multimedia-tr.pdf sox ve mplayer testlerini gerçekleştiriniz.

\item ccd2iso paketi kurulumu sonrası:
\begin{verbatim}
 # wget http://cekirdek.pardus.org.tr/~semen/dist/test/multimedia/converter/default.img
 # ccd2iso default.img test.iso
\end{verbatim}

test.iso dosyasının sorunsuz bir şekilde oluştuğunu gözlemleyin.

Sorunsuz bir şekilde iso dosyasının oluştuğunu gözlemleyin.
\item dvdbackup paketi kurulumu sonrası:
\begin{itemize}
 \item Aşağıdaki bağlantıda bulunan iso'yu DVD'ye yazdırın. 
\begin{verbatim}
 # wget http://cekirdek.pardus.org.tr/~semen/dist/test/hardware/optical/boot.iso
\end{verbatim}
 \item Eğer DVD yeniden yazdırabilir bir DVD ise /dev/dvdrw değil ise /dev/dvd girdilerini kullanarak aşağıdaki bulunan komutu çalıştırın ve sorunsuz bir şekilde DVD'nin yedeklendiğini gözlemleyin.
\begin{verbatim}
 # dvdbackup -i <girdiyolu> -o <çıktıyolu> -M
 Örnek 
 # dvdbackup -i /dev/dvdrw -o /home/pardus/dvd -M
\end{verbatim}
 
\end{itemize}
\item emovix paketi kurulumu sonrası:

Aşağıda bulunan dosyayı indirin:
\begin{verbatim}
 # wget http://cekirdek.pardus.org.tr/~semen/dist/test/multimedia/converter/default.img
\end{verbatim}
 
Komutların sorunsuz bir şeklilde çalıştıklarını gözlemleyin.
\begin{verbatim}
 # movix-version
 # movix-files
 # movix-conf
 # mkmovixiso default.img --output-file=default.iso
\end{verbatim}

\item ffmpeg2theora paketi kurulumu sonrası:
\begin{verbatim}
 # wget http://cekirdek.pardus.org.tr/~semen/dist/test/multimedia/video/cokluortam/DVD.mpg
 # ffmpeg2theora DVD.mpg
\end{verbatim}
\begin{itemize}
 \item Yukarıda bulunan komutun sorunsuz bir şekilde DVD.ogv dosyasını oluşturduğunu gözlemleyin.
 \item mplayer ile bu dosyayı çalıştırın ve sorunsuz bir şekilde çalıştığını gözlemleyin.
\end{itemize}
\item ffmpeg2theora paketi kurulumu sonrası:
\begin{verbatim}
 # wget http://cekirdek.pardus.org.tr/~semen/dist/test/multimedia/converter/lazarus.png
 # icns2png lazarus.icns
\end{verbatim}
\begin{itemize}
 \item Yukarıda bulunan komutun sorunsuz bir şekilde lazarus.png dosyasını oluşturduğunu gözlemleyin.
 \item gwenview ile bu dosyayı açın ve sorunsuz bir şekilde açıldığını gözlemleyin.
\end{itemize}
\item kaudiocreator paketi kurulumu sonrası:
\begin{itemize}
 \item Aşağıdaki bağlantıda bulunan ses dosyalarını k3b ile audio CD olarak yazdırın.
\begin{verbatim}
 # wget http://cekirdek.pardus.org.tr/~semen/dist/test/multimedia/video/cokluortam/
\end{verbatim}
 \item kaudiocreator uygulamasını çalıştırın:

Sorunsuz bir şekilde çalışıp CD'de bulunan dosyalarını listeleyebildiğini gözlemleyin.

Dosyaları seçin ve Rip butonuna basın ve bu işlemi düzgün bir şekilde yaptığını gözlemleyin.

Ev dizininiz altında rip ettiğiniz format adı (mp3, wav, ogg, flac olabilir) ile bir dizin yaratıldığını gözlemleyin, ve mplayer ile bu dosyaların sorunsuz bir şekilde çalıştığını gözlemleyin.
\end{itemize}

\item libnut paketi kurulumu sonrası:

multimedia-tr.pdf mplayer ve ffmeg testlerini gerçekleştirin.

\item mkvtoolnix paketi kurulumu sonrası:

Uygulamalar $\rightarrow$ Çokluortam $\rightarrow$ mkvmerge GUI uygulamasını açın:

Aşağıdaki bağlantıda bulunan dosyaları, uygulamayı kullanarak .mkv dosyasına dönüştürün ve mplayer ile bu dosyaların düzgün bir şekilde oluştuklarını gözlemleyin: (Ekle tuşuna basarak ilgili dosyayı ekleyin ve muxing'e başla tuşuna basın.)
\begin{verbatim}
  # wget http://cekirdek.pardus.org.tr/~semen/dist/test/multimedia/video/cokluortam.tar
 \end{verbatim}

\item mpeg2vidcodec paketi kurulumu sonrası:
\begin{verbatim}
  # mkdir flower
  # cd flower
  # wget http://cekirdek.pardus.org.tr/~semen/dist/test/multimedia/converter/flowgard.mpg
  # mpeg2decode -b flowgard.mpg -f -r -o0 sflowg.%d
  # cd ..
  # wget http://cekirdek.pardus.org.tr/~semen/dist/test/multimedia/converter/flower2.par
  # mpeg2encode flower2.par flowgard.m2v
  # mplayer flowgard.m2v
\end{verbatim}

Yukarıdaki komutların sorunsuz bir şekilde çalıştığını gözlemleyin.
\item ogmtools paketi kurulumu sonrası:

Aşağıdaki komutların düzgün bir şekilde çalıştıklarını gözlemleyin.
\begin{verbatim}
# wget http://cekirdek.pardus.org.tr/~semen/dist/test/multimedia/sound/sound/music.mp3 
# ogmmerge music.mp3 -o test.ogg
# mplayer test.ogg
\end{verbatim}

\item potrace paketi kurulumu sonrası:
\begin{verbatim}
# wget http://cekirdek.pardus.org.tr/~semen/dist/test/multimedia/converter/tepecik_01.pbm 
# potrace tepecik_01.pbm -o test.png
# gwenview test.png
\end{verbatim}

Sorunsuz bir şekilde png dosyasının oluştuğunu ve görüntülenebildiğini gözlemleyin.

\item shntool paketi kurulumu sonrası:
\begin{verbatim}
# wget http://cekirdek.pardus.org.tr/~semen/dist/test/multimedia/sound/sound/11k16bitpcm.wav
# wget http://cekirdek.pardus.org.tr/~semen/dist/test/multimedia/sound/sound/11k16bitpcm2.wav
# shncat 11k16bitpcm.wav
# shncmp 11k16bitpcm.wav 11k16bitpcm2.wav
\end{verbatim}
\item shorten paketi kurulumu sonrası:
\begin{verbatim}
 # wget http://cekirdek.pardus.org.tr/~semen/dist/test/multimedia/sound/sound/11k16bitpcm.wav
 # shorten 11k16bitpcm.wav
 # mplayer 11k16bitpcm.shn
\end{verbatim}

Sorunsuz bir şekilde .shn uzantılı dosyanın oluştuğunu ve çalıştırılabildiğini  gözlemleyin.
\item transcode paketi kurulumu sonrası:

\begin{itemize}
 \item hardware-tr.pdf k3b paketini test ediniz.
 \item Aşağıdaki komutları çalıştırın:
\begin{verbatim}
# wget http://cekirdek.pardus.org.tr/~semen/dist/test/multimedia/video/cokluortam/
Lake_dance_XviD.AVI
# transcode -i Lake_dance_XviD.AVI -y xvid -o test.avi -k -z
# mplayer test.avi
\end{verbatim}
 Oluşan .avi uzantılı dosyanın baş aşağı çalıştığını gözlemleyin.
\end{itemize}

\item vcdimager paketi kurulumu sonrası:

\begin{itemize}
 \item hardware-tr.pdf k3b paketini test ediniz.
 \item Aşağıdaki komutları çalıştırın ve sorunsuz bir şekilde çalıştıklarını gözlemleyin:
\begin{verbatim}
# wget http://cekirdek.pardus.org.tr/~semen/dist/test/multimedia/video/cokluortam/DVD.mpg
# vcdimager DVD.mpg
# vcd-info -b videocd.bin
# vcdxgen DVD.mpg
# vcdxminfo -i DVD.mpg
\end{verbatim}

\end{itemize}
\item vobcopy paketi kurulımu sonrası:


\end{enumerate}
\section{Graphics alt Bileşeni}
\begin{enumerate}
 \item gimp kurulumu sonrası:
  \begin{verbatim}
   http://cekirdek.pardus.org.tr/~semen/dist/test/multimedia/graphics/graphics.tar
  \end{verbatim}

 Yukarıda bulunan farklı formattaki dosyaları gimp ile açınız, sorunsuz bir şekilde açıldıklarını gözlemleyiniz.
\item digikam kurulumu sonrası:

  \begin{verbatim}
   http://cekirdek.pardus.org.tr/~semen/dist/test/multimedia/graphics/graphics.tar
  \end{verbatim}

 Digikam uygulaması için seçtiğiniz dosya dizinine yukarıda klasör içinde bulunan dosyaları kopyalayınız ve sorunsuz bir şekilde çalıştığını gözlemleyiniz.
\item imagemagick kurulumu sonrası:
  \begin{verbatim}
   # wget http://cekirdek.pardus.org.tr/~semen/dist/test/multimedia/graphics/graphics.tar
  \end{verbatim}

Yukarıdaki bağlantıda bulunan dosyaların aşağıda bulunan komutlar ile düzgün çalıştığını gözlemleyiniz.
  \begin{verbatim}
   # animate test_animate.gif
   # diplay test.*
  \end{verbatim}
\item tuxpaint, tuxpaint-stamps ve tuxpaint-doc paketleri kurulumu sonrası:
  \begin{itemize}
   \item  Uygulamayı açın çeşitli izimler yapın ve kaydedin, sorunsuz bir şekilde çalıştığını ve ilgili yerlerde ses dosyaları çalıştırdığını gözlemleyin.
   \item Stamps butonuna basın ve sağ tarafta bulunan pullardan birini eklemeye çalışın sorunsuz bir şekilde eklendiğini gözlemleyin.
   \item Aşağıda bulunan komutu çalıştırın ve daha sonra uygulamanın sol tarafında bulunan open butonuna basın ve import ettiğiniz resmin uygulmada görüntülenmiş olduğunu gözlemleyin.
\begin{verbatim}
   # tuxpaint-import /usr/share/tuxpaint/stamps/vehicles/ship/walnutBoat.png
\end{verbatim} 
\end{itemize}

\item inkscape paketi kurulumu sonrası:

Aşağida bulunan dosyayı inkscape uygulaması ile açın ve üzerinde birkaç değişiklik yapın. Sorunsuz bir şekilde çalışabildiklerini gözlemleyin.
\begin{verbatim}
# wget http://cekirdek.pardus.org.tr/~semen/dist/test/multimedia/graphics/drawing.svg 
\end{verbatim}

\item asymptote paketi kurulumu sonrası:

Aşağıda bulunan komutların sorunsuz bir şekilde çalıştığını gözlemleyin:
\begin{verbatim}
# wget http://cekirdek.pardus.org.tr/~semen/dist/test/multimedia/graphics/test_asymptote.tex
# latex test_asymptote
# asy test_asymptote
# latex test_asymptote
# okular test_asymptote.dvi
\end{verbatim}
\item  dcmtk paketi kurulumu sonrası:

Aşağıda bulunan komutlaruın düzgün bir şekilde çalıştırğını gözlemleyin:
\begin{verbatim}
# wget http://cekirdek.pardus.org.tr/~semen/dist/test/multimedia/graphics/test_dcmtk.dcm 
# dcmj2pnm test_dcmtk.dcm  test.png
# gwenview test.png
\end{verbatim}



\end{enumerate}
\section{Editor alt Bileşeni}
\begin{enumerate}
 \item avidemux-common paketi kurulumu sonrası:

 multimedia-tr.pdf avidemux testini gerçekleştiriniz.

 \item avidemux ve avidemux-qt paketleri kurulumu sonrası:

\begin{verbatim}
 # wget http://cekirdek.pardus.org.tr/~semen/dist/test/multimedia/video/cokluortam/Lake_dance_XviD.AVI
 # wget http://cekirdek.pardus.org.tr/~semen/dist/test/multimedia/video/cokluortam/MPEG-1_with_
VCD_extensions.mpeg
\end{verbatim}
Yukarıda bulunan dosyaları ilgili uygulama ile açın. Go $\rightarrow$ Play/Stop düymesine basın ve açtığınız videonun ses ve görüntü bakımından sorunsuz bir şekilde çalıştığını gözlemleyin.
\item avidemux-cli paketi kurulumu sonrası:

video.mpeg dosyasının sorunsuz bir şekilde oluşmuş olduğunu gözlemleyin.
\begin{verbatim}
 # wget http://cekirdek.pardus.org.tr/~semen/dist/test/multimedia/video/cokluortam/Lake_dance_XviD.AVI
 # avidemux2_cli --force-alt-h264 --load "Lake_dance_XviD.AVI" --save "video.mpeg" 
--output-format MPEG --quit 
 # mplayer video.mpeg
\end{verbatim}

\end{enumerate}

\end{document}

