\documentclass[a4paper,10pt]{article}
\usepackage[left=1cm,top=1cm,right=2cm,bottom=1cm]{geometry}
\title{Package Manager Test Cases}
\author{Semen Cirit}

\renewcommand{\labelenumi}{\arabic{enumi}.}
\renewcommand{\labelenumii}{\arabic{enumi}.\arabic{enumii}.}
\renewcommand{\labelenumiii}{\arabic{enumi}.\arabic{enumii}.\arabic{enumiii}.}
\renewcommand{\labelenumiv}{\arabic{enumi}.\arabic{enumii}.\arabic{enumiii}.\arabic{enumiv}.}

\begin{document}

\maketitle

\begin{enumerate}
    \item Try to open package manager from system settings
    \item Try to open package manager from Kmenu.
    

    The box on the left shows the package groups and each group contain related components' packages.

    These group-component couple examples is important for the following testcases.
      \begin{table}[h]
 	  \centering
 	  \begin{tabular}{|c|c|}
 		  \hline
 		  Group & Component \\
 		  Virtualization & hardware.virtualization \\
 		  Servers & server.database \\
 		  \hline
 	  \end{tabular} 
 	  \label{tab:tbl}
     \end{table}
  
   \item Toolboxes:
    \begin{enumerate}
        \item File toolbox
    
        Dropdown alternatives
        \begin{enumerate}
            \item Try to Show new packages
            
            Execute the following commands, with the example components above in the table.
            \begin{verbatim}
            $ pisi la -Uc <component>
            \end{verbatim}
            Compare package-manager and console results for related component.
            \item Try to Show installed packages
            
            Execute the following commands, with the example components above in the table.
            \begin{verbatim}
            $ pisi li -c <component>
            \end{verbatim}
            Compare package-manager and console results for related component.
    
            \item Try to Show upgradable packages
            
            Execute the following commands, with the example components above in the table.
            \begin{verbatim}
            $ pisi lu -c <component>
            \end{verbatim}
            Compare package-manager and console results for related component.
    
            \item Click quit dropdown menu. 
    
            Observe the situation.
        \end{enumerate}
    
        \item Settings Toolbox
        Dropdown alternatives
        \begin{enumerate}
            \item Try to enable and then disable show toolbar. Observe the situation.
            \item Try to enable and then disable show statusbar. Observe the situation.
            \item Configure shortcuts
            \begin{enumerate}
                \item Try to search for an action. Observe the situation.
    
                Select an alternative from the package-manager list
                \item Shortcut action
                \begin{itemize}
                    \item Try to change a shortcut. Observe the situation.
                    \item Try to add a custom shortcut. Observe the situation.
                    \item Try to remove shortcut. Observe the situation.
                \end{itemize}
    
                \item Alternative action
                \begin{itemize}
                    \item Try to change a shortcut. Observe the situation.
                    \item Try to add a custom shortcut. Observe the situation.
                    \item Try to remove shortcut. Observe the situation.
                \end{itemize}
            \end{enumerate}
            \item Configure toolbar
            \begin{enumerate}
                \item Try to add an action 
                \item Try to remove an action
                \item Try to take up an action
                \item Try to take down an action
                Observe all situations
            \end{enumerate}
            \item Configure package manager
            \begin{enumerate}
                \item General Settings tab
                \begin{itemize}
                    \item Try to show only desktop applications

                    Observe that "True" is assigned to "ShowOnlyGuiApp" variable in .kde/share/config/package-managerrc file.

                    \item Try to enable system tray icon

                    Observe that "True" is assigned to "SystemTray" variable in .kde/share/config/package-managerrc file.

                    \item Try to enable periodic update check
                    \begin{itemize}
                        \item Observe that "True" is assigned to "UpdateCheck" variable in .kde4/share/config/package-managerrc file.
                        \item Put a observable time interval, observe it from package manager,
                        \item Also observe this time interval is added to "UpdateCheckInterval" variable in .kde/share/config/package-managerrc file.
                    \end{itemize}

                    \item Try to enable bandwidth limit

                    Try to limit bandwidth and install some packages with package-manager, observe the limit from installation progressbar. And compare it with bandwith\_limit value in /etc/pisi/pisi.conf file.
                \end{itemize}
                \item Cache tab
                \begin{itemize}
                    \item Update check part
                    \begin{itemize}
                        \item Enable use hard disk cache for downloaded software

                        Observe \texttt{package\_cache = True} in /etc/pisi/pisi.conf file.

                        \item Try to give a value to maximum cache size

                        Observe maximum cache size is added to "package\_cache\_limit" variable in /etc/pisi/pisi.conf
                    \end{itemize}
                    \item Clear Cache part
                    \begin{itemize}
                        \item Try Clean disk cache now

                        Execute the following commands: 
                        \begin{verbatim}
                        # ls -l /var/pisi
                        \end{verbatim}

                        Look at size of them 

                        Press Clean disk cache now button

                        Execute the following commands again:

                        \begin{verbatim}
                        # ls -l /var/cache/pisi
                        \end{verbatim}

                        If you install any debug package, this directory will contain a packages-debug directory, if not it will be empty.
                    \end{itemize}
                \end{itemize} 
                \item Repositories tab
                \begin{itemize}
                    \item Try to add a repository
                    \item Try to remove a repository
                    \item Try to take up a repository
                    \item Try to take down a repository 

                    Observe all situations.
                    \item Execute the following command to list the repositories and compare the results with package-managers'.
                    \begin{verbatim}
                    # pisi lr
                    \end{verbatim}
                \end{itemize}
                \item Proxy tab
                \begin{itemize}
                    \item  Select No proxy server 
                    Look /etc/pisi/pisi.conf file, observe these variables are like that
                    \begin{verbatim}
                    http_proxy = None
                    https_proxy = None
                    ftp_proxy = None
                    \end{verbatim}
    
                    \item Select Use proxy server
    
                    \begin{itemize}
                        \item Try to add a HTTP server and enable use proxy server for all protocols.
        
                        Observe the related results of /etc/pisi/pisi.conf
        
                        \item Try to add a HTTP, a HTTPS and a FTP server
        
                        Observe the related results of /etc/pisi/pisi.conf
                    \end{itemize}
                \end{itemize}
                \item Help toolbox
    
                Dropdown alternatives
                \begin{itemize}
                    \item KDE help center (testcases will added later)
                    \item What's this? 
        
                    Try to take the description of each widget on the screen.
                    %\item Report bug
                    %\begin{itemize}
                    %    \item Choose a severity for the an example bug
                    %    \item Add a subject and description and send it
                    %    \item Control it from bugzilla
                    %\end{itemize}
                    \item Switch application language
                    \begin{itemize}
                        \item Change primary language and observe the situation.
                        \item Try to add a fallback language and observe the situation.
                        \item Try to remove the fallback language and observe the situation.
                    \end{itemize}
                \end{itemize}
        \end{enumerate}
    \end{enumerate}
\end{enumerate}

    \item Toolbar
    \begin{itemize}
        \item Try to right click on toolbar
        \item Try to change the oriantation, observe each result
        \item Try to change text position, observe each result.
    
            Observe also that selected style is assigned to "ToolButtonStyle" variable in .kde/share/config/package-managerrc file.
        \item Try to change the icon size, observe the each result.
        \item Try to lock toolbar, then unlock toolbar
    
            Then right click again and try to change some feature.
        \item Try to enable and disable toolbar, observe each result.
        \item Try to configure toolbar and test again toolboxes-> configure toolbar part of testcases.
    \end{itemize}
    \item Installing Package
	You can observe whether the package installed with this command:
	\begin{verbatim}
	 pisi info <package-name>
	\end{verbatim}

        \begin{itemize}
            \item After showing new packages. 
            \begin{itemize}
                \item Select a package from package list, try to install.

                Observe that the dependencies is listed on the installation window.

                \item Select all, try to install all

Use below command in order to see the selected packages were installed.
	\begin{verbatim}
	 pisi hs
	\end{verbatim}
                
            \end{itemize}
            \item After showing upgradable packages.
            \begin{itemize}
                \item Select a package from package list, try to install.
Use below command in order to see the selected packages were installed.
	\begin{verbatim}
	 pisi hs
	\end{verbatim}
                \item Select all, try to install all
Use below command in order to see the selected packages were installed.
	\begin{verbatim}
	 pisi hs
	\end{verbatim}
            \end{itemize}

	\item During some operation, cancel the authentication window.

	Observe that the operation remain in the previous state.

	\item After installing some package try to install a gui application (for example kaffein)and for this time enable remembering password when the system policykit screen open.
		
		Observe that after installing this package a screen is opened which contains this package icon. Click on it and try to open it.
	\item Try to install a package when your password remembered.
		
		Observe that the package manager doesn't ask password and install the package without any problem.
        \end{itemize}
    \item Removing packages
	You can observe whether the package removed with this command:
	\begin{verbatim}
	 pisi info <package-name>
	\end{verbatim}
        \begin{itemize}
            \item After showing installed packages.
            \begin{itemize}
                \item Select a package from package list, try to remove.
                \item Select all, try to remove all.
            \end{itemize}
        \end{itemize}
    \end{enumerate}

\end{document}
