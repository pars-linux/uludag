\documentclass[a4paper,10pt]{article}

\title{Network Manager Test Cases}	

\renewcommand{\labelenumi}{\arabic{enumi}.}
\renewcommand{\labelenumii}{\arabic{enumi}.\arabic{enumii}.}
\renewcommand{\labelenumiii}{\arabic{enumi}.\arabic{enumii}.\arabic{enumiii}.}
\renewcommand{\labelenumiv}{\arabic{enumi}.\arabic{enumii}.\arabic{enumiii}.\arabic{enumiv}.}

\begin{document}

\maketitle

\begin{enumerate}
  \item Push create profile button
    \begin{enumerate}
     \subsection*{For ethernet or wireless connections}
	
      \item Try to create a new profile.
	\begin{enumerate}
	  \item  Without adding a new name

		Observe that it gives a warning about the profile name and prevents to create the new profile.
	  \item Add a new name and try to create this profile.

		Observe that it can be added without any problem.
	\end{enumerate}
      
     \subsection*{For wireless connections}
     \item Try to scan for a wireless network
      
	Observe that it can find wireless networks.
     \item Disable hide password.
        
	Observe the situation.
      \item Try to add this wireless network password and 
	
      Observe the situation.
     
     \end{enumerate}
     \item Configuring Profiles (Push the profile configuration button on the right part of each profile.)
    
      \begin{enumerate}
      \item Observe your ethernet or wireless device is in the list of Device dropbox.

	  You can execute this commmand in order to observe this devices:
	  \begin{verbatim}
	  network devices
	  \end{verbatim}
	  Compare the output of the command with the devices in the combobox.

	  \item Try to change the profile name
	    \begin{verbatim}
	     network connections
	    \end{verbatim}
	    And observe that its name was changed.

	\newpage    
    !!!Apply each of the below cases randomly and observe the output of the below command after applying each case.
		\begin{verbatim}
		 ping 4.2.2.1
		\end{verbatim}
	If the network reachable it will give an output like this:
	\begin{verbatim}
	PING 4.2.2.1 (4.2.2.1) 56(84) bytes of data.
	64 bytes from 4.2.2.1: icmp_seq=1 ttl=244 time=74.8 ms
	64 bytes from 4.2.2.1: icmp_seq=2 ttl=244 time=83.9 ms
      	\end{verbatim}

	  \item Network settings

	    Execute this command for listing IP address and network mask:
		  \begin{verbatim}
		    ifconfig
		  \end{verbatim}
	   Then observe that those variables of the above command:
		   \begin{verbatim}
		    Variable "inet addr" means the IP address
		    Variable "Mask" means network mask
		  \end{verbatim}
	   Execute this command for listing gateway:
	    	   \begin{verbatim}
		     route -n 
		  \end{verbatim}
	   Interested Gateway is for the Genmask which is equal to 0.0.0.0

	      \begin{enumerate} 
	      \item Use DHCP
		\begin{itemize}
		  \item Use automatic IP
				      
		   Execute the above two command and observe that your related device takes the automatic settings. 

		  \item Use custom address and network mask. 
		
		   Execute the first above command and observe that your related device takes the IP address and network mask that you gave.
  
		  \item Use custum default gateway
		     
		      Execute the second above command and observe that your related device takes the gateway that you gave.
		\end{itemize}
	      \item Use Manual Settings
		    
		      Execute the above two command and observe that your related device takes the settings that you gave. 
	    \end{enumerate}
	  \item Name servers
	    \begin{itemize}
		 \item Use Default
		 
		  Observe the situation
		 \item Use Automatic
		  Observe the situation
	  
		 \item Use Custom

		  Give a custom nameserver and observe it from:
		   \begin{verbatim}
		     cat /etc/resolv.conf
		  \end{verbatim}
	     \end{itemize}
   
	  
	
      \end{enumerate}
      \item Try to remove a profile. ((Push the remove button on the right part of each profile.) )
      
	    Execute this command:
	    \begin{verbatim}
	     network connections
	    \end{verbatim}
	    And observe that it was removed from the output of this command.

\end{enumerate}
\end{document}
